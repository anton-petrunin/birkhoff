\documentclass{article}
\usepackage[english]{babel}
\usepackage[utf8]{inputenc}
\usepackage{lpic}
\usepackage[usenames,dvipsnames]{xcolor}
\usepackage{punk}
\usepackage[T1]{fontenc}
\pagestyle{empty}
\usepackage{tikz}
\usepackage[papersize={12.7094in,9.250in}, spine=0.432384in, cropgap=0.125in
%,cropmarks, cropframe
]{zwpagelayout}%0.125" + 6" + 192*0.002252" + 6" + .125" =  12.682384
\linespread{1}
\definecolor{orange}{RGB}{250,205,25}
\pagecolor{orange}
\begin{document}
\begin{lpic}[t(5in),b(0mm),r(0mm),l(7.5in)]{lambert-bug-white(1)}
\lbl[b]{17,105;{\hbox{
\hspace{3em}\parbox{.3\textwidth}{\begin{center}
        \ttfamily\bfseries\Huge  Euclidean plane and its relatives                           
                                  \end{center}
}}}}
\lbl{23,95;{\ttfamily\LARGE A minimalist introduction}}
\lbl{23,85;{\ttfamily\large Anton Petrunin}}
\lbl[]{-59.8,113,-90;{\ttfamily\bfseries\LARGE  EUCLIDEAN PLANE AND ITS RELATIVES}}
\lbl{-59.8,-5,-90;{\ttfamily\large Anton Petrunin}}
\lbl{21,-5;{\ttfamily\large semistable edition}}
\lbl{-150,100;{\hbox{
\hspace{3em}\parbox{.4\textwidth}
{\ttfamily\large 
The book is designed for a semester-long course in Foundations of Geometry and meant to be rigorous, conservative, elementary and minimalist.
\medskip
\\
Introduction\\
\hbox{\ \phantom{1}}1. Preliminaries
\medskip
\\ 
Euclidean geometry\\
\hbox{\ \phantom{1}}2. The Axioms\\
\hbox{\ \phantom{1}}3. Half-planes\\
\hbox{\ \phantom{1}}4. Congruent triangles\\
\hbox{\ \phantom{1}}5. Perpendicular lines\\
\hbox{\ \phantom{1}}6. Similar triangles\\
\hbox{\ \phantom{1}}7. Parallel lines\\
\hbox{\ \phantom{1}}8. Triangle geometry
\medskip
\\
Inversive geometry\\
\hbox{\ \phantom{1}}9. Inscribed angles\\
\hbox{\ }10. Inversion
\medskip
\\
Non-Euclidean geometry\\
\hbox{\ }11. Neutral plane\\
\hbox{\ }12. Hyperbolic plane\\
\hbox{\ }13. Geometry of h-plane
\medskip
\\
Additional topics
\\
\hbox{\ }14. Affine geometry\\
\hbox{\ }15. Projective geometry\\
\hbox{\ }16. Spherical geometry\\
\hbox{\ }17. Projective model\\
\hbox{\ }18. Complex coordinates\\
\hbox{\ }19. Geometric constructions\\
\hbox{\ }20. Area 
}}}}
\lbl{-59.8,32,90;{\includegraphics[scale=0.20]{lambert-bug-2-white}}}
\lbl[lb]{-217,-37;\parbox{.4\textwidth}
{\ttfamily\large {\ttfamily\large updated on 2019-01-17\\ http://anton-petrunin.github.io/birkhoff/
}}}
\end{lpic}

\end{document}

\chapter{Projective model}\label{chap:klein}
\addtocontents{toc}{\protect\begin{quote}}

The {}\emph{projective model} is another model of hyperbolic plane discovered by Beltrami; it is often called {}\emph{Klein model}.
The projective and conformal models are saying exactly the same thing but in two different languages. 
Some problems in hyperbolic geometry
admit simpler proof using the projective model 
and
others have simpler proof in the conformal model.
Therefore, it worth to know both. 

\section*{Special bijection of the h-plane to itself}
\addtocontents{toc}{Special bijection of the h-plane to itself.}

Consider the conformal disc model with the absolute at the unit circle $\Omega$ centered at~$O$.
Choose a coordinate system $(x,y)$ on the plane with the origin at $O$, 
so the circle $\Omega$ is described by the equation $x^2+y^2=1$.

\label{pic:stereographic_projection-klein}
\begin{wrapfigure}{o}{48mm}
\begin{lpic}[t(-4mm),b(-0mm),r(0mm),l(0mm)]{pics/stereographic_projection-klein(1)}
\lbl[rb]{23,24;$O$}
\lbl[br]{31,24;$P$}
\lbl[tr]{23,41.5;$N$}
\lbl[br]{23,4;$S$}
\lbl[lb]{37.5,39;$P'$}
\lbl[t]{36.5,21;$\hat P$}
\lbl[br]{8,37;$\Sigma$}
\lbl[t]{10,22;$\Pi$}
\end{lpic}
\caption*{The plane thru $P$, $O$ and $S$.}
\end{wrapfigure}

Let us think that our plane is the coordinate $xy$-plane in the Euclidean space; denote it by $\Pi$.
Let $\Sigma$ be the unit sphere centered at $O$;
it is described by the equation 
$$x^2+y^2+z^2=1.$$
Set $S=(0,0,-1)$ and $N=(0,0,1)$; 
these are the south and north poles of~$\Sigma$.

Consider stereographic projection $\Pi\to\Sigma$ from $S$;
given point $P\in\Pi$ denote its image in $\Sigma$ by~$P'$.
Note that the  h-plane is mapped to the {}\emph{north hemisphere};
that is, to the set of points $(x,y,z)$ in $\Sigma$ described by the inequality $z>0$.

For a point $P'\in \Sigma$ consider its foot point $\hat P$
on $\Pi$;
this is the closest point to~$P'$.

The composition $P\z\leftrightarrow P'\z\leftrightarrow\hat P$ of these two maps
is a bijection of the h-plane to itself.

Note that $P=\hat P$
 if and only if  $P\in \Omega$ or $P=O$.

\begin{thm}{Exercise}\label{ex:P-->hat-P}
Show that the bijection $P\leftrightarrow \hat P$ described above can be 
described the following way: 
set $\hat O=O$ and for any other point $P$ take $\hat P\in [OP)$
such that 
$$O\hat P=\frac{2\cdot x}{1+x^2},$$
where $x=OP$. 
\end{thm}

\begin{thm}{Lemma}\label{lem:P-hat-chord}
Let $(PQ)_h$ be an h-line with the ideal points $A$ and~$B$.
Then $\hat P,\hat Q\in[AB]$.

Moreover, 
$$\frac{A\hat Q\cdot B\hat P}{\hat QB\cdot \hat PA}
=
\left(\frac{AQ\cdot BP}{QB\cdot PA}\right)^2.
\eqlbl{eq:lem:P-hat-chord}$$
In particular, if $A,P,Q,B$ appear on the line in the same order, then
$$PQ_h=\tfrac12\cdot\ln\frac{A\hat Q\cdot B\hat P}{\hat QB\cdot \hat PA}.$$
\end{thm}

\parit{Proof.}
Consider the stereographic projection $\Pi\to \Sigma$ from the south pole $S$.
Let $P'$ and $Q'$ denotes the images of $P$ and~$Q$.


\begin{wrapfigure}[11]{o}{35mm}
\begin{lpic}[t(-0mm),b(-4mm),r(0mm),l(2mm)]{pics/lambda-plane(1)}
\lbl[rt]{0,15;$A$}
\lbl[lt]{32,15;$B$}
\lbl[t]{8,14;$\hat P$}
\lbl[br]{8,29;$P'$}
\end{lpic}
\caption*{The plane~$\Lambda$.}
\end{wrapfigure}

According to Theorem~\ref{thm:inverion-3d}\textit{\ref{thm:inverion-3d:cross-ratio}},
$$\frac{AQ\cdot BP}{QB\cdot PA}=\frac{AQ'\cdot BP'}{Q'B\cdot P'A}.\eqlbl{eq:(AB;PQ)=(AB;P'Q')}$$

By Theorem~\ref{thm:inverion-3d}\textit{\ref{thm:inverion-3d:angle}}, 
each circline in $\Pi$ which is perpendicular to $\Omega$ 
is mapped to a circle in $\Sigma$ which is still perpendicular to~$\Omega$.
It follows that the stereographic projection sends $(PQ)_h$ to the intersection of the north hemisphere of $\Sigma$ with a plane perpendicular to~$\Pi$.

Let $\Lambda$ denotes the plane;
it contains the points $A$, $B$, $P'$, $\hat P$  and the circle $\Gamma=\Sigma\cap\Lambda$.
(It also contains $Q'$ and $\hat Q$ but we will not use these points for a while.)



Note that 
\begin{itemize}
\item 
$A,B,P'\in\Gamma$,
\item $[AB]$ is a diameter of $\Gamma$,
\item $(AB)=\Pi\cap\Lambda$,
\item $\hat P\in [AB]$
\item $(P'\hat P)\perp (AB)$.
\end{itemize}

Since $[AB]$ is the diameter of $\Gamma$, 
by Corollary~\ref{cor:right-angle-diameter},
the angle $AP'B$ is right. 
Hence $\triangle A\hat PP'\z\sim \triangle AP'B\z\sim \triangle P'\hat PB$.
In particular
$$\frac{AP'}{BP'}=\frac{A\hat P}{P'\hat P}=\frac{P'\hat P}{B\hat P}.$$
Therefore
$$\frac{A\hat P}{B\hat P}=\left(\frac{AP'}{BP'}\right)^2.\eqlbl{eq:AP/BP}$$

The same way we get that
$$\frac{A\hat Q}{B\hat Q}=\left(\frac{AQ'}{BQ'}\right)^2.\eqlbl{eq:AQ/BQ}$$
Finally, note that
\ref{eq:(AB;PQ)=(AB;P'Q')}+\ref{eq:AP/BP}+\ref{eq:AQ/BQ} imply \ref{eq:lem:P-hat-chord}.

The last statement follows from \ref{eq:lem:P-hat-chord} and the definition of h-distance.
Indeed,
\begin{align*}
PQ_h&\df\ln\frac{A Q\cdot B P}{QB\cdot PA}=
\\
&=\ln\left(\frac{A \hat Q\cdot B \hat P}{\hat QB\cdot \hat PA}\right)^{\frac12}=
\\
&=\tfrac12\cdot\ln\frac{A \hat Q\cdot B \hat P}{\hat QB\cdot \hat PA}.
\end{align*}
\qedsf

{

\begin{wrapfigure}[10]{o}{51mm}
\begin{lpic}[t(-5mm),b(-0mm),r(0mm),l(0mm)]{pics/ex-hex(1)}
\lbl[t]{26,8;$A_1$}
\lbl[b]{26,31.5;$B_1$}
\lbl[tl]{31,19;$A_2$}
\lbl[br]{15,35;$B_2$}
\lbl[tr]{15,5;$A_3$}
\lbl[lw]{29,28;$B_3$}
\lbl[lt]{46,10;$\Gamma_1$}
\lbl[r]{42,38;$\Gamma_2$}
\lbl[b]{49,31;$\Gamma_3$}
\lbl[lb]{3,15;$\Omega$}
\end{lpic}
\end{wrapfigure}

\begin{thm}{Exercise}\label{ex:hex}
Let $\Gamma_1$, $\Gamma_2$ and $\Gamma_3$ 
be three circles perpendicular to the circle~$\Omega$.
Let $[A_1B_1]$, $[A_2B_2]$ and $[A_3B_3]$ denote
the common chords of $\Omega$ and $\Gamma_1$, $\Gamma_2$, $\Gamma_3$ correspondingly.
Show that the chords $[A_1B_1]$, $[A_2B_2]$ and $[A_3B_3]$ intersect at one point inside $\Omega$ if and only if $\Gamma_1$, $\Gamma_2$ and $\Gamma_3$ intersect at two points.
\end{thm}

}


\section*{Projective model}
\addtocontents{toc}{Projective  model.}

The following picture illustrates the map $P\mapsto \hat P$ described in the previous section --- if you take the picture on the left and apply the map $P\mapsto \hat P$,
you get the picture on the right.
The pictures are {}\emph{conformal} and \index{projective model}\emph{projective model} of the hyperbolic plane correspondingly.
The map $P\mapsto \hat P$ is a ``translation'' from one to another.

\begin{figure}[h!]
\centering
\begin{lpic}[t(0mm),b(3mm),r(0mm),l(0mm)]{pics/poincare-klein(1)}
\lbl[t]{23,0;Conformal model}
\lbl[t]{72,0;Projective model}
\end{lpic}
\end{figure}

In the projective model things look different;
some become simpler,
other things become more complicated.

\parbf{Lines.}
The h-lines in the projective model are chords of the absolute;
more precisely, chords without its endpoints.

\parbf{Circles and equidistants.}
The h-circles and equidistants in the projective model are certain type of ellipses and their open arcs.

It follows since the stereographic projection sends circles on the plane to circles on the unit sphere and the foot point projection of circle back to the plane is an ellipse.
(One may define ellipse as a foot point projection of a circle.)

{

\begin{wrapfigure}{r}{43mm}
\begin{lpic}[t(-0mm),b(-2mm),r(1mm),l(0mm)]{pics/h-klein-perp(1)}
\lbl[rb]{5,32;$A$}
\lbl[lb]{36,32;$B$}
\lbl[lt]{13,29;$P$}
\lbl[rt]{28,29;$Q$}
\end{lpic}
\end{wrapfigure}

\parbf{Distance.}
Consider a pair of h-points $P$ and $Q$.
Let $A$ and $B$ be the ideal point of the h-line in projective model;
that is, $A$ and $B$ are the intersections of the Euclidean line $(PQ)$ with the absolute.

Then by Lemma~\ref{lem:P-hat-chord},
$$PQ_h=\tfrac12\cdot\ln\frac{AQ\cdot BP}{QB\cdot PA},\eqlbl{eq:proj-h-dist}$$
assuming the points $A, P, Q, B$ appear on the line in the same order.

}

\parbf{Angles.}
The angle measures in the projective model are very different from the Euclidean angles and it is hard to figure out by looking on the picture.\label{klein-angles}
For example all the intersecting h-lines on the picture are perpendicular.
There are two useful exceptions:

\begin{itemize}
\item If $O$ is the center of the absolute, then 
$$\measuredangle_hAOB=\measuredangle AOB.$$
\item If $O$ is the center of the absolute 
and 
$\measuredangle OAB\z=\pm\tfrac\pi2$, then 
$$\measuredangle_h OAB=\measuredangle OAB=\pm\tfrac\pi2.$$
\end{itemize}

To find the angle measure in the projective model,
you may apply a motion of the h-plane which moves 
the vertex of the angle to the center of the absolute;
once it is done the hyperbolic and Euclidean angles have the same measure.

\parbf{Motions.}
The motions of the h-plane in the conformal and projective models are relevant to inversive transformations and projective transformation in the same way.
Namely: 
\begin{itemize}
\item Inversive transformations which preserve the h-plane describe motions of the h-plane in the conformal model.
\item Projective transformations which preserve h-plane describe motions of the h-plane in the projective model.
\end{itemize}

The following exercise is a hyperbolic analog of Exercise~\ref{ex:s-medians}. 
This is the first example of a statement which admits an easier proof using  the projective model.

\begin{thm}{Exercise}\label{ex:h-median}
Let $P$ and $Q$ be the points in h-plane which lie on the same distance from the center of the absolute.
Observe that in the projective model, h-midpoint of $[PQ]_h$ coincides with the Euclidean midpoint of $[PQ]_h$.

Conclude that if an h-triangle is inscribed in an h-circle, then its medians meet at one point.

Recall that an h-triangle might be also inscribed in a horocycle or an equidistant.
Think how to prove the statement in this case.
\end{thm}

\begin{wrapfigure}[6]{o}{31mm}
\begin{lpic}[t(-3mm),b(-0mm),r(0mm),l(0mm)]{pics/perp-klein(1)}
\lbl[t]{8,12.6,-10;$m$}
\lbl[l]{14.2,8;$\ell$}
\lbl[tl]{20,6;$s$}
\lbl[bl]{20,14;$t$}
\lbl[t]{26,8;$Z$}
\end{lpic}
\end{wrapfigure}

\begin{thm}{Exercise}\label{ex:klein-perp}
Let $\ell$ and $m$ are  h-lines in the projective model.
Let $s$ and $t$ denote the Euclidean lines tangent to the absolute
at the ideal points of $\ell$. 
Show that 
if the lines $s$, $t$ and the extension of $m$ intersect at one point, then $\ell$ and $m$ are perpendicular h-lines. 
\end{thm}

\begin{thm}{Exercise}\label{ex:klein-for-angle-parallelism}
Use the projective model to derive the formula for angle of parallelism  (Proposition~\ref{prop:angle-parallelism}). 
\end{thm}

\begin{thm}{Exercise}\label{ex:klein-inradius}
Use projective model to find the inradius of the ideal triangle.
\end{thm}

The projective model of h-plane can be used to give another proof of the hyperbolic Pythagorean theorem (\ref{thm:pyth-h-poincare}).

\begin{wrapfigure}{o}{24mm}
\begin{lpic}[t(-3mm),b(-0mm),r(0mm),l(0mm)]{pics/h-pyth-klein(1)}
\lbl[br]{3,23;$A$}
\lbl[lb]{15,28.5;$B$}
\lbl[tl]{15,21;$C$}
\lbl[l]{15,25;$s$}
\lbl[t]{8,20;$t$}
\lbl[b]{8,26,37;$u$}
\lbl[bl]{15,39;$X$}
\lbl[tl]{15,4;$Y$}
\end{lpic}
\end{wrapfigure}

First let us recall its statement:
\[\cosh c=\cosh a\cdot\cosh b,\eqlbl{eq:hyp-pyth-proj}\]
where $a\z=BC_h$, $b=CA_h$ and $c=AB_h$ and
$\triangle_hACB$ is a triangle in h-plane with right angle at~$C$.

Note that we can assume that $A$ is the center of the absolute.
Set 
$s=BC$, $t =CA$, $u= AB$.
According to the Euclidean Pythagorean theorem (\ref{thm:pyth}), we have
$$u^2=s^2+t^2.\eqlbl{eq:hyp-proj}$$
It remains to express $a$, $b$ and $c$ using $s$, $u$ and $t$ and show that \ref{eq:hyp-proj} implies \ref{eq:hyp-pyth-proj}.

\begin{thm}{Advanced exercise}\label{ex:pyth-h-proj}
Finish the proof of hyperbolic Pythagorean theorem (\ref{thm:pyth-h-poincare}) indicated above.
\end{thm}


\section*{Bolyai's construction}
\addtocontents{toc}{Bolyai's construction.}

Assume we need to construct a line thru $P$ asymptotically parallel to the given line $\ell$ in the h-plane.

If $A$ and $B$ are ideal points of $\ell$ in the projective model, 
then we could simply draw the Euclidean line $(PA)$.
However the ideal points do not lie in the h-plane, therefore there is no way to use them in the construction.

In the following construction we assume that you know a compass-and-ruler construction of the perpendicular line; see Exercise~\ref{ex:construction-perpendicular}.

\begin{thm}{Bolyai's construction}
\begin{enumerate}
\item Drop a perpendicular from $P$ to~$\ell$; denote it by~$m$.
Let $Q$ be the foot point of $P$ on~$\ell$.
\item Drop a perpendicular from $P$ to~$m$; denote it by~$n$.
\item Mark by $R$ a point on $\ell$ distinct from $Q$.
\item Drop a perpendicular from $R$ to~$n$; denote it by~$k$. 
\item Draw the circle $\Gamma_2$ with center $P$ and the radius $QR$. 
Mark by $T$ a point of intersection of $\Gamma_2$ with~$k$.
\item The line $(PT)_h$ is asymptotically parallel to~$\ell$.
\end{enumerate}
\end{thm}

\begin{thm}{Exercise}\label{ex:Boyai-in-Euclid}
Explain what happens if one performs the Bolyai construction in the Euclidean plane.
\end{thm}

To prove that Bolyai's construction gives the asymptotically parallel line in the h-plane,
it is sufficient to show the following.

\begin{thm}{Proposition}\label{prop:boyai}
Assume $P$, $Q$, $R$, $S$, $T$ be points in h-plane
such that 
\begin{itemize}
\item $S\in (RT)_h$,
\item $(PQ)_h\perp (QR)_h$,
\item $(PS)_h\perp(PQ)_h$,
\item $(RT)_h\perp (PS)_h$ and 
\item $(PT)_h$ and $(QR)_h$ are asymptotically parallel.
\end{itemize}
Then $QR_h=PT_h$.
\end{thm}


\parit{Proof.}
We will use the projective model.
Without loss of generality, we may assume that $P$ is the center of the absolute.
As it was noted on page \pageref{klein-angles},
in this case the corresponding Euclidean lines are also perpendicular;
that is, $(PQ)\perp (QR)$, $(PS)\perp(PQ)$ and $(RT)\z\perp (PS)$.

Let $A$ be the common ideal point of $(QR)_h$ and $(PT)_h$.
Let $B$ and $C$ denote the remaining ideal points of $(QR)_h$ and $(PT)_h$
correspondingly.

Note that the Euclidean lines $(PQ)$, $(TR)$ and $(CB)$ are parallel.

\begin{wrapfigure}{o}{44mm}
\begin{lpic}[t(-3mm),b(6mm),r(0mm),l(0mm)]{pics/boyai-constr(1)}
\lbl[bl]{21.5,21.5;$P$}
\lbl[tl]{21.5,12.5;$Q$}
\lbl[tr]{8.5,12.5;$R$}
\lbl[tr]{8.5,20;$S$}
\lbl[lb]{10.5,26;$T$}
\lbl[t]{40,12.5;$A$}
\lbl[tr]{1,12.5;$B$}
\lbl[br]{1,28;$C$}
\lbl[t]{31,12.5;$\ell$}
\lbl[b]{20,33,90;$m$}
\lbl[b]{31,21.5;$n$}
\lbl[b]{9,33,90;$k$}
\end{lpic}
\end{wrapfigure}

Therefore, 
\[\triangle AQP\sim \triangle ART \sim\triangle ABC.\]
In particular,
\[\frac{AC}{AB}=\frac{AT}{AR}=\frac{AP}{AQ}.\]

It follows that
\[\frac{AT}{AR}=\frac{AP}{AQ}=\frac{BR}{CT}=\frac{BQ}{CP}.\]
In particular,
\[\frac{AT\cdot CP}{TC\cdot PA}=\frac{AR\cdot BQ}{RB\cdot QA}.\]
Applying the formula for h-distance in the projective model \ref{eq:proj-h-dist}, we get that $QR_h=PT_h$.
\qeds
 

\addtocontents{toc}{\protect\end{quote}}

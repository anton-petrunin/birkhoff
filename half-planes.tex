\chapter{Half-planes}\label{chap:half-planes}
\addtocontents{toc}{\protect\begin{quote}}

This chapter contains long proofs of intuitively evident statements.
It is okay to skip it, but make sure you know definitions of positive/negative angles and that your intuition agrees with \ref{thm:signs-of-triug}, \ref{prop:half-plane}, \ref{cor:half-plane}, \ref{thm:pasch} and \ref{thm:abc}.

 
                            
\section*{Sign of an angle}
\addtocontents{toc}{Sign of an angle.}

The positive and negative angles can be visualized as {}\emph{counterclockwise} and  {}\emph{clockwise} directions; formally, they are defined the following way.

\begin{itemize}
\item The angle $A O B$ is called \index{angle!positive and negative angles}\emph{positive} 
if $0<\measuredangle A O B<\pi$;
\item The  angle $A O B$ is called {}\emph{negative} 
if $\measuredangle A O B<0$.
\end{itemize}

Note that according to the above definitions the straight angle as well as the zero angle 
are neither positive nor negative.

\begin{thm}{Exercise}\label{ex:AOB+<=>BOA-}
Show that $\angle A O B$ is positive if and only if $\angle B O A$ is negative.
\end{thm}

\begin{thm}{Lemma}\label{lem:straight-sign}
Let $\angle AOB$ be straight.
Then $\angle AOX$ is positive 
if and only if $\angle BOX$ is negative.
\end{thm}

\parit{Proof.}
Set $\alpha=\measuredangle AOX$ 
and 
$\beta=\measuredangle BOX$.
Since $\angle AOB$ is straight,
$$\alpha-\beta\equiv \pi.\eqlbl{eq:alpha-beta}$$

It follows that $\alpha=\pi$ $\Leftrightarrow$ $\beta=0$
and $\alpha=0$ $\Leftrightarrow$ $\beta=\pi$.
In these two cases the sing of $\angle AOX$ and $\angle BOX$ are undefined.

In the remaining cases we have $|\alpha|,|\beta|<\pi$.
If $\alpha$ and $\beta$ have the same sign, then $|\alpha-\beta|<\pi$
which contradicts \ref{eq:alpha-beta}.
Hence the statement follows.
\qeds

\begin{thm}{Exercise}\label{ex:PP(PN)}
Assume that the angles $AOB$ and $BOC$ are positive. 
Show that
$$\measuredangle AOB+\measuredangle BOC+\measuredangle COA=2\cdot\pi.$$
if $\angle COA$ is positive,
and
$$\measuredangle AOB+\measuredangle BOC+\measuredangle COA=0.$$
if $\angle COA$ is negative.
\end{thm}




\section*{Intermediate value theorem}
\addtocontents{toc}{Intermediate value theorem.}


\begin{thm}{Intermediate value theorem}\label{thm:intermidiate}
Let $f\:[a,b]\to \mathbb{R}$ be a continuous function.
Assume 
$f(a)$ and $f(b)$ have the opposite signs,
then $f(t_0)=0$ for some $t_0\in[a,b]$.
\end{thm}

\begin{wrapfigure}{o}{38mm}
\begin{lpic}[t(-6mm),b(0mm),r(0mm),l(5mm)]{pics/ivt(1)}
\lbl[r]{2,20;$f(b)$}
\lbl[r]{2,1;$f(a)$}
\lbl[tl]{11,8;$t_0$}
\lbl[t]{27.5,8;$b$}
\lbl[b]{7.5,10.5;$a$}
\end{lpic}
\end{wrapfigure}



The intermediate value theorem is assumed to be known;
it should be covered in any calculus course.
We will use the following corollary.

\begin{thm}[\abs]{Corollary}\label{cor:intermidiate}
Assume that for any $t\z\in [0,1]$ we have three points in the plane  $O_t$, $A_t$ and $B_t$, such that 
\begin{enumerate}[(a)]
\item Each  function $t\mapsto O_t$, $t\mapsto A_t$ and $t\mapsto B_t$ is continuous.
\end{enumerate}

\begin{enumerate}[(a)]\addtocounter{enumi}{1}
\item For for any $t\in [0,1]$, the points $O_t$, $A_t$ and $B_t$ do not lie on one line.  
\end{enumerate}
Then $\angle A_0O_0B_0$ and $\angle A_1O_1B_1$ have the same sign.
\end{thm}

\parit{Proof.}
Consider the function 
$f(t)=\measuredangle A_tO_tB_t$.

Since 
the points $O_t$, $A_t$ and $B_t$ do not lie on one line,
Theorem~\ref{thm:straight-angle} implies that $f(t)=\measuredangle A_tO_tB_t\ne 0$ nor $\pi$ for any $t\in[0,1]$.

Therefore, by Axiom~\ref{def:birkhoff-axioms:2c} and Exercise~\ref{ex:comp+cont},
$f$ is a continuous function.

Further,
by the intermediate value theorem, $f(0)$ and $f(1)$ have the same sign;
hence the result follows.
\qeds

\section*{Same sign lemmas}
\addtocontents{toc}{Same sign lemmas.}

\begin{thm}[\abs]{Lemma}\label{lem:signs}
Assume $Q'\in [PQ)$ and $Q'\z\ne P$.
Then for any $X\z\notin (PQ)$ the angles $PQX$ and $PQ'X$ have the same sign. 
\end{thm}

\begin{wrapfigure}{o}{33mm}
\begin{lpic}[t(-0mm),b(1mm),r(0mm),l(0mm)]{pics/PQX(1)}
\lbl[t]{30,1;$P$}
\lbl[t]{18,1;$Q'$}
\lbl[t]{1,1;$Q$}
\lbl[r]{11,27;$X$}
\end{lpic}
\end{wrapfigure}

\parit{Proof.}
By Proposition~\ref{prop:point-on-half-line},
for any $t\in [0,1]$ there is a unique point $Q_t\in[PQ)$ 
such that 
\[PQ_t=  (1-t)\cdot PQ+t\cdot PQ'.\]
Note that the map $t\mapsto Q_t$ is continuous,
\begin{align*}
Q_0&=Q,
&
Q_1&=Q'
\end{align*}
and for any $t\in [0,1]$, 
we have $P\z\ne Q_t$.

Applying Corollary \ref{cor:intermidiate},
for $P_t=P$, $Q_t$ and $X_t=X$, we get that
$\angle PQX$ has the same sign as $\angle PQ'X$.
\qeds



\begin{thm}[\abs]{Signs of angles of a triangle}\label{thm:signs-of-triug}
In any nondegenerate triangle $ABC$,
the angles $ABC$, $BCA$ and $CAB$ have the same sign. 
\end{thm}

{

\begin{wrapfigure}{o}{33mm}
\begin{lpic}[t(-3mm),b(1mm),r(0mm),l(0mm)]{pics/PQX-2(1)}
\lbl[t]{30,1;$Z$}
\lbl[t]{18,1;$A$}
\lbl[t]{2,1;$B$}
\lbl[b]{12.5,28.5;$C$}
\end{lpic}
\end{wrapfigure}

\parit{Proof.}
Choose a point $Z\in (AB)$ so that $A$ lies between $B$ and~$Z$.


According to Lemma~\ref{lem:signs},
the angles $ZBC$ and $ZAC$ have the same sign.


Note that $\measuredangle ABC=\measuredangle ZBC$
and 
$$\measuredangle ZAC+\measuredangle CAB\equiv \pi.$$
Therefore, $\angle CAB$ has the same sign as $\angle ZAC$
which in turn has the same sign as $\measuredangle ABC\z=\measuredangle ZBC$.

}

Repeating the same argument for $\angle BCA$ and $\angle CAB$,
we get the result.
\qeds

\begin{thm}[\abs]{Lemma}\label{lem:signsXY}
Assume $[XY]$ does not intersect $(PQ)$,
then the angles $PQX$ and $PQY$ 
have the same sign.
\end{thm}

\begin{wrapfigure}{r}{34mm}
\begin{lpic}[t(-5mm),b(3mm),r(0mm),l(0mm)]{pics/PQY(1)}
\lbl[t]{30,1;$P$}
\lbl[t]{1,1;$Q$}
\lbl[r]{11,27;$X$}
\lbl[l]{24,13;$Y$}
\end{lpic}
\end{wrapfigure}

The proof is nearly identical to the one above.

\parit{Proof.}
According to Proposition~\ref{prop:point-on-half-line},
for any $t\in [0,1]$ there is a point  $X_t\in[XY]$, 
such that 
\[XX_t= t\cdot XY.\]
Note that the map $t\mapsto X_t$ is continuous.
Moreover, $X_0=X$, $X_1=Y$ and $X_t\notin(QP)$ for any $t\in [0,1]$.

Applying Corollary \ref{cor:intermidiate},
for $P_t\z=P$, $Q_t\z=Q$ and $X_t$, we get that
$\angle PQX$ has the same sign as $\angle PQY$.
\qeds



\section*{Half-planes}
\addtocontents{toc}{Half-planes.}

%(???

\begin{thm}{Proposition}\label{prop:half-plane}
Assume $X,Y\notin(PQ)$.
Then the angles $PQX$ and $PQY$ have the same sign if and only if $[XY]$ does not intersect $(PQ)$.
\end{thm}

\begin{wrapfigure}{o}{30mm}
\begin{lpic}[t(-4mm),b(-0mm),r(0mm),l(0mm)]{pics/PQXYZ(1)}
\lbl[b]{11,14;$Q$}
\lbl[b]{3,13.3;$P$}
\lbl[l]{28,24;$X$}
\lbl[l]{27,1;$Y$}
\lbl[lt]{27,11;$Z$}
\end{lpic}
\end{wrapfigure}

\parit{Proof.} The if-part follows from Lemma~\ref{lem:signs}. 

Assume $[XY]$ intersects $(PQ)$;
denote by $Z$ the point of intersection.
Without loss of generality, we can assume $Z\ne P$.

Note that $Z$ lies between $X$ and $Y$.
By Lemma~\ref{lem:straight-sign}, $\angle PZX$ and $\angle PZY$ have opposite signs.
This proves the statement if $Z=Q$.

If $Z\ne Q$, then $\angle ZQX$ and $\angle QZX$ have opposite signs by \ref{thm:signs-of-triug}.
The same way we get that $\angle ZQY$ and $\angle QZY$ have opposite signs.

If $Q$ lies between $Z$ and $P$, then by Lemma~\ref{lem:straight-sign} two pairs of angles $\angle PQX$, $\angle ZQX$ and $\angle PQY$, $\angle ZQY$ have the opposite signs. 
It follows that $\angle PQX$ and $\angle PQY$ have opposite signs as required.

In the remaining case $[QZ)=[QP)$ and therefore $\angle PQX=\angle ZQX$ and $\angle PQY=\angle ZQY$. 
Hence again $\angle PQX$ and $\angle PQY$ have opposite signs as required.
\qeds

%)???

\begin{thm}[\abs]{Corollary}\label{cor:half-plane}
The complement of a line $(PQ)$ in the plane 
can be presented in a unique way as a union of two disjoint subsets 
called \index{half-plane}\emph{half-planes}
such that 
\begin{enumerate}[(a)]
\item\label{cor:half-plane:angle} Two points $X,Y\notin(PQ)$ lie in the same half-plane if and only if the angles $PQX$ and $PQY$ have the same sign.
\item\label{cor:half-plane:intersect} Two points $X,Y\notin(PQ)$ lie in the same half-plane if and only if $[XY]$ does not intersect~$(PQ)$.
\end{enumerate}

\end{thm}

\begin{wrapfigure}[6]{o}{25mm}
\begin{lpic}[t(-3mm),b(-5mm),r(0mm),l(0mm)]{pics/vert-intersect(1)}
\lbl[l]{10.5,13.5;$O$}
\lbl[tr]{20,3;$A$}
\lbl[t]{7.5,8.5;$B$}
\lbl[b]{6,17.5;$A'$}
\lbl[tl]{19,24;$B'$}
\end{lpic}
\end{wrapfigure}

We say that $X$ and $Y$ lie on  {}\emph{one side from} $(PQ)$ if they lie in one of the half-planes of $(PQ)$ and we say that  $P$ and $Q$ lie on the {}\emph{opposite sides from} $\ell$ if they lie in the different half-planes of~$\ell$.


\begin{thm}{Exercise}\label{ex:vert-intersect}
Assume that the angles $AOB$ and $A'OB'$ are vertical.
Show that the line $(AB)$ does not intersect the segment~$[A'B']$.
\end{thm}


Consider the triangle $ABC$.
The segments $[AB]$, $[BC]$ and $[CA]$ are called 
\index{side!side of the triangle}\emph{sides of the triangle}.

The following theorem follows from Corollary~\ref{cor:half-plane}.

{

\begin{wrapfigure}{o}{21mm}
\begin{lpic}[t(-0mm),b(-5mm),r(0mm),l(0mm)]{pics/pasch(1)}
\lbl[tr]{1,1;$A$}
\lbl[b]{7,18.5;$B$}
\lbl[tl]{19,4;$C$}
\lbl[b]{18,12;$\ell$}
\end{lpic}
\end{wrapfigure}

\begin{thm}[\abs]{Pasch's theorem}\label{thm:pasch}
Assume line $\ell$ does not pass thru any vertex a triangle.
Then it intersects either two or zero sides of the triangle.
\end{thm}

\parit{Proof.}
Assume that line $\ell$ intersects side $[AB]$ of the triangle $ABC$ and does not pass thru $A$, $B$ and $C$.

}

By Corollary~\ref{cor:half-plane}, the vertexes $A$ and $B$ lie on the opposite sides from $\ell$.

The vertex $C$ may lie on the same side with $A$ and on the opposite side with $B$ or the other way around.
By Corollary~\ref{cor:half-plane}, in the first case $\ell$ intersects side $[BC]$ and does not intersect $[AC]$ and in the second case $\ell$ intersects side $[AC]$ and does not intersect $[BC]$.
Hence the statement follows.
\qeds

\begin{thm}{Exercise}\label{ex:signs-PXQ-PYQ}
Show that two points $X,Y\notin(PQ)$ lie on the same side from $(PQ)$
if and only if the angles $PXQ$ and $PYQ$ have the same sign.
\end{thm}

\begin{multicols}{2}
\begin{center}
\begin{lpic}[t(0mm),b(0mm),r(0mm),l(0mm)]{pics/PQY-1(1)}
\lbl[t]{30,1;$P$}
\lbl[t]{1,1;$Q$}
\lbl[b]{12,28.5;$X$}
\lbl[rb]{17,12;$Y$}
\end{lpic}
\end{center}
\columnbreak
\begin{center}
\begin{lpic}[t(-4mm),b(0mm),r(0mm),l(0mm)]{pics/PQY-2(1)}
\lbl[t]{30,1;$B$}
\lbl[t]{1,1;$A$}
\lbl[lb]{23,15;$A'$}
\lbl[rb]{8,19;$B'$}
\lbl[b]{12,28.5;$C$}
\end{lpic} 
\end{center}
\end{multicols}

\begin{thm}{Exercise}\label{ex:chevinas}
Let $\triangle ABC$ be a nondegenerate triangle,
$A'\in[BC]$  and 
$B'\in [AC]$.
Show that the segments $[AA']$ and $[BB']$ intersect.
\end{thm}

\begin{thm}{Exercise}\label{ex:Z}
Assume that the points $X$ and $Y$ lie on opposite sides from the line~$(PQ)$.
Show that the half-line $[PX)$ does not interest~$[QY)$. 
\end{thm}

\begin{thm}{Advanced exercise}\label{ex:angle-measures}
Note that the following quantity 
$$\tilde\measuredangle ABC=\left[
\begin{aligned}
&\pi&&\text{if}&\measuredangle ABC&=\pi
\\
-&\measuredangle ABC&&\text{if}&\measuredangle ABC&<\pi
\end{aligned}
\right.$$
can serve as the angle measure; 
that is, the axioms hold if one exchanges $\measuredangle$ to $\tilde\measuredangle$ everywhere.

Show that $\measuredangle$ and $\tilde\measuredangle$ are the only possible angle measures on the plane. 

Show that without Axiom \ref{def:birkhoff-axioms:2c}, this is no longer true.
\end{thm}
 


\section*{Triangle with the given sides}
\addtocontents{toc}{Triangle with the given sides.}

Consider the triangle $ABC$.
Set 
\begin{align*}
a&=BC,
&
b&=CA,
&
c&=AB.
\end{align*}
Without loss of generality, we may assume that 
\[a\le b \le c.\]
Then all three triangle inequalities for $\triangle ABC$
hold if and only if 
\[c\le a+b.\]
The following theorem states that this is the only restriction on $a$, $b$ and~$c$.

\begin{thm}[\abs]{Theorem}\label{thm:abc}
Assume that $0<a\le b\le c\le a+b$.
Then there is a triangle $ABC$ 
such that $a=BC$, $b=CA$ and $c=AB$.
\end{thm}

The proof requires some preparation;
it is given in the end of section.

Assume $r>0$ and $\pi>\beta>0$.
Consider the triangle $ABC$ such that 
$AB=BC=r$ and $\measuredangle ABC=\beta$.
The existence of such a triangle follows from Axiom~\ref{def:birkhoff-axioms:2a} and Proposition~\ref{prop:point-on-half-line}.

\begin{wrapfigure}{o}{25mm}
\begin{lpic}[t(2mm),b(4mm),r(0mm),l(0mm)]{pics/sbr(1)}
\lbl[t]{2,0;$A$}
\lbl[t]{22,0;$C$}
\lbl[b]{12.5,29;$B$}
\lbl[w]{12.5,2.5;$\,s(\beta,r)\,$}
\lbl[W]{7.5,13;$r$}
\lbl[W]{19,13;$r$}
\lbl[t]{12.5,20;$\beta$}
\end{lpic}
\end{wrapfigure}

Note that according to Axiom~\ref{def:birkhoff-axioms:3}, 
the values
$\beta$ and $r$ define the triangle $ABC$ up to the congruence.
In particular, the distance $AC$ depends only on $\beta$ and~$r$.
Set 
$$s(\beta,r)\df AC.$$

\begin{thm}[\abs]{Proposition}\label{prop:f(r,a)}
Given $r>0$ and $\epsilon>0$, there is $\delta>0$ such that
if $0<\beta<\delta$, then 
\[s(r,\beta)<\epsilon.\]

\end{thm}


{

\begin{wrapfigure}{o}{33mm}
\begin{lpic}[t(-0mm),b(0mm),r(0mm),l(0mm)]{pics/fra(1)}
\lbl[t]{22.5,0;$A$}
\lbl[t]{2.5,0;$B$}
\lbl[b]{20,14;$C$}
\lbl[tl]{25,7;$D$}
\lbl[tl]{30,8;$Z$}
\lbl[r]{28,20;$Y$}
\lbl[rb]{30,27;$X$}
\lbl[w]{15,2;$\,r\,$}
\lbl[w]{13,8,30;$\,r\,$}
\end{lpic}
\end{wrapfigure}

\parit{Proof.}
Fix two points $A$ and $B$ such that $AB=r$.

Choose a point $X$ such that $\measuredangle ABX$ is positive.
Let $Y\in [AX)$ be the point such that $AY=\tfrac\epsilon8$;
it exists by Proposition~\ref{prop:point-on-half-line}.

Note that $X$ and $Y$ lie on the same side from $(AB)$;
therefore, $\angle ABY$ is positive. 
Set $\delta=\measuredangle ABY$.

Assume $0<\beta<\delta$,
$\measuredangle ABC=\beta$
and $BC\z=r$.




Applying Axiom~\ref{def:birkhoff-axioms:2a},
we can choose a half-line $[BZ)$ such that $\measuredangle ABZ=\tfrac12\cdot \beta$.
Note that $A$ and $Y$ lie on the opposite sides from~$(BZ)$.
Therefore, $(BZ)$ intersects $[AY]$;
denote by $D$ the point of intersection.

Since $D\in (BZ)$, we get $\measuredangle ABD=\tfrac \beta2$ or $\tfrac\beta2-\pi$.
The latter is impossible since $D$ and $Y$ lie on the same side from~$(AB)$.
Therefore, 
$$\measuredangle ABD=\measuredangle DBC=\tfrac \beta2.$$



By Axiom~\ref{def:birkhoff-axioms:3},
$\triangle ABD\cong \triangle CBD$.
In particular,
\begin{align*}
AC&\le AD+DC=
\\
&=2\cdot AD\le 
\\
&\le 2\cdot AY=
\\
&=\tfrac\epsilon4.
\end{align*}
Hence the result follows.
\qeds

}

\begin{thm}[\abs]{Corollary}\label{cor:C-cont}
Fix a real number $r>0$ 
and two distinct points $A$ and~$B$.
Then for 
any real number $\beta\in [0,\pi]$,
there is a unique point $C_\beta$ such that $BC_\beta=r$
and $\measuredangle ABC_\beta=\beta$.
Moreover, the map $\beta\mapsto C_\beta$ 
is a continuous map from $[0,\pi]$ to the plane.
\end{thm}

\parit{Proof.}
The existence and uniqueness of $C_\beta$ follows from Axiom~\ref{def:birkhoff-axioms:2a} and Proposition~\ref{prop:point-on-half-line}.

Note that if $\beta_1\ne\beta_2$, then
$$C_{\beta_1}C_{\beta_2}=s(r,|\beta_1-\beta_2|).$$

Therefore, Proposition~\ref{prop:f(r,a)} implies that  the map $\beta\mapsto C_\beta$ is continuous.
\qeds





\parit{Proof of Theorem~\ref{thm:abc}.}\label{page:proof:thm:abc}
Fix the points $A$ and $B$ such that $AB=c$.
Given $\beta\in [0,\pi]$,
denote by $C_\beta$ the point in the plane such that $BC_\beta\z=a$ and $\measuredangle ABC=\beta$.

According to Corollary~\ref{cor:C-cont},
the map
$\beta\mapsto C_\beta$ is continuous.
Therefore, the function $b(\beta)=AC_\beta$ is continuous
(formally, it follows from Exercise~\ref{ex:dist-cont} and Exercise~\ref{ex:comp+cont}).

Note that $b(0)=c-a$ and $b(\pi)=c+a$.
Since $c-a\le b\le c+a$,
by the intermediate value theorem (\ref{thm:intermidiate})
there is $\beta_0\in[0,\pi]$ such that
$b(\beta_0)=b$.
Hence the result follows. 
\qeds



\addtocontents{toc}{\protect\end{quote}}
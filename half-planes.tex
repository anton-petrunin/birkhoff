\chapter{Half-planes}\label{chap:half-planes}
\addtocontents{toc}{\protect\begin{quote}}

This chapter contains long proofs of intuitively evident statements.
It is okay to skip it, but make sure you know definitions of positive/negative angles and that your intuition agrees with \ref{prop:half-plane}, \ref{thm:pasch}, \ref{thm:signs-of-triug} and \ref{thm:abc}.

 
                            
\section*{Sign of angle}
\addtocontents{toc}{Sign of angle.}

\begin{itemize}
\item The angle $\angle A O B$ is called \index{angle!positive and negative angles}\emph{positive} 
if $0<\measuredangle A O B<\pi$;
\item The  angle $\angle A O B$ is called {}\emph{negative} 
if $\measuredangle A O B<0$.
\end{itemize}

Note that according to the above definitions the straight angle as well as zero angle 
are neither positive nor negative.

\begin{thm}{Exercise}\label{ex:AOB+<=>BOA-}
Show that $\angle A O B$ is positive if and only if $\angle B O A$ is negative.
\end{thm}

\begin{thm}{Exercise}\label{ex:straight-sign}
Let $\angle AOB$ be a straight angle.
Show that $\angle AOX$ is positive 
if and only if $\angle BOX$ is negative.
\end{thm}

\begin{thm}{Exercise}\label{ex:PP(PN)}
Assume that the angles $\angle AOB$ and $\angle BOC$ are positive. 
Show that
$$\measuredangle AOB+\measuredangle BOC+\measuredangle COA=2\cdot\pi.$$
if $\angle COA$ is positive
and
$$\measuredangle AOB+\measuredangle BOC+\measuredangle COA=0.$$
if $\angle COA$ is negative.
\end{thm}




\section*{Intermediate value theorem}
\addtocontents{toc}{Intermediate value theorem.}


\begin{thm}{Intermediate value theorem}\label{thm:intermidiate}
Let $f\:[a,b]\to \mathbb{R}$ be a continuous function.
Assume 
$f(a)$ and $f(b)$ have the opposite signs
then $f(t_0)=0$ for some $t_0\in[a,b]$.
\end{thm}

\begin{wrapfigure}{o}{37mm}
\begin{lpic}[t(-6mm),b(0mm),r(0mm),l(5mm)]{pics/ivt(1)}
\lbl[r]{2,20;$f(b)$}
\lbl[r]{2,1;$f(a)$}
\lbl[tl]{11,8;$t_0$}
\lbl[t]{27.5,8;$b$}
\lbl[b]{7.5,10.5;$a$}
\end{lpic}
\end{wrapfigure}



The Intermediate value theorem is assumed to be known;
it should be covered in any calculus course.
We will use the following corollary.

\begin{thm}{Corollary}\label{cor:intermidiate}
Assume that for any $t\z\in [0,1]$ we have three points in the plane  $O_t$, $A_t$ and $B_t$ such that 
\begin{enumerate}[(a)]
\item Each  function $t\mapsto O_t$, $t\mapsto A_t$ and $t\mapsto B_t$ is continuous.
\end{enumerate}

\begin{enumerate}[(a)]\addtocounter{enumi}{1}
\item For for any $t\in [0,1]$, the points $O_t$, $A_t$ and $B_t$ do not lie on one line.  
\end{enumerate}
Then the angles $\angle A_0O_0B_0$ 
and $\angle A_1O_1B_1$ have the same sign.
\end{thm}

\parit{Proof.}
Consider the function 
$f(t)=\measuredangle A_tO_tB_t$.

Since 
the points $O_t$, $A_t$ and $B_t$ do not lie on one line,
Theorem~\ref{thm:straight-angle} implies that $f(t)=\measuredangle A_tO_tB_t\ne 0$ nor $\pi$ for any $t\in[0,1]$.

Therefore by Axiom~\ref{def:birkhoff-axioms:2c} and Exercise~\ref{ex:comp+cont},
$f$ is a continuous function.

Further,
by Intermediate value theorem, $f(0)$ and $f(1)$ have the same sign;
hence the result follows.
\qeds

\section*{Same sign lemmas}
\addtocontents{toc}{Same sign lemmas.}

\begin{thm}{Lemma}\label{lem:signs}
Assume $Q'\in [PQ)$ and $Q'\z\ne P$.
Then for any $X\z\notin (PQ)$ the angles 
$\angle PQX$ and $\angle PQ'X$ have the same sign. 
\end{thm}

\begin{wrapfigure}{o}{33mm}
\begin{lpic}[t(-4mm),b(1mm),r(0mm),l(0mm)]{pics/PQX(1)}
\lbl[t]{30,1;$P$}
\lbl[t]{18,1;$Q'$}
\lbl[t]{1,1;$Q$}
\lbl[r]{11,27;$X$}
\end{lpic}
\end{wrapfigure}

\parit{Proof.}
By Proposition~\ref{prop:point-on-half-line},
for any $t\in [0,1]$ there is unique point $Q_t\in[PQ)$ 
such that 
\[PQ_t=  (1-t)\cdot PQ+t\cdot PQ'.\]
Note that the map $t\mapsto Q_t$ is continuous,
\begin{align*}
Q_0&=Q,
&
Q_1&=Q'
\end{align*}
and for any $t\in [0,1]$, 
we have $P\z\ne Q_t$.

Applying Corollary \ref{cor:intermidiate},
for $P_t=P$, $Q_t$ and $X_t=X$, we get that
$\angle PQX$ has the same sign as $\angle PQ'X$.
\qeds

\begin{thm}{Lemma}\label{lem:signsXY}
Assume $[XY]$ does not intersect $(PQ)$ 
then the angles $\angle PQX$ and $\angle PQY$ 
have the same sign.
\end{thm}

\begin{wrapfigure}{o}{34mm}
\begin{lpic}[t(-2mm),b(0mm),r(0mm),l(0mm)]{pics/PQY(1)}
\lbl[t]{30,1;$P$}
\lbl[t]{1,1;$Q$}
\lbl[r]{11,27;$X$}
\lbl[l]{24,13;$Y$}
\end{lpic}
\end{wrapfigure}

The proof is nearly identical to the one above.

\parit{Proof.}
According to Proposition~\ref{prop:point-on-half-line},
for any $t\in [0,1]$ there is a point  $X_t\in[XY]$ 
such that 
\[XX_t= t\cdot XY.\]
Note that the map $t\mapsto X_t$ is continuous,
$X_0=X$ and $X_1=Y$ and for any $t\in [0,1]$, 
we have $Y_t\notin(QP)$.

Applying Corollary \ref{cor:intermidiate},
for $P_t\z=P$, $Q_t\z=Q$ and $X_t$, we get that
$\angle PQX$ has the same sign as $\angle PQY$.
\qeds



\section*{Half-planes}
\addtocontents{toc}{Half-planes.}

\begin{thm}{Proposition}\label{prop:half-plane}
The complement of a line $(PQ)$ in the plane 
can be presented in the unique way as a union of two disjoint subsets 
called \index{half-plane}\emph{half-planes}
such that 
\begin{enumerate}[(a)]
\item\label{prop:half-plane:angle} Two points $X,Y\notin(PQ)$ lie in the same half-plane if and only if the angles $\angle PQX$ and $\angle PQY$ have the same sign.
\item\label{prop:half-plane:intersect} Two points $X,Y\notin(PQ)$ lie in the same half-plane if and only if $[XY]$ does not intersect  $(PQ)$.
\end{enumerate}

\end{thm}


Further we say that $X$ and $Y$ lie on 
{}\emph{one side from} $(PQ)$ if they lie in one of the half-planes of $(PQ)$ and we say that  $P$ and $Q$ lie on the {}\emph{opposite sides from} $\ell$ if they lie in the different half-planes of $\ell$.

\begin{wrapfigure}{o}{34mm}
\begin{lpic}[t(-4mm),b(-3mm),r(0mm),l(0mm)]{pics/PQY-3(1)}
\lbl[t]{30,8;$P$}
\lbl[t]{1,8;$Q$}
\lbl[]{22,15;$\mathcal{H}_+$}
\lbl[]{22,5;$\mathcal{H}_-$}
\end{lpic}
\end{wrapfigure}

\parit{Proof.}
Let us denote by $\mathcal{H}_+$ (correspondingly $\mathcal{H}_-$) 
the set of points $X$ in the plane such that
$\angle PQX$ is positive (correspondingly negative).

According to Theorem~\ref{thm:straight-angle},
$X\notin (PQ)$ if and only if 
$\measuredangle PQX\z\ne 0$ nor $\pi$.
Therefore $\mathcal{H}_+$ and $\mathcal{H}_-$
give the unique subdivision of the complement of $(PQ)$ which satisfies property (\ref{prop:half-plane:angle}).

\medskip

Now let us prove that the 
this subdivision depends only on the line $(PQ)$;
that is, if $(P'Q')=(PQ)$ and $X,Y\notin (PQ)$
then the angles 
$\angle PQX$ and $\angle PQY$ have the same sign
if and only if the angles $\angle P'Q'X$ and $\angle P'Q'Y$ have the same sign.

Applying Exercise~\ref{ex:straight-sign},
we can assume that $P=P'$ and $Q'\z\in [PQ)$.
It remains to apply Lemma \ref{lem:signs}.

\parit{(\ref{prop:half-plane:intersect}).}
Assume $[XY]$ intersects $(PQ)$.
Since the subdivision depends only on the line $(PQ)$, 
we can assume that $Q\in[XY]$.
In this case, by Exercise~\ref{ex:straight-sign},
the angles $\angle PQX$ and $\angle PQY$ have opposite signs.

\begin{wrapfigure}[6]{o}{25mm}
\begin{lpic}[t(-3mm),b(-5mm),r(0mm),l(0mm)]{pics/vert-intersect(1)}
\lbl[l]{10.5,13.5;$O$}
\lbl[tr]{20,3;$A$}
\lbl[t]{7.5,8.5;$B$}
\lbl[b]{6,17.5;$A'$}
\lbl[tl]{19,24;$B'$}
\end{lpic}
\end{wrapfigure}

Now assume $[XY]$ does not intersect $(PQ)$.
In this case, by Lemma~\ref{lem:signsXY},
$\angle PQX$ and $\angle PQY$ have the same sign.
\qeds


\begin{thm}{Exercise}\label{ex:vert-intersect}
Assume that the angles $\angle AOB$ and $\angle A'OB'$ are vertical.
Show that the line $(AB)$ does not intersect the segment $[A'B']$.
\end{thm}


Consider triangle $\triangle ABC$.
The segments $[AB]$, $[BC]$ and $[CA]$ are called 
\index{side!side of the triangle}\emph{sides of the triangle}.

The following theorem is a corollary of Proposition~\ref{prop:half-plane}.

\begin{thm}{Pasch's theorem}\label{thm:pasch}
Assume line $\ell$ does not pass thru any vertex a triangle.
Then it intersects either two or zero sides of the triangle.
\end{thm}

\begin{thm}{Signs of angles of triangle}\label{thm:signs-of-triug}
In any nondegenerate triangle $\triangle ABC$
the angles $\angle ABC$, $\angle BCA$ and $\angle CAB$ have the same sign. 
\end{thm}

\begin{wrapfigure}{o}{33mm}
\begin{lpic}[t(-3mm),b(1mm),r(0mm),l(0mm)]{pics/PQX-2(1)}
\lbl[t]{30,1;$Z$}
\lbl[t]{18,1;$A$}
\lbl[t]{2,1;$B$}
\lbl[b]{12.5,28.5;$C$}
\end{lpic}
\end{wrapfigure}


\parit{Proof.}
Choose a point $Z\in (AB)$ so that $A$ lies between $B$ and $Z$.


According to Lemma~\ref{lem:signs},
the angles $\angle ZBC$ and $\angle ZAC$ have the same sign.

Note that $\measuredangle ABC=\measuredangle ZBC$
and 
$$\measuredangle ZAC+\measuredangle CAB\equiv \pi.$$
Therefore $\angle CAB$ has the same sign as $\angle ZAC$
which in turn has the same sign as $\measuredangle ABC\z=\measuredangle ZBC$.


Repeating the same argument for $\angle BCA$ and $\angle CAB$,
we get the result.
\qeds

\medskip

\begin{multicols}{2}
\begin{center}
\begin{lpic}[t(-3mm),b(0mm),r(0mm),l(0mm)]{pics/PQY-1(1)}
\lbl[t]{30,1;$P$}
\lbl[t]{1,1;$Q$}
\lbl[b]{12,28.5;$X$}
\lbl[rb]{17,12;$Y$}
\end{lpic}
\end{center}
\columnbreak
\begin{center}
\begin{lpic}[t(-4mm),b(0mm),r(0mm),l(0mm)]{pics/PQY-2(1)}
\lbl[t]{30,1;$B$}
\lbl[t]{1,1;$A$}
\lbl[lb]{23,15;$A'$}
\lbl[rb]{8,19;$B'$}
\lbl[b]{12,28.5;$C$}
\end{lpic} 
\end{center}
\end{multicols}

\begin{thm}{Exercise}\label{ex:signs-PXQ-PYQ}
Show that two points $X,Y\notin(PQ)$ lie on the same side from $(PQ)$
if and only if the angles $\angle PXQ$ and $\angle PYQ$ have the same sign.
\end{thm}

\begin{thm}{Exercise}\label{ex:chevinas}
Let $\triangle ABC$ be a nondegenerate triangle,
$A'\in[BC]$  and 
$B'\in [AC]$.
Show that the segments $[AA']$ and $[BB']$ intersect.
\end{thm}

\begin{thm}{Exercise}\label{ex:Z}
Assume that the points $X$ and $Y$ lie on the opposite sides from the line $(PQ)$.
Show that the half-line $[PX)$ does not interests $[QY)$. 
\end{thm}

\begin{thm}{Advanced exercise}\label{ex:angle-measures}
Note that the following quantity 
$$\tilde\measuredangle ABC=\left[
\begin{aligned}
&\pi&&\text{if}&\measuredangle ABC&=\pi
\\
-&\measuredangle ABC&&\text{if}&\measuredangle ABC&<\pi
\end{aligned}
\right.$$
can serve as the angle measure; 
that is, the axioms hold if one changes everywhere $\measuredangle$ to $\tilde\measuredangle$.

Show that $\measuredangle$ and $\tilde\measuredangle$ are the only possible angle measures on the plane. 

Show that without Axiom \ref{def:birkhoff-axioms:2c}, this is not longer true.
\end{thm}
 


\section*{Triangle with the given sides}
\addtocontents{toc}{Triangle with the given sides.}

Consider a triangle $\triangle ABC$.
Set 
\begin{align*}
a&=BC,
&
b&=CA,
&
c&=AB.
\end{align*}
Without loss of generality we may assume that 
\[a\le b \le c.\]
Then all three triangle inequalities for $\triangle ABC$
hold if and only if 
\[c\le a+b.\]
The following theorem states that this is the only restriction on $a$, $b$ and $c$.

\begin{thm}{Theorem}\label{thm:abc}
Assume that $0<a\le b\le c\le a+b$.
Then there is a triangle $\triangle ABC$ such that $a=BC$, $b=CA$ and $c=AB$.
\end{thm}

The proof requires some preparation.

Assume $r>0$ and $\pi>\beta>0$.
Consider triangle $\triangle ABC$ such that 
$AB=BC=r$ and $\measuredangle ABC=\beta$.
The existence of such triangle follow from Axiom~\ref{def:birkhoff-axioms:2a} and Proposition~\ref{prop:point-on-half-line}.

\begin{wrapfigure}{o}{25mm}
\begin{lpic}[t(2mm),b(0mm),r(0mm),l(0mm)]{pics/sbr(1)}
\lbl[t]{2,0;$A$}
\lbl[t]{22,0;$C$}
\lbl[b]{12.5,29;$B$}
\lbl[w]{12.5,2.5;$\,s(\beta,r)\,$}
\lbl[W]{7.5,13;$r$}
\lbl[W]{19,13;$r$}
\lbl[t]{12.5,20;$\beta$}
\end{lpic}
\end{wrapfigure}

Note that according to Axiom~\ref{def:birkhoff-axioms:3}, 
the values
$\beta$ and $r$ define the triangle up to congruence.
In particular the distance $AC$ depends only on $\beta$ and $r$.
Set 
$$s(\beta,r)\df AC.$$

\begin{thm}{Proposition}\label{prop:f(r,a)}
Given $r>0$ and $\epsilon>0$ there is $\delta>0$ such that
if $0<\beta<\delta$ then $s(r,\beta)<\epsilon$.
\end{thm}


\parit{Proof.}
Fix two point $A$ and $B$ such that $AB=r$.

Choose a point $X$ such that $\measuredangle ABX$ is positive.
Let $Y\in [AX)$ be the point such that $AY=\tfrac\epsilon8$;
it exists by Proposition~\ref{prop:point-on-half-line}.

Note that $X$ and $Y$ lie on the same side from $(AB)$;
therefore $\angle ABY$ is positive. 
Set $\delta=\measuredangle ABY$.

Assume $0<\beta<\delta$,
$\measuredangle ABC=\beta$
and $BC\z=r$.




Applying Axiom~\ref{def:birkhoff-axioms:2a},
we can choose a half-line $[BZ)$ such that $\measuredangle ABZ=\tfrac12\cdot \beta$.
Note that $A$ and $Y$ lie on the opposite sides from $(BZ)$.
Therefore $(BZ)$ intersects $[AY]$;
denote by $D$ the point of intersection.

Since $D\in (BZ)$, we get $\measuredangle ABD=\tfrac \beta2$ or $\tfrac\beta2-\pi$.
The later is impossible since $D$ and $Y$ lie on the same side from $(AB)$.
Therefore 
$$\measuredangle ABD=\measuredangle DBC=\tfrac \beta2.$$

{
\begin{wrapfigure}{o}{33mm}
\begin{lpic}[t(-2mm),b(0mm),r(0mm),l(0mm)]{pics/fra(1)}
\lbl[t]{22.5,0;$A$}
\lbl[t]{2.5,0;$B$}
\lbl[b]{20,14;$C$}
\lbl[tl]{25,7;$D$}
\lbl[tl]{30,8;$Z$}
\lbl[r]{28,20;$Y$}
\lbl[rb]{30,27;$X$}
\lbl[w]{15,2;$\,r\,$}
\lbl[w]{13,8,30;$\,r\,$}
\end{lpic}
\end{wrapfigure}

By Axiom~\ref{def:birkhoff-axioms:3},
$\triangle ABD\cong \triangle CBD$.
In particular
\begin{align*}
AC&\le AD+DC=
\\
&=2\cdot AD\le 
\\
&\le 2\cdot AY=
\\
&=\tfrac\epsilon4.
\end{align*}
Hence the result follows.
\qeds

}

\begin{thm}{Corollary}\label{cor:C-cont}
Fix a real number $r>0$ 
and two distinct points $A$ and $B$.
Then for 
any real number $\beta\in [0,\pi]$,
there is unique point $C_\beta$ such that $BC_\beta=r$
and $\measuredangle ABC_\beta=\beta$.
Moreover, the map $\beta\mapsto C_\beta$ 
is a continuous map from $[0,\pi]$ to the plane.
\end{thm}

\parit{Proof.}
The existence and uniqueness of $C_\beta$ follows from Axiom~\ref{def:birkhoff-axioms:2a} and Proposition~\ref{prop:point-on-half-line}.

Note that if $\beta_1\ne\beta_2$ then
$$C_{\beta_1}C_{\beta_2}=s(r,|\beta_1-\beta_2|).$$

Therefore Proposition~\ref{prop:f(r,a)} implies that  the map $\beta\mapsto C_\beta$ is continuous.
\qeds





\parit{Proof of Theorem~\ref{thm:abc}.}
Fix points $A$ and $B$ such that $AB=c$.
Given $\beta\in [0,\pi]$,
denote by $C_\beta$ the point in the plane such that $BC_\beta=a$ and $\measuredangle ABC=\beta$.

According to Corollary~\ref{cor:C-cont},
the map
$\beta\mapsto C_\beta$ is a continuous.
Therefore function $b(\beta)=AC_\beta$ is continuous
(formally it follows from Exercise~\ref{ex:dist-cont} and Exercise~\ref{ex:comp+cont}).

Note that $b(0)=c-a$ and $b(\pi)=c+a$.
Since $c-a\le b\le c+a$,
by Intermediate value theorem (\ref{thm:intermidiate})
there is $\beta_0\in[0,\pi]$ such that
$b(\beta_0)=b$.
Hence the result follows. 
\qeds



\addtocontents{toc}{\protect\end{quote}}
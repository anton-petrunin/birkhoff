\chapter{Perpendicular lines}\label{chap:perp}
\addtocontents{toc}{\protect\begin{quote}}

\section*{Right, acute and obtuse angles}
\addtocontents{toc}{Right, acute and obtuse angles.}

\begin{itemize}
\item If $|\measuredangle A O B|=\tfrac\pi2$, we say that $\angle A O B$ is \index{angle!right angle}\emph{right};
%\item If $\measuredangle A O B\ne\pm\tfrac\pi2$, we say that the angle $\angle A O B$ is \index{angle!oblique angle}\emph{oblique};
\item If $|\measuredangle A O B|<\tfrac\pi2$, we say that $\angle A O B$ is 
\index{acute!angle}\emph{acute};
\item If $|\measuredangle A O B|>\tfrac\pi2$, we say that $\angle A O B$ is \index{angle!acute and obtuse angles}\index{obtuse angle}\emph{obtuse}.
\end{itemize}

\begin{wrapfigure}{o}{30mm}
\begin{lpic}[t(-0mm),b(0mm),r(0mm),l(2mm)]{pics/perp-notation(1)}
\end{lpic}
\end{wrapfigure}

On the diagrams,
the right angles will be marked with a little square, 
as shown.

If $\angle A O B$ is right,
we say also
that $[O A)$ is \index{perpendicular}\emph{perpendicular} to $[O B)$; 
it will be written as \index{38@$\perp$}$[O A)\z\perp [O B)$.

From Theorem~\ref{thm:straight-angle}, 
it follows that two lines $(O A)$
 and $(O B)$ are appropriately called {}\emph{perpendicular}, if $[O A)\z\perp [O B)$.
In this case we also write $(O A)\z\perp (O B)$.



\begin{thm}{Exercise}\label{ex:acute-obtuce}
Assume point $O$ lies between $A$ and $B$ and $X\ne O$.
Show that 
$\angle XOA$ is acute if and only if 
$\angle XOB$ is obtuse.
\end{thm}



\section*{Perpendicular bisector}
\addtocontents{toc}{Perpendicular bisector.}

Assume $M$ is the midpoint of the segment $[AB]$;
that is, $M\in(A B)$ and $A M \z= M B$.

The line $\ell$ which passes thru $M$ and perpendicular to $(AB)$,
is called the \index{bisector!perpendicular bisector}\index{perpendicular bisector}\emph{perpendicular bisector} to the segment~$[AB]$. 

\begin{thm}[\abs]{Theorem}\label{thm:perp-bisect}
Given distinct points $A$ and $B$,
all points equidistant from $A$ and $B$ and no
others lie on the perpendicular bisector to~$[A B]$.
\end{thm}

\begin{wrapfigure}{o}{40mm}
\begin{lpic}[t(-0mm),b(0mm),r(0mm),l(0mm)]{pics/perp-bisect(1)}
\lbl[t]{2,7;$A$}
\lbl[t]{38,7;$B$}
\lbl[tl]{20.5,7;$M$}
\lbl[lb]{20.5,36;$P$}
\end{lpic}
\end{wrapfigure}

\parit{Proof.} Let $M$ be the midpoint of~$[A B]$.

Assume $P A= P B$ and $P\ne M$.
According to SSS (\ref{thm:SSS}),
$\triangle A M P \z\cong\triangle B M P$.
Hence 
$$\measuredangle A M P=\pm \measuredangle B M P.$$  
Since $A\not=B$, we have ``$-$'' in the above formula.
Further,
\begin{align*}
\pi
&=
\measuredangle A M B
\equiv
\\
&\equiv\measuredangle A M P+\measuredangle P M B
\equiv
\\
&\equiv
2\cdot \measuredangle A M P.
\end{align*}
That is, $\measuredangle A M P
=
\pm
\tfrac\pi2$. 
Therefore, $P$ lies on the perpendicular bisector.


To prove the converse, 
suppose $P$ 
is any point on the perpendicular bisector to $[A B]$ and $P\ne M$.
Then $\measuredangle A M P=\pm \tfrac\pi2$, 
$\measuredangle B M P=\pm \tfrac\pi2$ and
$A M\z=B M$.
Therefore, $\triangle A M P\cong \triangle B M P$;
in particular, $A P\z= B P$.\qeds


\begin{thm}{Exercise}\label{ex:pbisec-side}
Let $\ell$ be the perpendicular bisector to the segment $[A B]$ and $X$ be an arbitrary point on the plane.

Show that 
$AX<BX$ if and only if $X$ and $A$ lie on the same side from~$\ell$.
\end{thm}

\begin{thm}{Exercise}\label{ex:side-angle}
Let $\triangle ABC$ be nondegenerate.
Show that $AC>BC$ if and only if $|\measuredangle ABC|>|\measuredangle CAB|$. 
\end{thm}



\section*{Uniqueness of a perpendicular}
\addtocontents{toc}{Uniqueness of perpendicular lines.}

\begin{thm}[\abs]{Theorem}\label{perp:ex+un}
There is one and only one line which passes thru a given point $P$ and is perpendicular to a given line~$\ell$.
\end{thm}

According to the above theorem, 
there is a unique point $Q\in\ell$ such that $(QP)\perp\ell$.
This point $Q$ is called the \index{foot point}\emph{foot point} of $P$ on~$\ell$. 

\parit{Proof.} 
If $P\in\ell$, then both, existence and uniqueness, follow from Axiom~\ref{def:birkhoff-axioms:2}.

\parit{Existence for $P\not\in\ell$.} 
Let $A$ and $B$ be two distinct points of~$\ell$. 
Choose $P'$ so that $AP'\z=AP$ and $\measuredangle P' A B\equiv -\measuredangle P A B$.
According to Axiom~\ref{def:birkhoff-axioms:3}, $\triangle A P' B\z\cong\triangle A P B$.
Therefore, $A P= A P'$ and $B P= B P'$.

{

\begin{wrapfigure}{o}{36mm}
\begin{lpic}[t(-0mm),b(-3mm),r(0mm),l(0mm)]{pics/perp(1)}
\lbl[rb]{1.5,18;$A$}
\lbl[lb]{33,18;$B$}
\lbl[t]{15,16;$\ell$}
\lbl[lb]{23,33;$P$}
\lbl[lt]{23,2;$P'$}
\end{lpic}
\end{wrapfigure}


According to Theorem~\ref{thm:perp-bisect}, $A$ and $B$ lie on the perpendicular bisector to~$[P P']$.
In particular, $(P P')\perp (A B)=\ell$.

\parit{Uniqueness for $P\not\in\ell$.} 
From above we can choose a point $P'$ in such a way that $\ell$ forms the perpendicular bisector to~$[PP']$.

Assume $m\perp \ell$ and $m\ni P$.
Then $m$ is a perpendicular bisector to some segment $[Q Q']$ of $\ell$;
in particular, $P Q= P Q'$.

}

{

\begin{wrapfigure}{i}{36mm}
\begin{lpic}[t(-2mm),b(-0mm),r(0mm),l(0mm)]{pics/perp-unique(1)}
\lbl[rb]{2,16;$Q$}
\lbl[lb]{33,16;$Q'$}
\lbl[lb]{18.5,28;$P$}
\lbl[lt]{18.5,2;$P'$}
\lbl[t]{14,14;$\ell$}
\lbl[b]{17,20,90;$m$}
\end{lpic}
\end{wrapfigure}

Since $\ell$ is the perpendicular bisector to $[P P']$,
we get that $PQ= P'Q$ and $PQ' \z= P'Q'$.
Therefore, 
$$P' Q=P Q=P Q'= P' Q'.$$
By Theorem~\ref{thm:perp-bisect}, 
$P'$ lies on the perpendicular bisector to $[QQ']$, which is~$m$.
By Axiom~\ref{def:birkhoff-axioms:1}, $m=(P P')$.
\qeds

}

\section*{Reflection}
\addtocontents{toc}{Reflection.}

Assume the point $P$ and the line $(AB)$ are given.
To find the \index{reflection}\emph{reflection} $P'$ of $P$ in $(AB)$,
one drops a perpendicular from $P$ onto $(AB)$, 
and continues it to the same distance on the other side.

According to Theorem~\ref{perp:ex+un}, $P'$ is uniquely determined by~$P$.

Note that $P=P'$ if and only if $P\in(AB)$.

\begin{thm}[\abs]{Proposition}\label{prop:reflection}
Assume $P'$ is a reflection of the point $P$ in the line~$(AB)$.
Then $AP'=AP$ 
and if $A\ne P$, 
then
$\measuredangle BAP'\equiv -\measuredangle BAP$.
\end{thm}

\parit{Proof.} 
Note that if $P\in (AB)$, 
then $P\z=P'$. 
By Corollary~\ref{cor:degenerate=pi}, $\measuredangle BAP=0$ or~$\pi$.
Hence the statement follows.

{
\begin{wrapfigure}{o}{43mm}
\begin{lpic}[t(-4mm),b(8mm),r(0mm),l(0mm)]{pics/reflection(1)}
\lbl[rb]{2,24;$A$}
\lbl[lb]{39,24;$B$}
\lbl[lb]{24.5,43.5;$P$}
\lbl[lt]{24.5,2.5;$P'$}
\end{lpic}
\end{wrapfigure}

If $P\notin (AB)$, then~$P'\ne P$.
By the construction of $P'$, 
the line $(AB)$ is perpendicular bisector of~$[PP']$.
Therefore, according to Theorem~\ref{thm:perp-bisect}, $AP'\z=AP$ and $BP'\z=BP$.
In particular, 
$\triangle ABP'\z\cong \triangle ABP$.
Therefore, $\measuredangle BAP'=\pm \measuredangle BAP$.

Since $P'\ne P$ and $AP'=AP$,
we get that $\measuredangle BAP'\ne \measuredangle BAP$.
That is, we are left with the case
$$\measuredangle BAP'=-\measuredangle BAP.$$
\qedsf

}
\newpage %???fix

\begin{thm}[\abs]{Corollary}\label{cor:reflection+angle}
The reflection in a line is a motion of the plane. 
Moreover, if $\triangle P'Q'R'$ is the reflection of $\triangle PQR$,
then 
$$\measuredangle Q'P'R'\equiv -\measuredangle QPR.$$

\end{thm}


\parit{Proof.}
From the construction, it follows that 
the composition of two reflections in the same line
is the identity map.
In particular, any reflection is a bijection.

Assume $P'$, $Q'$ and $R'$
denote the reflections of the points
$P$, $Q$ and $R$ in~$(AB)$. 
Let us show that
$$P'Q'=PQ
\quad
\text{and}
\quad
\measuredangle AP'Q'\equiv-\measuredangle APQ.
\eqlbl{eq:P'Q'=PQ}$$

Without loss of generality, we may assume that the points $P$ and $Q$ are distinct from $A$ and~$B$.
By Proposition~\ref{prop:reflection},
\begin{align*}
\measuredangle BAP'&\equiv -\measuredangle BAP,
&
\measuredangle BAQ'&\equiv -\measuredangle BAQ,
\\
AP'&=AP,
&
AQ'&=AQ.
\end{align*}
It follows that
$\measuredangle P'AQ'\equiv -\measuredangle PAQ$.
Therefore
$\triangle P'AQ'\cong\triangle PAQ$
and \ref{eq:P'Q'=PQ} follows.

Repeating the same argument for $P$ and $R$,
we get that
$$\measuredangle AP'R'\equiv-\measuredangle APR.$$
Subtracting the second identity in \ref{eq:P'Q'=PQ},
we get that
$$\measuredangle Q'P'R'\equiv-\measuredangle QPR.$$
\qedsf

\begin{thm}{Exercise}\label{ex:3-reflections}
Show that any motion of the plane can be presented as a 
composition of at most three reflections.
\end{thm}

Applying the exercise above 
and Corollary~\ref{cor:reflection+angle},
we can divide the motions of the plane in two types, 
\index{direct motion}\emph{direct} 
and 
\index{indirect motion}\emph{indirect motions}.
The motion $f$ is direct if 
$$\measuredangle Q'P'R'= \measuredangle QPR$$ 
for any $\triangle PQR$ and $P'=f(P)$, $Q'=f(Q)$ and $R'=f(R)$;
if instead we have 
$$\measuredangle Q'P'R'\equiv -\measuredangle QPR$$ 
for any $\triangle PQR$, then the motion $f$ is called indirect.

\begin{thm}{Exercise}\label{ex:2-reflections}
Let $X$ and $Y$ be the reflections of $P$ in the lines $(AB)$ and $(BC)$ correspondingly.
Show that 
$$\measuredangle XBY\equiv 2\cdot \measuredangle ABC.$$

\end{thm}

\section*{Perpendicular is shortest}
\addtocontents{toc}{Perpendicular is shortest.}

\begin{thm}[\abs]{Lemma}\label{lem:perp<oblique}
Assume $Q$ is the foot point of $P$ on the line~$\ell$.
Then the inequality
$$PX>PQ$$
holds for any point $X$ on $\ell$ distinct from~$Q$. 
\end{thm}

If $P$, $Q$ and $\ell$ are as above, 
then $PQ$ is called the \label{distance!from a point to a line}\index{distance!from a point to a line}\emph{distance from $P$ to $\ell$}. 

\begin{wrapfigure}[14]{o}{20mm}
\begin{lpic}[t(-2mm),b(0mm),r(0mm),l(0mm)]{pics/oblique(1)}
\lbl[tl]{17,21;$X$}
\lbl[t]{1,21;$\ell$}
\lbl[lb]{7.5,43;$P$}
\lbl[tl]{7,21;$Q$}
\lbl[lt]{7.5,2;$P'$}
\end{lpic}
\end{wrapfigure}

\parit{Proof.}
If $P\in \ell$, 
then the result follows since $PQ=0$.
Further we assume that $P\notin \ell$.

Let $P'$ be the reflection of $P$ in the line~$\ell$.
Note that $Q$ is the midpoint of $[PP']$
and $\ell$ is the perpendicular bisector of $[PP']$.
Therefore
$$PX=P'X
\quad
\text{and}
\quad
PQ=P'Q=\tfrac12\cdot PP'$$

Note that $\ell$ meets $[PP']$ only at the point $Q$.
Therefore, by the triangle inequality and Exercise~\ref{ex:degenerate-trig},
$$PX+P'X>PP'$$
and hence the result.
\qeds

\begin{thm}{Exercise}\label{ex:obtuce}
Assume $\angle ABC$ is right or obtuse.
Show that 
$$AC>AB.$$

\end{thm}






\section*{Circles}
\addtocontents{toc}{Circles.}

Given a positive real number $r$ and a point $O$,
the set $\Gamma$ of all points on distance $r$ from $O$ is called a \index{circle}\emph{circle} 
with \index{radius}\emph{radius} $r$ and \index{center}\emph{center}~$O$.

We say that a point $P$ lies \index{inside!inside a circle}\emph{inside} $\Gamma$ if $OP<r$; 
if $OP>r$, we say that $P$ lies \index{outside a circle}\emph{outside}~$\Gamma$.
\label{def:circle}

\begin{thm}{Exercise}\label{ex:inside-outside}
Let $\Gamma$ be a circle and $P\notin \Gamma$.
Assume a line $\ell$ is passing thru the point $P$
and intersects $\Gamma$ at two distinct points, $X$ and~$Y$.
Show that $P$ is inside $\Gamma$ if and only if $P$ lies between $X$ and~$Y$.
\end{thm}


A segment between two points on a circle is called a \index{chord}\emph{chord} of the circle.
A chord passing thru the center of the circle is called its \index{diameter}\emph{diameter}.
%A segment between a point on the circle and its center is called \index{radius}\emph{radius}.

\begin{thm}{Exercise}\label{ex:chord-perp}
Assume two distinct circles $\Gamma$ and $\Gamma'$ have a common chord~$[A B]$.
Show that the line between centers of $\Gamma$ and $\Gamma'$ forms a perpendicular bisector to~$[A B]$.
\end{thm}



\begin{thm}[\abs]{Lemma}\label{lem:line-circle}
A line and a circle can have at most two points of intersection.
\end{thm}

\parit{Proof.} Assume $A$, $B$ and $C$ are distinct points which lie on a line $\ell$ and a circle $\Gamma$ with the center~$O$.

Then $OA=OB=OC$; in particular, $O$ lies on the perpendicular bisectors 
$m$ and $n$ to $[A B]$ and $[B C]$ correspondingly.

Note that the midpoints of $[AB]$ and $[BC]$ are distinct.
Therefore, $m$ and $n$ are distinct.
The latter contradicts the uniqueness of the perpendicular (Theorem~\ref{perp:ex+un}).
\qeds

\begin{center}
\begin{lpic}[t(-0mm),b(0mm),r(0mm),l(0mm)]{pics/line-circ(1)}
\lbl[t]{2,3.5;$A$}
\lbl[t]{23,3.5;$B$}
\lbl[t]{63,3.5;$C$}
\lbl[b]{29,6.5;$\ell$}
\lbl[b]{12,17,90;$m$}
\lbl[b]{42,17,90;$n$}
\end{lpic}
\end{center}

\begin{thm}{Exercise}\label{ex:two-circ}
Show that two distinct circles can have at most two points of intersection.
\end{thm}

In consequence of the above lemma, 
a line $\ell$ and a circle $\Gamma$ might have 2, 1 or 0 points of intersections.
In the first case the line is called \index{secant line}\emph{secant line}, in the second case it is \index{tangent!line}\emph{tangent line};
if $P$ is the only point of intersection of $\ell$ and $\Gamma$,
we say that {}\emph{$\ell$ is tangent to $\Gamma$ at $P$}. 

Similarly, according Exercise \ref{ex:two-circ},
two circles might have 2, 1 or 0 points of intersections.
If $P$ is the only point of intersection of circles $\Gamma$ and $\Gamma'$,
we say that \index{tangent!circles}\emph{$\Gamma$ is tangent to $\Gamma$ at $P$}. 

\begin{thm}[\abs]{Lemma}\label{lem:tangent}
Let $\ell$ be a line and $\Gamma$ be a circle with the center~$O$.
Assume $P$ is a common point of $\ell$ and~$\Gamma$. 
Then $\ell$ is tangent to $\Gamma$ at $P$ if and only if $(PO)\perp \ell$.
\end{thm}

\parit{Proof.}
Let $Q$ be the foot point of $O$ on~$\ell$.

Assume~$P\ne Q$.
Let $P'$ denotes the reflection of $P$ in~$(OQ)$.

Note that $P'\in\ell$ and $(OQ)$ is the perpendicular bisector of~$[PP']$.
Therefore, $OP=OP'$.
Hence $P,P'\in \Gamma\cap \ell$;
that is, $\ell$ is secant to~$\Gamma$.

If $P=Q$, 
then according to Lemma~\ref{lem:perp<oblique},
$OP<OX$ for any point $X\in \ell$ distinct from~$P$.
Hence $P$ is the only point in the intersection $\Gamma\cap\ell$;
that is, $\ell$ is tangent to $\Gamma$ at~$P$. 
\qeds

\begin{thm}{Exercise}\label{ex:tangent-circles}
Let $\Gamma$ and $\Gamma'$ be two distinct circles with centers at $O$ and $O'$ correspondingly. 
Assume $\Gamma$ meets $\Gamma'$ at the point~$P$.
Show that $\Gamma$ is tangent to $\Gamma'$ if and only if $O$, $O'$ and $P$ lie on one line.
\end{thm}

\begin{thm}{Exercise}\label{ex:tangent-circles-2}
Let $\Gamma$ and $\Gamma'$ be two distinct circles with centers at $O$ and $O'$ and radii $r$ and~$r'$.

\begin{enumerate}[(a)]
\item\label{ex:tangent-circles-2:a} Show that $\Gamma$ is tangent to $\Gamma'$ if and only if
$$OO'=r+r'
\quad
\text{or}\quad
OO'=|r-r'|.$$
\item \label{ex:tangent-circles-2:b}
Show that $\Gamma$ intersects $\Gamma'$ if and only if
$$|r-r'|\le OO'\le r+r'.$$

\end{enumerate}

\end{thm}

{

\begin{wrapfigure}{r}{20mm}
\begin{lpic}[t(-10mm),b(0mm),r(0mm),l(1mm)]{pics/3-circ-2-point(1)}
\end{lpic}
\end{wrapfigure}

\begin{thm}{Exercise}\label{ex:tangent-circles-3}
Assume three circles intersect at two points as shown on the diagram.
Prove that the centers of these circles lie on one line.
\end{thm}

}

\section*{Geometric constructions}
\addtocontents{toc}{Geometric constructions.}

The \index{ruler-and-compass construction}\emph{ruler-and-compass constructions} in the plane is the construction of points, lines, and circles using only an idealized ruler and compass.
These construction problems provide a valuable source of exercises in geometry, 
which we will use further in the book.
In addition, Chapter~\ref{chap:car} is devoted completely to the subject.

The idealized ruler can be used only to draw a line thru the given two points.
The idealized compass can be used only to draw a circle with a given center and radius.
That is, given three points $A$, $B$ and $O$ 
we can draw the set of all points on distant $AB$ from~$O$.
We may also mark new points in the plane
as well as on the constructed lines, circles 
and their intersections (assuming that such points exist).

We can also look at the different set of construction tools.
For example,
we may only use the ruler or
we may invent a new tool, 
say a tool which produces a midpoint for any given two points.

As an example, let us consider the following problem:

\begin{thm}{Construction of midpoint}
Construct the midpoint of the given segment~$[AB]$.
\end{thm}

\parit{Construction.}
\begin{enumerate}[1.]
\item Construct the circle 
with center at $A$ 
which is passing thru~$B$.
\item Construct the circle 
with center at $B$ 
which is passing thru~$A$.
\item Mark both points of intersection of these circles, label them with $P$ and~$Q$.
\item Draw the line~$(PQ)$.
\item Mark the point of intersection of $(PQ)$ and $[AB]$; this is the midpoint.
\end{enumerate}

\begin{wrapfigure}[9]{o}{31mm}
\begin{lpic}[t(-0mm),b(0mm),r(0mm),l(1mm)]{pics/constr-perp-bisect(1)}
\lbl[b]{4.3,28.5;$A$}
\lbl[t]{26,12;$B$}
\lbl[t]{26,35;$P$}
\lbl[b]{4.5,5;$Q$}
\lbl[b]{14.5,22.5;$M$}
\end{lpic}
\end{wrapfigure}

\medskip

Typically, you need to prove that the construction produces what was expected. Here is a proof for the example above.

\parit{Proof.}
According to Theorem~\ref{thm:perp-bisect}, $(PQ)$ is the perpendicular bisector to~$[AB]$.
Therefore, $M\z=(AB)\cap(PQ)$ is the midpoint of~$[AB]$. 
\qeds

\begin{thm}{Exercise}\label{ex:construction-perpendicular}
Make a ruler-and-compass construction of a line thru a given point which is perpendicular to a given line.
\end{thm}

\begin{thm}{Exercise}\label{ex:center}
Make a ruler-and-compass construction of the center 
of a given circle.
\end{thm}

\begin{thm}{Exercise}\label{ex:tangent}
Make a ruler-and-compass construction of the lines tangent to a given circle which pass thru a given point.
\end{thm}

\begin{thm}{Exercise}\label{ex:tangent-circle}
Given two circles $\Gamma_1$ and $\Gamma_2$ and a segment $[AB]$
make a ruler-and-compass construction of a circle with the radius $AB$, 
which is tangent to each circle $\Gamma_1$ and~$\Gamma_2$.
\end{thm}

%??? add ``Locus intersection method''

\addtocontents{toc}{\protect\end{quote}}
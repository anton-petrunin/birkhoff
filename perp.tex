\chapter{Perpendicular lines}\label{chap:perp}

\section{Right, acute and obtuse angles}

\begin{itemize}
\item If $|\measuredangle A O B|=\tfrac\pi2$, we say that $\angle A O B$ is \index{angle!right angle}\emph{right};
%\item If $\measuredangle A O B\ne\pm\tfrac\pi2$, we say that the angle $\angle A O B$ is \index{angle!oblique angle}\emph{oblique};
\item If $|\measuredangle A O B|<\tfrac\pi2$, we say that $\angle A O B$ is 
\index{acute angle}\emph{acute};
\item If $|\measuredangle A O B|>\tfrac\pi2$, we say that $\angle A O B$ is \index{angle!acute and obtuse angles}\index{obtuse angle}\emph{obtuse}.
\end{itemize}

\begin{wrapfigure}[2]{r}{25mm}
\vskip-25mm
\centering
\includegraphics{mppics/pic-46}
\end{wrapfigure}

Right angles will be marked with a small square, as shown.

If $\angle A O B$ is right, we also may say that $[O A)$ is \index{perpendicular}\emph{perpendicular} to $[O B)$.
This can be written as \index{38@$\perp$}$[O A)\z\perp [O B)$.

Theorem~\ref{thm:straight-angle} allows us to call two lines $(O A)$ and $(O B)$ {}\emph{perpendicular} if $[O A)\z\perp [O B)$; so, we can write $(O A)\z\perp (O B)$.



\begin{thm}{Exercise}\label{ex:acute-obtuce}
Assume that point $O$ lies between $A$ and $B$ and $X\ne O$.
Show that 
$\angle XOA$ is acute if and only if 
$\angle XOB$ is obtuse.
\end{thm}



\section{Perpendicular bisector}

Assume $M$ is the midpoint of the segment $[AB]$;
equivalently, $M\in(A B)$ and $A M \z= M B$.

The line $\ell$ thru $M$ and perpendicular to $(AB)$,
is called the \index{bisector!perpendicular bisector}\index{perpendicular!bisector}\emph{perpendicular bisector} to the segment~$[AB]$. 

\begin{thm}[\abs]{Theorem}\label{thm:perp-bisect}
Given distinct points $A$ and $B$,
all points that are equidistant from $A$ and $B$ and no
others lie on the perpendicular bisector of~$[A B]$.
\end{thm}

\begin{wrapfigure}{o}{34mm}
\centering
\includegraphics{mppics/pic-48}
\end{wrapfigure}

\parit{Proof.} Let $M$ be the midpoint of~$[A B]$.

Assume $P A= P B$ and $P\ne M$.
According to SSS (\ref{thm:SSS}),
$\triangle A M P \z\cong\triangle B M P$.
Hence 
$$\measuredangle A M P=\pm \measuredangle B M P.$$  
Since $A\not=B$, we have ``$-$'' in the above formula.
Furthermore,
\begin{align*}
\pi
&=
\measuredangle A M B
\equiv
\\
&\equiv\measuredangle A M P+\measuredangle P M B
\equiv
\\
&\equiv
2\cdot \measuredangle A M P.
\end{align*}
That is, $\measuredangle A M P
=
\pm
\tfrac\pi2$. 
Therefore, $P$ lies on the perpendicular bisector.


To prove the converse, 
suppose $P$ 
is any point on the perpendicular bisector of $[A B]$ and $P\ne M$.
Then $\measuredangle A M P=\pm \tfrac\pi2$, 
$\measuredangle B M P=\pm \tfrac\pi2$ and
$A M\z=B M$.
By SAS, $\triangle A M P\cong \triangle B M P$;
in particular, $A P\z= B P$.\qeds


\begin{thm}{Exercise}\label{ex:pbisec-side}
Let $\ell$ be a perpendicular bisector of $[A B]$, and $X$ be an arbitrary point on the plane.

Show that 
$AX<BX$ if and only if $X$ and $A$ lie on the same side from~$\ell$.
\end{thm}

\begin{thm}{Exercise}\label{ex:side-angle}
Let $ABC$ be a nondegenerate triangle.
Show that 
\[AC>BC\iff|\measuredangle ABC|>|\measuredangle CAB|.\] 
\end{thm}

\section{Uniqueness of a perpendicular}

\begin{thm}[\abs]{Theorem}\label{perp:ex+un}
There is one and only one line that passes thru a given point $P$ and is perpendicular to a given line~$\ell$.
\end{thm}

According to the theorem above, 
there is a unique point $Q\in\ell$ such that $(QP)\perp\ell$.
This point $Q$ is called the \index{footpoint}\emph{footpoint} of $P$ on~$\ell$. 

\parit{Proof.} 
If $P\in\ell$, then both existence and uniqueness follow from Axiom~\ref{def:birkhoff-axioms:2}.

{

\begin{wrapfigure}{o}{30mm}
\vskip-4mm
\centering
\includegraphics{mppics/pic-50}
\end{wrapfigure}

\parit{Existence for $P\not\in\ell$.} 
Let $A$ and $B$ be two distinct points on~$\ell$. 
Choose $P'$ so that $AP'\z=AP$ and $\measuredangle  BAP' \equiv -\measuredangle   BAP$.
According to Axiom~\ref{def:birkhoff-axioms:3}, $\triangle A P' B\z\cong\triangle A P B$.
In particular, $A P= A P'$ and $B P= B P'$.

According to Theorem~\ref{thm:perp-bisect}, $A$ and $B$ lie on the perpendicular bisector of~$[P P']$.
In particular, $(P P')\perp (A B)=\ell$.

}

{

\begin{wrapfigure}{r}{44mm}
\centering
\includegraphics{mppics/pic-52}
\end{wrapfigure}

\parit{Uniqueness for $P\not\in\ell$.} 
From above we can choose a point $P'$ in such a way that $\ell$ forms the perpendicular bisector of~$[PP']$.

Assume $m\perp \ell$ and $m\ni P$.
Then $m$ is a perpendicular bisector of some segment $[Q Q']$ of $\ell$;
in particular, $P Q= P Q'$.

Since $\ell$ is the perpendicular bisector of $[P P']$,
we get that $PQ= P'Q$ and $PQ' \z= P'Q'$.
Therefore, 
$$P' Q=P Q=P Q'= P' Q'.$$
By Theorem~\ref{thm:perp-bisect}, 
$P'$ lies on the perpendicular bisector of $[QQ']$, which is~$m$.
By Axiom~\ref{def:birkhoff-axioms:1}, $m=(P P')$.
\qeds

}

\section{Reflection across a line}

Assume the point $P$ and the line $(AB)$ are given.
To find the \index{reflection!across a line}\emph{reflection} $P'$ of $P$ across $(AB)$,
one drops a perpendicular from $P$ onto $(AB)$, 
and continues it to the same distance on the other side.

According to Theorem~\ref{perp:ex+un}, $P'$ is uniquely determined by~$P$.

Note that $P=P'$ if and only if $P\in(AB)$.

\begin{thm}[\abs]{Proposition}\label{prop:reflection}
Assume $P'$ is the reflection of $P$ across $(AB)$.
Then $AP'=AP$, 
and if $A\ne P$, 
then
$\measuredangle BAP'\equiv -\measuredangle BAP$.
\end{thm}

\parit{Proof.} 
If $P\in (AB)$, 
then $P\z=P'$. 
By Corollary~\ref{cor:degenerate=pi}, $\measuredangle BAP\z=0$ or~$\pi$.
Hence the statement follows.

{

\begin{wrapfigure}{o}{37mm}
\vskip-2mm
\centering
\includegraphics{mppics/pic-54}
\end{wrapfigure}

If $P\notin (AB)$, then~$P'\ne P$.
By the construction of $P'$, 
the line $(AB)$ is a perpendicular bisector of~$[PP']$.
Therefore, according to Theorem~\ref{thm:perp-bisect}, $AP'\z=AP$ and $BP'\z=BP$.
In particular, 
$\triangle ABP'\z\cong \triangle ABP$.
Therefore, $\measuredangle BAP'=\pm \measuredangle BAP$.

Since $P'\ne P$ and $AP'=AP$,
we get that $\measuredangle BAP'\ne \measuredangle BAP$.
That is, we are left with the case $\measuredangle BAP'\equiv-\measuredangle BAP$.
\qeds

}

\begin{thm}{Exercise}\label{ex:2-reflections}
Let $X$ and $Y$ be the reflections of $P$ across the lines $(AB)$ and $(BC)$ respectively.
Show that 
$$\measuredangle XBY\equiv 2\cdot \measuredangle ABC.$$

\end{thm}


\begin{thm}[\abs]{Corollary}\label{cor:reflection+angle}
A reflection across a line is a motion of the plane. 
Moreover,
$$\measuredangle Q'P'R'\equiv -\measuredangle QPR$$
if $\triangle P'Q'R'$ is the reflection of $\triangle PQR$.
\end{thm}


\parit{Proof.}
The composition of two reflections across the same line
is the identity map.
In particular, any reflection is a bijection.

Fix a line $(AB)$ and two points $P$ and $Q$;
denote their reflections across $(AB)$ by $P'$ and $Q'$.
Let us show that
$$P'Q'=PQ;\eqlbl{eq:P'Q'=PQ}$$
that is, the reflection is distance-preserving,

\begin{wrapfigure}{r}{33mm}
\centering
\vskip-15mm
\includegraphics{mppics/pic-56}
\end{wrapfigure}

Without loss of generality, we may assume that the points $P$ and $Q$ are distinct from $A$ and~$B$.
By Proposition~\ref{prop:reflection}, we get that
\begin{align*}
\measuredangle BAP'&\equiv -\measuredangle BAP,
&
\measuredangle BAQ'&\equiv -\measuredangle BAQ,
\\
AP'&=AP,
&
AQ'&=AQ.
\end{align*}
It follows that
\[\measuredangle P'AQ'\equiv -\measuredangle PAQ.\eqlbl{eq:P'AQ'=PAQ}\]
By SAS, 
$\triangle P'AQ'\cong\triangle PAQ$
and \ref{eq:P'Q'=PQ} follows.
Moreover, we also get that 
\[\measuredangle AP'Q'\equiv\pm\measuredangle APQ.\]
From \ref{eq:P'AQ'=PAQ} and the theorem on the signs of angles of triangles (\ref{thm:signs-of-triug}) we get
\[\measuredangle AP'Q'\equiv-\measuredangle APQ.\eqlbl{eq:AP'Q'=APQ}\]

Repeating the same argument for a pair of points $P$ and $R$,
we get that
$$\measuredangle AP'R'\equiv-\measuredangle APR.\eqlbl{eq:AP'R'=APR}$$
Subtracting \ref{eq:AP'R'=APR} from \ref{eq:AP'Q'=APQ},
we get that
$$\measuredangle Q'P'R'\equiv-\measuredangle QPR.$$
\qedsf

\section{Direct and indirect motions}
\label{direct motion}

A motion $X\mapsto X'$ is called \index{direct motion}\emph{direct} if 
$$\measuredangle Q'P'R'= \measuredangle QPR$$ 
for any triangle $PQR$;
if instead we always have 
$$\measuredangle Q'P'R'\equiv -\measuredangle QPR,$$ 
then the motion $f$ is called \index{indirect motion}\emph{indirect}.

By Corollary~\ref{cor:reflection+angle}, any reflection across a line is an indirect motion.
Note that the composition of two reflections is a direct motion.
More generally, the composition of two indirect motions is direct,
the composition of two direct motions is direct,
and the composition of direct and indirect motions is indirect.

\begin{thm}{Exercise}\label{ex:3-reflections}
Show that any motion of the plane can be presented as a 
composition of at most three reflections across lines.

Conclude that any motion of the plane is either direct or indirect.
\end{thm} 

\section{Perpendicular is shortest}
\label{sec:perp<oblique}

\begin{thm}[\abs]{Lemma}\label{lem:perp<oblique}
Assume $Q$ is the footpoint of $P$ on the line~$\ell$.
Then 
$$PX>PQ$$
for any point $X$ on $\ell$ distinct from~$Q$. 
\end{thm}

If $P$, $Q$, and $\ell$ are as above, 
then $PQ$ is called the \label{distance!from a point to a line}\index{distance!from a point to a line}\emph{distance from $P$ to~$\ell$}. 

\parit{Proof.}
If $P\in \ell$, 
then the result follows since $PQ=0$.
So, we can assume that $P\notin \ell$.

\begin{wrapfigure}{o}{24mm}
\centering
\includegraphics{mppics/pic-58}
\vskip3mm
\end{wrapfigure}

Let $P'$ be the reflection of $P$ across the line~$\ell$.
Note that $Q$ is the midpoint of $[PP']$
and $\ell$ is the perpendicular bisector of $[PP']$.
Therefore
$$PX=P'X
\quad
\text{and}
\quad
PQ=P'Q=\tfrac12\cdot PP'.$$

The line $\ell$ meets $[PP']$ only at the point $Q$.
Therefore, $X\notin [PP']$.
By the triangle inequality and Corollary~\ref{cor:degenerate-trig}, we have 
$$PX+P'X>PP',$$
and hence the result: $PX>PQ$.
\qeds

\begin{thm}{Exercise}\label{ex:obtuce}
Assume $\angle ABC$ is right or obtuse.
Show that 
\[AC>AB.\]

\end{thm}

\begin{thm}{Exercise}\label{ex:right-triangle-inq}
Suppose that $\triangle ABC$ has a right angle at $C$.
Show that for any $X\in [AC]$ the distance from $X$ to $(AB)$ is smaller than $AB$.
\end{thm}


\section{Circles}

Recall that a circle with radius $r$ and center $O$ is the set of all points on distance $r$ from $O$.
We say that a point $P$ lies \index{inside!a circle}\emph{inside} of the circle if $OP<r$; 
if $OP>r$, we say that $P$ lies \index{outside a circle}\emph{outside} of the circle.
\label{def:circle}

\begin{thm}{Exercise}\label{ex:inside-outside}
Let $\Gamma$ be a circle and $P\notin \Gamma$.
Assume a line $\ell$ is passing thru the point $P$
and intersects $\Gamma$ at two distinct points, $X$ and~$Y$.
Show that $P$ is inside $\Gamma$ if and only if $P$ lies between $X$ and~$Y$.
\end{thm}


A segment between two points on a circle is called a \index{chord}\emph{chord} of the circle.
A chord passing thru the center of the circle is called its \index{diameter}\emph{diameter}.

\begin{thm}{Exercise}\label{ex:chord-perp}
Assume two distinct circles $\Gamma$ and $\Gamma'$ have a common chord~$[A B]$.
Show that the line between centers of $\Gamma$ and $\Gamma'$ forms a perpendicular bisector of~$[A B]$.
\end{thm}



\begin{thm}[\abs]{Lemma}\label{lem:line-circle}
A line and a circle can have at most two points of intersection.
\end{thm}

\begin{wrapfigure}{o}{54mm}
\vskip-4mm
\centering
\includegraphics{mppics/pic-60}
\end{wrapfigure}

\parit{Proof.} Assume $A$, $B$, and $C$ are distinct points that lie on a line $\ell$ and a circle $\Gamma$ with the center~$O$.
Then $OA=OB=OC$; in particular, $O$ lies on the perpendicular bisectors 
$m$ and $n$ to $[A B]$ and $[B C]$ respectively.
Note that the midpoints of $[AB]$ and $[BC]$ are distinct.
Therefore, $m$ and $n$ are distinct.
The latter contradicts the uniqueness of the perpendicular (Theorem~\ref{perp:ex+un}).
\qeds

\begin{thm}{Exercise}\label{ex:two-circ}
Show that two distinct circles can have at most two points of intersection.
\end{thm}

As a consequence of the above lemma, 
a line $\ell$ and a circle $\Gamma$ might have 2, 1, or 0 points of intersections.
In the first two cases, the line is called \index{secant line}\emph{secant} or \index{tangent!line}\emph{tangent} respectively;
if $P$ is the only point of intersection of $\ell$ and $\Gamma$,
we say that \textit{$\ell$ is tangent to $\Gamma$ at $P$}. 

Similarly, according to Exercise~\ref{ex:two-circ},
two distinct circles might have 2, 1, or 0 points of intersections.
If $P$ is the only point of intersection of circles $\Gamma$ and $\Gamma'$,
we say that \index{tangent!circles}\emph{$\Gamma$ is tangent to $\Gamma$ at $P$}; we also assume that a circle is tangent to itself at any of its points.

\begin{thm}[\abs]{Lemma}\label{lem:tangent}
Let $\ell$ be a line and $\Gamma$ be a circle centered at~$O$.
Assume $P$ is a common point of $\ell$ and~$\Gamma$. 
Then $\ell$ is tangent to $\Gamma$ at $P$ if and only if $(PO)\perp \ell$.
\end{thm}

\parit{Proof.}
Let $Q$ be the footpoint of $O$ on~$\ell$.

Assume~$P\ne Q$.
Let $P'$ be the reflection of $P$ across~$(OQ)$.
Note that $P'\in\ell$ and $(OQ)$ is the perpendicular bisector of~$[PP']$.
Therefore, $OP=OP'$.
Hence $P,P'\in \Gamma\cap \ell$;
that is, $\ell$ is secant to~$\Gamma$.

If $P=Q$, 
then according to Lemma~\ref{lem:perp<oblique},
$OP<OX$ for any point $X\in \ell$ distinct from~$P$.
Hence $P$ is the only point at the intersection $\Gamma\cap\ell$;
that is, $\ell$ is tangent to $\Gamma$ at~$P$. 
\qeds

\begin{thm}{Exercise}\label{ex:tangent-circles}
Let $\Gamma$ and $\Gamma'$ be two distinct circles with centers at $O$ and $O'$ respectively. 
Assume $\Gamma$ meets $\Gamma'$ at a point~$P$.
Show that $\Gamma$ is tangent to $\Gamma'$ if and only if $O$, $O'$, and $P$ lie on one line.
\end{thm}

\begin{thm}{Exercise}\label{ex:tangent-circles-2}
Let $\Gamma$ and $\Gamma'$ be two distinct circles with centers at $O$ and $O'$ and radiuses $r$ and~$r'$.
Show that $\Gamma$ is tangent to $\Gamma'$ if and only if
$$OO'=r+r'
\quad
\text{or}\quad
OO'=|r-r'|.$$

\end{thm}


{

\begin{wrapfigure}{r}{20mm}
\vskip-8mm
\centering
\includegraphics{mppics/pic-62}
\end{wrapfigure}

\begin{thm}{Exercise}\label{ex:tangent-circles-3}
Assume three circles have two points in common.
Prove that their centers lie on one line.
\end{thm}

}

\section{Geometric constructions}

A \index{ruler-and-compass construction}\emph{ruler-and-compass construction} in the plane is a construction of points, lines, and circles using only an idealized ruler and compass.
These construction problems provide a valuable source of exercises in geometry, 
which we will use further in the book.
In addition, Chapter~\ref{chap:car} is devoted completely to the subject.

The idealized ruler can be used only to draw a line thru the given two points.
The idealized compass can be used only to draw a circle with a given center and radius.
That is, given three points $A$, $B$, and $O$ 
we can draw the set of all points on distance $AB$ from~$O$.
We may also mark new points in the plane,
as well as on the constructed lines, circles, 
and their intersections (assuming that such points exist).

We can also look at the different sets of construction tools.
For example,
we may only use the ruler,
or we may invent a new tool, 
say a tool that produces a midpoint for any given two points.

As an example, let us consider the following problem:

\begin{thm}{Construction of midpoint}
Construct a midpoint of a given segment~$[AB]$.
\end{thm}

\parit{Construction.}
\begin{enumerate}[1.]
\item Construct the circle 
with center $A$ 
that is passing thru~$B$.
Construct the circle 
with center $B$ 
that is passing thru~$A$.
Mark both points of intersection of these circles; label them with $P$ and~$Q$.
\item Draw the line~$(PQ)$.
Mark by $M$ of the intersection point of $(PQ)$ and $[AB]$; this is the midpoint.
\end{enumerate}

\begin{figure}[!ht]
\centering
\includegraphics{mppics/pic-64}
\end{figure}

Typically, you need to prove that the construction produces what was expected. Here is a proof for the example above.

\parit{Proof.}
According to Theorem~\ref{thm:perp-bisect}, $(PQ)$ is the perpendicular bisector of~$[AB]$.
Therefore, $M\z=(AB)\cap(PQ)$ is the midpoint of~$[AB]$. 
\qeds



\begin{thm}{Exercise}\label{ex:construction-perpendicular}
Describe a ruler-and-compass construction of a line thru a given point that is perpendicular to a given line.
\end{thm}

\begin{thm}{Exercise}\label{ex:center}
Describe a ruler-and-compass construction of a center 
of a given circle.
\end{thm}

\begin{thm}{Exercise}\label{ex:tangent}
Describe a ruler-and-compass construction of lines tangent to a given circle that pass thru a given point.
\end{thm}

\begin{thm}{Exercise}\label{ex:tangent-circle}
Given two circles $\Gamma_1$ and $\Gamma_2$ and a segment $[AB]$
describe a ruler-and-compass construction of a circle with the radius $AB$ 
that is tangent to each circle $\Gamma_1$ and~$\Gamma_2$.
\end{thm}

%??? add ``Locus intersection method''

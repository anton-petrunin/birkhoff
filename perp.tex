\chapter{Perpendicular lines}\label{chap:perp}
\addtocontents{toc}{\protect\begin{quote}}

\section*{Right, acute and obtuse angles}
\addtocontents{toc}{Right, acute and obtuse angles.}

\begin{itemize}
\item If $|\measuredangle A O B|=\tfrac\pi2$, we say that the angle $\angle A O B$ is  \index{angle!right angle}\emph{right};
%\item If $\measuredangle A O B\ne\pm\tfrac\pi2$, we say that the angle  $\angle A O B$ is  \index{angle!oblique angle}\emph{oblique};
\item If $|\measuredangle A O B|<\tfrac\pi2$, we say that the angle  $\angle A O B$ is  
\index{acute!angle}\emph{acute};
\item If $|\measuredangle A O B|>\tfrac\pi2$, we say that the angle $\angle A O B$ is  \index{angle!acute and obtuse angles}\index{obtuse angle}\emph{obtuse}.
\end{itemize}

\begin{wrapfigure}{o}{30mm}
\begin{lpic}[t(-0mm),b(0mm),r(0mm),l(2mm)]{pics/perp-notation(1)}
\end{lpic}
\end{wrapfigure}

On the diagrams,
the right angles will be marked with a little square, 
as shown.

If $\angle A O B$ is right,
we say also
that $[O A)$ is \index{perpendicular}\emph{perpendicular} to $[O B)$; 
it will be written as \index{1rel@$\perp$}$[O A)\z\perp [O B)$.

From Theorem~\ref{thm:straight-angle}, 
it follows that two lines $(O A)$
 and $(O B)$ are appropriately called {}\emph{perpendicular}, if  $[O A)\z\perp [O B)$.
In this case we also write $(O A)\z\perp (O B)$.



\begin{thm}{Exercise}\label{ex:acute-obtuce}
Assume point $O$ lies between $A$ and $B$ and $X\ne O$.
Show that 
$\angle XOA$ is acute if and only if 
$\angle XOB$ is obtuse.
\end{thm}



\section*{Perpendicular bisector}
\addtocontents{toc}{Perpendicular bisector.}

Assume $M$ is the midpoint of the segment $[AB]$;
that is, $M\in(A B)$ and $A M \z=  M B$.


The line $\ell$ passing thru $M$ and perpendicular to $(AB)$
is called \index{bisector!perpendicular bisector}\index{perpendicular bisector}\emph{perpendicular bisector} to the segment~$[AB]$. 

\begin{thm}{Theorem}\label{thm:perp-bisect}
Given distinct points $A$ and $B$,
all points equidistant from $A$ and $B$ and no
others lie on the perpendicular bisector to~$[A B]$.
\end{thm}

\begin{wrapfigure}{o}{40mm}
\begin{lpic}[t(-0mm),b(0mm),r(0mm),l(0mm)]{pics/perp-bisect(1)}
\lbl[t]{2,7;$A$}
\lbl[t]{38,7;$B$}
\lbl[tl]{20.5,7;$M$}
\lbl[lb]{20.5,36;$P$}
\end{lpic}
\end{wrapfigure}

\parit{Proof.} Let $M$ be the midpoint of~$[A B]$.

Assume $P A= P B$ and $P\ne M$.
According to SSS (\ref{thm:SSS}),
$\triangle A M P \z\cong\triangle B M P$.
Hence 
$$\measuredangle A M P=\pm \measuredangle B M P.$$   
Since $A\not=B$, we have ``$-$'' in the above formula.
Further,
\begin{align*}
\pi
&=
\measuredangle A M B
\equiv
\\
&\equiv\measuredangle A M P+\measuredangle P M B
\equiv
\\
&\equiv
2\cdot \measuredangle A M P.
\end{align*}
That is, $\measuredangle A M P
=
\pm
\tfrac\pi2$. 
Therefore, $P$ lies on the perpendicular bisector.


To prove the converse, 
suppose $P\ne M$ 
is any point on the perpendicular bisector to $[A B]$.
Then $\measuredangle A M P=\pm \tfrac\pi2$, 
$\measuredangle B M P=\pm \tfrac\pi2$ and
$A M\z=B M$.
Therefore, $\triangle A M P\cong \triangle B M P$;
in particular $A P\z= B P$.\qeds


\begin{thm}{Exercise}\label{ex:pbisec-side}
Let $\ell$ be the perpendicular bisector to the segment $[A B]$ and $X$ be an arbitrary point on the plane.

Show that 
$AX<BX$ if and only if $X$ and $A$ lie on the same side from~$\ell$.
\end{thm}

\begin{thm}{Exercise}\label{ex:side-angle}
Let $\triangle ABC$ be nondegenerate.
Show that $AC>BC$ if and only if $|\measuredangle ABC|>|\measuredangle CAB|$.  
\end{thm}



\section*{Uniqueness of perpendicular}
\addtocontents{toc}{Uniqueness of perpendicular.}

\begin{thm}{Theorem}\label{perp:ex+un}
There is one and only one line  which pass thru a given point $P$ and perpendicular to a given line~$\ell$.
\end{thm}

\begin{wrapfigure}{o}{36mm}
\begin{lpic}[t(-0mm),b(-3mm),r(0mm),l(0mm)]{pics/perp(1)}
\lbl[rb]{1.5,18;$A$}
\lbl[lb]{33,18;$B$}
\lbl[t]{15,16;$\ell$}
\lbl[lb]{23,33;$P$}
\lbl[lt]{23,2;$P'$}
\end{lpic}
\end{wrapfigure}

According to the above theorem, 
there is unique point $Q\in\ell$ such that $(QP)\perp\ell$.
This point $Q$ is called \index{foot point}\emph{foot point} of $P$ on~$\ell$. 

\parit{Proof.} 
If $P\in\ell$, then both, existence and uniqueness, follow from Axiom~\ref{def:birkhoff-axioms:2}.

\parit{Existence for $P\not\in\ell$.} 
Let $A$, $B$ be two distinct points of~$\ell$. 
Choose $P'$ so that $AP'\z=AP$ and $\measuredangle P' A B\equiv -\measuredangle P A B$.
According to Axiom~\ref{def:birkhoff-axioms:3}, $\triangle A P' B\z\cong\triangle A P B$.
Therefore, $A P= A P'$ and $B P= B P'$.


According to Theorem~\ref{thm:perp-bisect}, $A$ and $B$ lie on the perpendicular bisector to~$[P P']$.
In particular $(P P')\perp (A B)=\ell$.

\begin{wrapfigure}{i}{36mm}
\begin{lpic}[t(-8mm),b(-3mm),r(0mm),l(0mm)]{pics/perp-unique(1)}
\lbl[rb]{2,16;$Q$}
\lbl[lb]{33,16;$Q'$}
\lbl[lb]{18.5,28;$P$}
\lbl[lt]{18.5,2;$P'$}
\lbl[t]{14,14;$\ell$}
\lbl[b]{17,20,90;$m$}
\end{lpic}
\end{wrapfigure}

\parit{Uniqueness for $P\not\in\ell$.} 
From above we can choose a point $P'$ in such a way that $\ell$ forms the perpendicular bisector to~$[PP']$.

Assume $m\perp \ell$ and $m\ni P$.
Then $m$ is a perpendicular bisector to some segment $[Q Q']$ of $\ell$;
in particular, $P Q= P Q'$.

Since $\ell$ is perpendicular bisector to $[P P']$,
we get $PQ= P'Q$ and $PQ' = P'Q'$.
Therefore, 
$$P' Q=P Q=P Q'= P' Q'.$$
By Theorem~\ref{thm:perp-bisect}, 
$P'$ lies on the perpendicular bisector to $[QQ']$ which is~$m$.
By Axiom~\ref{def:birkhoff-axioms:1}, $m=(P P')$.
\qeds


\section*{Reflection}
\addtocontents{toc}{Reflection.}

Assume a point $P$ and a line $(AB)$ are given.
To find the \index{reflection}\emph{reflection} $P'$ of $P$   in $(AB)$,
one drops a perpendicular from $P$ onto $(AB)$, 
and continues it to the same distance on the other side.

According to Theorem~\ref{perp:ex+un}, $P'$ is uniquely determined by~$P$.

Note that $P=P'$ if and only if $P\in(AB)$.

\begin{thm}{Proposition}\label{prop:reflection}
Assume $P'$ is a reflection of the point $P$ in the line~$(AB)$.
Then $AP'=AP$ 
and if $A\ne P$, 
then
$\measuredangle BAP'\equiv -\measuredangle BAP$.
\end{thm}

\parit{Proof.} 
Note that if $P\in (AB)$, 
then $P\z=P'$ 
and by Corollary~\ref{cor:degenerate=pi} $\measuredangle BAP=0$ or~$\pi$.
Hence the statement follows.

{
\begin{wrapfigure}{o}{43mm}
\begin{lpic}[t(-2mm),b(8mm),r(0mm),l(0mm)]{pics/reflection(1)}
\lbl[rb]{2,24;$A$}
\lbl[lb]{39,24;$B$}
\lbl[lb]{24.5,43.5;$P$}
\lbl[lt]{24.5,2.5;$P'$}
\end{lpic}
\end{wrapfigure}

If $P\notin (AB)$, then~$P'\ne P$.
By construction, the line $(AB)$ is perpendicular bisector of~$[PP']$.
Therefore, according to Theorem~\ref{thm:perp-bisect}, $AP'=AP$ and $BP'\z=BP$.
In particular, 
$\triangle ABP'\cong \triangle ABP$.
Therefore, $\measuredangle BAP'=\pm \measuredangle BAP$.

Since $P'\ne P$ and $AP'=AP$,
we get $\measuredangle BAP'\ne \measuredangle BAP$.
That is, we are left with the case
$$\measuredangle BAP'=-\measuredangle BAP.$$
\qedsf
}

\begin{thm}{Corollary}\label{cor:reflection+angle}
Reflection thru the line is a motion of the plane. 
More over if $\triangle P'Q'R'$ is the reflection of $\triangle PQR$
then 
$$\measuredangle Q'P'R'\equiv -\measuredangle QPR.$$

\end{thm}


\parit{Proof.}
From the construction it follows that 
the composition of two reflections thru 
the same line, say $(AB)$,
is the identity map.
In particular reflection is a bijection.

Assume $P'$, $Q'$ and $R'$
denote the reflections of the points
$P$, $Q$ and $R$ thru~$(AB)$. 
Let us first show that
$$P'Q'=PQ
\quad
\text{and}
\quad
\measuredangle AP'Q'\equiv-\measuredangle APQ.
\eqlbl{eq:P'Q'=PQ}$$

Without loss of generality we may assume that the points $P$ and $Q$ are distinct from $A$ and~$B$.
By Proposition~\ref{prop:reflection},
\begin{align*}
\measuredangle BAP'&\equiv -\measuredangle BAP,
&
\measuredangle BAQ'&\equiv -\measuredangle BAQ,
\\
AP'&=AP,
&
AQ'&=AQ.
\end{align*}
It follows that
$\measuredangle P'AQ'\equiv -\measuredangle PAQ$.
Therefore
$\triangle P'AQ'\cong\triangle PAQ$
and \ref{eq:P'Q'=PQ} follows.

Repeating the same argument for $P$ and $R$,
we get 
$$\measuredangle AP'R'\equiv-\measuredangle APR.$$
Subtracting the second identity in  \ref{eq:P'Q'=PQ},
we get 
$$\measuredangle Q'P'R'\equiv-\measuredangle QPR.$$
\qedsf

\begin{thm}{Exercise}\label{ex:3-reflections}
Show that any motion of the plane can be presented as a 
composition of at most three reflections.
\end{thm}

Applying the exercise above 
and Corollary~\ref{cor:reflection+angle},
we can divide the motions of the plane in two types, 
\index{direct motion}\emph{direct} 
and 
\index{indirect motion}\emph{indirect motions}.
The motion $m$ is direct if 
$$\measuredangle Q'P'R'= \measuredangle QPR$$ 
for any $\triangle PQR$ and $P'=m(P)$, $Q'=m(Q)$ and $R'=m(R)$;
if instead we have 
$$\measuredangle Q'P'R'\equiv -\measuredangle QPR$$ 
for any $\triangle PQR$ 
then the motion $m$ is called indirect.

\begin{thm}{Exercise}\label{ex:2-reflections}
Let $X$ and $Y$ be the reflections of $P$ 
thru the lines $(AB)$ and $(BC)$ correspondingly.
Show that 
$$\measuredangle XBY\equiv 2\cdot \measuredangle ABC.$$

\end{thm}

\section*{Perpendicular is shortest}
\addtocontents{toc}{Perpendicular is shortest.}

\begin{thm}{Lemma}\label{lem:perp<oblique}
Assume $Q$ is the foot point of $P$ on line~$\ell$.
Then 
$$PX>PQ$$
for any point $X$ on $\ell$ distinct from~$Q$. 
\end{thm}

If $P$, $Q$ and $\ell$ as above, 
then $PQ$ is called \index{distance!from a point to a line}\emph{distance from $P$ to $\ell$}. 

\begin{wrapfigure}[14]{o}{20mm}
\begin{lpic}[t(-2mm),b(0mm),r(0mm),l(0mm)]{pics/oblique(1)}
\lbl[tl]{17,21;$X$}
\lbl[t]{1,21;$\ell$}
\lbl[lb]{7.5,43;$P$}
\lbl[tl]{7,21;$Q$}
\lbl[lt]{7.5,2;$P'$}
\end{lpic}
\end{wrapfigure}

\parit{Proof.}
If $P\in \ell$, 
then the result follows since  $PQ=0$.
Further we assume that $P\notin \ell$.

Let $P'$ be the reflection of $P$ in~$\ell$.
Note that $Q$ is the midpoint of $[PP']$
and $\ell$ is perpendicular bisector of $[PP']$.
Therefore
$$PX=P'X
\quad
\text{and}
\quad
PQ=P'Q=\tfrac12\cdot PP'$$

Note that $\ell$ meets $[PP']$ at the point $Q$ only.
Therefore, by the triangle inequality and  Exercise~\ref{ex:degenerate-trig},
$$PX+P'X>PP'.$$
Hence the result follows.
\qeds

\begin{thm}{Exercise}\label{ex:obtuce}
Assume $\angle ABC$ is right or obtuse.
Show that $$BC>AB.$$

\end{thm}



\section*{Angle bisectors}
\addtocontents{toc}{Angle bisectors.}

If $\measuredangle A B X\equiv-\measuredangle C B X$, 
then we say that line $(BX)$ {}\emph{bisects angle} $\angle ABC$,
or line $(BX)$ is the \index{bisector!angle bisector}\emph{bisector} of $\angle ABC$.
If $\measuredangle A B X\equiv\pi-\measuredangle C B X$, then the line $(BX)$ is called \index{bisector!external bisector}\emph{external bisector} of $\angle ABC$.

Note that bisector and external bisector are uniquely defined by the angle.

\begin{wrapfigure}[7]{o}{39mm}
\begin{lpic}[t(-4mm),b(0mm),r(0mm),l(1mm)]{pics/bisectors(1)}
\lbl[tr]{30,5;$A$}
\lbl[rb]{9,10;$B$}
\lbl[br]{31,29;$C$}
\lbl[b]{30,16.7,17.5;bisector}
\lbl[b]{6.3,26.5,-72.5;external} 
\lbl[t]{4.7,26,-72.5;bisector}
\end{lpic}
\end{wrapfigure}



Note that if $\measuredangle ABA'=\pi$;
that is, if $B$ lies between $A$ and $A'$,
then bisector of $\angle ABC$ is the external bisector of $\angle A' B C$ and the other way around.




\begin{thm}{Exercise}\label{ex:perp-bisectors}
Show that for any angle, its bisector and external bisector are perpendicular.
\end{thm}


\begin{thm}{Lemma}\label{lem:angle-bisect-dist}
Assume $\measuredangle ABC\ne\pi$ nor~$0$.
Given angle $\angle ABC$ and a point $X$, 
consider foot points $Y$ and $Z$ of $X$ on $(AB)$ and~$(BC)$.

Then $X Y=X Z$ if and only if $X$ lies on the bisector or external bisector of $\angle ABC$.
\end{thm}


\parit{Proof.}
Let $Y'$ and $Z'$ be the reflections of $X$ thru $(AB)$ and $(BC)$ correspondingly.
By Proposition~\ref{prop:reflection},
$XB=Y'B=Z'B$.

\begin{wrapfigure}[8]{o}{26mm}
\begin{lpic}[t(2mm),b(0mm),r(0mm),l(1mm)]{pics/angle-bisect-lemma(1)}
\lbl[bl]{23,11;$A$}
\lbl[br]{1,18;$B$}
\lbl[rb]{21,36;$C$}
\lbl[t]{13.5,26;$Z$}
\lbl[l]{21,22;$X$}
\lbl[br]{16,13;$Y$}
\lbl[t]{14,0;$Y'$}
\lbl[rb]{8,35;$Z'$}
\end{lpic}
\end{wrapfigure}

Note that
$$
XY'=2\cdot XY
\quad
\text{and}
\quad 
XZ'=2\cdot XZ.
$$
Applying SSS and then SAS congruence conditions, we get
$$\begin{aligned}
XY&=XZ
\\
&\Updownarrow
\\
XY'&=XZ'
\\
&\Updownarrow
\\
\triangle BXY'&\cong\triangle BXZ'
\\
&\Updownarrow
\\
\measuredangle XBY'&= \pm \measuredangle BXZ'.
\end{aligned}
\eqlbl{eq:iff-chain}$$
According to Proposition~\ref{prop:reflection}, 
\begin{align*}
\measuredangle XBA&\equiv -\measuredangle Y'BA,
&
\measuredangle XBC&\equiv -\measuredangle Z'BC.
\end{align*}
Therefore, 
$$2\cdot \measuredangle XBA
\equiv 
\measuredangle XBY'
\quad 
\text{and}
\quad
2\cdot \measuredangle XBC
\equiv 
-XBZ'.$$
That is, we can continue the chain of equivalence conditions \ref{eq:iff-chain} the following way
$$\measuredangle XBY'
\equiv 
\pm \measuredangle BXZ'
\quad 
\iff
\quad 
2\cdot \measuredangle XBA 
\equiv
\pm 2\cdot \measuredangle XBC.$$


Since $(AB)\ne(BC)$, we have
$$2\cdot \measuredangle XBA
\not\equiv 
2\cdot \measuredangle XBC$$
(compare to Exercise~\ref{ex:ABCO-line}).
Therefore
$$XY=XZ
\quad
\iff
\quad  
2\cdot \measuredangle XBA 
\equiv
- 2\cdot \measuredangle XBC.$$

The last identity means either
\begin{align*}
\measuredangle XBA+\measuredangle XBC&\equiv 0
\intertext{or}
\measuredangle XBA+\measuredangle XBC&\equiv \pi.
\end{align*}
Hence the result follows.
\qeds


\section*{Circles}
\addtocontents{toc}{Circles.}

Given a positive real number $r$ and a point $O$,
the set $\Gamma$ of all points on distance $r$ from $O$ is called \index{circle}\emph{circle} 
with \index{radius}\emph{radius} $r$ and  \index{center}\emph{center}~$O$.

We say that a point $P$ lies \index{inside!inside a circle}\emph{inside} $\Gamma$ if $OP<r$ 
and if $OP>r$, we say that $P$ lies \index{outside a circle}\emph{outside}~$\Gamma$.
\label{def:circle}

\begin{thm}{Exercise}\label{ex:inside-outside}
Let $\Gamma$ be a circle and $P\notin \Gamma$.
Assume a line $\ell$ is passing thru the point $P$
intersects $\Gamma$ at two distinct points $X$ and~$Y$.
Show that $P$ is inside $\Gamma$ if and only if $P$ lies between $X$ and~$Y$.
\end{thm}


A segment between two points on $\Gamma$ is called \index{chord}\emph{chord} of~$\Gamma$.
A chord passing thru the center is called \index{diameter}\emph{diameter}.
%A segment between a point on the circle and its center is called \index{radius}\emph{radius}.

\begin{thm}{Exercise}\label{ex:chord-perp}
Assume two distinct circles $\Gamma$ and $\Gamma'$ have a common chord~$[A B]$.
Show that the line between centers of $\Gamma$ and $\Gamma'$ forms a perpendicular bisector to~$[A B]$.
\end{thm}



\begin{thm}{Lemma}\label{lem:line-circle}
A line and a circle can have at most two points of intersection.
\end{thm}

\begin{center}
\begin{lpic}[t(-0mm),b(0mm),r(0mm),l(0mm)]{pics/line-circ(1)}
\lbl[t]{2,3.5;$A$}
\lbl[t]{23,3.5;$B$}
\lbl[t]{63,3.5;$C$}
\lbl[b]{29,6.5;$\ell$}
\lbl[b]{12,17,90;$m$}
\lbl[b]{42,17,90;$n$}
\end{lpic}
\end{center}

\parit{Proof.} Assume $A$, $B$ and $C$ are distinct points which lie on a line $\ell$ and a circle $\Gamma$ with center~$O$.

Then $OA=OB=OC$; in particular $O$ lies on the perpendicular bisectors 
$m$ and $n$ to $[A B]$ and $[B C]$ correspondingly.

Note that the midpoints of $[AB]$ and $[BC]$ are distinct.
Therefore, $m$ and $n$ are distinct.
The latter contradicts the uniqueness of perpendicular (Theorem~\ref{perp:ex+un}).
\qeds

\begin{thm}{Exercise}\label{ex:two-circ}
Show that two distinct circles can have at most two points of intersection.
\end{thm}

In consequence of the above lemma, 
a line $\ell$ and a circle $\Gamma$ might have 2, 1 or 0 points of intersections.
In the first case the line is called \index{secant line}\emph{secant line}, in the second case it is \index{tangent!line}\emph{tangent line};
if $P$ is the only point of intersection of $\ell$ and $\Gamma$,
we say that {}\emph{$\ell$ is tangent to $\Gamma$ at $P$}. 

Similarly, according Exercise \ref{ex:two-circ},
two circles might have 2, 1 or 0 points of intersections.
If $P$ is the only point of intersection of circles $\Gamma$ and $\Gamma'$,
we say that \index{tangent!circles}\emph{$\Gamma$ is tangent to $\Gamma$ at $P$}. 

\begin{thm}{Lemma}\label{lem:tangent}
Let $\ell$ be a line and $\Gamma$ be a circle with center~$O$.
Assume $P$ is a common point of $\ell$ and~$\Gamma$. 
Then $\ell$ is tangent to $\Gamma$ at $P$ if and only if $(PO)\perp \ell$.
\end{thm}

\parit{Proof.}
Let $Q$ be the foot point of $O$ on~$\ell$.

Assume~$P\ne Q$.
Denote by $P'$ the reflection of $P$ thru~$(OQ)$.

Note that $P'\in\ell$ and $(OQ)$ is perpendicular bisector of~$[PP']$.
Therefore, $OP=OP'$.
Hence $P,P'\in \Gamma\cap \ell$;
that is, $\ell$ is  secant to~$\Gamma$.

If $P=Q$, 
then according to Lemma~\ref{lem:perp<oblique},
$OP<OX$ for any point $X\in \ell$ distinct from~$P$.
Hence $P$ is the only point in the intersection $\Gamma\cap\ell$;
that is, $\ell$ is tangent to $\Gamma$ at~$P$. 
\qeds

\begin{thm}{Exercise}\label{ex:tangent-circles}
Let $\Gamma$ and $\Gamma'$ be two distinct circles with centers at $O$ and $O'$ correspondingly. 
Assume $\Gamma$ and $\Gamma'$ intersect at point~$P$.
Show that $\Gamma$ is tangent to $\Gamma'$ if and only if $O$, $O'$ and $P$ lie on one line.
\end{thm}

\begin{thm}{Exercise}\label{ex:tangent-circles-2}
Let $\Gamma$ and $\Gamma'$ be two distinct circles with centers at $O$ and $O'$ and radii $r$ and~$r'$.

\begin{enumerate}[(a)]
\item\label{ex:tangent-circles-2:a} Show that $\Gamma$ is tangent to $\Gamma'$ if and only if
$$OO'=r+r'
\quad
\text{or}\quad
OO'=|r-r'|.$$
\item \label{ex:tangent-circles-2:b}
Show that $\Gamma$ intersects $\Gamma'$ if and only if
$$|r-r'|\le OO'\le r+r'.$$
\end{enumerate}

\end{thm}

\begin{thm}{Exercise}\label{ex:tangent-circles-3}
 Assume three circles  intersect at two points.
 Show that the centers of these circles lie on one line.
\end{thm}

\section*{Geometric constructions}
\addtocontents{toc}{Geometric constructions.}

The \index{ruler-and-compass construction}\emph{ruler-and-compass constructions} in the plane is the construction of points, lines, and circles using only an idealized ruler and compass.
These construction problems provide a valuable source of exercises in geometry 
which we will use further in the book.
In addition, Chapter~\ref{chap:car} is devoted completely to the subject.

The idealized ruler can be used only to draw a line thru the given two points.
The idealized compass can be used only to draw a circle with given center and radius.
That is, given three points $A$, $B$ and $O$ 
we can draw the set of all points on distant $AB$ from~$O$.
We may also mark new points in the plane
as well as on the constructed lines, circles 
and their intersections (assuming that such points exist).

We can also look at the different set of construction tools.
For example,
we may only use the ruler or
we may invent a new tool, 
say a tool which produce midpoint for given two points.

As an example, let us consider the following problem:

\begin{thm}{Construction of midpoint}
Construct the midpoint of the given segment~$[AB]$.
\end{thm}

\begin{wrapfigure}{o}{31mm}
\begin{lpic}[t(0mm),b(0mm),r(0mm),l(1mm)]{pics/constr-perp-bisect(1)}
\lbl[b]{4.3,28.5;$A$}
\lbl[t]{26,12;$B$}
\lbl[t]{26,35;$P$}
\lbl[b]{4.5,5;$Q$}
\lbl[b]{14.5,22.5;$M$}
\end{lpic}
\end{wrapfigure}

\parit{Construction.}
\begin{enumerate}[1.]
\item Construct the circle 
with center at $A$ 
which is passing thru~$B$.
\item Construct the circle 
with center at $B$ 
which is passing thru~$A$.
\item Mark both points of intersection of these circles, label them by $P$ and~$Q$.
\item Draw the line~$(PQ)$.
\item Mark the point of intersection of $(PQ)$ and $[AB]$; this is the midpoint.
\end{enumerate}

\medskip

Typically, you need to proof that the construction produce what was expected. Here is a proof for the example above.

\parit{Proof.}
According to Theorem~\ref{thm:perp-bisect}, $(PQ)$ is the perpendicular bisector to~$[AB]$.
Therefore, $M=(AB)\cap(PQ)$ is the midpoint of~$[AB]$. 
\qeds

\begin{thm}{Exercise}\label{ex:construction-perpendicular}
Make a ruler-and-compass construction of the line thru the given point which is perpendicular to the given line.
\end{thm}

\begin{thm}{Exercise}\label{ex:center}
Make a ruler-and-compass construction of the center 
of the given circle.
\end{thm}

\begin{thm}{Exercise}\label{ex:tangent}
Make a ruler-and-compass construction of the lines tangent to the given circle which pass thru the given point.
\end{thm}

\begin{thm}{Exercise}\label{ex:tangent-circle}
Given two circles $\Gamma_1$ and $\Gamma_2$ and a segment $[AB]$
make a ruler-and-compass construction of a circle of radius $AB$ 
which is tangent to each circle $\Gamma_1$ and~$\Gamma_2$.
\end{thm}




\addtocontents{toc}{\protect\end{quote}}
%\part*{Inversive geometry}
\addtocontents{toc}{\protect\begin{center}}
\addtocontents{toc}{\large{\bf Inversive geometry}}
\addtocontents{toc}{\protect\end{center}}
\chapter{Inscribed angles}\label{chap:inscribed-angle}
\addtocontents{toc}{\protect\begin{quote}}

\section*{Angle between a tangent line and a chord}
\addtocontents{toc}{Angle between a tangent line and a chord.}

\begin{thm}{Theorem}\label{thm:tangent-angle}
Let $\Gamma$ be a circle with center $O$ in the Euclidean plane.
Assume line $(XQ)$ is tangent to $\Gamma$ at $X$
and $[XY]$ is a chord of $\Gamma$.
Then 
$$2\cdot\measuredangle QXY
\equiv\measuredangle X O Y.
\eqlbl{eq:tangent-angle}$$
Equivalently, 
$$\measuredangle QXY\equiv\tfrac12\cdot\measuredangle X O Y
\ \ \text{or}\ \ 
\measuredangle QXY\equiv\tfrac12\cdot\measuredangle X O Y+\pi.$$

\end{thm}

\begin{wrapfigure}{o}{47mm}
\begin{lpic}[t(-9mm),b(0mm),r(0mm),l(0mm)]{pics/tangent-chord(01)}
\lbl[l]{43,5;$Q$}
\lbl[l]{43,21;$X$}
\lbl[rt]{12,2;$Y$}
\lbl[rb]{21,22;$O$}
\end{lpic}
\end{wrapfigure}

\parit{Proof.}
Note that $\triangle XOY$ is isosceles.
Therefore $\measuredangle YXO=\measuredangle OYX$.

Let us applying Theorem~\ref{thm:3sum}
to $\triangle XOY$. 
We get
\begin{align*}
\pi&\equiv\measuredangle YXO+\measuredangle OYX+\measuredangle XOY\equiv
\\
&\equiv 2\cdot \measuredangle YXO+\measuredangle XOY.
\end{align*}

By Lemma~\ref{lem:tangent}, $(OX)\z\perp(XQ)$.
Therefore 
$$\measuredangle QXY+\measuredangle YXO \equiv\pm\tfrac\pi2.$$

Therefore 
$$2\cdot\measuredangle QXY
\equiv \pi -2\cdot \measuredangle YXO
\equiv\measuredangle X O Y.
$$
\qedsf

\pagebreak
\section*{Inscribed angle}\label{sec:inscribed}
\addtocontents{toc}{Inscribed angle.}
\begin{wrapfigure}[11]{o}{45mm}
\begin{lpic}[t(-8mm),b(0mm),r(0mm),l(0mm)]{pics/inscribed-angle-5(1)}
\lbl[rb]{12,42;$P$}
\lbl[l]{43,23;$X$}
\lbl[rt]{11,5;$Y$}
\lbl[rb]{21,24;$O$}
\end{lpic}
\end{wrapfigure}

We say that triangle is \index{inscribed triangle}\emph{inscribed} in the circle $\Gamma$ if all its vertices lie on $\Gamma$.

\begin{thm}{Theorem}\label{thm:inscribed-angle}
Let $\Gamma$ be a circle with center $O$
in the Euclidean plane,
and $X,Y$ be two distinct points on $\Gamma$.
Then
$\triangle X P Y$ is inscribed in $\Gamma$ if and only if
$$2\cdot\measuredangle X P Y\equiv\measuredangle X O Y.
\eqlbl{eq:inscribed-angle}$$
Equivalently, if and only if
$$\measuredangle XPY\equiv\tfrac12\cdot\measuredangle X O Y\ \ \text{or}
\ \ \measuredangle XPY\equiv\tfrac12\cdot\measuredangle X O Y+\pi.$$

\end{thm}


\parit{Proof.}
Choose a point $Q$ such that $(PQ)\perp(OP)$.
By Lemma~\ref{lem:tangent}, $(PQ)$ is tangent to $\Gamma$.

According to Theorem~\ref{thm:tangent-angle},
\begin{align*}
2\cdot\measuredangle QPX&\equiv\measuredangle POX,
\\
2\cdot\measuredangle QPY&\equiv\measuredangle POY.
\end{align*}
Subtracting one identity from the other we get \ref{eq:inscribed-angle}.

\begin{wrapfigure}[11]{o}{40mm}
\begin{lpic}[t(-4mm),b(0mm),r(0mm),l(0mm)]{pics/inscribed-angle-new(1)}
\lbl[l]{39,20;$X$}
\lbl[rt]{15,1;$Y$}
\lbl[bl]{17,25;$O$}
\lbl[r]{21,16;$O'$}
\lbl[r]{8.5,27;$P$}
\lbl[w]{22,33;$\Gamma'$}
\lbl[w]{34,36;$\Gamma$}
\end{lpic}
\end{wrapfigure}

Let us prove the converse.
Assume that \ref{eq:inscribed-angle} holds for some $P\notin \Gamma$.
Note that $\measuredangle X O Y\ne 0$ and therefore $\measuredangle X P Y\ne 0$ nor $\pi$;
that is, $\triangle PXY$ is nondegenerate.

Let $\Gamma'$ be the circumcircle of $\triangle PXY$
and $O'$ be its circumcenter;
they exist by Exercise~\ref{ex:unique-cline}.


Note that $O'\ne O$.
From above, we have 
$$\measuredangle X O Y\equiv 2\cdot\measuredangle X P Y\equiv \measuredangle X O' Y.$$

Note that $OX=OY$ and $O'X=O'Y$.
By Theorem~\ref{thm:perp-bisect},
$(OO')$ is the perpendicular bisector to $[XY]$;
equivalently $X$ is the reflection of $Y$ in $(OO')$.
Applying Proposition~\ref{prop:reflection}, we get
\begin{align*}
\measuredangle X O O'&\equiv -\measuredangle Y O O',
&
\measuredangle X O' O&\equiv -\measuredangle Y O' O.
\end{align*}

Therefore 
\begin{align*}
2\cdot \measuredangle X O O'
&\equiv\measuredangle X O O'+\measuredangle O' O Y
\equiv
\\
&\equiv \measuredangle X O Y\equiv \measuredangle X O' Y\equiv
\\
&\equiv\measuredangle X O' O+\measuredangle O O' Y
\\
&\equiv 2\cdot \measuredangle X O' O.
\end{align*}
By Transversal property \ref{thm:parallel-2},
$(X O)\parallel (XO')$, a contradiction.
\qeds

{
\begin{wrapfigure}{o}{45mm}
\begin{lpic}[t(-0mm),b(0mm),r(0mm),l(2mm)]{pics/inscribed-angle-3(1)}
\lbl[r]{1,22;$Y'$}
\lbl[tl]{39,9;$Y$}
\lbl[rt]{13.5,16.5;$P$}
\lbl[t]{21,0;$X$}
\lbl[r]{7,36;$X'$}
\lbl[lb]{22.5,22.5;$O$}
\end{lpic}
\end{wrapfigure}

\begin{thm}{Exercise}\label{ex:inscribed-angle}
Let $[XX']$ and $[YY']$ be two chords of circle $\Gamma$ with center $O$ and radius $r$ in the Euclidean plane.
Assume $(XX')$ and $(YY')$ intersect at point $P$.
Show that 
\begin{enumerate}[(a)]
\item $2\cdot \measuredangle XPY=\measuredangle XOY+\measuredangle X'OY'$;
\item\label{ex:inscribed-angle:b} $\triangle PXY\sim \triangle PY'X'$;
\item $PX\cdot PX'=|OP^2-r^2|$.
\end{enumerate}

\end{thm}



\begin{thm}{Exercise}\label{ex:inscribed-hex}
Assume that the chords $[XX']$, $[YY']$ and $[ZZ']$
of the circle $\Gamma$ in the Euclidean plane intersect at one point.
Show that 
$$XY'\cdot ZX'\cdot YZ'=X'Y\cdot Z'X\cdot Y'Z.$$

\end{thm}
}
{
\begin{wrapfigure}{o}{32mm}
\begin{lpic}[t(-2mm),b(0mm),r(0mm),l(0mm)]{pics/altitudes-circumcircle(1)}
\lbl[b]{14,30.5;$C$}
\lbl[br]{1.5,9;$A$}
\lbl[b]{27,26.5;$A'$}
\lbl[t]{28,5;$B$}
\lbl[b]{3,23;$B'$}
\end{lpic}
\end{wrapfigure}

\begin{thm}{Exercise}\label{ex:altitudes-circumcircle}
Let $\Gamma$ be a circumcircle of $\triangle A B C$.
$A'\not=A$ and $B'\not=B$ be the points of intersection of altitudes from $A$ and $B$ with $\Gamma$.
Show that $\triangle A' B' C$ is isosceles.
\end{thm}

Recall that diameter of circle is its chord which pass thru the center.
Note that if $[XY]$ is diameter of circle with center $O$ then $\measuredangle X O Y=\pi$. 
Whence Theorem~\ref{thm:inscribed-angle} implies the following.

}

\begin{thm}{Corollary}\label{cor:right-angle-diameter}
Let $\Gamma$ be a circle with diameter $[XY]$.
Assume that point $P$ is distinct from $X$ and $Y$.
Then $P\in \Gamma$ if and only if $\angle XPY$ is right.
\end{thm}

\begin{thm}{Exercise}\label{ex:two-right}
Given four points $A$, $B$, $A'$ and $B'$
construct a point $Z$ such that both angles $\angle AZB$ and $\angle A'ZB'$ are right.
\end{thm}

{
\begin{wrapfigure}{o}{26mm}
\begin{lpic}[t(-7mm),b(0mm),r(0mm),l(0mm)]{pics/equileteral(1)}
\lbl[tr]{19,0;$L$}
\lbl[rb]{16,18;$M$}
\lbl[r]{3,7.5;$N$}
\lbl[t]{4,1.5;$O$}
\lbl[l]{23.5,13.5;$X$}
\end{lpic}
\end{wrapfigure}

\begin{thm}{Exercise}\label{ex:equilateral-2}
Assume three lines $\ell, m$ and $n$ intersect at point $O$ and form six equal angles at $O$. 
Let $X$ be a point distinct from $O$,
denote by $L$, $M$ and $N$ be the footpoints of $X$ on $\ell, m$ and $n$ correspondingly.
Show that $\triangle LMN$ is equilateral.
\end{thm}
}


\begin{thm}{Exercise}\label{ex:VVAA}
Let $\triangle A B C$ be a nondegenerate triangle in the Euclidean plane,
$A'$ and $B'$ be foot points of altitudes from $A$ and $B$.
Show that $A$, $B$, $A'$ and $B'$ lie on one circle.

What is the center of this circle?
\end{thm}

\begin{thm}{Exercise}\label{ex:perpendicular-ruler}
Assume a line $\ell$ and a circle with center on $\ell$ are given.
Make a ruler-only construction of the perpendicular to $\ell$
from the given point.
\end{thm}


\section*{Inscribed quadrilaterals}
\addtocontents{toc}{Inscribed quadrilaterals.}

A quadrilateral $\square ABCD$ is called 
\index{quadrilateral!inscribed quadrilateral}\emph{inscribed}
if all the points $A$, $B$, $C$ and $D$ lie on a circle or a line.

\begin{thm}{Theorem}\label{thm:inscribed-quadrilateral}
A quadrilateral $\square ABCD$ in the Euclidean plane is inscribed 
if and only if
$$2\cdot\measuredangle ABC+2\cdot\measuredangle CDA\equiv 0.
\eqlbl{eq:inscribed-4angle}$$
Equivalently, if and only if
$$\measuredangle ABC+\measuredangle CDA\equiv \pi
\ \ \text{or} \ \ 
\measuredangle ABC\equiv-\measuredangle CDA.$$

\end{thm}

\begin{wrapfigure}[13]{o}{47mm}
\begin{lpic}[t(-3mm),b(6mm),r(0mm),l(0mm)]{pics/inscribed-angle-4(1)}
\lbl[r]{1,21;$A$}
\lbl[rb]{5,35;$B$}
\lbl[lb]{40,32;$C$}
\lbl[tl]{30,1;$D$}
\end{lpic}
\end{wrapfigure}

\parit{Proof of Theorem~\ref{thm:inscribed-quadrilateral}.} 
Assume $\triangle ABC$ is degenerate.
By Corollary~\ref{cor:degenerate=pi},
$$2\cdot \measuredangle ABC\equiv 0;$$
From the same corollary, we get 
$$2\cdot \measuredangle CDA\equiv 0$$ 
if and only if $D\in (AB)$;
hence the result follows.

It remains to consider the case if $\triangle ABC$ is nondegenerate.

Denote by $\Gamma$ the circumcircle of  $\triangle ABC$ and let $O$ be the center of $\Gamma$.
According to Theorem~\ref{thm:inscribed-angle},
$$
2\cdot\measuredangle ABC
\equiv
\measuredangle AOC.
\eqlbl{eq:2<ABC=<AOB}
$$
From the same theorem, $D\in\Gamma$ if and only if 

$$
2\cdot\measuredangle CDA
\equiv\measuredangle COA.\eqlbl{eq:2<CDE=<BOA}
$$
Adding \ref{eq:2<ABC=<AOB} and \ref{eq:2<CDE=<BOA},
we get the result.
\qeds

{
\begin{wrapfigure}{o}{43mm}
\begin{lpic}[t(-0mm),b(-0mm),r(0mm),l(-1mm)]{pics/AB2XY(.9)}
\lbl[t]{20,4;$A$}
\lbl[rb]{11.5,23;$B$}
\lbl[l]{4.5,10;$Y$}
\lbl[b]{13.3,3.5;$X$}
\lbl[r]{43,20;$X'$}
\lbl[t]{24.5,34;$Y'$}
\lbl[blw]{21,17.5;$\Gamma$}
\lbl[blw]{41,33;$\Gamma'$}
\end{lpic}
\end{wrapfigure}

\begin{thm}{Exercise}\label{ex:secant-circles}
Let $\Gamma$ and $\Gamma'$
be two circles 
which intersect at two distinct points $A$ and $B$.
Assume $[XY]$ and $[X'Y']$ be the chords of $\Gamma$ and $\Gamma'$ correspondingly such that $A$ lies between $X$ and $X'$ and $B$ lies between $Y$ and $Y'$.
Show that $(XY)\parallel (X'Y')$.
\end{thm}

\begin{thm}{Exercise}\label{ex:two-chords}
Let $[XY]$ and $[X'Y']$
 be two parallel chords of a circle.
Show that $XX'=YY'$.
\end{thm}

}

\section*{Arcs}
\addtocontents{toc}{Arcs.}

A subset of a circle bounded by two points is called a circle arc.

More precisely,
let $\Gamma$ be a circle and $A$, $B$, $C$ be distinct points on $\Gamma$.
The subset  which includes the points $A$, $C$
as well as all the points on $\Gamma$ which lie with $B$ on the same side from $(AC)$ is called \index{circle arc}\emph{circle arc} $ABC$.

\begin{wrapfigure}{o}{40mm}
\begin{lpic}[t(-0mm),b(0mm),r(0mm),l(0mm)]{pics/tangent-half-line(1)}
\lbl[lb]{16,25;$A$}
\lbl[l]{24,13;$B$}
\lbl[tl]{18,3;$C$}
\lbl[t]{37,14;$X$}
\lbl[lb]{5,6;$\Gamma$}
\end{lpic}
\end{wrapfigure}


For the circle arc $ABC$, 
the points $A$ and $C$ are called 
\index{endpoint of arc}\emph{endpoints}. 
Note that there are two circle arcs of $\Gamma$ with the given endpoints.

A half-line $[AX)$ is called 
\index{tangent!half-line}\emph{tangent} 
to arc $ABC$ at $A$
if the line $(AX)$ is tangent to $\Gamma$ and the points $X$ and $B$ lie on the same side from the line $(AC)$.

If $B$ lies on the line $(AC)$, 
the arc $ABC$ degenerates to one of two following a subsets of line $(AC)$.
\begin{itemize}
\item If $B$ lies between $A$ and $C$ then we define the arc $ABC$ as the segment $[AC]$. 
In this case the half-line $[AC)$ is tangent to the arc $ABC$ at $A$.
\item If $B\in(AC)\backslash [AC]$ then we define the arc $ABC$ as the line $(AC)$ without all the points between $A$ and $C$.
If we choose points $X$ and $Y\in (AC)$ such that the points $X$, $A$, $C$ and $Y$ appear in the same order on the line then the arc $ABC$
is formed by two half-lines in $[AX)$ and $[CY)$.
The half-line $[AX)$ is tangent to the arc $ABC$ at $A$.
\item In addition, any half-line $[AB)$ will be regarded as an arc.
This degenerate arc has only one end point $A$
and it assumed to be tangent to itself at $A$.
\end{itemize}

The circle arcs together with the degenerate arcs will be called \index{arc}\emph{arcs}.


\begin{thm}{Proposition}\label{prop:arc(angle=angle)}
In the Euclidean plane,
a point $D$ lies on the arc $ABC$ if and only if 
$$\measuredangle ADC= \measuredangle ABC$$
or $D$ coincides with $A$ or $C$.
\end{thm}


\begin{wrapfigure}{o}{23mm}
\begin{lpic}[t(-3mm),b(0mm),r(0mm),l(0mm)]{pics/tangent-half-line-3(1)}
\lbl[rb]{8,23;$A$}
\lbl[l]{21,17;$B$}
\lbl[tr]{1,3;$C$}
\lbl[tl]{18,3;$D$}
\end{lpic}
\end{wrapfigure}

\parit{Proof.}
Note that if $A$, $B$ and $C$ lie on one line then 
the statement is evident.

Assume $\Gamma$ be the circle passing thru $A$, $B$ and $C$.

Assume $D$ is distinct from $A$ and $C$.
According to Theorem~\ref{thm:inscribed-quadrilateral},
$D\in\Gamma$ if and only if 
$$\measuredangle ADC= \measuredangle ABC\ \ \text{or}\ \ \measuredangle ADC\equiv \measuredangle ABC+\pi.$$


By Exercise~\ref{ex:signs-PXQ-PYQ},
the first identity holds then $B$ and $D$ lie on one side of $(AC)$;
that is, $D$ belongs to the arc $ABC$.
If the second identity holds then the points $B$ and $D$ lie on the opposite sides from $(AC)$,
in this case $D$ does not belong to the arc $ABC$.
\qeds

\begin{thm}{Proposition}\label{prop:arc(angle=tan)}
In the Euclidean plane, a half-lines $[AX)$ is tangent to the arc $ABC$ if and only if 
$$\measuredangle ABC+\measuredangle CAX\equiv \pi.$$

\end{thm}

\parit{Proof.}
For a degenerate arc $ABC$ 
the statement is evident.
Further we assume the arc $ABC$ is nondegenerate.

Applying theorems \ref{thm:tangent-angle}
and \ref{thm:inscribed-angle},
we get 
$$2\cdot \measuredangle ABC+2\cdot\measuredangle CAX\equiv 0.$$
Therefore either 
$$\measuredangle ABC+\measuredangle CAX\equiv \pi\ \ \ \text{or}\ \ \ \measuredangle ABC+\measuredangle CAX\equiv 0.$$

\begin{wrapfigure}[9]{o}{26mm}
\begin{lpic}[t(-0mm),b(3mm),r(0mm),l(0mm)]{pics/tangent-half-line-2(1)}
\lbl[r]{7,24;$A$}
\lbl[l]{18,13.5;$B$}
\lbl[br]{1,7;$C$}
\lbl[b]{23.5,22;$X$}
\end{lpic}
\end{wrapfigure}

Since $[AX)$ is the tangent half-line to the arc $ABC$,
$X$ and $B$ lie on the same side from $(AC)$.
Therefore the angles $\angle CAX$, $\angle CAB$ and $\angle ABC$ 
have the same sign.
In particular 
$\measuredangle ABC+\measuredangle CAX\not\equiv 0$;
that is, we are left with the case 
$$\measuredangle ABC+\measuredangle CAX\equiv \pi.$$
\qedsf

\begin{thm}{Exercise}\label{ex:arc-tan-straight}
Assume that in the Euclidean plane,
the half-lines $[AX)$ and $[AY)$
are tangent to the arcs $ABC$ and $ACB$ correspondingly.
Show that $\angle XAY$ is straight.
\end{thm}


\begin{thm}{Exercise}\label{ex:tangent-arc}
Show that in the Euclidean plane, there is unique arc 
with endpoints at the given points $A$ and $C$ 
which is tangent at $A$ to the given half line $[AX)$.
\end{thm}

\begin{wrapfigure}{r}{50mm}
\begin{lpic}[t(-11mm),b(-3mm),r(0mm),l(0mm)]{pics/two-arcs(1)}
\lbl[l]{21,35;$A$}
\lbl[bl]{42,42;$B_1$}
\lbl[tr]{13,17;$B_2$}
\lbl[bl]{32,20;$C$}
\lbl[t]{50,16;$Y_1$}
\lbl[l]{26,2;$Y_2$}
\lbl[tl]{25,51;$X_1$}
\lbl[b]{3,32;$X_2$}
\end{lpic}
\end{wrapfigure}

\begin{thm}{Exercise}\label{ex:two-arcs}
Given two arcs $AB_1C$ and $AB_2C$ in the Euclidean plane,
let $[AX_1)$ and $[AX_2)$ be the half-lines tangent to arcs $AB_1C$ and $AB_2C$ at $A$
and 
$[CY_1)$ and $[CY_2)$ be the half-lines tangent to arcs $AB_1C$ and $AB_2C$ at $C$.
Show that
$$\measuredangle X_1AX_2\equiv -\measuredangle Y_1CY_2 .$$

\end{thm}



\begin{thm}{Exercise}\label{ex:3x120}
Given an acute triangle $\triangle ABC$
make a compass-and-ruler construction of the point $Z$ such that
\[\measuredangle AZB
= \measuredangle BZC
= \measuredangle CZA
=\pm\tfrac23\cdot\pi\]

\end{thm}


\addtocontents{toc}{\protect\end{quote}}
%\part*{Inversive geometry}
\addtocontents{toc}{\protect\begin{center}}
\addtocontents{toc}{\large{\bf Inversive geometry}}
\addtocontents{toc}{\protect\end{center}}
\chapter{Inscribed angles}\label{chap:inscribed-angle}
\addtocontents{toc}{\protect\begin{quote}}

\section*{Angle between a tangent line and a chord}
\addtocontents{toc}{Angle between a tangent line and a chord.}

\begin{thm}{Theorem}\label{thm:tangent-angle}
Let $\Gamma$ be a circle with the center $O$.
Assume the line $(XQ)$ is tangent to $\Gamma$ at $X$
and $[XY]$ is a chord of~$\Gamma$.
Then 
$$2\cdot\measuredangle QXY
\equiv\measuredangle X O Y.
\eqlbl{eq:tangent-angle}$$
Equivalently, 
$$\measuredangle QXY
\equiv
\tfrac12\cdot\measuredangle X O Y
\quad 
\text{or}
\quad
\measuredangle QXY
\equiv
\tfrac12\cdot\measuredangle X O Y+\pi.$$

\end{thm}

\begin{wrapfigure}{o}{47mm}
\centering
\begin{lpic}[t(-9mm),b(0mm),r(0mm),l(0mm)]{pics/tangent-chord(01)}
\lbl[l]{43,5;$Q$}
\lbl[l]{43,21;$X$}
\lbl[rt]{12,2;$Y$}
\lbl[rb]{21,22;$O$}
\end{lpic}
\end{wrapfigure}

\parit{Proof.}
Note that $\triangle XOY$ is isosceles.
Therefore, $\measuredangle YXO=\measuredangle OYX$.

Let us apply Theorem~\ref{thm:3sum}
to $\triangle XOY$. 
We get
\begin{align*}
\pi&\equiv\measuredangle YXO+\measuredangle OYX+\measuredangle XOY\equiv
\\
&\equiv 2\cdot \measuredangle YXO+\measuredangle XOY.
\end{align*}

By Lemma~\ref{lem:tangent}, $(OX)\z\perp(XQ)$.
Therefore, 
$$\measuredangle QXY+\measuredangle YXO \equiv\pm\tfrac\pi2.$$

Therefore, 
$$2\cdot\measuredangle QXY
\equiv \pi -2\cdot \measuredangle YXO
\equiv\measuredangle X O Y.
$$
\qedsf

\section*{Inscribed angle}\label{sec:inscribed}
\addtocontents{toc}{Inscribed angle.}


We say that a triangle is \index{inscribed triangle}\emph{inscribed} in the circle $\Gamma$ if all its vertices lie on~$\Gamma$.

\begin{thm}{Theorem}\label{thm:inscribed-angle}
Let $\Gamma$ be a circle with the center $O$,
and $X,Y$ be two distinct points on~$\Gamma$.
Then
$\triangle X P Y$ is inscribed in $\Gamma$ if and only if
$$2\cdot\measuredangle X P Y\equiv\measuredangle X O Y.
\eqlbl{eq:inscribed-angle}$$
Equivalently, if and only if
$$\measuredangle XPY\equiv\tfrac12\cdot\measuredangle X O Y
\quad
\text{or}
\quad
\measuredangle XPY\equiv\pi+\tfrac12\cdot\measuredangle X O Y.$$

\end{thm}

\begin{wrapfigure}{o}{22mm}
\centering
\begin{lpic}[t(-8mm),b(0mm),r(0mm),l(0mm)]{pics/inscribed-angle-5(1)}
\lbl[rb]{5,19;$P$}
\lbl[lt]{14,22;$Q$}
\lbl[l]{20,11;$X$}
\lbl[rt]{5,1;$Y$}
\lbl[rb]{9.5,11;$O$}
\end{lpic}
\end{wrapfigure}


\parit{Proof; the ``only if'' part.}
Let $(PQ)$ be the tangent line to $\Gamma$ at~$P$.
By Theorem~\ref{thm:tangent-angle},
\begin{align*}
2\cdot\measuredangle QPX&\equiv\measuredangle POX,
&
2\cdot\measuredangle QPY&\equiv\measuredangle POY.
\end{align*}
Subtracting one identity from the other, we get \ref{eq:inscribed-angle}.

%%%%%%%%%%%%%%%%%%%%%%
\parit{``If'' part.}
Assume that \ref{eq:inscribed-angle} holds for some $P\notin \Gamma$.
Note that $\measuredangle X O Y\ne 0$. 
Therefore, $\measuredangle X P Y\ne 0$ nor $\pi$;
that is, $\triangle PXY$ is nondegenerate.

The line $(PX)$ might be tangent to $\Gamma$ at the point $X$ or intersect $\Gamma$ at another point;
it the latter case, let $P'$ denotes this point of intersection. 

\begin{figure}[h!]
\centering
\begin{lpic}[t(-0mm),b(0mm),r(0mm),l(0mm)]{pics/inscribed-angle-two-cases(1)}
\lbl[tr]{7.5,14.5;$X$}
\lbl[t]{22.5,7.5;$Y$}
\lbl[bl]{19,20;$O$}
\lbl[r]{4,25;$P$}
%%%%%
\lbl[tr]{43,14;$X$}
\lbl[tl]{60,9;$Y$}
\lbl[tr]{53.5,17.5;$O$}
\lbl[rb]{57,25;$P$}
\lbl[bl]{62,28;$P'$}
\end{lpic}
\end{figure}

In the first case, by Theorem~\ref{thm:tangent-angle}, we have
\begin{align*}
2\cdot \measuredangle PXY&\equiv \measuredangle XOY\equiv 
 2\cdot\measuredangle  XPY.
\end{align*}
Applying the transversal property (\ref{thm:parallel-2}), we get that
$(XY)\parallel (PY)$, which is impossible since $\triangle PXY$ is nondegenerate.

In the second case, 
applying the ``if'' part and that  $P$, $X$, and $P'$ lie on one line (see Exercise~\ref{ex:ABCO-line}) we get that 
\begin{align*}
2\cdot \measuredangle XPP'&\equiv
2\cdot \measuredangle XPY\equiv 
 \measuredangle  XOY\equiv
 2\cdot\measuredangle  XP'Y\equiv
 2\cdot\measuredangle  XP'P.
\end{align*}
Again, by transversal property,
$(PY)\parallel (P'Y)$, which is impossible since $\triangle PXY$ is nondegenerate.
\qeds

%%%%%%%%%%%%%%%%%%%%%%

{

\begin{wrapfigure}[14]{r}{44mm}
\centering
\begin{lpic}[t(-0mm),b(0mm),r(0mm),l(2mm)]{pics/inscribed-angle-3(1)}
\lbl[r]{1,22;$Y'$}
\lbl[tl]{39,9;$Y$}
\lbl[rt]{13.5,16.5;$P$}
\lbl[t]{21,0;$X$}
\lbl[r]{6.5,36;$X'$}
\lbl[lb]{23,23;$O$}
\end{lpic}
\end{wrapfigure}

\begin{thm}{Exercise}\label{ex:inscribed-angle}
Let $X$, $X'$, $Y$, and $Y'$ be distinct points on the circle $\Gamma$.
Assume $(XX')$ meets $(YY')$ at the point~$P$.
Show that 
\begin{enumerate}[(a)]
\item $2\cdot \measuredangle XPY=\measuredangle XOY+\measuredangle X'OY'$;
\item\label{ex:inscribed-angle:b} $\triangle PXY\sim \triangle PY'X'$;
\item $PX\cdot PX'=|OP^2-r^2|$, where $O$ is the center and $r$ is the radius of $\Gamma$.
\end{enumerate}

\end{thm}



\begin{thm}{Exercise}\label{ex:inscribed-hex}
%???(
Three chords $[XX']$, $[YY']$, and $[ZZ']$
of the circle $\Gamma$ intersect at one point.
%???)
Show that 
$$XY'\cdot ZX'\cdot YZ'=X'Y\cdot Z'X\cdot Y'Z.$$

\end{thm}
}

{
\begin{wrapfigure}{r}{32mm}
\centering
\begin{lpic}[t(-10mm),b(-4mm),r(0mm),l(0mm)]{pics/altitudes-circumcircle(1)}
\lbl[b]{14,30.5;$C$}
\lbl[br]{1.5,9;$A$}
\lbl[b]{27,26.5;$A'$}
\lbl[t]{28,5;$B$}
\lbl[b]{3,23;$B'$}
\end{lpic}
\end{wrapfigure}

\begin{thm}{Exercise}\label{ex:altitudes-circumcircle}
Let $\Gamma$ be a circumcircle of an acute triangle $A B C$.
Let $A'$ and $B'$ denote the second points of intersection of the altitudes from $A$ and $B$ with~$\Gamma$.
Show that $\triangle A' B' C$ is isosceles.
\end{thm}

Recall that the diameter of a circle is its chord which passes thru the center.
Note that if $[XY]$ is the diameter of a circle with center $O$, then $\measuredangle X O Y=\pi$. 
Hence Theorem~\ref{thm:inscribed-angle} implies the following:

}

\begin{thm}{Corollary}\label{cor:right-angle-diameter}
Let $\Gamma$ be a circle with the diameter~$[XY]$.
Assume that the point $P$ is distinct from $X$ and~$Y$.
Then $P\in \Gamma$ if and only if $\angle XPY$ is right.
\end{thm}

\begin{thm}{Exercise}\label{ex:two-right}
Given four points $A$, $B$, $A'$, and $B'$,
construct a point $Z$ such that both angles $AZB$ and $A'ZB'$ are right.
\end{thm}

\begin{thm}{Exercise}\label{ex:VVAA}
Let $\triangle A B C$ be a nondegenerate triangle,
$A'$ and $B'$ be foot points of altitudes from $A$ and~$B$ respectfully.
Show that the four points $A$, $B$, $A'$, and $B'$ lie on one circle.

What is the center of this circle?
\end{thm}

\begin{thm}{Exercise}\label{ex:perpendicular-ruler}
Assume a line $\ell$, 
a circle with its center on $\ell$ 
and a point $P\notin\ell$ are given.
Make a ruler-only construction of the perpendicular to $\ell$
from $P$.
\end{thm}

%???\begin{thm}{Exercise}
%The tangent lines at the points $A$ and $B$ to a given circle intersect at the point $P$.
%A line thru $P$ intersects the circle at two points $X$ and $Y$.
%Let $M$ be the midpoint of $[XY]$.
%Show that the points $P$, $A$, $B$, and $M$ lie on one circle.
%\end{thm}


\section*{Inscribed quadrilaterals}
\addtocontents{toc}{Inscribed quadrilaterals.}

A quadrilateral $ABCD$ is called 
\index{quadrilateral!inscribed quadrilateral}\emph{inscribed}
if all the points $A$, $B$, $C$, and $D$ lie on a circle or a line.

\begin{thm}{Theorem}\label{thm:inscribed-quadrilateral}
The quadrilateral $ABCD$ is inscribed 
if and only~if
$$2\cdot\measuredangle ABC+2\cdot\measuredangle CDA\equiv 0.
\eqlbl{eq:inscribed-4angle}$$
Equivalently, if and only if
$$\measuredangle ABC+\measuredangle CDA
\equiv 
\pi
\quad
\text{or}
\quad 
\measuredangle ABC
\equiv
-\measuredangle CDA.$$

\end{thm}

\begin{wrapfigure}{o}{21mm}
\centering
\begin{lpic}[t(-5mm),b(0mm),r(0mm),l(0mm)]{pics/inscribed-angle-4(1)}
\lbl[r]{0.5,9;$A$}
\lbl[b]{7,19;$B$}
\lbl[l]{20.5,9;$C$}
\lbl[t]{11,0;$D$}
\end{lpic}
\end{wrapfigure}

\parit{Proof of Theorem~\ref{thm:inscribed-quadrilateral}.} 
Assume $\triangle ABC$ is degenerate.
By Corollary~\ref{cor:degenerate=pi},
$$2\cdot \measuredangle ABC\equiv 0;$$
From the same corollary, we get that
$$2\cdot \measuredangle CDA\equiv 0$$ 
if and only if $D\in (AB)$;
hence the result follows.

It remains to consider the case if $\triangle ABC$ is nondegenerate.

Let $O$ and $\Gamma$ denote the circulcenter and circumcircle of $\triangle ABC$.
According to Theorem~\ref{thm:inscribed-angle},
$$
2\cdot\measuredangle ABC
\equiv
\measuredangle AOC.
\eqlbl{eq:2<ABC=<AOB}
$$
From the same theorem, $D\in\Gamma$ if and only if 

$$
2\cdot\measuredangle CDA
\equiv\measuredangle COA.\eqlbl{eq:2<CDE=<BOA}
$$
Adding \ref{eq:2<ABC=<AOB} and \ref{eq:2<CDE=<BOA},
we get the result.
\qeds

\begin{thm}{Exercise}\label{ex:two-chords}
Let $[XY]$ and $[X'Y']$
 be two parallel chords of a circle.
Show that $XX'=YY'$.
\end{thm}

{

\begin{wrapfigure}{r}{33mm}
\centering
\begin{lpic}[t(-8mm),b(-0mm),r(0mm),l(0mm)]{pics/AB2XY(1)}
\lbl[t]{13,2;$A$}
\lbl[rb]{8,16;$B$}
\lbl[r]{1,7;$Y$}
\lbl[t]{7,0;$X$}
\lbl[tl]{30,8;$X'$}
\lbl[b]{14.5,24;$Y'$}
\end{lpic}
\end{wrapfigure}

\begin{thm}{Exercise}\label{ex:secant-circles}
Let $\Gamma$ and $\Gamma'$
be two circles 
which intersect at two distinct points: $A$ and~$B$.
Assume $[XY]$ and $[X'Y']$ are the chords of $\Gamma$ and $\Gamma'$ correspondingly,
such that $A$ lies between $X$ and $X'$ and $B$ lies between $Y$ and~$Y'$.
Show that $(XY)\parallel (X'Y')$.
\end{thm}

}




\section*{Method of additional circle}
\addtocontents{toc}{Method of additional circle.}

\begin{thm}{Problem}\label{prob:add-circ}
Assume that two chords $[AA']$ and $[BB']$ intersect at the point $P$ inside their circle.
Let $X$ be a point such that both angles $XAA'$ and $XBB'$ are right.
Show that $(XP)\perp(A'B')$.
\end{thm}

\begin{wrapfigure}{o}{39mm}
\centering
\begin{lpic}[t(-2mm),b(6mm),r(0mm),l(0mm)]{pics/add-a-circle(1)}
\lbl[t]{30,9;$A$}
\lbl[r]{4,23;$B$}
\lbl[l]{25,43;$A'$}
\lbl[bl]{38,22;$B'$}
\lbl[rbw]{25,24;$P$}
\lbl[ltw]{6,6;$X$}
\lbl[bw]{33.5,29;$Y$}
\end{lpic}
\end{wrapfigure}

\parit{Solution.}
Set $Y=(A'B')\cap (XP)$.

Both angles $XAA'$ and $XBB'$ are right;
therefore
\[2\cdot\measuredangle XAA'
\equiv
2\cdot\measuredangle XBB'.\]
By Theorem~\ref{thm:inscribed-quadrilateral},  $\square XAPB$ is inscribed.
Applying this theorem again we get that
\[2\cdot\measuredangle AXP
\equiv
2\cdot\measuredangle ABP.\]

Since $\square ABA'B'$ is inscribed, 
\[2\cdot\measuredangle ABB'
\equiv
2\cdot\measuredangle AA'B'.\]

It follows that 
\[2\cdot\measuredangle AXY
\equiv
2\cdot\measuredangle AA'Y.\]
By the same theorem $\square XAYA'$ is inscribed,
and
therefore, 
\[2\cdot\measuredangle XAA'
\equiv
2\cdot\measuredangle XYA'.\]
Since $\angle XAA'$ is right, 
so is $\angle XYA'$. 
That is $(XP)\perp(A'B')$.
\qeds

\begin{thm}{Exercise}\label{ex:inaccuracy}
Find an inaccuracy in the solution of the problem \ref{prob:add-circ} and try to fix it.
\end{thm}

The method used in the solution 
is called {}\emph{method of additional circle},
since the circumcircles of the $\square XAPB$ and $\square XAPB$ 
 above can be considered as {}\emph{additional constructions}. 

{

\begin{wrapfigure}{r}{26mm}
\centering
\begin{lpic}[t(-6mm),b(0mm),r(0mm),l(0mm)]{pics/equileteral(1)}
\lbl[tr]{19,0;$L$}
\lbl[rb]{16,18;$M$}
\lbl[r]{3,7.5;$N$}
\lbl[t]{4,1.5;$O$}
\lbl[l]{23.5,13.5;$X$}
\end{lpic}
\end{wrapfigure}

\begin{thm}{Exercise}\label{ex:equilateral-2}
Assume three lines $\ell$, $m$, and $n$ intersect at point $O$ and form six equal angles at~$O$. 
Let $X$ be a point distinct from $O$.
Let $L$, $M$, and $N$ denote the foot points of perpendiculars from $X$ on $\ell$, $m$, and $n$ correspondingly.
Show that $\triangle LMN$ is equilateral.
\end{thm}
}

\begin{thm}{Advanced exercise}\label{ex:simson}
Assume that a point $P$ lies on the circumcircle the triangle $ABC$.
Show that three foot points of $P$ on the lines $(AB)$, $(BC)$, and $(CA)$ lie on one line
(which is called the \index{Simson line}\emph{Simson line} of $P$).
\end{thm}

\begin{thm}{Exercise}
Watch ``Why is pi here? And why is it squared? A geometric answer to the Basel problem'' by Grant Sanderson. (It is available on \href{https://youtu.be/d-o3eB9sfls}{YouTube}.) 
\end{thm}



\section*{Arcs}
\addtocontents{toc}{Arcs.}

A subset of a circle bounded by two points is called a circle arc.

More precisely,
let $\Gamma$ be a circle and $A$, $B$, $C$ be distinct points on~$\Gamma$.
The subset  which includes the points $A$, $C$
as well as all the points on $\Gamma$ which lie with $B$ on the same side of $(AC)$ is called \index{circle arc}\emph{circle arc}~$ABC$.

\begin{wrapfigure}[8]{o}{40mm}
\centering
\begin{lpic}[t(-0mm),b(0mm),r(0mm),l(0mm)]{pics/tangent-half-line(1)}
\lbl[lb]{16,25;$A$}
\lbl[l]{24,13;$B$}
\lbl[tl]{18,3;$C$}
\lbl[t]{37,14;$X$}
\lbl[lb]{5,6;$\Gamma$}
\end{lpic}
\end{wrapfigure}


For the circle arc $ABC$, 
the points $A$ and $C$ are called 
\index{endpoint of arc}\emph{endpoints}. 
There are two circle arcs of $\Gamma$ with the given endpoints.

A half-line $[AX)$ is called 
\index{tangent!half-line}\emph{tangent} 
to the circle arc $ABC$ at $A$
if the line $(AX)$ is tangent to $\Gamma$, and the points $X$ and $B$ lie on the same side of the line $(AC)$.

If $B$ lies on the line $(AC)$, then the arc $ABC$ degenerates to one of two following a subsets of the line~$(AC)$:
\begin{itemize}
\item If $B$ lies between $A$ and $C$, then we define the arc $ABC$ as the segment $[AC]$. 
In this case the half-line $[AC)$ is tangent to the arc $ABC$ at~$A$.
\item If $B\in(AC)\backslash [AC]$, then we define the arc $ABC$ as the line $(AC)$ without all the points between $A$ and~$C$.
If we choose points $X$ and $Y\in (AC)$ such that the points $X$, $A$, $C$, and $Y$ appear in the same order on the line, 
then the arc $ABC$ is the union of two half-lines in $[AX)$ and~$[CY)$.
In this case, the half-line $[AX)$ is tangent to the arc $ABC$ at~$A$.
\end{itemize}

In addition, any half-line $[AB)$ will be regarded as an arc.
This degenerate arc has only one end point $A$
and it assumed to be tangent to itself at~$A$.
The circle arcs together with the degenerate arcs will be called \index{arc}\emph{arcs}.

\begin{thm}{Proposition}\label{prop:arc(angle=angle)}
A point $D$ lies on the arc $ABC$ if and only if 
$$\measuredangle ADC= \measuredangle ABC$$
or $D$ coincides with $A$ or~$C$.
\end{thm}

\parit{Proof.}
If $A$, $B$, and $C$ lie on one line, 
then the statement is evident.

\begin{wrapfigure}{o}{23mm}
\centering
\begin{lpic}[t(-0mm),b(0mm),r(0mm),l(0mm)]{pics/tangent-half-line-3(1)}
\lbl[rb]{8,23;$A$}
\lbl[l]{21,17;$B$}
\lbl[tr]{1,3;$C$}
\lbl[tl]{18,3;$D$}
\end{lpic}
\end{wrapfigure}

Assume $\Gamma$ be the circle passing thru $A$, $B$, and~$C$.

Assume $D$ is distinct from $A$ and~$C$.
According to Theorem~\ref{thm:inscribed-quadrilateral},
$D\in\Gamma$ if and only if 
$$\measuredangle ADC
= \measuredangle ABC
\quad
\text{or}
\quad
\measuredangle ADC
\equiv
\measuredangle ABC+\pi.$$


By Exercise~\ref{ex:signs-PXQ-PYQ},
if the first identity holds, 
then $B$ and $D$ lie on one side of $(AC)$;
in this case $D$ belongs to the arc~$ABC$.
If the second identity holds, 
then the points $B$ and $D$ lie on opposite sides of $(AC)$,
in this case $D$ does not belong to the arc~$ABC$.
\qeds

\begin{thm}{Proposition}\label{prop:arc(angle=tan)}
The half-line $[AX)$ is tangent to the arc $ABC$ if and only if 
$$\measuredangle ABC+\measuredangle CAX\equiv \pi.$$

\end{thm}

\parit{Proof.}
For a degenerate arc $ABC$, 
the statement is evident.
Further we assume the arc $ABC$ is nondegenerate.

Applying theorems \ref{thm:tangent-angle}
and \ref{thm:inscribed-angle},
we get that
$$2\cdot \measuredangle ABC+2\cdot\measuredangle CAX\equiv 0.$$
Therefore, either 
$$\measuredangle ABC+\measuredangle CAX
\equiv 
\pi,
\quad
\text{or}
\quad
\measuredangle ABC+\measuredangle CAX
\equiv 
0.$$

\begin{wrapfigure}[9]{o}{26mm}
\centering
\begin{lpic}[t(-4mm),b(3mm),r(0mm),l(0mm)]{pics/tangent-half-line-2(1)}
\lbl[r]{7,24;$A$}
\lbl[l]{18,13.5;$B$}
\lbl[br]{1,7;$C$}
\lbl[b]{23.5,22;$X$}
\end{lpic}
\end{wrapfigure}

Since $[AX)$ is the tangent half-line to the arc $ABC$,
$X$ and $B$ lie on the same side of~$(AC)$.
Therefore, the angles $CAX$, $CAB$, and $ABC$ 
have the same sign.
In particular,
$\measuredangle ABC\z+\measuredangle CAX\not\equiv 0$;
that is, we are left with the case 
$$\measuredangle ABC+\measuredangle CAX\equiv \pi.$$
\qedsf

\begin{thm}{Exercise}\label{ex:arc-tan-straight}
Assume that
the half-lines $[AX)$ and $[AY)$
are tangent to the arcs $ABC$ and $ACB$ correspondingly.
Show that $\angle XAY$ is straight.
\end{thm}


\begin{thm}{Exercise}\label{ex:tangent-arc}
Show that there is a unique arc 
with endpoints at the given points $A$ and $C$, 
which is tangent at $A$ to the given half line~$[AX)$.
\end{thm}

%(???

\begin{thm}{Exercise}\label{ex:tangent-lim}
Let $[AX)$ be the tangent half-line to an arc $ABC$.
Assume $Y$ is a point on the arc $ABC$ which is distinct from $A$.
Show that $\measuredangle XAY\to 0$ as $AY\to 0$.

\end{thm}

%???)

{

\begin{wrapfigure}{r}{28mm}
\centering
\begin{lpic}[t(-3mm),b(-0mm),r(0mm),l(0mm)]{pics/two-arcs(1)}
\lbl[l]{11,16;$A$}
\lbl[bl]{22,24;$B_1$}
\lbl[tr]{5,5;$B_2$}
\lbl[bl]{16,11;$C$}
\lbl[t]{26,7;$Y_1$}
\lbl[l]{18,1;$Y_2$}
\lbl[lb]{8,25;$X_1$}
\lbl[t]{1,12;$X_2$}
\end{lpic}
\end{wrapfigure}


\begin{thm}{Exercise}\label{ex:two-arcs}
Given two circle arcs $AB_1C$ and $AB_2C$,
let $[AX_1)$ and $[AX_2)$ be the half-lines tangent to the arcs $AB_1C$ and $AB_2C$ at $A$,
and 
$[CY_1)$ and $[CY_2)$ be the half-lines tangent to the arcs $AB_1C$ and $AB_2C$ at~$C$.
Show that
$$\measuredangle X_1AX_2\equiv -\measuredangle Y_1CY_2.$$

\end{thm}

\begin{thm}{Exercise}\label{ex:3x120}
Given an acute triangle $ABC$
make a compass-and-ruler construction of the point $Z$ such that
\[\measuredangle AZB
= \measuredangle BZC
= \measuredangle CZA
=\pm\tfrac23\cdot\pi\]

\end{thm}

}


\addtocontents{toc}{\protect\end{quote}}

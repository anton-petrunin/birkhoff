\addtocontents{toc}{\protect\begin{center}}
\addtocontents{toc}{\large{\bf Inversive geometry}}
\addtocontents{toc}{\protect\end{center}}
\chapter{Inscribed angles}\label{chap:inscribed-angle}
\addtocontents{toc}{\protect\begin{quote}}

\section*{Angle between a tangent line and a chord}
\addtocontents{toc}{Angle between a tangent line and a chord.}

\begin{thm}{Theorem}\label{thm:tangent-angle}
Let $\Gamma$ be a circle with the center $O$.
Assume the line $(XQ)$ is tangent to $\Gamma$ at $X$
and $[XY]$ is a chord of~$\Gamma$.
Then 
$$2\cdot\measuredangle QXY
\equiv\measuredangle X O Y.
\eqlbl{eq:tangent-angle}$$
Equivalently, 
$$\measuredangle QXY
\equiv
\tfrac12\cdot\measuredangle X O Y
\quad 
\text{or}
\quad
\measuredangle QXY
\equiv
\tfrac12\cdot\measuredangle X O Y+\pi.$$

\end{thm}

\begin{wrapfigure}{o}{33mm}
\centering
\includegraphics{mppics/pic-120}
\end{wrapfigure}

\parit{Proof.}
Note that $\triangle XOY$ is isosceles.
Therefore, $\measuredangle YXO=\measuredangle OYX$.

Applying Theorem~\ref{thm:3sum}
to $\triangle XOY$,
we get
\begin{align*}
\pi&\equiv\measuredangle YXO+\measuredangle OYX+\measuredangle XOY\equiv
\\
&\equiv 2\cdot \measuredangle YXO+\measuredangle XOY.
\end{align*}

By Lemma~\ref{lem:tangent}, $(OX)\z\perp(XQ)$.
Therefore, 
$$\measuredangle QXY+\measuredangle YXO \equiv\pm\tfrac\pi2.$$

Therefore, 
$$2\cdot\measuredangle QXY
\equiv \pi -2\cdot \measuredangle YXO
\equiv\measuredangle X O Y.
$$
\qedsf

\section*{Inscribed angle}\label{sec:inscribed}
\addtocontents{toc}{Inscribed angle.}


We say that a triangle is \index{inscribed triangle}\emph{inscribed} in the circle $\Gamma$ if all its vertices lie on~$\Gamma$.

\begin{thm}{Theorem}\label{thm:inscribed-angle}
Let $\Gamma$ be a circle with the center $O$,
and $X,Y$ be two distinct points on~$\Gamma$.
Then
$\triangle X P Y$ is inscribed in $\Gamma$ if and only if
$$2\cdot\measuredangle X P Y\equiv\measuredangle X O Y.
\eqlbl{eq:inscribed-angle}$$
Equivalently, if and only if
$$\measuredangle XPY\equiv\tfrac12\cdot\measuredangle X O Y
\quad
\text{or}
\quad
\measuredangle XPY\equiv\pi+\tfrac12\cdot\measuredangle X O Y.$$

\end{thm}

\begin{wrapfigure}{o}{33mm}
\vskip-6mm
\centering
\includegraphics{mppics/pic-122}
\vskip4mm
\includegraphics{mppics/pic-124}
\vskip4mm
\includegraphics{mppics/pic-126}
\end{wrapfigure}


\parit{Proof; the ``only if'' part.}
Let $(PQ)$ be the tangent line to $\Gamma$ at~$P$.
By Theorem~\ref{thm:tangent-angle},
\begin{align*}
2\cdot\measuredangle QPX&\equiv\measuredangle POX,
&
2\cdot\measuredangle QPY&\equiv\measuredangle POY.
\end{align*}
Subtracting one identity from the other, we get~\ref{eq:inscribed-angle}.

\parit{``If'' part.}
Assume that \ref{eq:inscribed-angle} holds for some $P\notin \Gamma$.
Note that $\measuredangle X O Y\ne 0$. 
Therefore, $\measuredangle X P Y\ne 0$ nor $\pi$;
that is, $\triangle PXY$ is nondegenerate.

The line $(PX)$ might be tangent to $\Gamma$ at the point $X$ or intersect $\Gamma$ at another point;
in the latter case, let $P'$ denotes this point of intersection. 

In the first case, by Theorem~\ref{thm:tangent-angle}, we have
\begin{align*}
2\cdot \measuredangle PXY&\equiv \measuredangle XOY\equiv 
 2\cdot\measuredangle  XPY.
\end{align*}
Applying the transversal property (\ref{thm:parallel-2}), we get that
$(XY)\parallel (PY)$, which is impossible since $\triangle PXY$ is nondegenerate.

In the second case, 
applying the ``if'' part and that  $P$, $X$, and $P'$ lie on one line (see Exercise~\ref{ex:ABCO-line}) we get that 
\begin{align*}
2\cdot \measuredangle P'PY&\equiv
2\cdot \measuredangle XPY\equiv 
 \measuredangle  XOY\equiv
 \\
&\equiv 2\cdot\measuredangle  XP'Y\equiv
 2\cdot\measuredangle  XP'P.
\end{align*}
Again, by transversal property,
$(PY)\parallel (P'Y)$, which is impossible since $\triangle PXY$ is nondegenerate.
\qeds

{

\begin{wrapfigure}{r}{34mm}
\centering
\includegraphics{mppics/pic-128}
\vskip3mm
\includegraphics{mppics/pic-130}
\vskip3mm
\includegraphics{mppics/pic-132}
\end{wrapfigure}

\begin{thm}{Exercise}\label{ex:inscribed-angle}
Let $X$, $X'$, $Y$, and $Y'$ be distinct points on the circle $\Gamma$.
Assume $(XX')$ meets $(YY')$ at the point~$P$.
Show that 
\begin{enumerate}[(a)]
\item $2\cdot \measuredangle XPY=\measuredangle XOY+\measuredangle X'OY'$;
\item\label{ex:inscribed-angle:b} $\triangle PXY\sim \triangle PY'X'$;
\item\label{ex:inscribed-angle:power} $PX\cdot PX'=|OP^2-r^2|$, where $O$ is the center and $r$ is the radius of $\Gamma$.
\end{enumerate}

\end{thm}

(The value $OP^2-r^2$ is called the \index{power of a point}\emph{power} of the point $P$ with respect to the circle $\Gamma$. 
Part \textit{(\ref{ex:inscribed-angle:power})} of the exercise makes it a useful tool to study circles, but we are not going to consider it further in the book.) 

\begin{thm}{Exercise}\label{ex:inscribed-hex}
Three chords $[XX']$, $[YY']$, and $[ZZ']$
of the circle $\Gamma$ intersect at a point $P$.
Show that 
$$XY'\cdot ZX'\cdot YZ'=X'Y\cdot Z'X\cdot Y'Z.$$

\end{thm}

\begin{thm}{Exercise}\label{ex:altitudes-circumcircle}
Let $\Gamma$ be a circumcircle of an acute triangle $A B C$.
Let $A'$ and $B'$ denote the second points of intersection of the altitudes from $A$ and $B$ with~$\Gamma$.
Show that $\triangle A' B' C$ is isosceles.
\end{thm}

}

\begin{thm}{Exercise}\label{ex:two-chords}
Let $[XY]$ and $[X'Y']$
 be two parallel chords of a circle.
Show that $XX'=YY'$.
\end{thm}

\begin{thm}{Exercise}
Watch ``Why is pi here? And why is it squared? A geometric answer to the Basel problem'' by Grant Sanderson. (It is available on \href{https://youtu.be/d-o3eB9sfls}{YouTube}.) 
\end{thm}

\section*{Points on a common circle}
\addtocontents{toc}{Points on a common circle.}

Recall that the diameter of a circle is a chord that passes thru the center.
If $[XY]$ is the diameter of a circle with center $O$, then $\measuredangle X O Y\z=\pi$. 
Hence Theorem~\ref{thm:inscribed-angle} implies the following:


\begin{thm}{Corollary}\label{cor:right-angle-diameter}
Suppose $\Gamma$ is a circle with the diameter~$[AB]$.
A triangle $ABC$ has right angle ant $C$ if and only if $C\in\Gamma$.
\end{thm}

\begin{thm}{Exercise}\label{ex:two-right}
Given four points $A$, $B$, $A'$, and $B'$,
construct a point $Z$ such that both angles $AZB$ and $A'ZB'$ are right.
\end{thm}

\begin{thm}{Exercise}\label{ex:VVAA}
Let $\triangle A B C$ be a nondegenerate triangle,
$A'$ and $B'$ be foot points of altitudes from $A$ and~$B$ respectfully.
Show that the four points $A$, $B$, $A'$, and $B'$ lie on one circle.
What is the center of this circle?
\end{thm}

\begin{thm}{Exercise}\label{ex:perpendicular-ruler}
Assume a line $\ell$, 
a circle with its center on $\ell$ 
and a point $P\notin\ell$ are given.
Make a ruler-only construction of the perpendicular to $\ell$
from $P$.
\end{thm}

\begin{wrapfigure}{r}{34mm}
\vskip-4mm
\centering
\includegraphics{mppics/pic-133}
\end{wrapfigure}

\begin{thm}{Exercise}\label{ex:tnagents+midpoint}
Suppose that lines $\ell$, $m$ and $n$ pass thru a point $P$;
the lines $\ell$ and $m$ are tangent to a circle $\Gamma$ at $L$ and $M$;
the line $n$ intersects $\Gamma$ at two points $X$ and $Y$.
Let $N$ be the midpoint of $[XY]$.
Show that the points $P$, $L$, $M$, and $N$ lie on one circle.
\end{thm}

We say that a quadrilateral $ABCD$ is 
\index{quadrilateral!inscribed quadrilateral}\emph{inscribed in circle $\Gamma$}
if all the points $A$, $B$, $C$, and $D$ lie on $\Gamma$.

\begin{thm}{Corollary}\label{cor:inscribed-quadrilateral}
A nondegenerate quadrilateral $ABCD$ is inscribed in a circle if and only if 
\[2\cdot\measuredangle ABC\equiv 2\cdot\measuredangle ADC.\]

\end{thm}

\parit{Proof.}
Since $\square ABCD$ is nondegenerate, so is $\triangle ABC$.
Let $O$ and $\Gamma$ denote the circulcenter and circumcircle of $\triangle ABC$ (they exist by Exercise~\ref{ex:unique-cline}).

{

\begin{wrapfigure}[10]{o}{24mm}
\vskip-0mm
\centering
\includegraphics{mppics/pic-134}
\end{wrapfigure}

According to Theorem~\ref{thm:inscribed-angle},
$$
2\cdot\measuredangle ABC
\equiv
\measuredangle AOC.
$$
From the same theorem, $D\in\Gamma$ if and only if 

$$
2\cdot\measuredangle ADC
\equiv\measuredangle AOC,
$$
hence the result.
\qeds

}


{

\begin{wrapfigure}{r}{33mm}
\vskip-6mm
\centering
\includegraphics{mppics/pic-136}
\end{wrapfigure}

\begin{thm}{Exercise}\label{ex:secant-circles}
Let $\Gamma$ and $\Gamma'$
be two circles 
that intersect at two distinct points: $A$ and~$B$.
Assume $[XY]$ and $[X'Y']$ are the chords of $\Gamma$ and $\Gamma'$ correspondingly,
such that $A$ lies between $X$ and $X'$ and $B$ lies between $Y$ and~$Y'$.
Show that $(XY)\parallel (X'Y')$.
\end{thm}

}

\section*{Method of additional circle}
\addtocontents{toc}{Method of additional circle.}

\begin{thm}{Problem}\label{prob:add-circ}
Assume that two chords $[AA']$ and $[BB']$ intersect at the point $P$ inside their circle.
Let $X$ be a point such that both angles $XAA'$ and $XBB'$ are right.
Show that $(XP)\perp(A'B')$.
\end{thm}

\parit{Solution.}
Set $Y=(A'B')\cap (XP)$.

Both angles $XAA'$ and $XBB'$ are right;
therefore
\[2\cdot\measuredangle XAA'
\equiv
2\cdot\measuredangle XBB'.\]
By Corollary~\ref{cor:inscribed-quadrilateral},  $\square XAPB$ is inscribed.
Applying this theorem again we get that
\[2\cdot\measuredangle AXP
\equiv
2\cdot\measuredangle ABP.\]

\begin{wrapfigure}[9]{o}{32mm}
\vskip-7mm
\centering
\includegraphics{mppics/pic-138}
\end{wrapfigure}

Since $\square ABA'B'$ is inscribed, 
\[2\cdot\measuredangle ABB'
\equiv
2\cdot\measuredangle AA'B'.\]

It follows that 
\[2\cdot\measuredangle AXY
\equiv
2\cdot\measuredangle AA'Y.\]
By the same theorem $\square XAYA'$ is inscribed,
and
therefore, 
\[2\cdot\measuredangle XAA'
\equiv
2\cdot\measuredangle XYA'.\]
Since $\angle XAA'$ is right, 
so is $\angle XYA'$. 
That is $(XP)\perp(A'B')$.
\qeds

\begin{thm}{Exercise}\label{ex:inaccuracy}
Find an inaccuracy in the solution of Problem \ref{prob:add-circ} and try to fix it.
\end{thm}

The method used in the solution 
is called {}\emph{method of additional circle},
since the circumcircles of the quadrilaterals $XAPB$ and $XAPB$ 
 above can be considered as {}\emph{additional constructions}. 

{

\begin{wrapfigure}{r}{27mm}
\vskip-8mm
\centering
\includegraphics{mppics/pic-140}
\end{wrapfigure}

\begin{thm}{Exercise}\label{ex:equilateral-2}
Assume three lines $\ell$, $m$, and $n$ intersect at point $O$ and form six equal angles at~$O$. 
Let $X$ be a point distinct from $O$.
Let $L$, $M$, and $N$ denote the foot points of perpendiculars from $X$ on $\ell$, $m$, and $n$ correspondingly.
Show that $\triangle LMN$ is equilateral.
\end{thm}
}

\begin{thm}{Advanced exercise}\label{ex:simson}
Assume that a point $P$ lies on the circumcircle the triangle $ABC$.
Show that three foot points of $P$ on the lines $(AB)$, $(BC)$, and $(CA)$ lie on one line.
(This line is called the \index{Simson line}\emph{Simson line} of $P$).
\end{thm}

\section*{Arcs of circlines}
\addtocontents{toc}{Arcs of circlines.}

A subset of a circle bounded by two points is called a circle arc.

More precisely,
suppose $A$, $B$, $C$ are distinct points on a circle $\Gamma$.
The \index{circle arc}\emph{circle arc}~$ABC$ is the subset that includes the points $A$, $C$
as well as all the points on $\Gamma$ that lie with $B$ on the same side of $(AC)$.

For the circle arc $ABC$, 
the points $A$ and $C$ are called 
\index{endpoint of arc}\emph{endpoints}. 
There are precisely two circle arcs of $\Gamma$ with the given endpoints; they are \index{opposite arc}\emph{opposite} to each other.

Suppose $X$ be another point on $\Gamma$.
By Corollary~\ref{cor:inscribed-quadrilateral} we have
that $2\cdot\measuredangle AXC\equiv 2\cdot\measuredangle ABC$;
that is,
\[\measuredangle AXC\equiv\measuredangle ABC
\quad\text{or}\quad
\measuredangle AXC\equiv\measuredangle ABC+\pi.\]

\begin{wrapfigure}{r}{33mm}
\vskip-2mm
\centering
\includegraphics{mppics/pic-142}
\end{wrapfigure}

Recall that $X$ and $B$ lie on the same side from $(AC)$ if and only if $\angle AXC$ and $\angle ABC$ have the same sign (see Exercise~\ref{ex:signs-PXQ-PYQ}).
It follows that 
\begin{itemize}
\item $X$ lies on the arc $ABC$ if and only if 
\[\measuredangle AXC\equiv\measuredangle ABC;\]
\item $X$ lies on the arc opposite to $ABC$ if 
\[\measuredangle AXC\equiv\measuredangle ABC+\pi.\]
\end{itemize}

Note that a circle arc $ABC$ is defined if $\triangle ABC$ is not degenerate.
If $\triangle ABC$ is degenerate, then arc $ABC$ is defined as a subset of line bounded by $A$ and $C$ that contain $B$.


\begin{wrapfigure}{r}{43mm}
\vskip-2mm
\centering
\includegraphics{mppics/pic-144}
\vskip4mm
\includegraphics{mppics/pic-146}
\end{wrapfigure}

More precisely,  if $B$ lies between $A$ and $C$, then the \emph{arc} $ABC$ is defined as 
the line segment $[AC]$.
If $B'$ lies on the extension of $[AC]$,
then the arc $AB'C$ is defined as a union of disjoint half-lines $[AX)$ and $[CY)$ in $(AC)$.
In this case the arcs $ABC$ and $AB'C$ are called opposite to each other.

\begin{wrapfigure}{r}{43mm}
\vskip-2mm
\includegraphics{mppics/pic-148}
\end{wrapfigure}

In addition, any half-line $[AB)$ will be regarded as an arc.
If $A$ lies between $B$ and $X$, then $[AX)$ will be called opposite to $[AB)$.
This degenerate arc has only one end point $A$.

It will be convenient to use the notion of 
\index{circline}\emph{circline},
that means {}\emph{circle or line}.
For example any arc is a subset of a circline;
we also may use term \index{circline arc}\emph{circline arc} if we want to emphasise that the arc might be degenerate.
Note that for any three distinct points $A$, $B$, and $C$ there is unique circline arc $ABC$.

The following statement summarizes the discussion above.

\begin{thm}{Proposition}\label{prop:arcs}
Let $ABC$ be a circline arc and $X$ be a point distinct from $A$ and $C$.
Then 
\begin{enumerate}[(a)]
\item $X$ lies on the arc $ABC$ if and only if 
 \[\measuredangle AXC=\measuredangle ABC;\]
\item $X$ lies on the arc opposite to $ABC$ if and only if 
 \[\measuredangle AXC\equiv\measuredangle ABC+\pi;\]
\end{enumerate}
\end{thm}

\begin{thm}{Exercise}\label{ex:3x120}
Given an acute triangle $ABC$
make a compass-and-ruler construction of the point $Z$ such that
\[\measuredangle AZB
= \measuredangle BZC
= \measuredangle CZA
=\pm\tfrac23\cdot\pi\]

\end{thm}

\begin{thm}{Exercise}\label{ex:a+b=c}
Suppose that point $P$ lies on the circumcircle of a equilateral triangle $ABC$
and $PA\le PB\le PC$.
Show that $PA+PB=PC$.
\end{thm}

A quadrilateral $ABCD$ is 
\index{quadrilateral!inscribed quadrilateral}\emph{inscribed}
if all the points $A$, $B$, $C$, and $D$ lie on a circline $\Gamma$.
If the arcs $ABC$ and $ADC$ are opposite, then we say that the points $A$, $B$, $C$, and $D$ appear on $\Gamma$ in the same \index{cyclic order}\emph{cyclic order}.

This definition makes possible to formulate the following refinement of Corollary~\ref{cor:inscribed-quadrilateral} which includes the degenerate quadrilaterals.
It follows directly from \ref{prop:arcs}.

\begin{thm}{Proposition}\label{prop:inscribed-quadrilateral}
A quadrilateral $ABCD$ is inscribed in a circline if and only if 
\[\measuredangle ABC=\measuredangle ADC\quad\text{or}\quad \measuredangle ABC\equiv\measuredangle ADC+\pi.\]
Moreover, the second identity holds if and only if the points $A,B,C,D$ appear on the circline in the same cyclic order.
\end{thm}


\section*{Tangent half-lines}
\addtocontents{toc}{Tangent half-lines.}

\begin{wrapfigure}{o}{30mm}
\centering
\includegraphics{mppics/pic-150}
\end{wrapfigure}

Suppose $ABC$ is an arc of a circle $\Gamma$.
A half-line $[AX)$ is called 
\index{tangent!half-line}\emph{tangent} 
to the arc $ABC$ at $A$
if the line $(AX)$ is tangent to $\Gamma$, and the points $X$ and $B$ lie on the same side of the line $(AC)$.

If the arc is formed by the line segment $[AC]$, then the half-line $[AC)$ is considered to be tangent at $A$.
If the arc is formed by a union of two half lines $[AX)$ and $[BY)$ in $(AC)$,
then the half-line $[AX)$ is considered to be tangent to the arc at $A$.

\begin{thm}{Proposition}\label{prop:arc(angle=tan)}
The half-line $[AX)$ is tangent to the arc $ABC$ if and only if 
$$\measuredangle ABC+\measuredangle CAX\equiv \pi.$$

\end{thm}

\parit{Proof.}
For a degenerate arc $ABC$, 
the statement is evident.
Further we assume the arc $ABC$ is nondegenerate.

\begin{wrapfigure}{o}{24mm}
\vskip-0mm
\centering
\includegraphics{mppics/pic-152}
\end{wrapfigure}

Applying theorems \ref{thm:tangent-angle}
and \ref{thm:inscribed-angle},
we get that
$$2\cdot \measuredangle ABC+2\cdot\measuredangle CAX\equiv 0.$$
Therefore, either 
$$\measuredangle ABC+\measuredangle CAX
\equiv 
\pi,
\quad
\text{or}
\quad
\measuredangle ABC+\measuredangle CAX
\equiv 
0.$$

Since $[AX)$ is the tangent half-line to the arc $ABC$,
$X$ and $B$ lie on the same side of~$(AC)$.
By \ref{cor:half-plane} and \ref{thm:signs-of-triug}, the angles $CAX$, $CAB$, and $ABC$ 
have the same sign.
In particular,
$\measuredangle ABC\z+\measuredangle CAX\not\equiv 0$;
that is, we are left with the case 
$$\measuredangle ABC+\measuredangle CAX\equiv \pi.$$
\qedsf

\begin{thm}{Exercise}\label{ex:tangent-arc}
Show that there is a unique arc 
with endpoints at the given points $A$ and $C$, 
that is tangent to the given half line~$[AX)$ at $A$.
\end{thm}

\begin{thm}{Exercise}\label{ex:tangent-lim}
Let $[AX)$ be the tangent half-line to an arc $ABC$.
Assume $Y$ is a point on the arc $ABC$ that is distinct from $A$.
Show that $\measuredangle XAY\to 0$ as $AY\to 0$.

\end{thm}

{

\begin{wrapfigure}{r}{33mm}
\vskip-8mm
\centering
\includegraphics{mppics/pic-154}
\end{wrapfigure}


\begin{thm}{Exercise}\label{ex:two-arcs}
Given two circle arcs $AB_1C$ and $AB_2C$, let $[AX_1)$ and $[AX_2)$ be the half-lines tangent to the arcs $AB_1C$ and $AB_2C$ at $A$, and $[CY_1)$ and $[CY_2)$ be the half-lines tangent to the arcs $AB_1C$ and $AB_2C$ at~$C$.
Show that
$$\measuredangle X_1AX_2\equiv -\measuredangle Y_1CY_2.$$

\end{thm}









\addtocontents{toc}{\protect\end{quote}}

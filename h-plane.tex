\chapter{Geometry of the h-plane}\label{chap:h-plane}
\addtocontents{toc}{\protect\begin{quote}}

In this chapter, we study the geometry of the plane described by the conformal disc model.
For briefness, this plane will be called the {}\emph{h-plane}.

We can work with this model directly from inside of the Euclidean plane. 
We may also use the axioms of neutral geometry since they all hold in the h-plane; the latter proved in the previous chapter.

\section*{Angle of parallelism}
\addtocontents{toc}{Angle of parallelism.}

Let $P$ be a point off an h-line~$\ell$. 
Drop a perpendicular $(PQ)_h$ from $P$ to $\ell$;
let $Q$ be its foot point.
Let  $\phi$ be the smallest value such that the h-line $(PZ)_h$ with $|\measuredangle_h Q P Z|=\phi$  does not intersect~$\ell$.

The value $\phi$ is called  the \index{angle!angle of parallelism}\emph{angle of parallelism} of $P$ to~$\ell$.
Clearly, $\phi$ depends only on the h-distance $s=PQ_h$.
Further, $\phi(s)\to \pi/2$ as  $s\to 0$, 
and $\phi(s)\to0$ as $s\to\infty$.
(In the Euclidean geometry, the angle of parallelism is identically equal to~$\pi/2$.)

\begin{wrapfigure}[11]{o}{44mm}
\begin{lpic}[t(-5mm),b(-0mm),r(0mm),l(0mm)]{pics/ultra-parallel(1)}
\lbl[bl]{13.6,25.5;$P$}
\lbl[lb]{26,25;$Q$}
\lbl[br]{20,34;$Z$}
\lbl[lb]{29,18;$\ell$}
\lbl[]{18.5,25;{\small $\phi$}}
\end{lpic}
\end{wrapfigure}

If $\ell$, $P$ and $Z$ are as above, then  the h-line $m=(PZ)_h$ is called \index{asymptotically parallel lines}\emph{asymptotically parallel} to~$\ell$.
In other words, two h-lines are asymptotically parallel if they share one ideal point.
(In hyperbolic geometry, the term {}\emph{parallel lines} is often used for \index{asymptotically parallel lines}\emph{asymptotically parallel lines}; we do not follow this convention.)

Given $P\not\in\ell$, there are exactly two asymptotically parallel lines thru $P$  to $\ell$; 
the remaining parallel lines are called \index{parallel lines!ultra parallel lines}\emph{ultra parallel}.


On the diagram, the two solid h-lines passing thru $P$ are asymptotically parallel to~$\ell$;
the  dashed h-line is ultra parallel to~$\ell$.








\begin{thm}{Proposition}\label{prop:angle-parallelism}
Let $Q$ be the foot point of $P$ on h-line~$\ell$.
Denote by $\phi$ the angle of parallelism of $P$ to~$\ell$.
Then
$$PQ_h=\tfrac12\cdot\ln \tfrac{1+\cos\phi}{1-\cos\phi}.$$

\end{thm}

\begin{wrapfigure}{o}{63mm}
\begin{lpic}[t(-5mm),b(0mm),r(-1mm),l(0mm)]{pics/absolute-triangle-2(1)}
\lbl[t]{30,2.5;$A$}
\lbl[b]{30,40;$B$}
\lbl[tr]{20,20.5;$P$}
\lbl[tl]{32,20.5;$X$}
\lbl[t]{60,19.5;$Z$}
\lbl[tl]{27,20.5;$Q$}
\lbl[tl]{33,34;$\phi$}
\end{lpic}
\end{wrapfigure}


\parit{Proof.} Applying a motion of the h-plane if necessary,
we may assume $P$ is the center of the absolute.
Then the h-lines thru $P$ are the intersections of Euclidean lines with the h-plane.

Let us denote by $A$ and $B$ the ideal points of~$\ell$.
Without loss of generality, we may assume that $\angle APB$ 
is positive.
In this case 
$$\phi=\measuredangle QPB=\measuredangle APQ=\tfrac12 \cdot\measuredangle APB.$$

Let $Z$ be the center of the circle  $\Gamma$ containing the h-line~$\ell$.
Set $X$ to be the point of intersection of the Euclidean segment $[AB]$ and the line~$(PQ)$.

Note that, $PX=\cos\phi$.
Therefore, by Lemma~\ref{lem:O-h-dist},
$$PX_h=\ln \tfrac{1+\cos\phi}{1-\cos\phi}.$$

Note that both angles $PBZ$ and $BXZ$ are right.
Since the angle $PZB$ is shared,  $\triangle ZBX\sim \triangle ZPB$.
In particular, 
$$ZX\cdot ZP=ZB^2;$$
that is, $X$ is the inverse of $P$ in~$\Gamma$.

The inversion in $\Gamma$ is the reflection of the h-plane in~$\ell$. 
Therefore
\begin{align*}
PQ_h&=QX_h=
\\
&=\tfrac12\cdot PX_h=
\\
&=\tfrac12\cdot\ln \tfrac{1+\cos\phi}{1-\cos\phi}.
\end{align*}
\qedsf


\begin{thm}{Exercise}\label{ex:small-angle}
Let $ABC$ be an equilateral h-triangle with side $100$.
Show that $|\measuredangle_h ABC|<10^{-10}$.
\end{thm}

\section*{Inradius of h-triangle}
\addtocontents{toc}{Inradius of triangle.}

\begin{thm}{Theorem}\label{thm:h-inradius}
The inradius of any h-triangle 
is less than $\tfrac12\cdot\ln3$.
\end{thm}

\parit{Proof.}
Fix an h-triangle $XYZ$.
Denote by $A$, $B$ and $C$ the ideal points of the h-half-lines
$[XY)_h$, $[YZ)_h$ and~$[ZX)_h$.

\begin{wrapfigure}[11]{o}{43mm}
\begin{lpic}[t(-3mm),b(4mm),r(0mm),l(0mm)]{pics/h-triangle-absolute(1)}
\lbl[tl]{25,23;{\small $X$}}
\lbl[r]{21,19;{\small $Y$}}
\lbl[b]{20,25;{\small $Z$}}
\lbl[tr]{17,0;$A$}
\lbl[br]{8,38;$B$}
\lbl[bl]{41,29;$C$}
\end{lpic}
\end{wrapfigure}

Consider the \index{triangle!ideal triangle}\emph{ideal triangle} $ABC$;
its \emph{sides} are the three pairwise asymptotically parallel lines with ideal points $A$, $B$ and $C$.

It should be clear that the inradius of the ideal triangle $ABC$
is bigger than the inradius of $\triangle_hXYZ$.


Applying an inversion if necessary,
we can assume that the h-incenter ($O$)
of the ideal triangle is the center of the absolute (see \ref{thm:main-observ}). 
Therefore, without loss of generality, we may assume 
$$\measuredangle AOB=\measuredangle BOC=\measuredangle COA=\tfrac23\cdot\pi.$$

It remains to find the inradius.
Denote by $Q$ the foot point of $O$ on~$(AB)_h$.
Then $OQ_h$ is the inradius.
Note that the angle of parallelism of $(AB)_h$ at $O$ is equal to $\tfrac\pi3$.

{

\begin{wrapfigure}[7]{o}{44mm}
\begin{lpic}[t(-5mm),b(-5mm),r(0mm),l(0mm)]{pics/absolute-triangle(1)}
\lbl[tl]{32,3;$A$}
\lbl[bl]{32,40;$B$}
\lbl[r]{0,20;$C$}
\lbl[r]{20.5,21;{\small $O$}}
\lbl[l]{29,20;$Q$}
\end{lpic}
\end{wrapfigure}

By Proposition~\ref{prop:angle-parallelism},
\begin{align*}
OQ_h&=\tfrac12\cdot\ln\frac{1+\cos\frac{\pi}{3}}{1-\cos\frac{\pi}{3}}=
\\
&=\tfrac12\cdot\ln\frac{1+\tfrac12}{1-\tfrac12}=
\\
&=\tfrac12\cdot\ln 3.
\end{align*}
\qedsf

%\begin{thm}{Exercise}\label{ex:right-h-triangle} Find the optimal upper for the inradius of right h-triangles. \end{thm}

\begin{thm}{Exercise}\label{ex:side-sup}
Let $\square_h ABCD$ be a quadrilateral in the h-plane 
such that the h-angles at $A$, $B$ and $C$ are right and $AB_h=BC_h$.
Find the optimal upper bound for~$AB_h$.
\end{thm}

}


\section*{Circles, horocycles and equidistants}
\addtocontents{toc}{Circles, horocycles and equidistants.}

%\begin{wrapfigure}{o}{35mm}
%\begin{lpic}[t(0mm),b(0mm),r(0mm),l(0mm)]{pics/h-circle(1)}
%\lbl[lt]{15,27;$P$}
%\lbl[t]{4,31;$P'$}
%\lbl[lt]{25,14;$\Gamma$}
%\end{lpic}
%\end{wrapfigure}

Note that according to Lemma~\ref{lem:h-circle=circle},
any h-circle is a Euclidean circle which lies completely in the h-plane.
Further, any h-line is an intersection of the h-plane with the circle 
perpendicular to the absolute.

In this section we will describe the 
h-geometric meaning of the intersections 
of the other circles with the h-plane.

You will see that all these intersections have a {}\emph{perfectly round shape} in the h-plane.

One may think of these curves as trajectories of a car which drives in the plane with a fixed position of the steering wheel.

In the Euclidean plane, 
this way you either run along a circle or along a line.

In the hyperbolic plane, the picture is different.
If you turn the steering wheel to the far right, you will run along a circle.
If you turn it less, at a certain position of the wheel,  you will never come back to the same point, but the path will be different from the line.
If you turn the wheel further a bit, you start to run along a path which stays at some fixed distance from an h-line.

\begin{wrapfigure}{o}{46mm}
\begin{lpic}[t(-3mm),b(-3mm),r(0mm),l(0mm)]{pics/equidistant(1)}
\lbl[b]{14,23,60;$m$}
\lbl[lt]{25,17;$g$}
\lbl[b]{15,40;$A$}
\lbl[rt]{3,15;$B$}
\lbl[lb]{23.5,35;$P$}
\lbl[t]{16,10.5;$P'$}
\lbl[br]{7,37;$\Gamma$}
\lbl[tl]{38,23;$\Delta$}
\end{lpic}
\end{wrapfigure}

\parbf{Equidistants of h-lines.}
Consider the h-plane with the absolute~$\Omega$.
Assume a circle $\Gamma$ intersects $\Omega$ in two distinct points, $A$ and~$B$. 
Denote by $g$ the intersection of $\Gamma$ with the h-plane.
Let us draw an  h-line $m$ with the ideal points $A$ and~$B$.
According to Exercise~\ref{ex:ideal-line-unique}, $m$ is uniquely defined.

Consider any h-line $\ell$ perpendicular to~$m$;
let $\Delta$ be the circle containing~$\ell$.

Note that $\Delta\perp \Gamma$.
Indeed,
according to Corollary~\ref{cor:perp-inverse-clines}, $m$ and $\Omega$ invert to themselves in~$\Delta$.
It follows that $A$ is the inverse of $B$ in~$\Delta$.
Finally, by Corollary~\ref{cor:perp-inverse}, we get that $\Delta\perp \Gamma$.

Therefore, inversion in $\Delta$ sends both $m$ and $g$ to themselves.
For any two points $P',P\in g$ there is a choice of $\ell$ and $\Delta$ as above such that
$P'$ is the invese of $P$ in $\Delta$.
By the main observation (\ref{thm:main-observ}) the inversion in $\Delta$ is a motion of the h-plane. Therefore, all points of $g$ lie on the same distance from~$m$.

In other words, $g$ is the set of points which lie on a fixed h-distance and on the same side from~$m$.

Such a curve $g$ is called 
\index{equidistant}\emph{equidistant} to h-line~$m$.
In Euclidean geometry, the equidistant from a line is a line;
apparently in hyperbolic geometry the picture is different.



\parbf{Horocycles.}
If the circle $\Gamma$ touches the absolute from inside at one point $A$, then the complement $h=\Gamma\backslash\{A\}$ lies in the h-plane.
This set is called a \index{horocycle}\emph{horocycle}.
It also has a perfectly round shape in the sense described above.

\begin{wrapfigure}[10]{o}{43mm}
\begin{lpic}[t(-0mm),b(-3mm),r(0mm),l(0mm)]{pics/oricircle(1)}
\lbl[rb]{16,15;$\Gamma$}
\lbl[rb]{2,30;$A$}
\end{lpic}
\end{wrapfigure}

Horocycles are the border case between circles and equidistants  to h-lines.
A horocycle might be considered as a limit of circles 
thru a fixed point
with the centers running to infinity along a line.
The same horocycle is a limit of equidistants which pass thru fixed point to the h-lines running to infinity.

\begin{thm}{Exercise}\label{ex:right-trig-horocycle}
Find the leg of an isosceles right h-triangle inscribed in a horocycle.
\end{thm}



\section*{Hyperbolic triangles}
\addtocontents{toc}{Hyperbolic triangles.}

\begin{thm}{Theorem}\label{thm:3sum-h}
Any nondegenerate hyperbolic triangle has a positive defect.
\end{thm}


\begin{wrapfigure}{o}{35mm}
\begin{lpic}[t(-5mm),b(-0mm),r(0mm),l(-0mm)]{pics/h-triangle(1)}
\lbl[tr]{11.5,5;$A$}
\lbl[tr]{33,9;$C$}
\lbl[r]{7,20;$B$}
\lbl[t]{15,11.5;$D$}
\end{lpic}
\end{wrapfigure}

\parit{Proof.}
Fix an h-triangle $ABC$.
According to Theorem~\ref{thm:3sum-a},
$$\defect(\triangle_hABC)\ge 0.\eqlbl{eq:defect<0}$$
It remains to show that in the case of equality, $\triangle_hABC$ degenerates.

Without loss of generality, we may assume that $A$ is the center of the absolute;
in this case 
$\measuredangle_h CAB\z=\measuredangle CAB$.
Yet we may assume that 
$$\measuredangle_h CAB,
\quad 
\measuredangle_h ABC,
\quad
\measuredangle_h BCA,
\quad
\measuredangle ABC,
\quad
\measuredangle BCA\ge 0.$$

Let $D$ be an arbitrary point in $[CB]_h$ distinct from $B$ and~$C$.
From Proposition~\ref{prop:arc(angle=tan)}, we have
$$\measuredangle ABC-\measuredangle_h ABC \equiv 
\pi-\measuredangle CDB
\equiv \measuredangle BCA-\measuredangle_h BCA.$$

From Exercise~\ref{ex:|3sum|}, we get
$$\defect(\triangle_hABC)=2\cdot(\pi-\measuredangle CDB).$$
Therefore, if we have equality in \ref{eq:defect<0}, then $\measuredangle CDB=\pi$.
In particular, the h-segment $[BC]_h$ coincides with the Euclidean segment~$[BC]$.
By Exercise~\ref{ex:line/h-line},
the latter can happen only if the h-line $(BC)_h$ passes thru the center of the absolute ($A$);
that is,  if $\triangle_hABC$ degenerates.
\qeds

The following theorem states, in particular, that nondegenerate hyperbolic triangles are congruent if their corresponding angles are equal.
In particular, in hyperbolic geometry, similar triangles have to be congruent.

\begin{thm}{AAA congruence condition}\label{thm:AAA}
Two nondegenerate triangles
 $\triangle_hABC$ and $\triangle_hA'B'C'$
in the h-plane are congruent if
$\measuredangle_hABC\z=\pm\measuredangle_hA'B'C'$,
$\measuredangle_hBCA\z=\pm\measuredangle_hB'C'A'$
and  
$\measuredangle_hCAB=\pm\measuredangle_hC'A'B'$.
\end{thm}

\parit{Proof.}
Note hat if $AB_h=A'B'_h$, then the theorem follows from ASA.

\begin{wrapfigure}{o}{32mm}
\begin{lpic}[t(-3mm),b(-0mm),r(2mm),l(3mm)]{pics/AAA(1)}
\lbl[r]{3,33;$A'$}
\lbl[lb]{26,9;$B'$}
\lbl[rb]{0.5,1;$C'$}
\lbl[lb]{20,15;$B''$}
\lbl[rb]{3,9.5;$C''$}
\end{lpic}
\end{wrapfigure}

Assume the contrary. 
Without loss of generality, we may assume that $AB_h<A'B'_h$.
Therefore, we can choose the point $B''\in [A'B']_h$  such that $A'B''_h=AB_h$.

Choose an h-half-line $[B''X)$ so that 
\[\measuredangle_h A'B''X=\measuredangle_h A'B'C'.\]
According to Exercise~\ref{ex:parallel-abs}, $(B''X)_h\parallel(B'C')_h$.

By Pasch's theorem (\ref{thm:pasch}),
$(B''X)_h$ intersects~$[A'C']_h$.
Denote by $C''$ the point of intersection.

According to ASA, $\triangle_h ABC\cong\triangle_h A'B''C''$;
in particular, 
$$\defect(\triangle_h ABC)=\defect(\triangle_h A'B''C'').
\eqlbl{eq:defect=defect}$$

Applying Exercise~\ref{ex:defect} twice, we get
$$\begin{aligned}
\defect(\triangle_h A'B'C')
&=
\defect(\triangle_h A'B''C'')
+
\\
&+\defect(\triangle_h B''C''C')+\defect(\triangle_h  B''C'B').
\end{aligned}
\eqlbl{eq:defect+defect}$$
By Theorem~\ref{thm:3sum-h}, all the defects have to be positive.
Therefore
$$\defect(\triangle_h A'B'C')
>\defect(\triangle_h ABC).$$
On the other hand,
$$\begin{aligned}
\defect(\triangle_h A'B'C')
&= |\measuredangle_hA'B'C'|+|\measuredangle_hB'C'A'|+|\measuredangle_hC'A'B'|=
\\
&=|\measuredangle_hABC|+|\measuredangle_hBCA|+|\measuredangle_hCAB|=
\\
&=\defect(\triangle_h ABC)
  \end{aligned}$$
--- a contradiction.
\qeds

Recall that a bijection from a plane to itself is called \emph{angle preserving} if 
\[\measuredangle ABC= \measuredangle A'B'C'\]
for any $\triangle ABC$ and its image $\triangle A'B'C'$.

\begin{thm}{Exercise}\label{ex:angle-preserving-hyp}
Show that any angle-preserving transformation of the h-plane is a motion.
\end{thm}

\section*{Conformal interpretation}
\addtocontents{toc}{Conformal interpretation.}

Let us give another interpretation of the h-distance.

\begin{thm}{Lemma}\label{lem:conformal}
Consider the h-plane with the unit circle centered at~$O$ as the absolute.
Fix a point $P$ and let $Q$ be another point in the h-plane.
Set $x=PQ$ and $y=PQ_h$, then
$$\lim_{x\to 0}\frac{y}{x}=\frac{2}{1-OP^2}.$$

\end{thm}

The above formula tells us that the h-distance from $P$ to a nearby point $Q$ is almost proportional to the Euclidean distance
with the coefficient $\tfrac{2}{1-OP^2}$.   
The value $\lambda(P)=\tfrac{2}{1-OP^2}$ is called the \index{conformal factor}\emph{conformal factor} of the h-metric.

The value $\tfrac1{\lambda(P)}=\tfrac12\cdot(1-OP^2)$
can be interpreted as the {}\emph{speed limit} at the given point~$P$. 
In this case the h-distance is the minimal time needed to travel from one point of the h-plane to another point.

\begin{wrapfigure}{o}{44mm}
\begin{lpic}[t(-0mm),b(0mm),r(0mm),l(0mm)]{pics/POQ-Gamma(1)}
\lbl[b]{40,35;$\Gamma$}
\lbl[br]{19,20;$O$}
\lbl[bl]{31,20;$P$}
\lbl[t]{31,14;$Q$}
\lbl[t]{23,12;$Q'$}
\lbl[b]{42,20;$P'$}
\end{lpic}
\end{wrapfigure}

\parit{Proof.}
If $P=O$, then according to Lemma~\ref{lem:O-h-dist}
$$\frac{y}{x}=\frac{\ln \tfrac{1+x}{1-x}}{x}\to 2\eqlbl{eq:O=P}$$
as $x\to0$.

If $P\ne O$,
denote by $P'$ the inverse of $P$ in the absolute.
Denote by $\Gamma$ the circle with the center $P'$ 
perpendicular to the absolute.

According to the main observation (\ref{thm:main-observ}) and Lemma~\ref{lem:P-->O}, 
the inversion in $\Gamma$ is a motion of the h-plane which sends $P$ to~$O$.
In particular, if we denote by $Q'$ the inverse of $Q$ in $\Gamma$, then $OQ'_h=PQ_h$.

Set $x'=OQ'$.
According to Lemma~\ref{lem:inversion-sim},
$$\frac{x'}{x}=\frac{OP'}{P'Q}.$$
Since $P'$ is the inverse of $P$ in the absolute, we have $PO\cdot OP'=1$.
Therefore, 
$$\frac{x'}{x}\to \frac{OP'}{P'P}=\frac{1}{1-OP^2}$$
as $x\to 0$.

Together with \ref{eq:O=P},
it implies that
$$\frac{y}{x}=\frac{y}{x'}\cdot \frac{x'}{x}\to \frac{2}{1-OP^2}$$
as $x\to 0$.\qeds

Here is an application of the lemma above.

\begin{thm}{Proposition}\label{prop:circum}
The circumference of an h-circle of the h-radius $r$ is 
$$2\cdot\pi\cdot\sinh r,$$
where \index{sh@$\sinh$}$\sinh r$ denotes the \index{hyperbolic sine}\emph{hyperbolic sine} of $r$;
that is,
$$\sinh r\df \frac{e^r-e^{-r}}{2}.$$

\end{thm}



Before we proceed with the proof, let us discuss the same problem in the Euclidean plane.

The circumference of a circle in the Euclidean plane
can be defined as the limit of perimeters of regular $n$-gons inscribed in the circle as $n\to \infty$.



Namely, let us fix $r>0$.
Given a positive integer $n$, consider $\triangle AOB$
such that
$\measuredangle AOB=\tfrac{2\cdot\pi}{n}$ and $OA=OB=r$.
Set $x_n=AB$.
Note that $x_n$ is the side of а regular $n$-gon inscribed in the circle of radius $r$. %???$n$-gon is undefined
Therefore, the perimeter of the $n$-gon is  $n\cdot x_n$.

\begin{wrapfigure}[11]{o}{43mm}
\begin{lpic}[t(-2mm),b(-0mm),r(0mm),l(-0mm)]{pics/7-gon(1)}
\lbl[l]{42,21;$A$}
\lbl[bl]{34,38;$B$}
\lbl[r]{19,21;$O$}
\lbl[]{30,26,25.7;$\tfrac{2\cdot\pi}{n}$}
\lbl[w]{33,21.5;$\,r\,$}
\lbl[w]{29,31.5,51.4;$\,r\,$}
\end{lpic}
\end{wrapfigure}

The circumference of the circle with the radius $r$ 
might be defined as the limit
$$\lim_{n\to\infty} n\cdot x_n=2\cdot\pi\cdot r.\eqlbl{eq:2pir}$$
(This limit can be taken as the definition of~$\pi$.)

In the following proof, we repeat the same construction in the h-plane.

\parit{Proof.}
Without loss of generality, we can assume that the center $O$ of the circle is the center of the absolute.

By Lemma~\ref{lem:O-h-dist}, 
the h-circle with the h-radius $r$ is the Euclidean circle with the center $O$ and the radius 
$$a=\frac{e^r-1}{e^r+1}.$$

Denote by $x_n$ and $y_n$ the side lengths of the regular $n$-gons inscribed in the circle in the Euclidean and hyperbolic plane correspondingly.

Note that $x_n\to0$ as $n\to\infty$.
By Lemma~\ref{lem:conformal},
\begin{align*}
\lim_{n\to\infty}\frac{y_n}{x_n}
&=\frac{2}{1-a^2}.
\end{align*}

Applying \ref{eq:2pir},
we get that the circumference of the h-circle can be found the following way:
\begin{align*}
\lim_{n\to\infty}n\cdot y_n
&=\frac{2}{1-a^2}\cdot\lim_{n\to\infty}n\cdot x_n=
\\
&=\frac{4\cdot\pi\cdot a}{1-a^2}=
\\
&=\frac{4\cdot\pi\cdot\left(\frac{e^r-1}{e^r+1}\right)}{1-\left(\frac{e^r-1}{e^r+1}\right)^2}=
\\
&=2\cdot\pi\cdot\frac{e^{r}-e^{-r}}{2}=
\\
&=2\cdot\pi\cdot\sinh r.
\end{align*}
\qedsf

\begin{thm}{Exercise}\label{ex:circum}
Denote by $\circum_h(r)$ the circumference of the h-circle of the radius~$r$.
Show that 
$$\circum_h(r+1)>2\cdot \circum_h(r)$$
for all $r>0$.
\end{thm}

\section*{Hyperbolic Pythagorean theorem}
\addtocontents{toc}{Hyperbolic Pythagorean theorem.}

Recall that $\cosh$ denotes \index{ch@$\cosh$}\index{hyperbolic cosine}\emph{hyperbolic cosine};
that is, the function defined by
$$\cosh x\df \tfrac{e^x+e^{-x}}2.$$

\begin{thm}{Hyperbolic Pythagorean theorem}\label{thm:pyth-h-poincare}
Assume that $ACB$ is an h-triangle with right angle at~$C$.
Set $a=BC_h$, $b=CA_h$ and $c=AB_h$.
Then
\[\cosh c=\cosh a\cdot\cosh b.
\eqlbl{eq:thm:pyth-h-poincare}\]

\end{thm}

The formula \ref{eq:thm:pyth-h-poincare} is proved by means of direct calculations.
Let us discuss the limit cases of this formula before going into the proof.

Note that $\cosh x$ can be written using the Taylor expansion
\[\cosh x=1+\tfrac1{2}\cdot x^2+\tfrac1{24}\cdot x^4+\dots.\]

It follows that if $a$ and $b$ are small and $c^2=a^2+b^2$ then
\begin{align*}
\cosh c &\approx 1+\tfrac1{2}\cdot c^2\approx
\\
&\approx(1+\tfrac1{2}\cdot a^2)\cdot (1+\tfrac1{2}\cdot b^2)
\approx 
\\
&\approx
\cosh a\cdot\cosh b.
\end{align*}
These approximations show that the original Pythagorean theorem (\ref{thm:pyth}) is a limit case of Hyperbolic Pythagorean theorem for small triangles.

For large $a$ and $b$ the values $e^{-a}$ and $e^{-b}$ are neglectable.
In this case we have the following approximations:
\begin{align*}
\cosh a\cdot\cosh b&\approx \tfrac{e^a}2\cdot\tfrac{e^b}2=
\\
&=\tfrac{e^{a+b-\ln 2}}{2}\approx
\\
&\approx \cosh(a+b-\ln 2).
\end{align*}
Therefore $c\approx a+b-\ln 2$. 

\begin{thm}{Exercise}\label{ex:c+1>a+b}
Assume that $ACB$ is an h-triangle with right angle at~$C$.
Set $a=BC_h$, $b=CA_h$ and $c=AB_h$.
Show that
\[c+\ln 2>a+b.\]

\end{thm}

In the proof of Hyperbolic Pythagorean theorem we use the following formula from Exercise~\ref{ex:cosh}:
\[\cosh PQ_h=\frac{PQ\cdot  P'Q'+PQ'\cdot  P'Q}{PP'\cdot QQ'},\]
Here $P$, $Q$ are h-points distinct from the absolute and $P'$, $Q'$ are their inversions in the absolute.
A complete proof of this formula is given in the hints.

\begin{wrapfigure}{o}{32mm}
\begin{lpic}[t(-0mm),b(0mm),r(0mm),l(0mm)]{pics/pyth-h-poincare(1)}
\lbl[r]{9.5,18;$A$}
\lbl[r]{10.3,27;$A'$}
\lbl[t]{16,9.7;$B$}
\lbl[t]{30,9.9;$B'$}
\lbl[tr]{10.2,10.2;$C$}
\end{lpic}
\end{wrapfigure}

\parit{Proof.}
We assume that absolute is a unit circle.
By the main observation (\ref{thm:main-observ}) we can assume that $C$ is the center of absolute.
Denote by $A'$ and $B'$ the inversions of $A$ and $B$ in the absolute.

Set $x=BC$, $y=AC$.
By Lemma~\ref{lem:O-h-dist}
\begin{align*}
a&=\ln \tfrac{1+x}{1-x} 
&
b&=\ln \tfrac{1+y}{1-y}.
\end{align*}
Therefore
\[\begin{aligned}
\cosh a&=\tfrac12\cdot (\tfrac{1+x}{1-x}+\tfrac{1-x}{1+x})=
&&&&&
\cosh b&=\tfrac12\cdot (\tfrac{1+y}{1-y}+\tfrac{1-y}{1+y})=
\\
&=\tfrac{1+x^2}{1-x^2},
&&&&&
&=\tfrac{1+y^2}{1-y^2}.
\end{aligned}
\eqlbl{cosha+coshb}
\]

Note that 
\begin{align*}
B'C&=\tfrac1x
&
A'C&=\tfrac1y
\intertext{Therefore}
BB'&=\tfrac1x-x,
&
AA'&=\tfrac1y-y,
\intertext{Since the triangles $ABC$, $A'BC$, $AB'C$, $A'B'C$ are right, by original Pythagorean theorem (\ref{thm:pyth}) implies}
AB&=\sqrt{x^2+y^2},
&
AB'&=\sqrt{\tfrac1{x^2}+y^2},
\\
A'B&=\sqrt{x^2+\tfrac1{y^2}},
&
A'B'&=\sqrt{\tfrac1{x^2}+\tfrac1{y^2}}.
\end{align*}

According to Exercise~\ref{ex:cosh},
\[
\begin{aligned}
\cosh c&= \frac{AB\cdot A'B'+AB'\cdot A'B}{AA'\cdot BB'}=
\\
&=
\frac{\sqrt{x^2+y^2}\cdot \sqrt{\tfrac1{x^2}
+\tfrac1{y^2}}+\sqrt{\tfrac1{x^2}+y^2}\cdot \sqrt{x^2+\tfrac1{y^2}}}
{(\tfrac1y-y)\cdot (\tfrac1x-x)}=
\\
&=\frac{x^2+y^2+1+x^2\cdot y^2}
{(1-y^2)\cdot (1-x^2)}
\\
&=\tfrac{1+x^2}
{1-x^2}\cdot
\tfrac{1+y^2}
{1-y^2}
\end{aligned}
\eqlbl{coshc}
\]
Finally note that \ref{cosha+coshb} and \ref{coshc} imply \ref{eq:thm:pyth-h-poincare}.
\qeds

\addtocontents{toc}{\protect\end{quote}}
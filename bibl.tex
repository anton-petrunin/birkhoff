\renewcommand{\bibname}{Used resources}
\begin{thebibliography}{52}

\bibitem{akopyan}
Akopyan, A. V. 
\textit{Geometry in figures,} 2017.


\bibitem{alexandrov}
Aleksandrov, A. D.
Minimal foundations of geometry,
\textit{Siberian Math. J.} 35 (1994), no. 6, 1057--1069.

\bibitem{bachmann} Bachmann, F.
\textit{Aufbau der Geometrie aus dem Spiegelungsbegriff,} 1959. 

\bibitem{arnold-rogness} Arnold, D; Rogness J., \textit{M\"obius transformations revealed}
\begin{verbatim}http://www-users.math.umn.edu/~arnold/moebius/\end{verbatim}


\bibitem{beltrami}  Beltrami, E.
\textit{Teoria fondamentale degli spazii di curvatura costante,} 
Annali. di Mat., ser II, 2 (1868), 232--255.

\bibitem{birkhoff} 
Birkhoff, G. D.
A set of postulates for plane geometry,
based on scale and protractors, 
\textit{Annals of Mathematics} 33 (1932), 329--345.

\bibitem{bolyai} Bolyai, J. \textit{Appendix.} 1832

\bibitem{euclid}\textit{Euclid's Elements.}

\bibitem{greenberg}
Greenberg, M. J.
\textit{Euclidean and non-Euclidean geometries: Development and history,}
4th ed., New York: W. H. Freeman, 2007.

\bibitem{kiselev}
Kiselev, A. P.
\textit{Kiselev's geometry.
Book I. Planimetry,}
Adapted from Russian by Alexander Givental.


\bibitem{lambert} 
Lambert, J. H.
\textit{Theorie der Parallellinien,}
%F. Engel, P. St\"ackel (Eds.) 
1786.
%Leipzig.

\bibitem{legendre} 
Legendre, A.-M.
\textit{El\'ements de g\'eom\'etrie}, 1794.

\bibitem{lobachevsky}\begin{otherlanguage}{russian}
Лобачевский, Н. И.  
О началах геометрии, 
\textit{Казанский вестник,} вып. 25--28 (1829--1830 гг.).
\end{otherlanguage}

\bibitem{lobachevsky-1840} 
Lobachevsky, N. I.
\textit{Geometrische Untersuchungen zur Theorie der Parallellinien,} 
%Berlin: F. Fincke,
1840.


\bibitem{moise} 
Moise, E.
\textit{Elementary geometry from an advanced standpoint,}  
%3rd ed. Boston: Addison-Wesley. 
1990.

\bibitem{prasolov}
Prasolov, V.
\textit{Problems in plane and solid geometry,} 
translated and edited by Dimitry Leites. 2006.

\bibitem{saccheri} 
Saccheri, G. G.
\textit{Euclides ab omni n\ae vo vindicatus,} 
1733.


\bibitem{sharygin}
\begin{otherlanguage}{russian}
Шарыгин, И. Ф.
\textit{Геометрия 7--9}, %М.: Дрофа, 
1997.
\end{otherlanguage}

\bibitem{tarski} Tarski, A.
What is elementary geometry? in
\textit{The axiomatic method,} edited by L. Henkin, P. Suppes, and A. Tarski,
1959.



\end{thebibliography}

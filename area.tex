\chapter{Area}
\label{chap:area}
\addtocontents{toc}{\protect\begin{quote}}

Any rigorous introduction to area 
is tedious.
On the other hand, 
there is a more general notion called \label{Lebesgue measure}\emph{Lebesgue measure}.
Historically, the notion of area inspire the development of Lebesgue measure;
but we will use Lebesgue measure to define of area.

The construction of Lebesgue measure typically use 
the method of coordinates 
and it is included in any textbook in Real Analysis.
The tedious part based of the properties of Lebesgue measure is packed in proof Theorem~\ref{thm:area} 
and it is given without proof.

\section*{Solid triangles}
\addtocontents{toc}{Solid  triangles.}

We say that a point $X$ 
lies \index{inside!inside a triangle}\emph{inside} a nondegenerate triangle $\triangle ABC$
if the following three condition hold:
\begin{itemize}
\item $A$ and $X$ lie on the same side from the line $(BC)$;
\item $B$ and $X$ lie on the same side from the line $(CA)$;
\item $C$ and $X$ lie on the same side from the line $(AB)$.
\end{itemize}

\begin{wrapfigure}[5]{o}{20 mm}
\begin{lpic}[t(-4 mm),b(0mm),r(0mm),l(0mm)]{pics/inside(1)}
\lbl[br]{2.5,4.5;$A$}
\lbl[t]{17,3;$B$}
\lbl[l]{9,15;$C$}
\lbl[t]{9,8;$X$}
\end{lpic}
\end{wrapfigure}

The set of all points inside $\triangle ABC$ 
and on its sides $[AB]$, $[BC]$, $[CA]$
will be called \index{triangle!solid triangle}\index{solid!triangle}\emph{solid triangle} and denoted by \index{1fig@$\solidtriangle$}$\solidtriangle ABC$.

\begin{thm}{Exercise}\label{ex:triangle-convex}
Show that solid triangle is \index{convex set}\emph{convex};
that is if $X,Y\in\solidtriangle ABC$
then $[XY]\subset \solidtriangle ABC$.
\end{thm}


The notations $\triangle ABC$ and $\solidtriangle ABC$ look similar,
they also have close but different meanings, 
which better not to be confused.
Recall that $\triangle ABC$ is an ordered triple of distinct points
(see page \pageref{page:def:triangle}),
while $\solidtriangle ABC$ is an infinite set of points.

In particular $\solidtriangle ABC=\solidtriangle BAC$; indeed any point which belong to the set $\solidtriangle ABC$ 
also belongs to the set $\solidtriangle BAC$
and the other way around.
On the other hand,
$\triangle ABC\ne\triangle BAC$ simply because the sequence of points $(A,B,C)$ is distinct from the sequence $(B,A,C)$.

Also, in general  $\triangle ABC\ncong\triangle BAC$, but it is always true that $\solidtriangle ABC\cong\solidtriangle BAC$, where congruence of the sets $\solidtriangle ABC$ and $\solidtriangle BAC$ 
is understood the following way.

\begin{thm}{Definition}\label{def:cong-sets}
Two sets $\mathcal{S}$ and $\mathcal{T}$ in the plane  
are called \index{congruent!sets}\emph{congruent} 
(briefly \index{1rel@$\cong$}$\mathcal{S}\cong \mathcal{T}$)
if 
$\mathcal{T}=f(\mathcal{S})$ for some motion $f$ of the plane.
\end{thm}

If $\triangle ABC$ is not degenerate
and \[\solidtriangle ABC\cong \solidtriangle A'B'C',\]
then after relabeling the vertices of $\triangle ABC$ 
we will have 
\[\triangle ABC\cong \triangle A'B'C'.\]
The existence of such relabeling follow from the exercise.

\begin{thm}{Exercise}\label{ex:vertex}
Let $\triangle ABC$ be nondegenerate and $X\z\in \solidtriangle ABC$.
Show that $X$ is a vertex of $\triangle ABC$
if and only if there is a line $\ell$ which intersects $\solidtriangle ABC$
at the single point $X$.
\end{thm}

\section*{Polygonal sets}
\addtocontents{toc}{Polygonal sets.}

\index{elementary set}\emph{Elementary set} on the plane 
is a set of one of the following three types:
\begin{itemize}
 \item one-point set;
 \item segment;
 \item solid triangle.
\end{itemize}

\begin{wrapfigure}{r}{24 mm}
\begin{lpic}[t(-13mm),b(0mm),r(0mm),l(0mm)]{pics/polygonal-set(1)}
\end{lpic}
\end{wrapfigure}

A set in the plane is called \index{polygonal set}\emph{polygonal} if it can be presented as a union of finite collection of elementary sets.

According to this definition, empty set $\emptyset$
is a polygonal set.
Indeed, $\emptyset$ is a union of an empty collection of elementary sets.

A polygonal set is called \index{polygonal set!degenerate polygonal set}\index{degenerate!polygonal set}\emph{degenerate} if it can be presented as union of finite number of one-point sets and segments.

If $X$ and $Y$ lie on the opposite sides of the line $(AB)$
then the union
$\solidtriangle AXB\cup \solidtriangle BYA$
is a polygonal set which is called \index{solid!quadrilateral, parallelogram, rectangle, square}\index{quadrilateral!solid quadrilateral}\emph{solid quadrilateral} and denoted as 
\index{1fig@$\solidsquare$}$\solidsquare AXBY$.
In particular, 
we can talk about \index{parallelogram!solid parallelogram}\emph{solid parallelograms}, \index{rectangle!solid rectangle}\emph{rectangles} and \index{square!solid square}\emph{squares} in the Euclidean plane.

\begin{wrapfigure}{o}{38 mm}
\begin{lpic}[t(-3mm),b(0mm),r(0mm),l(0mm)]{pics/two-subdivisions(1)}
\end{lpic}
\end{wrapfigure}

Typically a polygonal set admits many 
presentation as union of finite collection of elementary sets.
For example, if $\square AXBY$ is a parallelogram, then
\[\solidsquare AXBY=\solidtriangle AXB\cup \solidtriangle BYA=\solidtriangle XAY\cup \solidtriangle YBX.\]

\begin{thm}{Exercise}\label{ex:solid-square}
Show that a solid square is not degenerate.
\end{thm}

\begin{thm}{Exercise}\label{ex:poly-circ}
Show that a circle is not a polygonal set.
\end{thm}



\section*{Definition of area}
\addtocontents{toc}{Definition of area.}

\begin{thm}{Claim}\label{clm:poly-ring}
For any two polygonal sets $\mathcal{P}$ and $\mathcal{Q}$,
the union $\mathcal{P}\cup\mathcal{Q}$ 
as well as the intersection $\mathcal{P}\cap\mathcal{Q}$ 
are also polygonal sets.
\end{thm}

A class of sets which closed with respect to union and intersection  is called a {}\emph{ring}.
The claim above therefore states that polygonal sets in the plane form a ring.

\parit{Semi-proof.}
Let us present $\mathcal{P}$ and $\mathcal{Q}$
as a union of finite collection of elementary sets $\mathcal{P}_1,\dots,\mathcal{P}_k$ 
and $\mathcal{Q}_1,\dots,\mathcal{Q}_n$ correspondingly.

Note that
\[\mathcal{P}\cup\mathcal{Q}
=
\mathcal{P}_1
\cup
\dots
\cup
\mathcal{P}_k
\cup
\mathcal{Q}_1
\cup
\dots
\cup
\mathcal{Q}_n.\]
Therefore $\mathcal{P}\cup\mathcal{Q}$ is polygonal.

{
\begin{wrapfigure}{o}{22 mm}
\begin{lpic}[t(-2 mm),b(0mm),r(0mm),l(0mm)]{pics/intersection(1)}
\end{lpic}
\end{wrapfigure}

Note that the union of all sets $\mathcal{P}_i\cap \mathcal{Q}_j$ 
forms $\mathcal{P}\cap \mathcal{Q}$.

Therefore in order to show that $\mathcal{P}\cap \mathcal{Q}$ is polygonal,
it is sufficient to show that each $\mathcal{P}_i\cap \mathcal{Q}_j$ is polygonal for any pair $i, j$. 
The diagram should suggest an idea for the proof of the latter statement.
\qeds

}

The following theorem defines the area
as a function which returns a real number for any polygonal set and satisfying certain conditions.
We omit the proof of this theorem.
It follows from the construction of Lebesgue measure 
which can be found in any text book on Real Analysis.

\begin{thm}{Theorem}\label{thm:area}
For each polygonal set $\mathcal{P}$ in the Euclidean plane 
there is a real number $s$ 
called \index{area}\emph{area of $\mathcal{P}$} 
(briefly $s=\area \mathcal{P}$) such that 
\[\area\emptyset=0\ \ \text{and}\ \ \area\mathcal{K}=1\]
where  $\mathcal{K}$ the solid square with unit side
and the conditions
\begin{align*}
\mathcal{P}\cong\mathcal{Q}\ \ &\Rightarrow\ \ \area\mathcal{P}=\area\mathcal{Q};
\\
\mathcal{P}\subset\mathcal{Q}\ \ &\Rightarrow\ \ \area\mathcal{P}\le\area\mathcal{Q};
\\
\area\mathcal{P}+\area\mathcal{Q}
&=
\area(\mathcal{P}\cup\mathcal{Q})+\area(\mathcal{P}\cap\mathcal{Q})
\end{align*}
hold 
for any two polygonal sets $\mathcal{P}$ and $\mathcal{Q}$.

Moreover the area function 
\[\mathcal{P}\mapsto \area \mathcal{P}\]
is uniquely defined by the above conditions.
\end{thm}

Further you will see
that
based on this theorem, 
the concept of area 
can be painlessly developed to a dissent level 
without cheating.

It is also psossible to add Theorem~\ref{thm:area} as an extra axiom,
then the rest of the section becomes completely rigorous.




\begin{thm}{Proposition}\label{prop:area-positive}
For any polygonal set $\mathcal{P}$ in the Euclidean plane, 
we have
\[\area\mathcal{P}\ge 0.\]

\end{thm}

\parit{Proof.} 
Since $\emptyset \subset \mathcal{P}$,
we get
\[\area\emptyset\le \area\mathcal{P}.\]

Since $\area\emptyset=0$ the result follows.\qeds



\section*{Vanishing area and subdivisions}
\addtocontents{toc}{Vanishing area and subdivisions.}

\begin{thm}{Proposition}\label{prop:area-segment}
Any one-point set as well as any segment in the Euclidean plane have  vanishing area.
\end{thm}

\parit{Proof.}
Fix a line segment $[AB]$.
Consider a sold square $\solidsquare ABCD$.

Note that given a positive integer $n$,
there are $n$ disjoint segments $[A_1B_1],\dots,[A_nB_n]$ 
in $\solidsquare ABCD$,
such that each $[A_iB_i]$ is congruent to $[AB]$ in the sense of the Definition~\ref{def:cong-sets}.


\begin{wrapfigure}{o}{26 mm}
\begin{lpic}[t(-0mm),b(0mm),r(0mm),l(0mm)]{pics/segments-in-square(1)}
\lbl[b]{1,25;$A_1$}
\lbl[t]{1,0;$B_1$}
\lbl[t]{13,27;$\dots$}
\lbl[b]{13,-2;$\dots$}
\lbl[b]{24,25;$A_n$}
\lbl[t]{24,0;$B_n$}
\end{lpic}
\end{wrapfigure}


Applying the last identity in Theorem \ref{thm:area} few times, 
we get 
\begin{align*}
n\cdot \area[AB]
&=\area\left([A_1B_1]\cup\dots\cup[A_nB_n]\right)\le
\\
&\le \area(\solidsquare ABCD)              
\end{align*}
That is
\[\area[AB]\le \tfrac1n\cdot\area(\solidsquare ABCD)\] 
for any positive integer $n$.
Therefore $\area[AB]\le 0$.

On the other hand, by Proposition~\ref{prop:area-positive},
\[\area[AB]\ge 0.\]
Hence the result follows.

For any one-point set $\{A\}$ 
we have $\emptyset\subset \{A\}\subset [AB]$.
Therefore 
\[0\le \area\{A\}\le \area[AB]=0.\]
Whence $\area\{A\}=0$.
\qeds

\begin{thm}{Corollary}\label{cor:degenerate}
Any degenerate polygonal set in the Euclidean plane has vanishing area.
\end{thm}

\parit{Proof.}
Let $\mathcal P$ be a degenerate set,
say
\[\mathcal{P}=[A_1B_1]\cup\dots\cup[A_nB_n]\cup\{C_1,\dots,C_k\}.\]
Applying Theorem~\ref{thm:area} 
together with Proposition~\ref{prop:area-positive},
we get
\begin{align*}
\area\mathcal{P}\le
& \area[A_1B_1]+\dots+\area[A_nB_n]+
\\
&+\area\{C_1\}+\dots+\area\{C_k\}.
\end{align*}
By Proposition~\ref{prop:area-segment}, the right hand side vanish.
Hence the statement follows.
\qeds


We say that polygonal set $\mathcal{P}$ is \index{subdivision of polygonal set}\emph{subdivided} 
into two polygonal sets $\mathcal{Q}_1$ and $\mathcal{Q}_2$ 
if $\mathcal{P}=\mathcal{Q}_1\cup\mathcal{Q}_2$ 
and the intersection $\mathcal{Q}_1\cap\mathcal{Q}_2$ is degenerate.
(Recall that according to Claim~\ref{clm:poly-ring},
the set $\mathcal{Q}_1\cap\mathcal{Q}_2$ is polygonal.)

\begin{thm}{Proposition}\label{prop:subdivision}
Assume polygonal sets $\mathcal{P}$ is subdivided into two polygonal set $\mathcal{Q}_1$ and $\mathcal{Q}_2$.
Then 
\[\area\mathcal{P}=\area\mathcal{Q}_1+\area\mathcal{Q}_2.\]

\end{thm}

\parit{Proof.}
By Theorem~\ref{thm:area},
\[\area\mathcal{P}=\area\mathcal{Q}_1+\area\mathcal{Q}_2-\area(\mathcal{Q}_1\cap\mathcal{Q}_2).\]

Since $\mathcal{Q}_1\cap\mathcal{Q}_2$ is degenerate,
by Corollary~\ref{cor:degenerate},
\[\area(\mathcal{Q}_1\cap\mathcal{Q}_2)=0.\]
Hence the result follows.
\qeds


\section*{Area of solid rectangles}
\addtocontents{toc}{Area of solid rectangles,}

\begin{thm}{Theorem}\label{thm:area-rect}
The solid rectangle in the Euclidean plane 
with sides $a$ and $b$ has area $a\cdot b$.
\end{thm}

\begin{thm}{Algebraic lemma}\label{lem:alg-area}
Assume that a function $s$ 
returns a nonnegative real number $s(a,b)$ 
for any pair of positive real numbers $(a,b)$ 
and satisfies the following identities
\begin{align*}
s(1,1)&=1;
\\
s(a,b+c)&=s(a,b)+s(a,c)
\\
s(a+b,c)&=s(a,c)+s(b,c)
\end{align*}
for any $a,b,c>0$.
Then 
\[s(a,b)=a\cdot b\] 
for any $a,b>0$.
\end{thm}

The proof is similar to the proof of Lemma \ref{lem:R-auto};
we omit it.

\parit{Proof of Theorem~\ref{thm:area-rect}.}
Denote by $\mathcal{R}_{a,b}$ the solid rectangle with sides $a$ and $b$.
Set 
\[s(a,b)=\area \mathcal{R}_{a,b}.\]

By theorem \ref{thm:area}, 
$s(1,1)=1$.
That is, the first identity in the Algebraic Lemma holds.


\begin{wrapfigure}{o}{26 mm}
\begin{lpic}[t(-4 mm),b(4 mm),r(0mm),l(0mm)]{pics/two-rectangles(1)}
\lbl[]{6,7;$\mathcal{R}_{a,c}$}
\lbl[]{17,7;$\mathcal{R}_{b,c}$}
\lbl[t]{12,0;$\mathcal{R}_{a+b,c}$}
\end{lpic}
\end{wrapfigure}

Note that the rectangle $\mathcal{R}_{a+b,c}$
can be subdivided into two rectangle 
congruent to $\mathcal{R}_{a,c}$
and $\mathcal{R}_{b,c}$.
Therefore by Proposition~\ref{prop:subdivision}, 
\[
\area \mathcal{R}_{a+b,c}=\area \mathcal{R}_{a,c}+\area \mathcal{R}_{b,c}
\]
That is, the second identity in the Algebraic Lemma holds.
The proof of the third identity is analogues.

It remains to apply Algebraic lemma.
\qeds


\section*{Area of solid parallelograms}
\addtocontents{toc}{parallelograms}

\begin{thm}{Proposition}\label{prop:area-parallelogram}
Let $\square ABCD$ be a parallelogram in the Euclidean plane, $a=AB$ and $h$ be the distance between the lines $(AB)$ and $(CD)$.
Then 
\[\area(\solidsquare ABCD)=a\cdot h.\]

\end{thm}


\parit{Proof.}
Let us denote by $A'$ and $B'$ the foot points of $A$ and $B$
on the line $(CD)$.

Note that $ABB'A'$ is a rectangle with sides $a$ and $h$.
By Proposition~\ref{thm:area-rect},
\[\area( \solidsquare ABB'A')=h\cdot a.\eqlbl{eq:ABBA}\]

\begin{wrapfigure}{o}{41 mm}
\begin{lpic}[t(-3mm),b(0mm),r(0mm),l(0mm)]{pics/par-rect(1)}
\lbl[t]{1,-1;$A$}
\lbl[t]{17,-1;$B$}
\lbl[b]{39,12.5;$C$}
\lbl[b]{24,12.5;$D$}
\lbl[b]{1.5,12.5;$A'$}
\lbl[b]{17.5,12.5;$B'$}
\end{lpic}
\end{wrapfigure}

Without loss of generality we may assume that  $\solidsquare ABCA'$ 
contains $\solidsquare ABCD$ and $\solidsquare ABB'A'$.

In this case $\solidsquare ABB'D$ admits two subdivisions.
First into  $\solidsquare ABCD$ and $\solidtriangle AA'D$.
Second into $\solidsquare ABB'A'$ and $\solidtriangle BB'C$.

By Proposition~\ref{prop:subdivision},
\[\begin{aligned}
\area( \solidsquare ABCD)&+\area(\solidtriangle AA'D)=
\\
&=
\area(\solidsquare ABB'A')+ \area (\solidtriangle BB'C).   
  \end{aligned}
\eqlbl{eq:area-sum}\]

Note that 
\[\triangle AA'D\cong \triangle BB'C.\eqlbl{eq:cong-area}\]
Indeed, since $\square ABB'A'$ and $\square ABCD$ are parallelograms, 
by Lemma~\ref{lem:parallelogram},
we have $AA'=BB'$, $AD=BC$ and $DC=AB=A'B'$.
It follows that $A'D=B'C$.
Applying SSS congruence condition, we get \ref{eq:cong-area}.

In particular,
\[\area(\solidtriangle BB'C)=\area (\solidtriangle AA'D).
\eqlbl{eq:area-trinagles}\]

\begin{wrapfigure}{o}{22 mm}
\begin{lpic}[t(3 mm),b(0mm),r(0mm),l(0mm)]{pics/two-parallelograms(1)}
\lbl[tr]{1,7;$A$}
\lbl[t]{12,0;$B$}
\lbl[l]{21,20;$C$}
\lbl[b]{9,27.5;$D$}
\lbl[tl]{16,9;$B'$}
\lbl[bl]{18,28;$C'$}
\lbl[br]{4,26;$D'$}
\end{lpic}
\end{wrapfigure}

Subtracting \ref{eq:area-trinagles} from \ref{eq:area-sum},
we get
\[\area (\solidsquare ABCD)=\area(\solidsquare ABB'D).\]
From \ref{eq:ABBA}, the statement follows.
\qeds

\begin{thm}{Exercise}\label{ex:two-parallelograms}
Assume $\square ABCD$ and $\square AB'C'D'$ are two parallelograms such that $B'\in[BC]$ and $D\z\in [C'D']$.
Show that
\[\area(\solidsquare ABCD)=\area(\solidsquare AB'C'D').\]

\end{thm}


\section*{Area of solid triangles}
\addtocontents{toc}{and triangles.}


\begin{thm}{Theorem}\label{thm:area-of-triangle}
Let $a=BC$ and $h_A$ to be the altitude from $A$
in  $\triangle ABC$.
Then 
\[\area(\solidtriangle ABC)=\tfrac12\cdot a\cdot h_A.\]
\end{thm}

\parbf{Remark.}
Since $\triangle ABC$ completely determines the solid triangle $\solidtriangle ABC$,
it is acceptable to write 
$\area(\triangle ABC)$ for $\area(\solidtriangle ABC)$.

\parit{Proof.}
Draw the line $m$ thru $A$ which is parallel to $(BC)$
and line $n$ thru $C$ parallel to $(AB)$.
Note that $m\nparallel n$;
set $D=m\cap n$.
By construction, $\square ABCD$ is a parallelogram.

\begin{wrapfigure}{o}{24 mm}
\begin{lpic}[t(-0mm),b(0mm),r(0mm),l(0mm)]{pics/two-triangles(1)}
\lbl[t]{5,0;$A$}
\lbl[t]{21,0;$B$}
\lbl[b]{17.5,18;$C$}
\lbl[b]{2,18;$D$}
\lbl[b]{4.5,10,-75;$m$}
\lbl[b]{10,17;$n$}
\end{lpic}
\end{wrapfigure}

Note that $\solidsquare ABCD$ admits a subdivision into $\solidtriangle ABC$ and $\solidtriangle CDA$.
Therefore 
\[\area(\solidsquare ABCD)
=
\area(\solidtriangle ABC)
+
\area(\solidtriangle CDA)\]

Since $\square ABCD$ is a parallelogram, by Lemma~\ref{lem:parallelogram} we have
\[AB=CD\ \ \text{and}\ \  BC=DA.\]
Therefore by SSS congruence condition, we have
$\triangle ABC\cong\triangle CDA$.
In particular
\[\area(\solidtriangle ABC)
=
\area(\solidtriangle CDA).\]

From above and Proposition \ref{prop:area-parallelogram}, we get
\begin{align*}
\area(\solidtriangle ABC)
&=\tfrac12\cdot\area(\solidsquare ABCD)=
\\
&=\tfrac12\cdot h_A\cdot a
\end{align*}
\qedsf

\begin{thm}{Exercise}\label{ex:three-trig}
Denote by $h_A$, $h_B$ and $h_C$
the altitudes of $\triangle ABC$ from vertices $A$, $B$ and $C$ correspondingly.
Note that from Theorem~\ref{thm:area-of-triangle},
it follows that
\[h_A\cdot BC=h_B\cdot CA=h_C\cdot AB.\]

Give a proof of this statement without using area.
\end{thm}

\begin{thm}{Exercise}\label{ex:half-parallelogram}
Assume $M$ lies inside the parallelogram $\square ABCD$;
that is, $M$ belongs to the solid parallelogram $\solidsquare ABCD$, but does not lie on its sides.
Show that
\[\area(\solidtriangle ABM)+\area(\solidtriangle CDM)
=\tfrac12\cdot \area(\solidsquare ABCD).\]
\end{thm}


\begin{thm}{Exercise}\label{ex:area-diag}
Assume that diagonals 
of a nondegenerate quadrilateral $\square ABCD$ 
intersect at point $M$.
Show that 
\[\area(\solidtriangle ABM)\cdot\area(\solidtriangle CDM)
=
\area(\solidtriangle BCM)\cdot\area(\solidtriangle DAM).\]
 
\end{thm}

\begin{thm}{Exercise}\label{ex:area-inradius}
Let $r$ be the inradius of $\triangle ABC$
and $p$ be its semiperimeter; 
that is $p=\tfrac12\cdot(AB+BC+CA)$.
Show that
\[\area(\solidtriangle ABC)=p\cdot r.\]

\end{thm}

\begin{thm}{Advanced exercise}\label{ex:area-affine}
Show that for any affine transformation $\beta$ there is a constant $k>0$
such that the equality 
\[\area[\beta(\solidtriangle)]=k\cdot\area\solidtriangle.\]
holds for any solid triangle $\solidtriangle$.

Moreover, if $\beta$ has the matrix form $\left(\begin{smallmatrix}
x\\ y
\end{smallmatrix} \right)
  \mapsto
  \left(\begin{smallmatrix}
a&b\\ c&d
\end{smallmatrix} \right)
  \cdot
  \left(\begin{smallmatrix}
x\\ y
\end{smallmatrix} \right)
  +
\left(\begin{smallmatrix}
v\\ w
\end{smallmatrix} \right)$
then 
\[k=|\det\left(\begin{smallmatrix}
a&b\\ c&d
\end{smallmatrix} \right)|=|a\cdot d-b\cdot c|.\]

\end{thm}





\section*{Area method}
\addtocontents{toc}{Area method.}

In this section we will give examples of
slim proofs using area.
Note that these proofs are not truly elementary --- the price one pays to introduce the area function is high.

We start with the proof of the Pythagorean theorem.
In the Elements of Euclid, Pythagorean theorem was formulated as equality  \ref{eq:pyth+area} below
and the proof used a similar technique.

\parit{Proof.}
We need to show that if $a$ and $b$ are legs and $c$ is the hypotenuse 
of a right triangle  then
\[a^2+b^2=c^2.\]

Denote by $\solidtriangle$ the right solid triangle with legs $a$ and $b$
and  by $\solidsquare_{x}$ be the solid square 
with side $x$.

Let us construct two subdivisions of $\solidsquare_{a+b}$.

\begin{wrapfigure}{o}{45 mm}
\begin{lpic}[t(-3mm),b(-3mm),r(0mm),l(0mm)]{pics/pyth-area(1)}
\end{lpic}
\end{wrapfigure}

1. Subdivide $\solidsquare_{a+b}$ into two solid squares congruent to $\solidsquare_a$ and $\solidsquare_b$
and 4 solid triangles congruent to $\solidtriangle$,
see the left diagram.

2. Subdivide $\solidsquare_{a+b}$ into one solid square congruent to $\solidsquare_c$
and 4 solid right triangles congruent to $\solidtriangle$,
see the right diagram.

Applying Proposition \ref{prop:subdivision} few times,
we get.
\begin{align*}
\area\solidsquare_{a+b}
&=
\area\solidsquare_{a}+\area\solidsquare_{b}+4\cdot\area\solidtriangle=
\\
&=\area\solidsquare_{c}+4\cdot\area\solidtriangle.
\end{align*}
Therefore 
\[\area\solidsquare_{a}+\area\solidsquare_{b}=\area\solidsquare_{c}.\eqlbl{eq:pyth+area}\]

Since 
\[\area\solidsquare_x=x^2,\] 
the statement follows.\qeds

{
\begin{wrapfigure}{r}{18.5mm}
\begin{lpic}[t(-5mm),b(0mm),r(0mm),l(0mm)]{pics/pyth-area-2(1)}
\end{lpic}
\end{wrapfigure}

\begin{thm}{Exercise}\label{ex:pyth-2}
Build an other proof of Pythagorean theorem
based on the diagram. 

(In the notations above it shows a subdivision of $\solidsquare_c$ into $\solidsquare_{a-b}$ and 4 solid triangles congruent to $\solidtriangle$.)
\end{thm}

} 

\begin{thm}{Exercise}\label{ex:sum-3-dist}
Show that the sum of distances from a point to the sides of an equilateral triangle is the same for all points inside the triangle.
\end{thm}


The following claim is simple but very useful.

\begin{thm}{Claim}\label{clm:area-ratio}
Assume  that two triangles $\triangle ABC$
and $\triangle A'B'C'$ in the Euclidean plane 
have equal altitudes dropped from $A$ and $A'$ correspondingly.
Then
\[\frac{\solidtriangle A'B'C'}{\solidtriangle ABC}
=
\frac{B'C'}{BC}.\]

In particular, the same identity holds if $A=A'$ and the bases $[BC]$ and $[B'C']$ lie on one line.
\end{thm}

\parit{Proof.}
Let $h$ be the altitude.
By Theorem~\ref{thm:area-of-triangle},
\[\frac{\solidtriangle A'B'C'}{\solidtriangle ABC}
=
\frac{\frac12 \cdot h\cdot B'C'}{\frac12 \cdot h\cdot BC}
=
\frac{B'C'}{BC}.\]
\qedsf

\begin{wrapfigure}{r}{30 mm}
\begin{lpic}[t(-4mm),b(0mm),r(0mm),l(0mm)]{pics/area-bisector(1)}
\lbl[b]{6.5,20.5;$A$}
\lbl[lt]{27,0;$B$}
\lbl[rt]{1,0;$C$}
\lbl[t]{12,0;$D$}
\end{lpic}
\end{wrapfigure}

\parit{Lemma~\ref{lem:bisect-ratio} via Area method.}
We have to show that if $\triangle A B C$ is nondegenerate
and the bisector of $\angle BAC$ 
intersects $[BC]$ at the point $D$.
Then 
$$\frac{AB}{AC}=\frac{DB}{DC}.$$

Applying  Claim~\ref{clm:area-ratio}, we get
\[\frac{\area(\solidtriangle ABD)}{\area(\solidtriangle ACD)}
=\frac{BD}{CD}.\]

By Lemma~\ref{lem:angle-bisect-dist} the triangles $\triangle ABD$ and $\triangle ACD$ have equal altitudes from $D$.
Applying  Claim~\ref{clm:area-ratio} again, we get
\[\frac{\area(\solidtriangle ABD)}{\area(\solidtriangle ACD)}=\frac{AB}{AC}.\]
Hence the result follows.
\qeds

\begin{thm}{Exercise}\label{ex:area-medians}
Let $X$ lies inside a nondegenerate triangle $\triangle ABC$.
Show that $X$ lies on the median from $A$ if and only if 
\[\area(\solidtriangle ABX)=\area(\solidtriangle ACX).\]
\end{thm}

\begin{thm}{Exercise}\label{ex:area-medians-2} 
Build a proof of Theorem~\ref{thm:centroid} via area method based on the Exercise~\ref{ex:area-medians}.

Namely, show that medians of nondegenerate triangle intersect at one point and the point of their intersection  divides each median in the ration 1:2.
\end{thm}

\section*{Area in
the absolute planes and spheres}
\addtocontents{toc}{Area in absolute planes and spheres.}

The theorem Theorem \ref{thm:area} will hold in the absolute planes and spheres if the solid unit square $\mathcal{K}$
exchanged to a fixed nondegenerate polygonal set.
One has to make such change for good reason --- 
hyperbolic plane and sphere have no unit squares.

The set $\mathcal{K}$ in this case plays role of the unit measure for the area
and changing $\mathcal{K}$ will require conversion of area units.

\begin{wrapfigure}{o}{17 mm}
\begin{lpic}[t(-3 mm),b(0mm),r(0mm),l(0mm)]{pics/lambert-square(1)}
\lbl[rb]{1,14;$A$}
\lbl[rt]{1,1;$B$}
\lbl[lt]{13.5,.5;$C$}
\lbl[bl]{15,15;$D$}
\lbl[]{9,7;{\color{white} $\mathcal{K}_n$}}
\lbl[t]{8,1;$\tfrac1n$}
\lbl[r]{1.5,7;$\tfrac1n$}
\end{lpic}
\end{wrapfigure}

According to the standard convention, the set $\mathcal{K}$
is taken so that on small scales area behaves like area in the Euclidean plane.
Say if $\mathcal{K}_n$ denotes the solid quadrilateral $\solidsquare ABCD$ 
with right angles at $A$, $B$ and $C$ and $AB=BC=\tfrac1n$ then we may assume that
\[n^2\cdot\area \mathcal{K}_n\to 1 \ \ \text{as}\ \ n\to\infty.
\eqlbl{eq:lim-Kn}\]

This convention works equally well for spheres and absolute planes, including Euclidean plane.
In spherical geometry  equivalently we may assume that if $r$ is the radius of the sphere then 
the area of whole sphere is $4\cdot\pi\cdot r^2$.

Recall that {}\emph{defect of triangle} $\triangle ABC$ is defined as 
$$\defect(\triangle ABC)
\df 
\pi-|\measuredangle ABC|+|\measuredangle BCA|+|\measuredangle CAB|.$$
It turns out that any absolute plane and  sphere
there is a real number $k$
such that 
$$k\cdot\area(\triangle ABC)+\defect(\triangle ABC)=0
\eqlbl{eq:curv-defect}$$
for any triangle $\triangle ABC$.

This number $k$ is called \index{curvature}\emph{curvature};
$k=0$ for the Euclidean plane,
$k=-1$ for the h-plane and $k=1$ for the unit sphere
and $k=\tfrac1{r^2}$ for the sphere of radius $r$.
In particular it follows that any ideal triangle in h-plane has area $\pi$.

The identity \ref{eq:curv-defect} suggest an alternative way to introuduce area function which works on spheres and all absolute planes except the Euclidean plane.



\section*{Quadrable sets}
\addtocontents{toc}{Quadrable  sets.}

A set $\mathcal{S}$ 
in the plane is called \index{quadrable set}\emph{quadrable}
if for any $\epsilon>0$ there are two polygonal sets 
$\mathcal{P}$ and $\mathcal{Q}$
such that 
\[\mathcal{P}\subset\mathcal{S}\subset\mathcal{Q}
\ \ \ \ \text{and}\ \ \ \ 
\area\mathcal{Q}-\area\mathcal{P}<\epsilon.\]

If $\mathcal{S}$ is quadrable,
its area  can be defined 
as the (necessary unique) real number $\area\mathcal{S}$ 
such that the inequality
\[\area\mathcal{Q}\le \area\mathcal{S}\le \area\mathcal{P}
\]
holds if
$\mathcal{P}$ and $\mathcal{Q}$ are polygonal sets and $\mathcal{P}\subset\mathcal{S}\subset\mathcal{Q}$.

\begin{thm}{Exercise}\label{ex:circle-is-quadrable}
Let $\mathcal{D}$ be the unit disk;
that is $\mathcal{D}$ is a set which contains 
the unit circle $\Gamma$ and all the points inside $\Gamma$.

Show that $\mathcal{D}$ is a quadrable set.
\end{thm}

Since $\mathcal{D}$ is quadrable, 
the expression $\area\mathcal{D}$ makes sense.
The constant $\pi$ can be defined as 
$\pi=\area\mathcal{D}$.

\medskip

It turns out that the class of quadrable sets is the largest class for which 
the area can be defined in such a way that it satisfies all the conditions in Theorem~\ref{thm:area} incliding uniqueness.

There is a way to define area for all bounded sets
which satisfies all the conditions in the Theorem~\ref{thm:area} except uniqueness.
(A set in the plane is called \index{bounded set}\emph{bounded} if it lies inside a circle.)

In the hyperbolic plane and in the sphere
there is no similar construction.
Read about Hausdorff--Banach--Tarski Paradox
if you wonder why.


\addtocontents{toc}{\protect\end{quote}}                          

\chapter{Area}
\label{chap:area}

We will define area as a function that satisfies certain conditions (Section~\ref{sec:def(area)}).
The so-called {}\emph{Lebesgue measure} provides an example of such a function.
In particular, the existence of the Lebesgue measure implies the existence of an area function.
This construction is included in any textbook on real analysis.

Based solely on the existence, we develop the concept of area with no cheating.
We choose this approach because any rigorous introduction to area is either tedious or comes with cheating.
We do not want to cheat while also respecting your time. 
Sooner or later, you will need to learn about the Lebesgue measure if you haven't already.


\section{Solid triangles}

{

\begin{wrapfigure}{r}{27 mm}
\vskip-8mm
\centering
\includegraphics{mppics/pic-292}
\end{wrapfigure}

We say that a point $X$ lies \index{inside!a triangle}\emph{inside} a nondegenerate triangle $ABC$ if the following three conditions hold:

\begin{itemize}
\item $A$ and $X$ lie on the same side of the line~$(BC)$;
\item $B$ and $X$ lie on the same side of the line~$(CA)$;
\item $C$ and $X$ lie on the same side of the line~$(AB)$.
\end{itemize}

}

The set of all points inside $\triangle ABC$ 
and on its sides $[AB]$, $[BC]$, $[CA]$
will be called \index{triangle!solid triangle}\index{solid!triangle}\emph{solid triangle} $ABC$ and denoted by \index{25@$\solidtriangle$}$\solidtriangle ABC$.

\begin{thm}{Exercise}\label{ex:triangle-convex}
Show that any solid triangle is \index{convex set}\emph{convex};
that is, for any pair of points $X,Y\z\in\solidtriangle ABC$,
the line segment $[XY]$ lies in  $\solidtriangle ABC$.
\end{thm}



%\begin{thm}{Exercise}\label{ex:solid-triangle-sum}
%Show that  $X\in\solidtriangle ABC$ if and only if
%\[|\measuredangle AXB|+|\measuredangle BXC|+|\measuredangle CXA|=2\cdot\pi.\]
%\end{thm}


The notations $\triangle ABC$ and $\solidtriangle ABC$ look similar, 
and they also have closely related but different meanings, which it is better not to confuse.
Recall that $\triangle ABC$ is an ordered triple of distinct points
(see Section~\ref{sec:cong-triangles}),
while $\solidtriangle ABC$ is an infinite set of points.

In particular, $\solidtriangle ABC=\solidtriangle BAC$ for any triangle $ABC$.
Indeed, any point that belongs to the set $\solidtriangle ABC$ 
also belongs to the set $\solidtriangle BAC$
and the other way around.
On the other hand,
$\triangle ABC\ne\triangle BAC$ simply because the ordered triple of points $(A,B,C)$ is distinct from the ordered triple $(B,A,C)$.

Note that $\solidtriangle ABC\cong\solidtriangle BAC$ even if $\triangle ABC\ncong\triangle BAC$, where congruence of the sets $\solidtriangle ABC$ and $\solidtriangle BAC$ 
is understood the following way:

\begin{thm}{Definition}\label{def:cong-sets}
Two sets $\mathcal{S}$ and $\mathcal{T}$ in the plane  
are called \index{congruent!sets}\emph{congruent} 
(briefly \index{32@$\cong$}$\mathcal{S}\cong \mathcal{T}$)
if 
$\mathcal{T}=f(\mathcal{S})$ for some motion $f$ of the plane.
\end{thm}

If $\triangle ABC$ is not degenerate
and \[\solidtriangle ABC\cong \solidtriangle A'B'C',\]
then after relabeling the vertices of $\triangle ABC$ 
we will have 
\[\triangle ABC\cong \triangle A'B'C'.\]

Indeed it is sufficient to show that 
if $f$ is a motion that maps $\solidtriangle ABC$ to $\solidtriangle A'B'C'$,
then $f$ maps each vertex of $\triangle ABC$ to a vertex $\triangle A'B'C'$.
The latter follows from the characterization of vertices of solid triangles given in the following exercise:

\begin{thm}{Exercise}\label{ex:vertex}
Let $\triangle ABC$ be nondegenerate and $X\z\in \solidtriangle ABC$.
Show that $X$ is a vertex of $\triangle ABC$
if and only if there is a line $\ell$ that intersects $\solidtriangle ABC$
at the single point~$X$.
\end{thm}

\section{Polygonal sets}

An e\index{elementary set}\emph{elementary set} on the plane 
is a set of one of the following three types:
\begin{itemize}
 \item one-point set;
 \item segment;
 \item solid triangle.
\end{itemize}

\begin{wrapfigure}{r}{24 mm}
\vskip-12mm
\centering
\includegraphics{mppics/pic-294}
\end{wrapfigure}

A set in the plane is called \index{polygonal set}\emph{polygonal} if it can be presented as a union of a finite collection of elementary sets.

According to this definition, the empty set $\emptyset$
is a polygonal set.
Indeed, $\emptyset$ is a union of an empty collection of elementary sets.

A polygonal set is called \index{polygonal set!degenerate polygonal set}\index{degenerate!polygonal set}\emph{degenerate} if it can be presented as a union of a finite collection of one-point sets and segments.

If $X$ and $Y$ lie on opposite sides of the line $(AB)$,
then the polygonal set
$\solidtriangle AXB\cup \solidtriangle BYA$
is called \index{solid!quadrangle, parallelogram,\\ rectangle, square}\index{quadrangle!solid quadrangle}\emph{solid quadrangle} $AXBY$ and denoted by 
\index{27@$\solidsquare$}$\solidsquare AXBY$.
In particular, 
we can talk about \index{parallelogram!solid parallelogram}\emph{solid parallelograms}, \index{rectangle!solid rectangle}\emph{rectangles}, and \index{square!solid square}\emph{squares}.

\begin{wrapfigure}{o}{38 mm}
\centering
\includegraphics{mppics/pic-296}
\end{wrapfigure}

Typically a polygonal set admits many 
presentations as a union of a finite collection of elementary sets.
For example, if $\square AXBY$ is a parallelogram, then
\[\solidsquare AXBY=\solidtriangle AXB\cup \solidtriangle AYB=\solidtriangle XAY\cup \solidtriangle XBY.\]

\begin{thm}{Exercise}\label{ex:solid-square}
Show that a solid square is not degenerate.
\end{thm}

\begin{thm}{Exercise}\label{ex:poly-circ}
Show that a circle is not a polygonal set.
\end{thm}

\begin{thm}{Claim}\label{clm:poly-ring}
For any two polygonal sets $\mathcal{P}$ and $\mathcal{Q}$,
the union $\mathcal{P}\cup\mathcal{Q}$, 
as well as the intersection $\mathcal{P}\cap\mathcal{Q}$, 
are also polygonal sets.
\end{thm}

A class of sets that is closed with respect to union and intersection is called a {}\emph{ring of sets}.
The claim above, therefore, states that polygonal sets in the plane form a ring of sets.

\parit{Informal proof.}
Let us present $\mathcal{P}$ and $\mathcal{Q}$
as a union of a finite collection of elementary sets $\mathcal{P}_1,\dots,\mathcal{P}_k$ 
and $\mathcal{Q}_1,\dots,\mathcal{Q}_n$ respectively.

{

\begin{wrapfigure}{o}{22 mm}
\centering
\includegraphics{mppics/pic-298}
\end{wrapfigure}

Note that
\[\mathcal{P}\cup\mathcal{Q}
=
\mathcal{P}_1
\cup
\dots
\cup
\mathcal{P}_k
\cup
\mathcal{Q}_1
\cup
\dots
\cup
\mathcal{Q}_n.\]
Therefore, $\mathcal{P}\cup\mathcal{Q}$ is polygonal.

Note that $\mathcal{P}\cap \mathcal{Q}$ is the union of $\mathcal{P}_i\cap \mathcal{Q}_j$ for all $i$ an~$j$.
Therefore, to show that $\mathcal{P}\cap \mathcal{Q}$ is polygonal,
it is sufficient to show that $\mathcal{P}_i\cap \mathcal{Q}_j$ is polygonal for any $i$ and $j$.

The picture suggests a proof of the latter statement for solid triangles $\mathcal{P}_i$ and $\mathcal{Q}_j$.
The other cases are simpler; a formal proof can be built on Exercise~\ref{ex:triangle-convex}.
\qeds

\section{Definition of area}
\label{sec:def(area)}

\index{area}\emph{Area} is defined as a function $\mathcal{P}\mapsto \area\mathcal{P}$
that returns a nonnegative real number $\area\mathcal{P}$ for any polygonal set $\mathcal{P}$ and satisfies the following conditions:\label{page:area-def}
\textit{
\begin{enumerate}[(a)]
\item
$\area\mathcal{K}_1=1$
where  $\mathcal{K}_1$ is a solid unit square;
\item\label{area-3} the conditions
\begin{align*}
\mathcal{P}\cong\mathcal{Q}
\quad 
&\Rightarrow
\quad \area\mathcal{P}=\area\mathcal{Q};
\\
\mathcal{P}\subset\mathcal{Q}
\quad
&\Rightarrow
\quad 
\area\mathcal{P}\le\area\mathcal{Q};
\\
\area\mathcal{P}+\area\mathcal{Q}
&=
\area(\mathcal{P}\cup\mathcal{Q})+\area(\mathcal{P}\cap\mathcal{Q})
\end{align*}
hold 
for any two polygonal sets $\mathcal{P}$ and $\mathcal{Q}$.
\end{enumerate}
}

The first condition is called {}\emph{normalization}; essentially it says that a solid unit square is used as a unit to measure area.
The three conditions in \textit{(\ref{area-3})} are called {}\emph{invariance}, {}\emph{monotonicity}, and {}\emph{additivity}.

The Lebesgue measure is an example of an area function. 
Namely, if one takes the area of $\mathcal{P}$ to be its Lebesgue measure,
then the function $\mathcal{P}\mapsto\area\mathcal{P}$ satisfies the above conditions.

The construction of the Lebesgue measure can be found in any textbook on real analysis.
We do not discuss it here.

If the reader is not familiar with the Lebesgue measure, then the existence of area function can be taken as granted; 
it might be added to axioms altho it follows from the axioms \ref{def:birkhoff-axioms:0}--\ref{def:birkhoff-axioms:4}.

\section{Vanishing area and subdivisions}

\begin{thm}{Proposition}\label{prop:area-segment}
For any two points $A$ and $B$, we have
\[\area[AB]=0\quad\text{and}\quad\area\{A\}=0.\]
\end{thm}

\parit{Proof.}
Fix a line segment~$[AB]$.
Consider a sold square $\solidsquare ABCD$.

For any positive integer $n$,
there are $n$ disjoint segments $[A_1B_1],\z\dots,[A_nB_n]$ 
in $\solidsquare ABCD$,
such that each $[A_iB_i]$ is congruent to $[AB]$ in the sense of the Definition~\ref{def:cong-sets}.

\begin{wrapfigure}[10]{o}{37 mm}
\vskip-2mm
\centering
\includegraphics{mppics/pic-300}
\vskip0mm
\end{wrapfigure}

Applying the invariance, additivity, and monotonicity of the area function, 
we get
\begin{align*}
n\cdot \area[AB]
&=\area\left([A_1B_1]\cup\dots\cup[A_nB_n]\right)\le
\\
&\le \area(\solidsquare ABCD)              
\end{align*}
In other words,
\[\area[AB]\le \tfrac1n\cdot\area(\solidsquare ABCD)\] 
for any positive integer~$n$.
It follows that $\area[AB]\le 0$. 
Since area cannot be negative, we get
\[\area[AB]= 0.\]

For any one-point set $\{A\}$ 
we have that $\{A\}\subset [AB]$.
Therefore, 
\[0\le \area\{A\}\le \area[AB]=0.\]
Hence $\area\{A\}=0$.
\qeds

\begin{thm}{Corollary}\label{cor:degenerate}
Any degenerate polygonal set has vanishing area.
\end{thm}

\parit{Proof.}
Let $\mathcal P$ be a degenerate set,
say
\[\mathcal{P}=[A_1B_1]\cup\dots\cup[A_nB_n]\cup\{C_1,\dots,C_k\}.\]
Since area is nonnegative by definition, applying additivity several times, we get that
\begin{align*}
\area\mathcal{P}\le
& \area[A_1B_1]+\dots+\area[A_nB_n]+
\\
&+\area\{C_1\}+\dots+\area\{C_k\}.
\end{align*}
By Proposition~\ref{prop:area-segment}, the right-hand side vanishes.

On the other hand, 
$\area\mathcal{P}\ge 0$,
hence the result.
\qeds

We say that a polygonal set $\mathcal{P}$ is \index{subdivision of polygonal set}\emph{subdivided} 
into two polygonal sets $\mathcal{Q}_1,\dots,\mathcal{Q}_n$ 
if $\mathcal{P}=\mathcal{Q}_1\cup\dots\cup \mathcal{Q}_n$ 
and the intersection $\mathcal{Q}_i\cap\mathcal{Q}_j$ is degenerate for any pair $i$ and $j$.
(Recall that according to Claim~\ref{clm:poly-ring},
the intersections $\mathcal{Q}_i\cap\mathcal{Q}_j$ are polygonal.)

\begin{thm}{Proposition}\label{prop:subdivision}
Assume that a polygonal set $\mathcal{P}$ is subdivided into polygonal sets $\mathcal{Q}_1,\dots,\mathcal{Q}_n$.
Then 
\[\area\mathcal{P}=\area\mathcal{Q}_1+\dots+\area\mathcal{Q}_n.\]

\end{thm}

\begin{wrapfigure}{o}{26 mm}
\vskip-4mm
\centering
\includegraphics{mppics/pic-302}
\end{wrapfigure}

\parit{Proof.}
Assume $n=2$; by the additivity of area,
\[\area\mathcal{P}=\area\mathcal{Q}_1+\area\mathcal{Q}_2-\area(\mathcal{Q}_1\cap\mathcal{Q}_2).\]

Since $\mathcal{Q}_1\cap\mathcal{Q}_2$ is degenerate,
by Corollary~\ref{cor:degenerate},
\[\area(\mathcal{Q}_1\cap\mathcal{Q}_2)=0.\]

Applying this formula a few times we get the general case.
Indeed, if $\mathcal{P}$ is subdivided into $\mathcal{Q}_1,\dots,\mathcal{Q}_n$, then
\begin{align*}
\area\mathcal{P}&=\area\mathcal{Q}_1+\area(\mathcal{Q}_2\cup\dots\cup\mathcal{Q}_n)=
\\
&=\area\mathcal{Q}_1+\area \mathcal{Q}_2+\area( \mathcal{Q}_3\cup\dots\cup\mathcal{Q}_n)=
\\
&\ \ \ \vdots
\\
&=\area\mathcal{Q}_1+\area \mathcal{Q}_2+\dots+\area\mathcal{Q}_n.
\end{align*}
\qedsf

\parbf{Remark.}
Two polygonal sets $\mathcal{P}$ and $\mathcal{P}'$ are called \index{equidecomposable sets}\emph{equidecomposable} if they admit subdivisions into polygonal sets $\mathcal{Q}_1,\dots,\mathcal{Q}_n$ and $\mathcal{Q}'_1,\dots,\mathcal{Q}'_n$ such that 
$\mathcal{Q}_i\cong\mathcal{Q}'_i$ for each $i$.

According to the proposition, if $\mathcal{P}$ and $\mathcal{P}$ are equidecomposable, then $\area \mathcal{P}=\area\mathcal{P}'$.
A converse to this statement also holds;
namely, \textit{if two nondegenerate polygonal sets have equal area, then they are equidecomposable.}

The last statement was proved by William Wallace, Farkas Bolyai, and Paul Gerwien.
The analogous statement in three dimensions, known as {}\emph{Hilbert's third problem}, is false; it was proved by Max Dehn.



\section{Rectangles}

\begin{thm}{Theorem}\label{thm:area-rect}
The area of a rectangle is the product of its adjacent sides.
\end{thm}

\begin{thm}{Algebraic lemma}\label{lem:alg-area}
Assume that a function $s$ 
returns a nonnegative real number $s(a,b)$ 
for any pair of positive real numbers $(a,b)$ 
and it satisfies the following identities:
\[\begin{aligned}
s(1,1)&=1,
\\
s(a,b+c)&=s(a,b)+s(a,c),
\\
s(a+b,c)&=s(a,c)+s(b,c)
\end{aligned}
\eqlbl{3s}
\]
for any $a,b,c>0$.
Then 
\[s(a,b)=a\cdot b\] 
for any $a,b>0$.
\end{thm}

The proof is similar to the proof of Lemma~\ref{lem:R-auto}.

\parit{Proof.}
If $a>a'$ and $b>b'$ then 
\[s(a,b)\ge s(a',b').\eqlbl{s(a,b)>s(a',b')}\]
Indeed, since $s$ returns nonnegative numbers, we get that
\begin{align*}
s(a,b)&=s(a',b)+s(a-a',b)\ge s(a',b)=
\\
&= s(a',b')+s(a',b-b')\ge s(a',b').
\end{align*}

Applying the second and third identity in \ref{3s} a few times we get that
\begin{align*}
m\cdot s(a,b)&=s(a,m\cdot b)=
\\&=s(m\cdot a,b)
\end{align*}
for any positive integer $m$. Therefore
\begin{align*}
s(\tfrac kl,\tfrac mn)
&=k \cdot s(\tfrac 1l,\tfrac mn)=
\\
&=k\cdot m \cdot s(\tfrac 1l,\tfrac 1n)=
\\
&=k\cdot m\cdot \tfrac 1l\cdot s(1, \tfrac 1n)=
\\
&=k\cdot m\cdot \tfrac 1l\cdot \tfrac 1n\cdot s(1,1)=
\\
&=\tfrac kl\cdot\tfrac mn
\end{align*}
for any positive integers $k$, $l$, $m$, and $n$.
That is, the needed identity holds for any pair of rational numbers $a=\tfrac kl$ and $b=\tfrac mn$.

Arguing by contradiction, assume $s(a,b)\ne a\cdot b$ for a pair of positive real numbers $(a,b)$;
so, $s(a,b)> a\cdot b$ or $s(a,b)\z< a\cdot b$.

Recall that \index{63@$\lfloor x \rfloor$, $\lceil x \rceil$}$\lfloor x \rfloor$ and $\lceil x \rceil$  denote \index{floor and ceiling}\emph{floor} and \emph{ceiling} of $x$;
that is, $\lfloor x \rfloor$ is the greatest integer such that $\lfloor x \rfloor\le x$,
and $\lceil x \rceil$ is the least integer such that $\lceil x \rceil\ge x$.

\raggedcolumns\setlength{\multicolsep}{.5mm}
\setlength{\columnseprule}{1pt}
\begin{multicols}{2}
If $s(a,b)> a\cdot b$,
we can choose a positive integer $n$ such that
\[s(a,b)> (a+\tfrac1n)\cdot (b+\tfrac1n).\eqlbl{s(a,b)>}\]
Set $k=\lfloor a\cdot n \rfloor+1$ and $m=\lfloor b\cdot n \rfloor+1$;
equivalently, $k$ and $m$ are positive integers such that
\begin{align*}
a< \tfrac kn&\le a+\tfrac1n,
\\
b<\tfrac mn&\le b+\tfrac1n.
\end{align*}
By \ref{s(a,b)>s(a',b')}, we get that
\begin{align*}
s(a,b)&\le s(\tfrac kn,\tfrac mn)=
\\
&=\tfrac kn\cdot\tfrac mn\le
\\
&\le (a+\tfrac1n)\cdot(b+\tfrac1n),
\end{align*}
which contradicts \ref{s(a,b)>}.

\columnbreak

If $s(a,b)< a\cdot b$, choose a positive integer $n$ such that $a>\tfrac1n$, $b>\tfrac1n$, and
\[s(a,b)< (a-\tfrac1n)\cdot (b-\tfrac1n).\eqlbl{s(a,b)<}\]
Set $k=\lceil a\cdot n \rceil-1$ and $m=\lceil b\cdot n \rceil-1$; that is,
\begin{align*}
a> \tfrac kn&\ge a-\tfrac1n,
\\ 
b>\tfrac mn&\ge b-\tfrac1n.
\end{align*}
By \ref{s(a,b)>s(a',b')}, we get that
\begin{align*}
s(a,b)&\ge s(\tfrac kn,\tfrac mn)=
\\
&=\tfrac kn\cdot\tfrac mn\ge
\\
&\ge (a-\tfrac1n)\cdot(b-\tfrac1n),
\end{align*}
which contradicts \ref{s(a,b)<}.\qeds
\end{multicols}
\setlength{\columnseprule}{0pt}








\parit{Proof of Theorem~\ref{thm:area-rect}.}
Suppose that $\mathcal{R}_{a,b}$ denotes the solid rectangle with sides $a$ and~$b$.
Let 
\[s(a,b)=\area \mathcal{R}_{a,b}.\]

By the definition of area, 
$s(1,1)=\area(\mathcal{K})=1$.
That is, the first identity in the algebraic lemma holds.


\begin{wrapfigure}{o}{30 mm}
\vskip-0mm
\centering
\includegraphics{mppics/pic-304}
\end{wrapfigure}

The rectangle $\mathcal{R}_{a+b,c}$
can be subdivided into two rectangles $\mathcal{R}_{a,c}$
and~$\mathcal{R}_{b,c}$.
By Proposition~\ref{prop:subdivision}, 
\[
\area \mathcal{R}_{a+b,c}=\area \mathcal{R}_{a,c}+\area \mathcal{R}_{b,c}.
\]
That is, the second identity in the algebraic lemma holds.
The proof of the third identity is similar.

It remains to apply the algebraic lemma.
\qeds


\section{Parallelograms}

\begin{thm}{Proposition}\label{prop:area-parallelogram}
Let $\square ABCD$ be a parallelogram in the Euclidean plane.
Then 
\[\area(\solidsquare ABCD)=a\cdot h,\]
where $a=AB$, and $h$ is the distance between the lines $(AB)$ and~$(CD)$.
\end{thm}


{

\begin{wrapfigure}{r}{41 mm}
\vskip-0mm
\centering
\includegraphics{mppics/pic-306}
\end{wrapfigure}

\parit{Proof.}
Let $A'$ and $B'$ denote the footpoints of $A$ and $B$
on the line~$(CD)$.

Note that $ABB'A'$ is a rectangle with sides $a$ and $h$.
By Proposition~\ref{thm:area-rect},
\[\area( \solidsquare ABB'A')=h\cdot a.
\eqlbl{eq:ABBA}\]

}

Without loss of generality, we may assume that  $\solidsquare ABCA'$ 
contains $\solidsquare ABCD$ and $\solidsquare ABB'A'$.
In this case, $\solidsquare ABCA'$ admits two subdivisions: 
\[\solidsquare ABCA'=\solidsquare ABCD\cup\solidtriangle AA'D=\solidsquare ABB'A'\cup\solidtriangle BB'C.\]

By Proposition~\ref{prop:subdivision},
\[\begin{aligned}
\area( \solidsquare ABCD)&+\area(\solidtriangle AA'D)=
\\
&=
\area(\solidsquare ABB'A')+ \area (\solidtriangle BB'C).   
  \end{aligned}
\eqlbl{eq:area-sum}\]




Note that 
\[\triangle AA'D\cong \triangle BB'C.\eqlbl{eq:cong-area}\]
Indeed, since the quadrangles $ABB'A'$ and $ABCD$ are parallelograms, 
by Lemma~\ref{lem:parallelogram},
we have that $AA'=BB'$, $AD=BC$, and $DC=AB\z=A'B'$.
It follows that $A'D=B'C$.
Applying the SSS congruence condition, we get \ref{eq:cong-area}.


In particular,
\[\area(\solidtriangle BB'C)=\area (\solidtriangle AA'D).
\eqlbl{eq:area-trinagles}\]

Subtracting \ref{eq:area-trinagles} from \ref{eq:area-sum},
we get that
\[\area (\solidsquare ABCD)=\area(\solidsquare ABB'D).\]
It remains to apply \ref{eq:ABBA}.
\qeds

{

\begin{wrapfigure}{r}{26 mm}
\vskip-5mm
\centering
\includegraphics{mppics/pic-308}
\end{wrapfigure}

\begin{thm}{Exercise}\label{ex:two-parallelograms}
Assume $\square ABCD$ and $\square AB'C'D'$ are two parallelograms such that $B'\in[BC]$ and $D\z\in [C'D']$.
Show that
\[\area(\solidsquare ABCD)=\area(\solidsquare AB'C'D').\]

\end{thm}

}

\section{Triangles}


\begin{thm}{Theorem}\label{thm:area-of-triangle}
Let $h_A$ be the altitude from $A$
in  $\triangle ABC$ and $a=BC$.
Then 
\[\area(\solidtriangle ABC)=\tfrac12\cdot a\cdot h_A.\]
\end{thm}

\parit{Proof.}
Draw the line $m$ thru $A$ that is parallel to $(BC)$
and line $n$ thru $C$ parallel to~$(AB)$.
Note that the lines $m$ and $n$ are not parallel;
denote by $D$ their point of intersection.
By construction, $\square ABCD$ is a parallelogram.

Note that $\solidsquare ABCD$ admits a subdivision into $\solidtriangle ABC$ and $\solidtriangle CDA$.
Therefore, 
\[\area(\solidsquare ABCD)
=
\area(\solidtriangle ABC)
+
\area(\solidtriangle CDA).\]

\begin{wrapfigure}{o}{28 mm}
\centering
\includegraphics{mppics/pic-310}
\end{wrapfigure}

Since $\square ABCD$ is a parallelogram,  Lemma~\ref{lem:parallelogram} implies that
\[AB=CD
\quad
\text{and}
\quad
BC=DA.\]
Therefore, by the SSS congruence condition, we have
$\triangle ABC\z\cong\triangle CDA$.
In particular
\[\area(\solidtriangle ABC)
=
\area(\solidtriangle CDA).\]

From above and Proposition~\ref{prop:area-parallelogram}, we get that
\begin{align*}
\area(\solidtriangle ABC)
&=\tfrac12\cdot\area(\solidsquare ABCD)=
\\
&=\tfrac12\cdot h_A\cdot a.
\end{align*}
\qedsf

\begin{thm}{Exercise}\label{ex:three-trig}
Let $h_A$, $h_B$, and $h_C$ denote the altitudes of $\triangle ABC$ from vertices $A$, $B$, and $C$ respectively.
Note that from Theorem~\ref{thm:area-of-triangle},
it follows that
\[h_A\cdot BC=h_B\cdot CA=h_C\cdot AB.\]

Give a proof of this statement without using Theorem~\ref{thm:area-of-triangle}.
\end{thm}

\begin{thm}{Exercise}\label{ex:half-parallelogram}
Assume $M$ lies inside the parallelogram $ABCD$;
that is, $M$ belongs to the solid parallelogram $\solidsquare ABCD$ but does not lie on its sides.
Show that
\[\area(\solidtriangle ABM)+\area(\solidtriangle CDM)
=\tfrac12\cdot \area(\solidsquare ABCD).\]
\end{thm}


\begin{thm}{Exercise}\label{ex:area-diag}
Assume that diagonals 
of a nondegenerate quadrangle $ABCD$ 
intersect at point $M$.
Show that 
\[\area(\solidtriangle ABM)\cdot\area(\solidtriangle CDM)
=
\area(\solidtriangle BCM)\cdot\area(\solidtriangle DAM).\]
 
\end{thm}

\begin{thm}{Exercise}\label{ex:area-inradius}
Let $r$ be the inradius of $\triangle ABC$
and $p$ be its {}\emph{semiperimeter}; 
that is, $p=\tfrac12\cdot(AB+BC+CA)$.
Show that
\[\area(\solidtriangle ABC)=p\cdot r.\]

\end{thm}



\begin{thm}{Exercise}\label{ex:subdivision}
Show that any polygonal set admits a subdivision into a finite collection of solid triangles and a degenerate set.
Conclude that for any polygonal set, its area is uniquely defined.
\end{thm}

\section{Area method}

In this section, we give examples of proofs using the properties of the area function.
These proofs are not truly elementary --- the price for the existence of the area function is high.

{

\begin{wrapfigure}{r}{20 mm}
\vskip-2mm
\centering
\includegraphics{mppics/pic-312}
\end{wrapfigure}

We start with the proof of the \index{Pythagorean theorem}Pythagorean theorem.
In the Elements of Euclid, the Pythagorean theorem was formulated as equality  \ref{eq:pyth+area} below,
and the proof used a similar technique.

\parit{Proof.}
We need to show that 
\[a^2+b^2=c^2,\]
where $a$ and $b$ are legs and $c$ is the hypotenuse 
of a right triangle.

Let $\mathcal{K}_{x}$ be the solid square with side~$x$.
Denote by $\mathcal{T}$ the right solid triangle with legs $a$ and $b$.

}

Let us construct two subdivisions of $\mathcal{K}_{a+b}$:
\begin{enumerate}[1.]
\item Subdivide $\mathcal{K}_{a+b}$ into two solid squares congruent to $\mathcal{K}_a$ and $\mathcal{K}_b$
and four solid triangles congruent to $\mathcal{T}$;
see the first picture.

\item Subdivide $\mathcal{K}_{a+b}$ into one solid square congruent to $\mathcal{K}_c$
and four solid right triangles congruent to $\mathcal{T}$;
see the second picture.

\end{enumerate}

Applying Proposition~\ref{prop:subdivision} a few times,
we get that
\begin{align*}
\area\mathcal{K}_{a+b}
&=
\area\mathcal{K}_{a}+\area\mathcal{K}_{b}+4\cdot\area\mathcal{T}=
\\
&=\area\mathcal{K}_{c}+4\cdot\area\mathcal{T}.
\end{align*}
Therefore, 
\[\area\mathcal{K}_{a}+\area\mathcal{K}_{b}=\area\mathcal{K}_{c}.\eqlbl{eq:pyth+area}\]
By Theorem~\ref{thm:area-rect}, we know that 
\[\area\mathcal{K}_x=x^2,\] 
for any $x>0$. 
Hence the statement follows.\qeds

{

\begin{wrapfigure}{r}{19mm}
\vskip-0mm
\centering
\includegraphics{mppics/pic-314}
\end{wrapfigure}

\begin{thm}{Exercise}\label{ex:pyth-2}
Build another proof of the Pythagorean theorem
based on the picture.

(In the notations above it shows a subdivision of $\mathcal{K}_c$ into $\mathcal{K}_{a-b}$ and four copies of~$\mathcal{T}$ if $a>b$.)
\end{thm}

} 

\begin{thm}{Exercise}\label{ex:sum-3-dist}
Show that the sum of distances from a point to the sides of an equilateral triangle is the same for all points inside the triangle.
\end{thm}

\begin{thm}{Claim}\label{clm:area-ratio}
Assume  that two triangles $ABC$ and $A'B'C'$ in the Euclidean plane 
have equal altitudes dropped from $A$ and $A'$ respectively.
Then
\[\frac{\area(\solidtriangle A'B'C')}{\area(\solidtriangle ABC)}
=
\frac{B'C'}{BC}.\]

In particular, the same identity holds if $A=A'$ and the bases $[BC]$ and $[B'C']$ lie on one line.
\end{thm}

\parit{Proof.}
Let $h$ be the altitude.
By Theorem~\ref{thm:area-of-triangle},
\[\frac{\area(\solidtriangle A'B'C')}{\area(\solidtriangle ABC)}
=
\frac{\frac12 \cdot h\cdot B'C'}{\frac12 \cdot h\cdot BC}
=
\frac{B'C'}{BC}.\]
\qedsf

Now let us show how to use this claim to prove Lemma~\ref{lem:bisect-ratio}.
First, let us recall its statement:

\begin{thm*}{Lemma}
If $\triangle A B C$ is nondegenerate and its angle bisector at $A$ intersects $[BC]$ at point~$D$, then 
$$\frac{AB}{AC}=\frac{DB}{DC}.$$
\end{thm*}

\begin{wrapfigure}{r}{30 mm}
\vskip-0mm
\centering
\includegraphics{mppics/pic-316}
\end{wrapfigure}

\parit{Proof.}
Applying  Claim~\ref{clm:area-ratio}, we get that
\[\frac{\area(\solidtriangle ABD)}{\area(\solidtriangle ACD)}
=\frac{BD}{CD}.\]

By Proposition~\ref{prop:angle-bisect-dist} the triangles $ABD$ and $ACD$ have equal altitudes from~$D$.
Applying  Claim~\ref{clm:area-ratio} again, we get that
\[\frac{\area(\solidtriangle ABD)}{\area(\solidtriangle ACD)}=\frac{AB}{AC}.\]
Hence the result follows.
\qeds

The second statement in the following exercise is a partial case of \index{Ceva's theorem}Ceva's theorem; see \ref{thm:ceva-affine}.



\begin{thm}{Exercise}\label{ex:ceva}
Let $ABC$ be a nondegenerate triangle.
Assume $A'$ lies between $B$ and $C$,
point $B'$ lies between $C$ and $A$,
point $C'$ lies between $A$ and $B$.
Suppose that line segments $[AA']$, $[BB']$, and $[CC']$ meet at a point $X$.
Show that 

\vskip-3mm

{

\begin{wrapfigure}[6]{r}{40 mm}
\vskip4mm
\centering
\includegraphics{mppics/pic-318}
\end{wrapfigure}

\begin{align*}
\frac{\area(\solidtriangle ABX)}{\area(\solidtriangle BCX)}&=\frac{AB'}{B'C},
\\
\frac{\area(\solidtriangle BCX)}{\area(\solidtriangle CAX)}&=\frac{BC'}{C'A},
\\
\frac{\area(\solidtriangle CAX)}{\area(\solidtriangle ABX)}&=\frac{CA'}{A'B} .
\end{align*}

}

Conclude that 
\[\frac{AB'\cdot CA'\cdot BC'}{B'C\cdot A'B\cdot C'A}=1.\]
\end{thm}

\begin{wrapfigure}[5]{r}{40 mm}
\vskip-6mm
\centering
\includegraphics{mppics/pic-319}
\end{wrapfigure}

\begin{thm}{Exercise}\label{ex:cross-ratio-area}
Suppose that points $L_1$, $L_2$, $L_3$, $L_4$ lie on a line $\ell$ 
and points $M_1$, $M_2$, $M_3$, $M_4$ lie on a line $m$. 
Assume that the lines $(L_1M_1)$, $(L_2M_2)$, $(L_3M_3)$, and $(L_4M_4)$ pass thru point $O$ that does not lie on $\ell$ nor~$m$.

\begin{enumerate}[(a)]
 \item\label{ex:cross-ratio-area:a} Apply Claim~\ref{clm:area-ratio} to show that
 \[\frac{\area\solidtriangle OL_iL_j}{\area\solidtriangle OM_iM_j}=\frac{OL_i\cdot OL_j}{OM_i\cdot OM_j}\]
 for any $i\ne j$.
 \item\label{ex:cross-ratio-area:b} Use \textit{(\ref{ex:cross-ratio-area:a})} to prove that 
 \[\frac{L_1L_2\cdot L_3L_4}{L_2L_3\cdot L_4L_1}=\frac{M_1M_2\cdot M_3M_4}{M_2M_3\cdot M_4M_1};\]
 that is, the quadruples $(L_1, L_2, L_3, L_4)$ and $(M_1, M_2, M_3, M_4)$ have the same cross-ratio.
 
\end{enumerate}

\end{thm}


\section{Neutral planes and spheres}

Area can be defined in the neutral planes and spheres.
In this definition,
the solid unit square $\mathcal{K}_1$ has to be  
exchanged for a fixed nondegenerate polygonal set $\mathcal{U}$.
Such a change is necessary for a good reason --- 
hyperbolic plane and sphere have no squares.

In this case, the set $\mathcal{U}$ serves as the unit measure for area;
changing $\mathcal{U}$ would require a conversion of area units.

\begin{wrapfigure}{o}{26 mm}
\vskip-0mm
\centering
\includegraphics{mppics/pic-320}
\end{wrapfigure}

According to the standard convention, the set $\mathcal{U}$
is chosen so that on small scales the area behaves like in the Euclidean plane.
For example, if $\mathcal{K}_a$ denotes the solid quadrangle $\solidsquare ABCD$ 
with right angles at $A$, $B$, and $C$ such that  $AB=BC=a$, 
then we may assume that
\[\tfrac{1}{a^2}\cdot\area \mathcal{K}_a\to 1
\quad
\text{as}
\quad 
a\to0.\]

This convention works equally well for spheres and neutral planes, including the Euclidean plane.
In spherical geometry  equivalently we may assume that if $r$ is the radius of the sphere, 
then the area of the whole sphere is $4\cdot\pi\cdot r^2$.

Recall that the \index{defect}\emph{defect of triangle} $\triangle ABC$ is defined as 
$$\defect(\triangle ABC)
\df 
\pi-|\measuredangle ABC|-|\measuredangle BCA|-|\measuredangle CAB|.$$
It turns out that for any neutral plane or sphere,
there is a real number $k$
such that 
$$k\cdot\area(\solidtriangle ABC)+\defect(\triangle ABC)=0
\eqlbl{eq:curv-defect}$$
for any $\triangle ABC$.

This number $k$ is called \index{curvature}\emph{curvature};
$k=0$ for the Euclidean plane,
$k=-1$ for the h-plane, 
$k=1$ for the unit sphere,
and $k=\tfrac1{r^2}$ for the sphere of radius~$r$.

Since the angles of an ideal triangle vanish, any ideal triangle in the h-plane has area~$\pi$.
Similarly, in the unit sphere, the area of an equilateral triangle with right angles has an area of $\tfrac\pi2$;
since the whole sphere can be subdivided into eight such triangles, we get that the area of the unit sphere is $4\cdot\pi$.

The identity \ref{eq:curv-defect} can be used as an alternative way to introduce the area function; it works on spheres and all neutral planes, except for the Euclidean plane.

\section{Quadrable sets}

A set $\mathcal{S}$ 
in the plane is called \index{quadrable set}\emph{quadrable}
if, for any $\epsilon>0$, there are two polygonal sets 
$\mathcal{P}$ and $\mathcal{Q}$
such that 
\[\mathcal{P}
\subset
\mathcal{S}\subset\mathcal{Q}
\quad
\text{and}
\quad
\area\mathcal{Q}-\area\mathcal{P}
<
\epsilon.\]

If $\mathcal{S}$ is quadrable,
its area  can be defined 
as the necessarily unique real number $s=\area\mathcal{S}$
such that the inequality
\[\area\mathcal{Q}\le s\le \area\mathcal{P}
\]
holds for any polygonal sets $\mathcal{P}$ and $\mathcal{Q}$ such that $\mathcal{P}\subset\mathcal{S}\subset\mathcal{Q}$.

\begin{thm}{Exercise}\label{ex:circle-is-quadrable}
Let $\mathcal{D}$ be a \emph{unit disc};
that is, $\mathcal{D}$ is a set that contains 
the unit circle $\Gamma$ and all the points inside~$\Gamma$.

Show that $\mathcal{D}$ is a quadrable set.
\end{thm}

Since $\mathcal{D}$ is quadrable, the expression $\area\mathcal{D}$ makes sense and the constant $\pi$ can be defined as $\pi\df\area\mathcal{D}$.

\medskip

It turns out that the class of quadrable sets is the largest class for which 
the area function satisfying the conditions in Section~\ref{sec:def(area)} is \textit{uniquely} defined.

If you do not require uniqueness, then there are ways to extend the area function to all bounded sets.
(A set in the plane is called \index{bounded set}\emph{bounded} if it lies inside a circle.)
On the sphere and hyperbolic plane, 
there is no similar construction.
If you wonder why,
read about the \index{doubling the ball}\emph{doubling the ball} --- a paradox of Felix Hausdorff, Stefan Banach, and Alfred Tarski.                         

\chapter{Spherical geometry}
\label{chap:sphere}
\addtocontents{toc}{\protect\begin{quote}}

Spherical geometry studies the surface of a unit sphere.
This geometry has applications in cartography, navigation, and astronomy.

The spherical geometry is a close relative of the Euclidean and hyperbolic geometries.
Most of the theorems of hyperbolic geometry have spherical analogs,
but spherical geometry is easier to visualize. 

\section*{Euclidean space}
\addtocontents{toc}{Euclidean  space.}

Recall that Euclidean space is the set $\mathbb{R}^3$ of all triples $(x,y,z)$ of real numbers
such that the distance between a pair of points
$A\z=(x_A,y_A,z_A)$ and $B=(x_B,y_B,z_B)$
is defined by the following formula:
$$AB
\df
\sqrt{(x_A-x_B)^2+(y_A-y_B)^2+(z_A-z_B)^2}.$$

The planes in the space are defined as the set of solutions of equation 
$$a\cdot x+b\cdot y+c\cdot z+d=0$$ 
for real numbers $a$, $b$, $c$, and $d$ such that at least one of the numbers $a$, $b$ or $c$ is not zero.
Any plane in the Euclidean space is isometric to the Euclidean plane.

A sphere in the space is the direct analog of circle in the plane.
Formally, \index{sphere}\emph{sphere} with center $O$ and radius $r$ is the set of points in the space that lie on the distance $r$ from~$O$.

Let $A$ and $B$ be two points on the unit sphere centered at~$O$.
The \index{spherical distance}\emph{spherical distance} from $A$ to $B$
(briefly $AB_s$)
 is defined as  $|\measuredangle AOB|$. 

In spherical geometry, the role of lines play the \index{great circle}\emph{great circles};
that is, the intersection of the sphere with a plane passing thru~$O$.

Note that the great circles do not form lines in the sense of Definition~\ref{def:line}.
Also, any two distinct great circles intersect at two antipodal points.
In particular, the sphere does not satisfy the axioms of the neutral plane.







\section*{Pythagorean theorem}
\addtocontents{toc}{Pythagorean theorem.}

Here is an analog of the Pythagorean theorems (\ref{thm:pyth} and \ref{thm:pyth-h-poincare}) in spherical geometry.

\begin{thm}{Spherical Pythagorean Theorem}\label{thm:s-pyth}
Let $\triangle_sABC$ be a spherical triangle with a right angle at~$C$.
Set $a=BC_s$, $b=CA_s$, and $c=AB_s$.
Then
$$\cos c=\cos a\cdot\cos b.$$

\end{thm}

In the proof, we will use the notion of the scalar product which we are about to discuss.

Let $v_A\z=(x_A,y_A,z_A)$ and $v_B=(x_B,y_B,z_B)$ denote the position vectors of points $A$ and $B$.
The \index{scalar product}\emph{scalar product} of the two vectors $v_A$ and $v_B$ in $\mathbb{R}^3$
is defined as 
$$\langle v_A,v_B\rangle
\df
x_A\cdot x_B+y_A\cdot y_B+z_A\cdot z_B.\eqlbl{eq:scal-def}$$

Assume both vectors $v_A$ and $v_B$ are nonzero;
let $\phi$ denotes the angle measure between them.
Then the scalar product can be expressed the following way:
$$\langle v_A,v_B\rangle=|v_A|\cdot|v_B|\cdot\cos\phi,
\eqlbl{eq:scal-angle}$$
where 
\begin{align*}
|v_A|&=\sqrt{x_A^2+y_A^2+z_A^2},
&
|v_B|&=\sqrt{x_B^2+y_B^2+z_B^2}.
\end{align*}

Now, assume that the points $A$ and $B$ 
lie on the unit sphere $\Sigma$ in $\mathbb{R}^3$ centered at the origin.
In this case $|v_A|=|v_B|=1$.
By \ref{eq:scal-angle} we get that
$$\cos AB_s=\langle v_A,v_B\rangle.
\eqlbl{eq:scalar-s-dist}$$

\parit{Proof of the spherical Pythagorean Theorem.}
Since the angle at $C$ is right,
we can choose the coordinates in $\mathbb{R}^3$ so that 
$v_C\z=(0,0,1)$, $v_A$ lies in the $xz$-plane, so $v_A\z=(x_A,0,z_A)$,
and $v_B$ lies in $yz$-plane, so $v_B=(0,y_B,z_B)$.

Applying, \ref{eq:scalar-s-dist},
we get that
\begin{align*}
z_A&=\langle v_C,v_A\rangle
=\cos b,
\\
z_B&=\langle v_C,v_B\rangle
=\cos a.
\end{align*}

{

\begin{wrapfigure}{r}{43mm}
\centering
\includegraphics{mppics/pic-248}
\end{wrapfigure}

Applying, \ref{eq:scal-def} and \ref{eq:scalar-s-dist}, we get that
\begin{align*}
\cos c &=\langle v_A,v_B\rangle=
\\
&=x_A\cdot 0+0\cdot y_B+z_A\cdot z_B=
\\
&=\cos b\cdot\cos a.
\end{align*}
\qedsf

\begin{thm}{Exercise}\label{ex:2(pi/4)=pi/3}
Show that 
if $\triangle_sABC$ is a spherical triangle with a right angle at $C$,
and $AC_s=BC_s=\tfrac\pi4$, then $AB_s=\tfrac\pi3$.
\end{thm}

}

\section*{Inversion of the space}
\addtocontents{toc}{Inversion of the space.}

The inversion in a sphere is defined the same way as we define the inversion in a circle.

Formally, let $\Sigma$ be the sphere with the center $O$ and radius~$r$.
The \index{inversion!inversion in a sphere}\emph{inversion} in $\Sigma$ of a point $P$ is the point $P'\in[OP)$ such that
$$OP\cdot OP'=r^2.$$
In this case, the sphere $\Sigma$  will be called the 
\index{inversion!sphere of inversion}\emph{sphere of inversion} 
and its center is called the \index{inversion!center of inversion}\emph{center of inversion}.

We also add $\infty$ to the space and assume that the center of inversion is mapped to $\infty$ and the other way around. 
The space $\mathbb{R}^3$ with the point $\infty$ will be called \index{inversive space}\emph{inversive space}.

The inversion of the space 
has many properties 
of the inversion of the plane.
Most important for us are the analogs of theorems \ref{lem:inverse-4-angle}, \ref{thm:inverse-cline}, \ref{thm:angle-inversion} which can be summarized as follows:

\begin{thm}{Theorem}\label{thm:inversion-3d}
The inversion in the sphere has the following properties:
\begin{enumerate}[(a)]
\item\label{thm:inversion-3d:a} Inversion maps a sphere or a plane into a sphere or a plane.
\item\label{thm:inversion-3d:b} Inversion maps a circle or a line into a circle or a line. 
\item\label{thm:inversion-3d:cross-ratio} Inversion preserves the cross-ratio;
that is, if $A'$, $B'$, $C'$, and $D'$ are the inverses of the points $A$, $B$, $C$ and $D$ respectively,
then
$$\frac{AB\cdot CD}{BC\cdot DA}= \frac{A'B'\cdot C'D'}{B'C'\cdot D'A'}.$$
\item Inversion maps arcs into arcs.
\item\label{thm:inversion-3d:angle} Inversion preserves the absolute value of the angle
measure between tangent half-lines to the arcs.
\end{enumerate}
\end{thm}


We do not present the proofs here, but
they nearly repeat the corresponding proofs in plane geometry.
To prove \textit{(\ref{thm:inversion-3d:a})}, you will need in addition the following lemma;
its proof is left to the reader.

\begin{thm}{Lemma}
Let $\Sigma$ be a subset of the Euclidean space
that contains at least two points.
Fix a point $O$ in the space.

Then $\Sigma$ is 
a sphere 
if and only if
for any plane $\Pi$ passing thru $O$,
the intersection $\Pi\cap \Sigma$ is either empty set,
one point set or a circle.
\end{thm}  

The following observation helps to reduce part~\textit{(\ref{thm:inversion-3d:b})} to part~\textit{(\ref{thm:inversion-3d:a})}.

\begin{thm}{Observation}
Any circle in the space is an intersection of two spheres.
\end{thm}

{

\begin{wrapfigure}{o}{25mm}
\centering
\includegraphics{mppics/pic-250}
\end{wrapfigure}

Let us define a \index{circular cone}\emph{circular cone} as a set formed by line segments from a fixed point, called the \index{tip of cone}\emph{tip} of the cone, to all the points on a fixed circle, called the \index{base of cone}\emph{base} of the cone;
we always assume that the base does not lie in the same plane as the tip.
We say that the cone is \index{right circular cone}\emph{right} 
if the center of the base circle is the foot point of the tip on the base plane;
otherwise we call it \index{oblique circular cone}\emph{oblique}.

}

\begin{thm}{Exercise}\label{ex:cone}
Let $K$ be an oblique circular cone. Show that there is a plane $\Pi$ that is not parallel to the base plane of $K$ such that the intersection $\Pi\cap K$ is a circle.
\end{thm}


\section*{Stereographic projection}
\addtocontents{toc}{Stereographic projection.}

{

\begin{wrapfigure}{r}{48mm}
\vskip-10mm
\centering
\includegraphics{mppics/pic-252}
\caption*{The plane thru $P$, $O$, and~$S$.}
\end{wrapfigure}

Consider the unit sphere $\Sigma$ 
centered at the origin $(0,0,0)$.
This sphere can be described by the equation $x^2+y^2+z^2\z=1$. 

Let $\Pi$ denotes the $xy$-plane;
it is defined by the equation $z = 0$.
Clearly, $\Pi$
runs thru the center of~$\Sigma$.

Let $N = (0, 0, 1)$ and $S\z=(0, 0, -1)$ denote the ``north'' and ``south'' poles of $\Sigma$;
these are the points on the sphere that have extremal distances to~$\Pi$.
Let $\Omega$ denotes the ``equator'' of $\Sigma$;
it is the intersection $\Sigma\cap\Pi$.

}

For any point $P\ne S$ on $\Sigma$,
consider the line $(SP)$ in the space. 
This line intersects $\Pi$ in exactly one point, denoted by~$P'$. 
Set $S'=\infty$.


The map $\xi_s\: P\mapsto P'$ is called the \index{stereographic projection}\emph{stereographic projection from $\Sigma$ to $\Pi$ with respect to the south pole}.
The inverse of this map $\xi^{-1}_s\: P'\z\mapsto P$ is called the {}\emph{stereographic projection from $\Pi$ to $\Sigma$ with respect to the south pole}.

The same way, one can define the
{}\emph{stereographic projections $\xi_n$ and $\xi^{-1}_n$ with respect to the north pole}~$N$.

Note that $P=P'$ if and only if $P\in\Omega$.


Note that if $\Sigma$ and $\Pi$ are as above,
then the composition of the stereographic projections 
$\xi_s: \Sigma\to\Pi$ and  $\xi^{-1}_s: \Pi\to\Sigma$
are the restrictions to $\Sigma$ and $\Pi$ respectively of the inversion in the sphere $\Upsilon$ with the center $S$ and radius $\sqrt{2}$.


From above and Theorem~\ref{thm:inversion-3d},
it follows that the stereographic projection preserves 
the angles between arcs;
more precisely {}\emph{the absolute value of the angle measure} between arcs on the sphere.

This makes it particularly useful in cartography.
A map of a big region of earth cannot be done in a constant scale,
but using a stereographic projection, one can keep the angles between roads the same as on earth.

In the following exercises,
we assume that $\Sigma$, $\Pi$, $\Upsilon$, $\Omega$, $O$, $S$, and $N$ are as above.
  
\begin{thm}{Exercise}\label{ex:two-stereographics}
Show that $\xi_n \circ \xi^{-1}_s$, the composition of stereographic projections 
from $\Pi$ to $\Sigma$ from  $S$, and
from $\Sigma$ to $\Pi$ from  $N$ is 
the inverse of the plane $\Pi$ in~$\Omega$.
\end{thm}

\begin{thm}{Exercise}\label{ex:great-circ}
Show that  a stereographic projection $\Sigma\to\Pi$
sends the great circles to plane circlines that intersect $\Omega$ at opposite points.
\end{thm}

The following exercise is analogous to Lemma~\ref{lem:conformal}.

\begin{thm}{Exercise}\label{ex:conform-sphere}
Fix a point $P\in \Pi$  and let $Q$ be another point in~$\Pi$.
Let $P'$ and $Q'$ denote their stereographic projections to~$\Sigma$.
Set $x=PQ$ and $y=P'Q'_s$.
Show that
$$\lim_{x\to 0}\, \frac{y}{x}=\frac{2}{1+OP^2}.$$
\end{thm}



\section*{Central projection}
\addtocontents{toc}{Central projection.}

The central projection is analogous to the projective model of hyperbolic plane which is discussed in Chapter~\ref{chap:klein}.

Let $\Sigma$ be the unit sphere centered at the origin which will be denoted by~$O$.
Let $\Pi^+$ denotes the plane defined by the equation $z=1$.
This plane is parallel to the $xy$-plane and it passes thru 
the north pole $N\z=(0,0,1)$ of~$\Sigma$.

{

\begin{wrapfigure}{r}{40mm}
\centering
\includegraphics{mppics/pic-254}
\end{wrapfigure}

Recall that the northern hemisphere of $\Sigma$,
is the subset of points $(x,y,z)\z\in \Sigma$ such that $z>0$.
The northern  hemisphere will be denoted by~$\Sigma^+$.

Given a point $P\in \Sigma^+$, consider the half-line $[OP)$. 
Let $P'$ denotes the intersection of $[OP)$ and~$\Pi^+$.
Note that 
if $P=(x,y,z)$, then $P'=(\tfrac xz,\tfrac yz,1)$.
It follows that $P\leftrightarrow P'$ is a bijection between $\Sigma^+$ and~$\Pi^+$.

}

The described bijection $\Sigma^+\leftrightarrow \Pi^+$ is called the \index{central projection}\emph{central projection} of 
the hemisphere~$\Sigma^+$.

Note that the central projection sends the intersections of the great circles with $\Sigma^+$ to the lines in~$\Pi^+$.
The latter follows since the great circles are intersections of $\Sigma$ with planes passing thru the origin
as well as the lines in $\Pi^+$ are the intersection of $\Pi^+$ with these planes.

The following exercise 
is analogous to Exercise~\ref{ex:h-median}
in hyperbolic geometry.

\begin{thm}{Exercise}\label{ex:s-medians}
Let $\triangle_sABC$ be a nondegenerate spherical triangle.
Assume that the plane $\Pi^+$ is parallel to the plane passing thru $A$, $B$, and~$C$.
Let $A'$, $B'$, and $C'$ denote the central projections of $A$, $B$ and~$C$.
\begin{enumerate}[(a)]
\item\label{ex:s-medians:a} Show that the midpoints of $[A'B']$, $[B'C']$, and $[C'A']$
are central projections of the midpoints of $[AB]_s$, $[BC]_s$, and $[CA]_s$ respectively.
\item\label{ex:s-medians:b} Use part (\ref{ex:s-medians:a}) to show that the medians of a spherical triangle intersect at one point.
\end{enumerate}

\end{thm}





\addtocontents{toc}{\protect\end{quote}}

\chapter{Spherical geometry}
\label{chap:sphere}

Spherical geometry studies the surface of a unit sphere.
This geometry has applications in cartography, navigation, and astronomy.

Spherical geometry is a close relative of Euclidean and hyperbolic geometries.
Most of the theorems in hyperbolic geometry have spherical analogs,
but spherical geometry is easier to visualize. 

\section{Sphere}

Recall that Euclidean space is the set $\mathbb{R}^3$ of all triples $(x,y,z)$ of real numbers
such that the distance between a pair of points
$A\z=(x_A,y_A,z_A)$ and $B=(x_B,y_B,z_B)$
is defined by the following formula:
$$AB
\df
\sqrt{(x_A-x_B)^2+(y_A-y_B)^2+(z_A-z_B)^2}.$$

The planes in this space are defined as the set of solutions of 
$$a\cdot x+b\cdot y+c\cdot z+d=0$$ 
for constants $a$, $b$, $c$, and $d$, where at least one of the numbers $a$, $b$, or $c$ is not zero.
Any such plane is isometric to the Euclidean plane.

A \index{sphere}\emph{sphere} with center $O$ and radius $r$ is the set of points in the space that lie at the distance $r$ from~$O$.

Let $A$ and $B$ be two points on the unit sphere centered at~$O$.
The \index{distance!spherical}\index{spherical distance}\emph{spherical distance} from $A$ to $B$
(briefly $AB_s$)
 is defined as  $|\measuredangle AOB|$. 

A \index{great circle}\emph{great circle} is the intersection of the sphere with a plane passing thru~$O$.
In spherical geometry, great circles play the role of lines despite not meeting Definition~\ref{def:line}.
Note that great circles are circles of radius $\tfrac\pi2$;
it has two antipodal centers.

Let $A$ and $B$ be two distinct and not antipodal points on the unit sphere.
Then there is a unique great circle thru $A$, and $B$;
it will be denoted by $(AB)_s$.
The shorter of the two arcs of $(AB)_s$ with endpoints $A$ and $B$ will be denoted by $[AB]_s$.
We can talk about spherical triangles and their sides;
their angle measures are defined as the angle between the tangent half-lines to the arcs.

\section{Central projection}

Any map from a sphere (or its part) to a plane distorts the surface in some way.
In this section we describe the so-called central projection; it maps great circles to straight lines on the plane.
This property makes it convenient in navigation;
while the spherical shapes are distorted, the shortest path on the plane corresponds to a shortest path on the sphere.

The central projection is analogous to the projective model of the hyperbolic plane discussed in Chapter~\ref{chap:klein}.

Let $\Sigma$ be the unit sphere centered at the origin~$O$.
Suppose that $\Pi^+$ denotes the plane defined by the equation $z=1$.
This plane is parallel to the $(x,y)$-plane and passes thru 
the north pole $N\z=(0,0,1)$ of~$\Sigma$.

The northern hemisphere of $\Sigma$
is the subset of points $(x,y,z)\z\in \Sigma$ such that $z>0$; it will be denoted by~$\Sigma^+$.

\begin{wrapfigure}{o}{40mm}
\centering
\vskip-3mm
\includegraphics{mppics/pic-254}
\end{wrapfigure}

Given a point $P\in \Sigma^+$, consider the half-line $[OP)$. 
Suppose that $P'$ denotes the intersection of $[OP)$ and~$\Pi^+$.
Note that 
if $P=(x,y,z)$, then $P'=(\tfrac xz,\tfrac yz,1)$.
It follows that $P\leftrightarrow P'$ is a bijection between $\Sigma^+$ and~$\Pi^+$.



The described bijection $\Sigma^+\leftrightarrow \Pi^+$ is called the \index{central projection}\emph{central projection} of 
the hemisphere~$\Sigma^+$.

Note that the central projection sends the intersections of the great circles with $\Sigma^+$ to the lines in~$\Pi^+$.
This follows since the great circles are intersections of $\Sigma$ with planes passing thru the origin,
and the lines in $\Pi^+$ are the intersection of $\Pi^+$ with these planes.

The following exercise 
is analogous to Exercise~\ref{ex:h-median}.

\begin{thm}{Exercise}\label{ex:s-medians}
Let $\triangle_sABC$ be a nondegenerate spherical triangle.
Assume that the plane $\Pi^+$ is parallel to the plane passing thru $A$, $B$, and~$C$.
Let $A'$, $B'$, and $C'$ denote the central projections of $A$, $B$, and~$C$.
\begin{enumerate}[(a)]
\item\label{ex:s-medians:a} Show that the midpoints of $[A'B']$, $[B'C']$, and $[C'A']$
are central projections of the midpoints of $[AB]_s$, $[BC]_s$, and $[CA]_s$ respectively.
\item\label{ex:s-medians:b} Use part (\ref{ex:s-medians:a}) to show that the medians of a spherical triangle intersect at one point.
\end{enumerate}

\end{thm}


\begin{thm}{Exercise}\label{ex:s-altitudes}
Let $P\leftrightarrow P'$ be the central projection described above
and $N$ be the north pole; so, $N'=N$.
Show that $|\measuredangle_s NPQ|=\tfrac\pi2$ if and only if $|\angle N'P'Q'|=\tfrac\pi2$.
\end{thm}

\section{Stereographic projection}

In this section, we define stereographic projection;
it is closely related to the conformal model of the hyperbolic plane, which is discussed in Chapter~\ref{chap:poincare}.

{

\begin{wrapfigure}{r}{48mm}
\vskip-6mm
\centering
\includegraphics{mppics/pic-252}
\caption*{The plane thru $P$, $O$, and~$S$.}
\end{wrapfigure}

Consider the unit sphere $\Sigma$ 
centered at the origin $(0,0,0)$.
This sphere can be described by the equation $x^2+y^2\z+z^2\z=1$. 

Suppose that $\Pi$ denotes the $(x,y)$-plane;
it is defined by the equation $z \z= 0$.
Clearly, $\Pi$
runs thru the center of~$\Sigma$.

Let $N = (0, 0, 1)$ and $S\z=(0, 0, -1)$ denote the ``north'' and ``south'' poles of $\Sigma$;
these are the points on the sphere that have extremal distances to~$\Pi$.
Suppose that $\Omega$ denotes the ``equator'' of $\Sigma$;
it is the intersection $\Sigma\cap\Pi$.

}

For any point $P\ne S$ on $\Sigma$,
consider the line $(SP)$ in the space. 
This line intersects $\Pi$ at exactly one point, denoted by~$P'$. 
Set $S'=\infty$.


The map $\xi_s\: P\mapsto P'$ is called the \index{stereographic projection}\emph{stereographic projection from $\Sigma$ to $\Pi$ with respect to the south pole}.
The inverse of this map $\xi^{-1}_s\: P'\z\mapsto P$ is called the {}\emph{stereographic projection from $\Pi$ to $\Sigma$ with respect to the south pole}.

In the same way, one can define the
{}\emph{stereographic projections $\xi_n$ and $\xi^{-1}_n$ with respect to the north pole}~$N$.

Note that $P=P'$ if and only if $P\in\Omega$.

Exercise~\ref{ex:stereographic-inversion} below states that the stereographic projection preserves 
the angles between arcs;
more precisely, \textit{the absolute value of the angle measure} between arcs on the sphere.
This is a useful property in cartography;
the curves on the sphere meet at the same angles as their stereographic projections.


\section[Inversion]{Inversion across a sphere}

The inversion across a sphere is defined the same way as the inversion across a circle.

Formally, let $\Sigma$ be the sphere with center $O$ and radius~$r$.
The \index{inversion!inversion across a sphere}\emph{inversion} across $\Sigma$ of a point $P$ is the point $P'\in[OP)$ such that
$$OP\cdot OP'=r^2.$$
In this case, the sphere $\Sigma$  will be called the 
\index{inversion!sphere of inversion}\emph{sphere of inversion},
and its center is called the \index{inversion!center of inversion}\emph{center of inversion}.

We also add $\infty$ to the space and assume that the center of inversion is mapped to $\infty$ and the other way around. 
The space $\mathbb{R}^3$ with the point $\infty$ will be called \index{inversive!space}\emph{inversive space}.

The inversion of space shares many properties with the inversion of the plane.
Most importantly, analogs of theorems \ref{lem:inverse-4-angle}, \ref{thm:inverse-cline}, \ref{thm:angle-inversion} can be summarized as follows:

\begin{thm}{Theorem}\label{thm:inversion-3d}
The inversion across the sphere has the following properties:
\begin{enumerate}[(a)]
\item\label{thm:inversion-3d:a} Inversion maps a sphere or a plane into a sphere or a plane.
\item\label{thm:inversion-3d:b} Inversion maps a circline into a circline. 
\item\label{thm:inversion-3d:cross-ratio} Inversion preserves the cross-ratio;
that is, if $A'$, $B'$, $C'$, and $D'$ are the inverses of the points $A$, $B$, $C$, and $D$ respectively,
then
$$\frac{AB\cdot CD}{BC\cdot DA}= \frac{A'B'\cdot C'D'}{B'C'\cdot D'A'}.$$
\item Inversion maps arcs into arcs.
\item\label{thm:inversion-3d:angle}
Inversion preserves the absolute value of the angle
measure between tangent half-lines to the arcs.
\end{enumerate}
\end{thm}

In the following exercises, we apply the theorem above to tie the stereographic projection with the inversion. 
We assume that $\Sigma$, $\Pi$, $\Omega$, $O$, $S$, $N$ and $\xi_s$ are as in the previous section.

{

\begin{wrapfigure}{r}{32mm}
\vskip-6mm
\centering
\includegraphics{mppics/pic-253}
\end{wrapfigure}

\begin{thm}{Exercise}\label{ex:stereographic-inversion}
Show that the stereographic projections 
$\xi_s\: \Sigma\to\Pi$ and $\xi^{-1}_s\: \Pi\to\Sigma$
are the restrictions to $\Sigma$ and $\Pi$ respectively, of the inversion across the sphere $\Upsilon$ with the center $S$ and radius $\sqrt{2}$.

Conclude that the stereographic projection preserves 
the angles between arcs;
more precisely, the absolute value of the angle measure between arcs on the sphere.
\end{thm}

\begin{thm}{Exercise}\label{ex:great-circ}
Show that the stereographic projection $\xi_s\:\Sigma\to\Pi$
sends the great circles to plane circlines that intersect $\Omega$ at opposite points.
\end{thm}

}

\begin{thm}{Exercise}\label{ex:conform-sphere}
Fix a point $P\in \Pi$  and let $Q$ be another point in~$\Pi$.
Let $P'$ and $Q'$ denote their stereographic projections to~$\Sigma$.
Set $x=PQ$ and $y=P'Q'_s$.
Show that
$$\lim_{x\to 0}\, \frac{y}{x}=\frac{2}{1+OP^2}.$$
\end{thm}

The last exercise is analogous to Lemma~\ref{lem:conformal}.

The proof of \ref{thm:inversion-3d} resembles the corresponding proofs in plane geometry.
Let us give couple of a hints to the reader who wants to reconstruct its proof.
To prove \ref{thm:inversion-3d}\textit{\ref{thm:inversion-3d:a}}, one needs the following lemma;
its proof is left to the reader.

\begin{thm}{Lemma}
Let $\Sigma$ be a subset of the Euclidean space
that contains at least two points.
Fix a point $O$ in the space.

Then $\Sigma$ is 
a sphere 
if and only if
for any plane $\Pi$ passing thru $O$,
the intersection $\Pi\cap \Sigma$ is either an empty set,
a one-point set, or a circle.
\end{thm}  

The following observation reduces part~\textit{(\ref{thm:inversion-3d:b})} to part~\textit{(\ref{thm:inversion-3d:a})}.

\begin{thm}{Observation}
Any circle in the space is an intersection of two spheres.
\end{thm}

{

\begin{wrapfigure}{o}{25mm}
\centering
\includegraphics{mppics/pic-250}
\end{wrapfigure}

Let us define a \index{circular cone}\emph{circular cone} as a set formed by line segments from a fixed point, called the \index{tip of cone}\emph{tip} of the cone, to all the points on a fixed circle, called the \index{base!of cone}\emph{base} of the cone;
we always assume that the base does not lie in the same plane as the tip.
We say that the cone is \index{right circular cone}\emph{right} 
if the center of the base circle is the footpoint of the tip on the base plane;
otherwise, we call it \index{oblique circular cone}\emph{oblique}.

}

\begin{thm}{Exercise}\label{ex:cone}
Let $K$ be an oblique circular cone.
Show that there is a plane $\Pi$ that is not parallel to the base plane of $K$ such that the intersection $\Pi\cap K$ is a circle.
\end{thm}

\section{Pythagorean theorem}

Here is an analog of the Pythagorean theorems (\ref{thm:pyth} and \ref{thm:pyth-h-poincare}) in spherical geometry.

\begin{thm}{Spherical Pythagorean theorem}\label{thm:s-pyth}\index{Pythagorean theorem}
Let $\triangle_sABC$ be a spherical triangle with a right angle at~$C$.
Set $a=BC_s$, $b=CA_s$, and $c=AB_s$.
Then
$$\cos c=\cos a\cdot\cos b.$$

\end{thm}

In the proof, we will use the notion of the scalar product which we are about to discuss.

Let $v_A\z=(x_A,y_A,z_A)$ and $v_B=(x_B,y_B,z_B)$ denote the position vectors of points $A$ and $B$.
The \index{scalar product}\emph{scalar product} of the two vectors $v_A$ and $v_B$ in $\mathbb{R}^3$
is defined as 
$$\langle v_A,v_B\rangle
\df
x_A\cdot x_B+y_A\cdot y_B+z_A\cdot z_B.\eqlbl{eq:scal-def}$$

Assume both vectors $v_A$ and $v_B$ are nonzero;
suppose that $\phi$ denotes the angle measure between them.
Then the scalar product can be expressed the following way:
$$\langle v_A,v_B\rangle=|v_A|\cdot|v_B|\cdot\cos\phi,
\eqlbl{eq:scal-angle}$$
where 
\begin{align*}
|v_A|&=\sqrt{x_A^2+y_A^2+z_A^2},
&
|v_B|&=\sqrt{x_B^2+y_B^2+z_B^2}.
\end{align*}

Now, assume that the points $A$ and $B$ 
lie on the unit sphere $\Sigma$ in $\mathbb{R}^3$ centered at the origin.
In this case, $|v_A|=|v_B|=1$.
By \ref{eq:scal-angle} we get that
$$\cos AB_s=\langle v_A,v_B\rangle.
\eqlbl{eq:scalar-s-dist}$$

\parit{Proof of the spherical Pythagorean theorem.}
Since the angle at $C$ is right,
we can choose the coordinates in $\mathbb{R}^3$ so that 
$v_C\z=(0,0,1)$, $v_A$ lies in the $(x,y)$-plane, so $v_A\z=(x_A,0,z_A)$,
and $v_B$ lies in the $(y,z)$-plane, so $v_B=(0,y_B,z_B)$.

{

\begin{wrapfigure}{r}{40mm}
\vskip-4mm
\centering
\includegraphics{mppics/pic-248}
\end{wrapfigure}


Applying, \ref{eq:scalar-s-dist},
we get that
\begin{align*}
z_A&=\langle v_C,v_A\rangle
=\cos b,
\\
z_B&=\langle v_C,v_B\rangle
=\cos a.
\end{align*}

Applying, \ref{eq:scal-def} and \ref{eq:scalar-s-dist}, we get that
\begin{align*}
\cos c &=\langle v_A,v_B\rangle=
\\
&=x_A\cdot 0+0\cdot y_B+z_A\cdot z_B=
\\
&=\cos b\cdot\cos a.
\end{align*}
\qedsf

}

\begin{thm}{Exercise}\label{ex:2(pi/4)=pi/3}
Show that 
if $\triangle_sABC$ is a spherical triangle with a right angle at $C$,
and $AC_s=BC_s=\tfrac\pi4$, then $AB_s=\tfrac\pi3$.
\end{thm}



\section{Imaginary distance}

Recall that  
\[
\cosh x=\frac {e^{x}+e^{-x}}2
\qquad\text{and}\qquad
\cos x=\frac {e^{i\cdot x}+e^{-i\cdot x}}2.
\]
The first formula is the definition of the hyperbolic cosine (see \ref{sec:hyp-trig});
the second one is called Euler's formula (see \ref{sec:Euler's formula}).
It follows that 
\[\cosh x=\cos (i\cdot x)
\qquad\text{and}\qquad
\cos x=\cosh (i\cdot x).\]

Let us compare the formulas in hyperbolic and spherical Pythagorean theorems (see \ref{thm:s-pyth} and \ref{thm:pyth-h-poincare}):
\begin{align*}
\cosh c&=\cosh a\cdot \cosh b,
&
\cos c&=\cos a\cdot \cos b.
\end{align*}
Note that if we change $a$, $b$, and $c$ to $i\cdot a$, $i\cdot b$, and $i\cdot c$,
then the first formula transforms into the second one, and the other way around.

This is not a coincidence;
the same holds for all analytic formulas ---
changing every distance $d$ to $i\cdot d$ transforms a valid spherical formula into a valid hyperbolic formula;
the angle measures need no change.
This magic substitution was found by Franz Taurinus \cite{taurinus}; we are not going to prove it.

\begin{thm}{Advanced exercise}\label{ex:taurinus}
Consider a spherical triangle $ABC$; set
$a\z=BC_s$, $b=CA_s$, $c=AB_s$,
$\alpha=\measuredangle_sCAB$, $\beta=\measuredangle_sABC$, $\gamma=\measuredangle_sBCA$.
Use the Taurinus substitution to write the hyperbolic analogs to the following formulas in spherical geometry:

\begin{enumerate}[(a)]
\item The spherical cosine rule:
\[\cos c=\cos a \cdot \cos b+\sin a\cdot \sin b\cdot \cos\gamma.\]
\item The dual spherical cosine rule:
\[\cos \gamma=-\cos \alpha \cdot \cos \beta+\sin \alpha\cdot \sin \beta \cdot \cos c.\]
\item
The spherical sine rule:
\[\frac{\sin \alpha}{\sin a}=\frac{\sin \beta}{\sin b}=\frac{\sin \gamma}{\sin c}.\]
\end{enumerate}
 
\end{thm}

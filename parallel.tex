\chapter[Parallel lines]{Parallel lines}\label{chap:angle-sum}
\addtocontents{toc}{\protect\begin{quote}}

\section*{Parallel lines}
\addtocontents{toc}{Parallel lines.}



{

\begin{wrapfigure}{r}{19mm}
\begin{lpic}[t(-3mm),b(0mm),r(0mm),l(-0mm)]{pics/parallel-notation(1)}
\end{lpic}
\end{wrapfigure}

In consequence of Axiom~\ref{def:birkhoff-axioms:1}, 
any two distinct lines $\ell$ and $m$ have either one point
in common or none. 
In the first case they are \index{intersecting lines}\emph{intersecting} (briefly \index{36@$\parallel$}$\ell\nparallel m$); 
in the second case, $\ell$ and $m$ are said to be \index{parallel lines}\emph{parallel} (briefly \index{36@$\parallel$}$\ell\parallel m$);
in addition, a line is always regarded as parallel to itself.

}

To emphasize that two lines on a diagram are parallel we will mark them with arrows of the same type.



\begin{thm}[\abs]{Proposition}\label{prop:perp-perp} Let $\ell$, $m$ and $n$ be three lines.
Assume that $n\perp m$ and $m\perp \ell$.
Then $\ell\parallel n$. 
\end{thm}

\parit{Proof.}
Assume the contrary; 
that is, $\ell\nparallel n$.
Then there is a point, say $Z$, of intersection of $\ell$ and~$n$.
Then by Theorem~\ref{perp:ex+un},
$\ell=n$.

Since any line is parallel to itself, we have $\ell\parallel n$, a contradiction.
\qeds

\begin{thm}{Theorem}\label{thm:parallel}
For any point $P$ and any line $\ell$
there is a unique line $m$
which passes thru $P$ and is parallel to~$\ell$.
\end{thm}

The above theorem has two parts, existence and uniqueness.
In the proof of uniqueness we will use the method of similar triangles.

\parit{Proof; existence.} 
Apply Theorem~\ref{perp:ex+un} two times,
first to construct the line $m$ thru $P$ which is perpendicular to $\ell$,
and second to construct the line $n$ thru $P$ which is perpendicular to~$m$.
Then apply Proposition~\ref{prop:perp-perp}.

\parit{Uniqueness.}
If $P\in\ell$, then $m=\ell$ by the definition of parallel lines.
Further we assume $P\notin\ell$.

Let us construct the lines $n\ni P$ and $m\ni P$ as in the proof of existence, so $m\parallel \ell$.

Assume there is yet another line $s\ni P$ which is distinct from $m$ and parallel to~$\ell$.
Choose a point $Q\in s$ which lies with $\ell$ on the same side from~$m$.
Let $R$ be the foot point of $Q$ on~$n$.

Let $D$ be the point of intersection of $n$ and~$\ell$.
According to Proposition~\ref{prop:perp-perp} $(QR)\parallel m$. 
Therefore, $Q$, $R$ and $\ell$ lie on the same side of~$m$. 
In particular, $R\in [P D)$.

\begin{center}
 \begin{lpic}[t(0mm),b(0mm),r(0mm),l(0mm)]{pics/parallel(1)}
\lbl[br]{2,24.5;$P$}
\lbl[r]{1,15;$R$}
\lbl[rt]{2,.5;$D$}
\lbl[bl]{23,16;$Q$}
\lbl[t]{55,.5;$Z$}
\lbl[t]{35,1.5;$\ell$}
\lbl[b]{35,23;$m$}
\lbl[b]{39,9.5,-15;$s$}
\lbl[b]{2,9,90;$n$}
\end{lpic}
\end{center}

Choose $Z\in [P Q)$ such that 
$$\frac{PZ}{PQ}=\frac{PD}{PR}.$$
By SAS similarity condition (or equivalently by Axiom~\ref{def:birkhoff-axioms:4})
we have $\triangle RPQ\sim \triangle DPZ$;
therefore $(Z D)\perp(P D)$.
It follows that $Z$ lies on $\ell$ and $s$ --- a contradiction.\qeds

\begin{thm}{Corollary}\label{cor:parallel-1}
Assume $\ell$, $m$ and $n$ are lines
such that $\ell\parallel m$ and $m\parallel n$.
Then $\ell\parallel n$.
\end{thm}

\parit{Proof.}
Assume the contrary; that is, $\ell\nparallel n$.
Then there is a point $P\in \ell\cap n$.
By Theorem~\ref{thm:parallel},
$n=\ell$ --- a contradiction.
\qeds

Note that from the definition, we have $\ell\parallel m$ if and only if $m\z\parallel \ell$.
Therefore, according to the above corollary, ``$\parallel$'' is an 
\index{equivalence relation}\emph{equivalence relation}.
That is, for any lines $\ell$, $m$ and $n$ the following conditions hold:
\begin{enumerate}[(i)]
\item $\ell\parallel \ell$;
\item if $\ell\parallel m$, then $m\parallel \ell$;
\item if $\ell\parallel m$ and $m\parallel n$, then 
$\ell\parallel n$.
\end{enumerate}

\begin{thm}{Exercise}\label{ex:perp-perp}
Let $k$, $\ell$, $m$ and $n$ be lines such that $k\perp \ell$, $\ell\perp m$ and $m\perp n$.
Show that $k\nparallel n$.
\end{thm}

\begin{thm}{Exercise}\label{ex:construction-parallel}
Make a ruler-and-compass construction of a line thru a given point which is parallel to a given line.
\end{thm}

{

\begin{wrapfigure}[6]{r}{30mm}
\begin{lpic}[t(-0mm),b(0mm),r(0mm),l(0mm)]{pics/parallel-2(1)}
\lbl[b]{14.5,18.5;$A$}
\lbl[b]{3,15;$B$}
\lbl[b]{18,7.5;$C$}
\lbl[b]{3,3;$D$}
\end{lpic}
\end{wrapfigure}

\section*{Transversal property}
\addtocontents{toc}{Transversal property.}

If the line $t$ intersects each line $\ell$ and $m$ at one point, then we say that $t$ is a \index{transversal}\emph{transversal} to $\ell$ and~$m$.
For example, on the diagram, line $(CB)$ is a transversal 
to $(AB)$ and~$(CD)$.

}

\begin{thm}{Transversal property}\label{thm:parallel-2} 
$(AB)\parallel(C D)$ if and only if 
$$2\cdot(\measuredangle A B C+\measuredangle B C D)\equiv 0.
\eqlbl{A B C + B C D}$$ 
Equivalently 
$$\measuredangle A B C+\measuredangle B C D
\equiv 
0
\quad
\text{or}
\quad
\measuredangle A B C+\measuredangle B C D
\equiv
\pi.$$ 
Moreover, if $(AB)\ne(C D)$, then in the first case 
$A$ and $D$ lie on the opposite sides of $(BC)$,
in the second case 
$A$ and $D$ lie on the same sides of~$(BC)$.
\end{thm}

In the proof of the transversal property we will use the following lemma.

%(???

\begin{thm}[\abs]{Lemma}\label{lem:2angles1}
Let $ABC$ be a nondegenerate triangle.
Then there is a point $X$ that lies with $B$ on the same side of $(AC)$ such that 
\[\measuredangle BAX=\measuredangle ABC.\]
\end{thm}

\begin{wrapfigure}{r}{23mm}
\begin{lpic}[t(-6mm),b(0mm),r(0mm),l(0mm)]{pics/2sum1(1)}
\lbl[l]{16,2;$A$}
\lbl[b]{6,23;$B$}
\lbl[r]{0,2;$C$}
\lbl[b]{20,23;$X$}
\lbl[l]{11.5,12;$M$}
\end{lpic}
\end{wrapfigure}

\parit{Proof.} 
Let $M$ be the midpoint of $[AB]$.
Choose $X\z\in (CM)$ distinct from $C$ so that $XM=CM$.


Note that the angles $CMB$ and $XMA$
are vertical;
in particular, 
$$\measuredangle CMB=\measuredangle XMA.\eqlbl{eq:CMB=XMA}$$
By construction, $AM\z=BM$ and $CM\z=XM$.
Therefore, 
$$\triangle CMB\cong \triangle XMA;$$ 
in particular $\measuredangle ABC\equiv\pm \measuredangle BAX$.
From \ref{eq:CMB=XMA} and \ref{thm:signs-of-triug}, we can not have a minus sign in the last equality; therefore
$$\measuredangle ABC\equiv\measuredangle BAX.$$

By construction, the line segments $[BM]$ and $[MX]$ do not cross the line $(AC)$.
Therefore the points $X$ and $B$ lie on the same side of $(AC)$ (see \ref{cor:half-plane}).
\qeds


\parit{Proof of transversal property; ``only-if'' part.} 
We need to show that 
if $(AB)\z\nparallel(C D)$,
then \ref{A B C + B C D} does not hold.

Since $(AB)\nparallel(C D)$, there is a unique point $Z\z\in (AB)\z\cap(C D)$.
Since $Z$ lies on $(AB)$ and on $(CD)$, we have
\begin{align*}
2\cdot \measuredangle ABC&\equiv 2\cdot \measuredangle ZB,
&
2\cdot \measuredangle BCD&\equiv 2\cdot \measuredangle BCZ.
\end{align*}
Therefore 
\[2\cdot(\measuredangle A B C+\measuredangle B C D)\equiv 2\cdot(\measuredangle Z B C+\measuredangle B C Z).\]

Note that the triangle $ZBC$ is nondegenerate.
By Lemma \ref{lem:2angles1}, 
there is a point $X$ that lies with $C$ on the same side of $(ZB)$ such that 
$\measuredangle CBX=\measuredangle BCZ$.
In particular 
\[2\cdot \measuredangle ZBX\equiv 2\cdot (\measuredangle ZBC+\measuredangle BCZ).\]

\begin{wrapfigure}{r}{43mm}
\begin{lpic}[t(-0mm),b(0mm),r(0mm),l(0mm)]{pics/onlyif-transversal(1)}
\lbl[b]{25,20;$A$}
\lbl[bl]{13,20;$B$}
\lbl[br]{16,6;$C$}
\lbl[br]{9,2;$D$}
\lbl[t]{4,11;$X$}
\lbl[b]{38,20;$Z$}
\end{lpic}
\end{wrapfigure}

Since $X$ does not lie on $(ZB)$ we have
\[2\cdot \measuredangle ZBX\not\equiv 0.\]
Hence the ``only-if'' part follows.


\parit{``If''-part.}
Given points $A$, $B$ and $C$ there is unique line $(CD)$ such that \ref{A B C + B C D} holds.
Indeed if 
$$2\cdot(\measuredangle A B C+\measuredangle B C D)\equiv 2\cdot(\measuredangle A B C+\measuredangle B C D')\equiv0.
$$ 
then 
$$2\cdot\measuredangle B C D\equiv 2\cdot\measuredangle B C D'.$$
Therefore $2\cdot\measuredangle D C D'\equiv0$; that is, $D'\in (CD)$, or equivalently the line $(CD)$ coinsides with the line $(CD')$.

From the ``only-if'' part we know that if \ref{A B C + B C D} holds then $(CD)\parallel(AB)$.
By Theorem \ref{thm:parallel} there is unique line thru $C$ that is parallel to $(AB)$.
Therefore it must be the same line $(CD)$ such that \ref{A B C + B C D} hold.
\qeds

%???)

\begin{thm}{Exercise}\label{ex:smililar+parallel}
Let $\triangle ABC$ be a nondegenerate triangle.
Assume $B'$ and $C'$ are points on sides $[AB]$ and $[AC]$ such that $(B'C')\parallel(BC)$.
Show that $\triangle ABC\sim\triangle AB'C'$.
\end{thm}

\begin{thm}{Exercise}\label{ex:trisection}
Trisect a given segment with a ruler and a compass.
\end{thm}

\section*{Angles of triangles}
\addtocontents{toc}{Angles of triangle.}

\begin{thm}{Theorem}\label{thm:3sum}
In any $\triangle A B C$, we have
$$\measuredangle A B C+ \measuredangle B C A + \measuredangle C A B \equiv \pi.$$

\end{thm}

%(???

\parit{Proof.} 
First note that 
if $\triangle A B C$ is degenerate, then the equality follows from Theorem~\ref{thm:straight-angle}.
Further we assume that $\triangle A B C$ is nondegenerate.

\begin{wrapfigure}{o}{23mm}
\begin{lpic}[t(-0mm),b(0mm),r(0mm),l(0mm)]{pics/2sum2(1)}
\lbl[t]{14,.5;$A$}
\lbl[r]{4,22;$B$}
\lbl[t]{2,.5;$C$}
\lbl[r]{18,22;$X$}
\end{lpic}
\end{wrapfigure}

By Lemma~\ref{lem:2angles1} there is a point $X$ that lies with $B$ on the same side of $(AC)$ such that 
$\measuredangle BAX\z=\measuredangle ABC$.
In particular 
\[2\cdot (\measuredangle XAB+\measuredangle ABC)\equiv 0.\]
By transversal property, we get that $(AX)\parallel (CB)$.

Now, let us apply the transversal property (\ref{thm:parallel-2}) to the parallel lines $(BC)$ and $(AX)$ and their transversal $(CA)$.
Since $B$ and $X$ lie on the same side of $(CA)$, we have 
\[\measuredangle BCA+\measuredangle CAX\equiv \pi.\eqlbl{eq:ABC+CAB}\]

Since $\measuredangle BAX=\measuredangle ABC$,
we have 
\[\measuredangle CAX\equiv\measuredangle CAB+\measuredangle ABC\]
The latter identity and \ref{eq:ABC+CAB} imply the theorem.\qeds

%???)

{

\begin{wrapfigure}{r}{43mm}
\begin{lpic}[t(-3mm),b(0mm),r(0mm),l(0mm)]{pics/bisector-BA=AD=DC(1)}
\lbl[b]{9,26;$A$}
\lbl[t]{2,1;$B$}
\lbl[t]{38,1;$C$}
\lbl[t]{16,1;$D$}
\end{lpic}
\end{wrapfigure}

\begin{thm}{Exercise}\label{ex:pent}
Let $\triangle ABC$ be a nondegenerate triangle.
Assume there is a point $D\in [BC]$ 
such that 
\[\measuredangle BAD\equiv \measuredangle DAC,
\quad
BA=AD\z=DC.\]
Find the angles of $\triangle ABC$. 
\end{thm}

}

\begin{thm}{Exercise}\label{ex:|3sum|}
Show that 
$$|\measuredangle A B C|+ |\measuredangle B C A| + |\measuredangle C A B| = \pi$$
for any $\triangle ABC$.
\end{thm} 



{

\begin{wrapfigure}[5]{r}{30mm}
\begin{lpic}[t(2mm),b(0mm),r(0mm),l(0mm)]{pics/right-isos(1)}
\lbl[t]{3,0;$A$}
\lbl[br]{15,10;$B$}
\lbl[t]{28,0;$C$}
\lbl[br]{28,19;$D$}
\end{lpic}
\end{wrapfigure}

\begin{thm}{Exercise}\label{ex:right-isos}
Let $\triangle ABC$ be an isosceles nondegenerate triangle with the base~$[AC]$.
Consider the point $D$ on the extension of the side $[AB]$ 
 such that $AB=BD$.
Show that $\angle ACD$ is right.
\end{thm}

\begin{thm}{Exercise}\label{ex:pi/4-isos}
Let $\triangle ABC$ be an isosceles nondegenerate triangle with base~$[AC]$. 
Assume that a circle is passing thru $A$,
centered at a point on $[AB]$ and tangent to $(BC)$ at the point~$X$.
Show that $\measuredangle CAX\z=\pm\tfrac\pi4$.
\end{thm}

}

\begin{thm}{Exercise}\label{ex:quadrilateral}
Show that for any quadrilateral $ABCD$, we have
$$\measuredangle ABC+\measuredangle BCD+\measuredangle CDA+\measuredangle DAB\equiv 0.$$

\end{thm}







\section*{Parallelograms}
\addtocontents{toc}{Parallelograms.}

A quadrilateral $ABCD$ in the Euclidean plane is called \index{quadrilateral!degenerate quadrilateral}\index{degenerate!quadrilateral}\emph{nondegenerate} if any three points from $A,B,C,D$ do not lie on one line.

A nondegenerate quadrilateral $ABCD$ is called a \index{parallelogram}\emph{parallelogram}
if $(AB)\z\parallel (CD)$ and $(BC)\z\parallel (DA)$.

{

\begin{wrapfigure}{o}{32mm}
\begin{lpic}[t(-0mm),b(0mm),r(1mm),l(1mm)]{pics/parallelogram(1)}
\lbl[bl]{29,27.5;$A$}
\lbl[br]{7,27.5;$B$}
\lbl[t]{-.5,.5;$C$}
\lbl[t]{22.5,.5;$D$}
\end{lpic}
\end{wrapfigure}

\begin{thm}{Lemma}\label{lem:parallelogram}
If $\square A B C D$ is a parallelogram, then
\begin{enumerate}[(a)]
\item $\measuredangle D A B= \measuredangle B C D$;
\item $AB=CD$.
\end{enumerate}
\end{thm}


\parit{Proof.}
Since $(AB)\parallel (CD)$,
the points $C$ and $D$ lie on the same side of~$(AB)$.
Hence $\angle ABD$ and $\angle ABC$ have the same sign.

Analogously, 
$\angle CBD$ and $\angle CBA$ have the same sign. 

}

Since $\measuredangle ABC\equiv -\measuredangle CBA$,
we get that the angles $DBA$ and $DBC$ have opposite signs; 
that is, $A$ and $C$ lie on opposite sides of~$(BD)$.


According to the transversal property (\ref{thm:parallel-2}), 
$$\measuredangle B D C
\equiv 
-\measuredangle DBA
\quad
\text{and}
\quad 
\measuredangle DBC
\equiv 
-\measuredangle BDA.$$
By the ASA condition
$\triangle A B D\cong \triangle C D B$.
The latter implies both statements in the lemma.
\qeds



\begin{thm}{Exercise}\label{ex:romb}
Assume $ABCD$ is a quadrilateral such that
\[AB=CD=BC=DA.\]
Show that $ABCD$ is a parallelogram.
\end{thm}

A quadrilateral as in the exercise above is called a \index{rhombus}\emph{rhombus}.

\begin{thm}{Exercise}\label{ex:diad-par}
Show that diagonals of a parallelogram intersect each other at their midpoints.
\end{thm}

A quadraliteral $ABCD$ is called a \index{rectangle}\emph{rectangle} if the angles $ABC$, $BCD$, $CDA$ and $DAB$ are right.
Note that according to the transversal property \ref{thm:parallel-2},
any rectangle is a parallelogram.

A rectangle with equal sides is called a \index{square}\emph{square}.

\begin{thm}{Exercise}\label{ex:rectangle}
Show that the parallelogram $ABCD$ is a rectangle
if and only if $AC=BD$.
\end{thm}

\begin{thm}{Exercise}\label{ex:romb2}
Show that the parallelogram $ABCD$ is a rhombus
if and only if $(AC)\perp (BD)$.
\end{thm}

Assume $\ell\parallel m$, and $X,Y\in m$.
Let $X'$ and $Y'$ denote the foot points of $X$ and $Y$ on~$\ell$.
Note that $\square XYY'X'$ is a rectangle.
By Lemma~\ref{lem:parallelogram}, $XX'=YY'$.
That is, any point on $m$ lies on the same distance from $\ell$.
This distance is called the \index{distance!between parallel lines}\emph{distance between} $\ell$ and~$m$.


\section*{Method of coordinates}
\addtocontents{toc}{Method of coordinates.}

The following exercise is important;
it shows that our axiomatic definition agrees with the model definition described on page \pageref{def:d_2}.


\begin{thm}{Exercise}\label{ex:coordinates} 
Let $\ell$ and $m$ be perpendicular lines in the Euclidean plane.
Given a point $P$, let $P_\ell$ and $P_m$ denote the foot points of $P$ on $\ell$ and $m$ correspondingly.


\begin{enumerate}[(a)]
\item Show that for any $X\in \ell$ and $Y\in m$ there is a unique point $P$ such that $P_\ell=X$ and $P_m=Y$.
\end{enumerate}

\begin{enumerate}[(a)]\addtocounter{enumi}{1}
\item
Show that 
$PQ^2=P_\ell Q_\ell^2+P_mQ_m^2$
for any pair of points $P$ and~$Q$.
\end{enumerate}

\begin{enumerate}[(a)]\addtocounter{enumi}{2}
\item Conclude that the plane is isometric to $(\mathbb{R}^2,d_2)$; see page \pageref{def:d_2}.
\end{enumerate}

\end{thm}

\begin{wrapfigure}[8]{o}{33mm}
\begin{lpic}[t(-2mm),b(0mm),r(0mm),l(4mm)]{pics/PQlm(1)}
\lbl[br]{12,10;$P$}
\lbl[bl]{23,22;$Q$}
\lbl[t]{11,0;$P_\ell$}
\lbl[t]{22,0;$Q_\ell$}
\lbl[r]{1,8;$P_m$}
\lbl[r]{1,21;$Q_m$}
\lbl[t]{28,1;$\ell$}
\lbl[b]{1,28,90;$m$}
\end{lpic}
\end{wrapfigure}

Once this exercise is solved, we can apply 
the method of coordinates
to solve any problem in Euclidean plane geometry.
This method is powerful, 
but it is often considered as a bad style.

\begin{thm}{Exercise}\label{ex:abc}
Use the Exercise~\ref{ex:coordinates}
to give an alternative proof of Theorem~\ref{thm:abc} in the Euclidean plane.

That is, prove that given the real numbers $a$, $b$ and $c$ such that 
 $$0<a\le b\le c\le a+c,$$
there is a triangle $ABC$
such that $a=BC$, $b=CA$ and $c=AB$.
\end{thm} 

%(???

\begin{thm}{Exercise}\label{ex:circle-coord}
Show that for fixed real values $a$, $b$ and $c$ the equation 
\[x^2+y^2+a\cdot x+b\cdot y+c=0\]
describes a circle, one-point set or empty set.

Show that if it is a circle then it has center $(-\tfrac a2,-\tfrac b2)$ and the radius $r=\tfrac12\cdot \sqrt{a^2+b^2-4\cdot c}$
\end{thm}

\begin{thm}{Exercise}\label{ex:apolonnius}
Use the previous exercise to show that given two distinct point $A$ and $B$ and positive real number $k\ne1$,
the locus of points $X$ such that $AX=k\cdot BX$ is a circle. 
\end{thm}

The circle in the exercise above is an example of the so called \index{Apollonian circle}\emph{Apollonian circle}.

%???)










\addtocontents{toc}{\protect\end{quote}}

\chapter[Parallel lines and similar triangles]{Parallel lines and\\ similar triangles}\label{chap:parallel}
\addtocontents{toc}{\protect\begin{quote}}

\section*{Parallel lines}
\addtocontents{toc}{Parallel lines.}

In consequence of Axiom~\ref{def:birkhoff-axioms:1}, 
any two distinct lines $\ell$ and $m$ have either one point
in common or none. 
In the first case they are \index{intersecting lines}\emph{intersecting}; 
in the second case, $\ell$ and $m$ are said to be \index{parallel lines}\emph{parallel} (briefly \index{36@$\parallel$}$\ell\parallel m$);
in addition, a line is always regarded as parallel to itself.


\begin{thm}[\abs]{Proposition}\label{prop:perp-perp}Let $\ell$, $m$ and $n$ be three lines.
Assume that $n\perp m$ and $m\perp \ell$.
Then $\ell\parallel n$. 
\end{thm}

\parit{Proof.}
Assume the contrary; 
that is, $\ell\nparallel n$.
Then there is a point, say $Z$, of intersection of $\ell$ and~$n$.
Then by Theorem \ref{perp:ex+un},
$\ell=n$.

Since any line is parallel to itself, we have $\ell\parallel n$, a contradiction.
\qeds

\begin{thm}{Theorem}\label{thm:parallel}
For any point $P$ and any line $\ell$
there is a unique line $m$
which passes thru $P$ and is parallel to~$\ell$.
\end{thm}

The above theorem has two parts, existence and uniqueness.
In the proof of uniqueness we will use Axiom~\ref{def:birkhoff-axioms:4} for the first time in this book.

\parit{Proof; existence.} 
Apply Theorem \ref{perp:ex+un} two times,
first to construct the line $m$ thru $P$ which is perpendicular to $\ell$,
and second to construct the line $n$ thru $P$ which is perpendicular to~$m$.
Then apply Proposition~\ref{prop:perp-perp}.

\parit{Uniqueness.}
If $P\in\ell$, then $m=\ell$ by the definition of parallel lines.
Further we assume $P\notin\ell$.

Let us construct the lines $n\ni P$ and $m\ni P$ as in the proof of existence, so $m\parallel \ell$.

Assume there is yet another line $s\ni P$ which is distinct from $m$ and parallel to~$\ell$.
Choose a point $Q\in s$ which lies with $\ell$ on the same side from~$m$.
Let $R$ be the foot point of $Q$ on~$n$.

Let $D$ be the point of intersection of $n$ and~$\ell$.
According to Proposition~\ref{prop:perp-perp} $(QR)\parallel m$. 
Therefore, $Q$, $R$ and $\ell$ lie on the same side from~$m$. 
In particular, $R\in [P D)$.

\begin{center}
 \begin{lpic}[t(0mm),b(0mm),r(0mm),l(0mm)]{pics/parallel(1)}
\lbl[br]{2,24.5;$P$}
\lbl[r]{1,15;$R$}
\lbl[rt]{2,1;$D$}
\lbl[bl]{23,16;$Q$}
\lbl[t]{55,1;$Z$}
\lbl[t]{35,1.5;$\ell$}
\lbl[b]{35,23;$m$}
\lbl[b]{39,9.5,-15;$s$}
\lbl[b]{2,9,90;$n$}
\end{lpic}
\end{center}

Choose $Z\in [P Q)$ such that 
$$\frac{PZ}{PQ}=\frac{PD}{PR}.$$
Then
by Axiom~\ref{def:birkhoff-axioms:4},  $(Z D)\perp(P D)$; 
that is, $Z\in \ell\cap s$, a contradiction.\qeds

\begin{thm}{Corollary}\label{cor:parallel-1}
Assume $\ell$, $m$ and $n$ are lines
such that $\ell\parallel m$ and $m\parallel n$.
Then $\ell\parallel n$.
\end{thm}

\parit{Proof.}
Assume the contrary; that is, $\ell\nparallel n$.
Then there is a point $P\in \ell\cap n$.
By Theorem \ref{thm:parallel},
$n=\ell$, a contradiction.
\qeds

Note that from the definition, we have $\ell\parallel m$ if and only if $m\z\parallel \ell$.
Therefore, according to the above corollary, ``$\parallel$'' is an 
\index{equivalence relation}\emph{equivalence relation}.
That is, for any lines $\ell$, $m$ and $n$ the following conditions hold:
\begin{enumerate}[(i)]
\item $\ell\parallel \ell$;
\item if $\ell\parallel m$, then $m\parallel \ell$;
\item if $\ell\parallel m$ and $m\parallel n$, then 
$\ell\parallel n$.
\end{enumerate}

\begin{thm}{Exercise}\label{ex:perp-perp}
Let $k$, $\ell$, $m$ and $n$ be  lines such that $k\perp \ell$ and $m\perp n$.
Show that if $k\parallel m$, then $\ell\parallel n$.
\end{thm}

\begin{thm}{Exercise}\label{ex:construction-parallel}
Make a ruler-and-compass construction of a line thru a given point which is parallel to a given line.
\end{thm}

\section*{Similar triangles}
\addtocontents{toc}{Similar triangles.}

Two triangles $A' B' C'$ and $A B C$ are called
\index{triangle!similar triangles}\index{similar triangles}\emph{similar} (briefly  \index{30@$\sim$}$\triangle A' B' C'\z\sim\triangle A B C$) if their sides are proportional; 
that is, 
$$A' B'
=
k\cdot A B,
\quad
B' C'=k\cdot B C
\quad
\text{and}
\quad
C' A'
=
k\cdot C A
\eqlbl{dist}
$$
for some $k>0$, and 
$$
\begin{aligned}
\measuredangle A' B' C'&=\pm\measuredangle A B C,
\\
\measuredangle B' C' A'&=\pm\measuredangle B  C A,
\\ 
\measuredangle C' A' B'&=\pm\measuredangle CAB.
\end{aligned}
\eqlbl{angles}
$$

\parbf{Remarks.}
\begin{itemize}
\item According to \ref{thm:signs-of-triug}, in the above three equalities, the signs can be assumed to be the same.

\item If $\triangle A' B' C'\sim\triangle A B C$ with $k=1$ in \ref{dist}, 
 then $\triangle A' B' C'\z\cong\triangle A B C$.

\item Note that ``$\sim$'' is an 
\index{equivalence relation}\emph{equivalence relation}.
That is, 
\begin{enumerate}[(i)]
\item $\triangle A B C\sim\triangle A B C$
for any $\triangle A B C$.
\item If $\triangle A' B' C'\sim\triangle A B C$, then
$$\triangle A B C\sim\triangle A' B' C'.$$
\item If $\triangle A'' B'' C''\sim\triangle A' B' C'$ and $\triangle A' B' C'\z\sim\triangle A B C$, then 
$$\triangle A'' B'' C''\sim\triangle A B C.$$
\end{enumerate}
\end{itemize}

Using the notation ``$\sim$'', Axiom~\ref{def:birkhoff-axioms:4} can be formulated the following way.

\begin{thm}{Reformulation of Axiom~\ref{def:birkhoff-axioms:4}}
If for the two triangles 
$\triangle ABC$, 
$\triangle AB'C'$,
and $k>0$ we have
$B'\in [AB)$,
$C'\in [AC)$,
$AB'=k\cdot AB$ and
$AC'=k\cdot AC$,
then $\triangle ABC\sim\triangle AB'C'$.
\end{thm}

In other words, the Axiom~\ref{def:birkhoff-axioms:4} provides 
a condition which guarantees that two triangles are similar.
Let us formulate three more such {}\emph{similarity conditions}.

\begin{thm}{Similarity conditions}\label{prop:sim}
Two triangles 
$\triangle ABC$ and $\triangle A'B'C'$
are similar if one of the following conditions hold.

(SAS)\index{SAS similarity condition} For some constant $k>0$ we have
$$A B=k\cdot A' B',
\quad  
A C=k\cdot A' C'$$
$$
\text{and}
\quad 
\measuredangle B A C=\pm\measuredangle B' A' C'.$$

(AA)\index{AA similarity condition} The triangle $A' B' C'$ is nondegenerate
and 
$$\measuredangle  A B  C
=
\pm\measuredangle A' B' C',
\quad 
\measuredangle B A C
=
\pm\measuredangle B' A' C'.$$

(SSS)\index{SSS similarity condition} For some constant $k>0$ we have
$$A B=k\cdot A' B',
\quad
A C=k\cdot A' C',
\quad
CB=k\cdot C'B'.$$

\end{thm}

Each of these conditions is proved by applying Axiom~\ref{def:birkhoff-axioms:4} with the SAS, ASA and SSS congruence conditions correspondingly
(see Axiom~\ref{def:birkhoff-axioms:3} and the conditions \ref{thm:ASA}, \ref{thm:SSS}).


\parit{Proof.}
Set $k=\tfrac{AB}{A'B'}$.
Choose points $B''\in [A'B')$ and $C''\in [A'C')$,
so that $A'B''=k\cdot A'B'$ and $A'C''=k\cdot A'C'$.
By Axiom~\ref{def:birkhoff-axioms:4},
$\triangle A'B'C'\z\sim \triangle A'B''C''$.

Applying the SAS, ASA or SSS congruence condition, depending on the case, 
we get $\triangle A'B''C''\cong \triangle ABC$.
Hence the result follows.
\qeds



A bijection from a plane to itself is called \index{angle preserving transformation}\emph{angle preserving transformation} if 
\[\measuredangle ABC= \measuredangle A'B'C'\]
for any triangle $ABC$ and its image $\triangle A'B'C'$.


\begin{thm}{Exercise}\label{ex:angle-preserving-euclid}
Show that any angle-preserving transformation of the plane multiplies all the distance by a fixed constant.
\end{thm}

\section*{Method of similar triangles}
\addtocontents{toc}{Method of similar triangles.}

%???

The similarity of triangles provides a method for solving geometrical problems.
To apply this method, we have to search for pairs of similar triangles and then use the proportionality of corresponding sides and/or equalities of corresponding angles.

{

\begin{wrapfigure}{o}{27mm}
\begin{lpic}[t(-4mm),b(0mm),r(0mm),l(-0mm)]{pics/alt-two(1)}
\lbl[rt]{1,1;$A$}
\lbl[lt]{24,1;$B$}
\lbl[b]{15,24.5;$C$}
\lbl[lb]{22,10;$A'$}
\lbl[rb]{8,12;$B'$}
\end{lpic}
\end{wrapfigure}

The following exercise is a classical illustration of this method.

\begin{thm}{Exercise}\label{ex:sim+foots}
Let $\triangle A B C$ be an \index{acute!triangle}\emph{acute} triangle; that is, all its angles are acute. 
Denote by $A'$ the foot point of $A$ on $(BC)$ and
by $B'$ the foot point of $B$ on~$(AC)$.
Prove that $\triangle A'B'C\sim \triangle ABC$.
\end{thm}

}

\section*{Pythagorean theorem}
\addtocontents{toc}{Pythagorean theorem.}

A triangle is called \index{triangle!right triangle}\emph{right} if one of its angles is right.
The side opposite the right angle is called the \index{hypotenuse}\emph{hypotenuse}. 
The sides adjacent to the right angle are called \index{leg}\emph{legs}.  


\begin{thm}{Theorem}\label{thm:pyth}
Assume $\triangle ABC$ is a right triangle with the right angle at~$C$.
Then
$$AC^2+BC^2=AB^2.$$ 

\end{thm}

\parit{Proof.}
Let $D$ be the foot point of $C$ on~$(AB)$.

\begin{wrapfigure}[4]{o}{44mm}
\begin{lpic}[t(-0mm),b(-1mm),r(0mm),l(2mm)]{pics/pyth(1)}
\lbl[rt]{1,0;$A$}
\lbl[lt]{40,0;$B$}
\lbl[b]{9,19;$C$}
\lbl[t]{9,0;$D$}
\end{lpic}
\end{wrapfigure}

According to Lemma~\ref{lem:perp<oblique},
\begin{align*}
AD&<AC<AB
\intertext{and}
BD&<BC<AB.
\end{align*}
Therefore, $D$ lies between $A$ and $B$;
in particular, 
$$AD+BD=AB.\eqlbl{AD+BD=AB}$$

Note that by the AA similarity condition, we have
$$\triangle ADC\sim\triangle ACB\sim \triangle CDB.$$
In particular,
$$
\frac{A D}{A C}=\frac{A C}{A B}
\quad
\text{and}
\quad
\frac{B D}{B C}=\frac{B C}{B A}.
\eqlbl{BCD}$$

Let us rewrite the two identities in \ref{BCD}:
\begin{align*}
AC^2=AB\cdot AD
\quad
\text{and}
\quad
BC^2=AB\cdot B D.
\end{align*}
Summing up these two identities and applying \ref{AD+BD=AB}, we get
$$AC^2 +BC^2=AB\cdot (AD+ B D)=AB^2.$$
\qedsf

The idea in the proof above appears in the Elements, see \cite[X.33]{euclid},
but the proof given there \cite[I.47]{euclid} is different; 
it uses area method discussed in Chapter \ref{chap:area}.


\begin{thm}{Exercise}\label{ex:pyth}
Assume $A$, $B$, $C$ and $D$ are as in the proof above.
Show that 
$$CD^2=AD\cdot BD.$$

\end{thm}

The following exercise is the converse to the Pythagorean theorem.

\begin{thm}{Exercise}\label{ex:pyth-conv}
Assume $\triangle ABC$ is a triangle such that  
$$AC^2+BC^2=AB^2.$$ 
Prove that the angle at $C$ is right.
\end{thm}


\section*{Angles of triangles}
\addtocontents{toc}{Angles of triangle.}

\begin{thm}{Theorem}\label{thm:3sum}
In any $\triangle A B C$, we have
$$\measuredangle A B C+ \measuredangle  B C A + \measuredangle  C A B \equiv \pi.$$

\end{thm}

\parit{Proof.} 
First note that 
if $\triangle A B C$ is degenerate, then the equality follows from Theorem~\ref{thm:straight-angle}.
Further we assume that $\triangle A B C$ is nondegenerate.

\begin{wrapfigure}{o}{43mm}
\begin{lpic}[t(3mm),b(0mm),r(0mm),l(0mm)]{pics/3-sum(1)}
\lbl[tr]{1,1;$A$}
\lbl[lt]{40,1;$B$}
\lbl[b]{23,25;$C$}
\lbl[lb]{7,3;{\small $\alpha$}}
\lbl[lb]{25.8,3;{\small $\alpha$}}
\lbl[rb]{34.5,2.5;{\small $\beta$}}
\lbl[t]{22.5,17;{\small $\gamma$}}
\lbl[rb]{15.5,2.5;{\small $\beta$}}
\lbl[b]{21,7;{\small $\pm\gamma$}}
\lbl[t]{20,0;$M$}
\lbl[lb]{32,13;$K$}
\lbl[rb]{11,13;$L$}
\end{lpic}
\end{wrapfigure}

Set 
\begin{align*}
\alpha&=\measuredangle C A B,
\\
\beta&=\measuredangle A B C,
\\
\gamma&=\measuredangle B C A.
\end{align*}
We need to prove that 
$$\alpha+\beta+\gamma\equiv\pi.
\eqlbl{eq:a+b+c}$$

Let $K$, $L$, $M$ be the midpoints of the sides $[B C]$, $[C A]$, $[A B]$ respectively.
By
Axiom~\ref{def:birkhoff-axioms:4},
\begin{align*}
\triangle A M L&\sim \triangle A B C,
&
\triangle M B K&\sim \triangle A B C,
&
\triangle L K C &\sim  \triangle A B C
\intertext{and}
L M &=\tfrac12\cdot B C,
&
M K  &=\tfrac12\cdot   C A,
&
 K L &=  \tfrac12\cdot A B.
\end{align*}
According to the SSS condition (\ref{prop:sim}),
$\triangle K L M\sim \triangle ABC$. 
Thus,
$$\measuredangle M K L=\pm \alpha,
\quad  
\measuredangle K L M=\pm\beta,
\quad  
\measuredangle L M K=\pm\gamma.
\eqlbl{eq:pm-a-b-c}$$
According to \ref{thm:signs-of-triug}, the ``$+$'' or ``$-$''  sign is to be
the same in \ref{eq:pm-a-b-c}.

If in \ref{eq:pm-a-b-c} we have ``$+$'', then \ref{eq:a+b+c} follows since
$$\beta+\gamma+\alpha
\equiv
\measuredangle  A M L + \measuredangle L M K + \measuredangle K M B
\equiv
\measuredangle  A M B 
\equiv
\pi
$$

It remains to show that we cannot have ``$-$'' in \ref{eq:pm-a-b-c}.
If this is the case, then the same argument as above gives
$$\alpha+\beta-\gamma\equiv\pi.$$
The same way we get 
$$\alpha-\beta+\gamma\equiv\pi.$$
Adding the last two identities, we get 
$$2\cdot\alpha\equiv0.$$
Equivalently 
$\alpha\equiv\pi$ or $0$;
that is, $\triangle A B C$ is degenerate, a contradiction.
\qeds

{

\begin{wrapfigure}[6]{i}{43mm}
\begin{lpic}[t(-0mm),b(0mm),r(0mm),l(0mm)]{pics/bisector-BA=AD=DC(1)}
\lbl[b]{9,26;$A$}
\lbl[t]{2,1;$B$}
\lbl[t]{38,1;$C$}
\lbl[t]{16,1;$D$}
\end{lpic}
\end{wrapfigure}

\begin{thm}{Exercise}\label{ex:pent}
Let $\triangle ABC$ be a nondegenerate triangle.
Assume there is a point $D\in [BC]$ 
such that $(AD)$ bisects $\angle BAC$ and $BA\z=AD\z=DC$.
Find the angles of $\triangle ABC$. 
\end{thm}


\begin{thm}{Exercise}\label{ex:|3sum|}
Show that 
$$|\measuredangle A B C|+ |\measuredangle  B C A| + |\measuredangle  C A B| = \pi$$
for any $\triangle ABC$.
\end{thm} 

}

\begin{wrapfigure}[5]{o}{30mm}
\begin{lpic}[t(-6mm),b(0mm),r(0mm),l(0mm)]{pics/right-isos(1)}
\lbl[t]{3,0;$A$}
\lbl[br]{15,10;$B$}
\lbl[t]{28,0;$C$}
\lbl[br]{28,19;$D$}
\end{lpic}
\end{wrapfigure}

\begin{thm}{Exercise}\label{ex:right-isos}
Let $\triangle ABC$ be an isosceles nondegenerate triangle with the base~$[AC]$.
Consider the point $D$ on the extension of the side $[AB]$ 
 such that $AB=BD$.
Show that $\angle ACD$ is right.
\end{thm}

\begin{thm}{Exercise}\label{ex:pi/4-isos}
Let $\triangle ABC$ be an isosceles nondegenerate triangle with base~$[AC]$. 
Assume that a circle is passing thru $A$,
centered at a point on $[AB]$ and tangent to $(BC)$ at the point~$X$.
Show that $\measuredangle CAX=\pm\tfrac\pi4$.
\end{thm}



\section*{Transversal property}
\addtocontents{toc}{Transversal property.}

If the line $t$ intersects each line $\ell$ and $m$ at one point, then we say that $t$ is a \index{transversal}\emph{transversal} to $\ell$ and~$m$.
On the diagram below, line $(CB)$ is a transversal 
to $(AB)$ and~$(CD)$.

\begin{thm}{Transversal property}\label{thm:parallel-2} 
$(AB)\parallel(C D)$ if and only if 
$$2\cdot(\measuredangle A B C+\measuredangle B C D)\equiv 0.
\eqlbl{A B C + B C D}$$ 
Equivalently 
$$\measuredangle A B C+\measuredangle B C D
\equiv 
0
\quad
\text{or}
\quad
\measuredangle A B C+\measuredangle B C D
\equiv
\pi;$$ 
in the first case 
$A$ and $D$ lie on the opposite sides of $(BC)$,
in the second case  
$A$ and $D$ lie on the same sides of~$(BC)$.
\end{thm}

\begin{wrapfigure}[6]{o}{22mm}
\begin{lpic}[t(1mm),b(0mm),r(0mm),l(0mm)]{pics/parallel-2(1)}
\lbl[b]{14.5,18;$A$}
\lbl[b]{3,15;$B$}
\lbl[b]{18,7.5;$C$}
\lbl[b]{3,3;$D$}
\end{lpic}
\end{wrapfigure}

\parit{Proof.} 
If $(AB)\nparallel(C D)$,
then there is $Z\in (AB)\cap(C D)$ 
and $\triangle BCZ$ is nondegenerate.

According to Theorem~\ref{thm:3sum}, 
\begin{align*}
\measuredangle ZBC+\measuredangle BCZ&\equiv \pi-\measuredangle CZB\not\equiv 
\\&\not\equiv 0
\quad
\text{nor}
\quad
\pi.
\end{align*}
Note that $2\cdot\measuredangle Z B C\equiv 2\cdot\measuredangle A B C$ and $2\cdot\measuredangle B C Z\equiv2\cdot\measuredangle B C D$.
Therefore, 
$$2\cdot(\measuredangle A B C+\measuredangle B C D)\equiv 2\cdot\measuredangle Z B C +2\cdot\measuredangle B C Z\not\equiv 0;$$
that is, \ref{A B C + B C D} does not hold.

Note that if the points $A$, $B$ and $C$ are fixed,
the identity \ref{A B C + B C D} uniquely defines the line~$(C D)$.
By Theorem~\ref{thm:parallel}, it follows that
if $(AB)\z\parallel(C D)$, then the equality \ref{A B C + B C D} holds.

Applying Corollary~\ref{cor:half-plane},
we get the last part of the transversal property.
\qeds


\begin{thm}{Exercise}\label{ex:smililar+parallel}
Let $\triangle ABC$ be a nondegenerate triangle.
Assume $B'$ and $C'$ are points on sides $[AB]$ and $[AC]$ such that $(B'C')\parallel(BC)$.
Show that $\triangle ABC\sim\triangle AB'C'$.
\end{thm}

\begin{thm}{Exercise}\label{ex:trisection}
Trisect a given segment with a ruler and a compass.
\end{thm}





\section*{Parallelograms}
\addtocontents{toc}{Parallelograms.}

A \index{quadrilateral}\emph{quadrilateral} is an ordered quadruple of distinct points in the plane.
The quadrilateral $ABCD$ will be also denoted by \index{25@$\square$}$\square ABCD$.

Given a quadrilateral $ABCD$,
the four segments $[AB]$, $[BC]$, $[CD]$ and $[DA]$ are called \index{side!side of quadrilateral}\emph{sides of $\square ABCD$};
the remaining two segments $[AC]$ and $[BD]$ are called \index{diagonals of quadrilateral}\emph{diagonals of $\square ABCD$}.


\begin{thm}{Exercise}\label{ex:quadrilateral}
Show that for any quadrilateral $ABCD$, we have
$$\measuredangle ABC+\measuredangle BCD+\measuredangle CDA+\measuredangle DAB\equiv 0.$$

\end{thm}

A quadrilateral $ABCD$ in the Euclidean plane is called \index{quadrilateral!degenerate quadrilateral}\index{degenerate!quadrilateral}\emph{nondegenerate} if any three points from $A,B,C,D$ do not lie on one line.

A nondegenerate quadrilateral $ABCD$ is called a \index{parallelogram}\emph{parallelogram}
if $(AB)\z\parallel (CD)$ and $(BC)\z\parallel (DA)$.

{

\begin{wrapfigure}{o}{32mm}
\begin{lpic}[t(-2mm),b(0mm),r(1mm),l(1mm)]{pics/parallelogram(1)}
\lbl[bl]{29,27.5;$A$}
\lbl[br]{7,27.5;$B$}
\lbl[t]{-.5,.5;$C$}
\lbl[t]{22.5,.5;$D$}
\end{lpic}
\end{wrapfigure}

\begin{thm}{Lemma}\label{lem:parallelogram}
If $\square A B C D$ is a parallelogram, then
\begin{enumerate}[(a)]
\item $\measuredangle D A B= \measuredangle B C D$;
\item $AB=CD$.
\end{enumerate}
\end{thm}


\parit{Proof.}
Since $(AB)\parallel (CD)$,
the points $C$ and $D$ lie on the same side from~$(AB)$.
Hence $\angle ABD$ and $\angle ABC$ have the same sign.

Analogously, 
$\angle CBD$ and $\angle CBA$ have the same sign. 

}

Since $\measuredangle ABC\equiv -\measuredangle CBA$,
we get that the angles $DBA$ and $DBC$ have opposite signs; 
that is,  $A$ and $C$ lie on the opposite sides of~$(BD)$.


According to the transversal property (\ref{thm:parallel-2}), 
$$\measuredangle B D C
\equiv 
-\measuredangle DBA
\quad
\text{and}
\quad 
\measuredangle DBC
\equiv 
-\measuredangle BDA.$$
By the ASA condition
$\triangle A B D\cong \triangle C D B$.
The latter implies both statements in the lemma.
\qeds



\begin{thm}{Exercise}\label{ex:romb}
Assume $ABCD$ is a quadrilateral such that
\[AB=CD=BC=DA.\]
Show that $ABCD$ is a parallelogram.
\end{thm}

A quadrilateral as in the exercise above is called a \index{rhombus}\emph{rhombus}.

\begin{thm}{Exercise}\label{ex:diad-par}
Show that diagonals of a parallelogram intersect each other at their midpoints.
\end{thm}

A quadraliteral $ABCD$ is called a \index{rectangle}\emph{rectangle} if the angles $ABC$, $BCD$, $CDA$ and $DAB$ are right.
Note that according to the transversal property \ref{thm:parallel-2},
any rectangle is a parallelogram.

A rectangle with equal sides is called a \index{square}\emph{square}.

\begin{thm}{Exercise}\label{ex:rectangle}
Show that the parallelogram $ABCD$ is a rectangle
if and only if $AC=BD$.
\end{thm}

\begin{thm}{Exercise}\label{ex:romb2}
Show that the parallelogram $ABCD$ is a rhombus
if and only if $(AC)\perp (BD)$.
\end{thm}

Assume $\ell\parallel m$, and $X,Y\in m$.
Denote by $X'$ and $Y'$ the foot points of $X$ and $Y$ on~$\ell$.
Note that $\square XYY'X'$ is a rectangle.
By Lemma~\ref{lem:parallelogram}, $XX'=YY'$.
That is, any point on $m$ lies on the same distance from $\ell$.
This distance is called the \index{distance!between parallel lines}\emph{distance between} $\ell$ and~$m$.


\section*{Method of coordinates}
\addtocontents{toc}{Method of coordinates.}

The following exercise is important;
it shows that our axiomatic definition agrees with the model definition described on page \pageref{def:d_2}.


\begin{thm}{Exercise}\label{ex:coordinates} 
Let $\ell$ and $m$ be perpendicular lines in the Euclidean plane.
Given a point $P$,  denote by $P_\ell$ and $P_m$ the foot points of $P$ on $\ell$ and $m$ correspondingly.


\begin{enumerate}[(a)]
\item Show that for any $X\in \ell$ and $Y\in m$ there is a unique point $P$ such that $P_\ell=X$ and $P_m=Y$.
\end{enumerate}

\begin{enumerate}[(a)]\addtocounter{enumi}{1}
\item
Show that 
$PQ^2=P_\ell Q_\ell^2+P_mQ_m^2$
for any pair of points $P$ and~$Q$.
\end{enumerate}

\begin{enumerate}[(a)]\addtocounter{enumi}{2}
\item Conclude that the plane is isometric to $(\mathbb{R}^2,d_2)$; see page \pageref{def:d_2}.
\end{enumerate}

\end{thm}

\begin{wrapfigure}[8]{o}{33mm}
\begin{lpic}[t(2mm),b(0mm),r(0mm),l(4mm)]{pics/PQlm(1)}
\lbl[br]{12,10;$P$}
\lbl[bl]{23,22;$Q$}
\lbl[t]{11,0;$P_\ell$}
\lbl[t]{22,0;$Q_\ell$}
\lbl[r]{1,8;$P_m$}
\lbl[r]{1,21;$Q_m$}
\lbl[t]{28,1;$\ell$}
\lbl[b]{1,28,90;$m$}
\end{lpic}
\end{wrapfigure}

Once this exercise is solved, we can apply 
the method of coordinates
to solve any problem in Euclidean plane geometry.
This method is powerful, 
but it is often considered as a bad style.

\begin{thm}{Exercise}\label{ex:abc}
Use the Exercise~\ref{ex:coordinates}
to give an alternative proof of Theorem~\ref{thm:abc} in the Euclidean plane.

That is, prove that given the real numbers $a$, $b$ and $c$ such that 
 $$0<a\le b\le c\le a+c,$$
there is a triangle $ABC$
such that $a=BC$, $b=CA$ and $c=AB$.
\end{thm} 

\begin{thm}{Exercise}\label{ex:line-coord}
Let $(x_A,y_A)$ and $(x_B,y_B)$ be the coordinates of distinct points $A$ and $B$ in the Euclidean plane.
Show that the line $(AB)$ is formed by points with coordinates $(x,y)$ which satisfy the following equation
\[(x-x_A)\cdot (y_B-y_A)=(y-y_A)\cdot (x_B-x_A).\]

\end{thm}










\addtocontents{toc}{\protect\end{quote}}
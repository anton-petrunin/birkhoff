\chapter[Parallel lines]{Parallel lines}\label{chap:angle-sum}
\addtocontents{toc}{\protect\begin{quote}}

\section*{Parallel lines}
\addtocontents{toc}{Parallel lines.}



{

\begin{wrapfigure}{r}{19mm}
\vskip-4mm
\centering
\includegraphics{mppics/pic-72}
\end{wrapfigure}

In consequence of Axiom~\ref{def:birkhoff-axioms:1}, 
any two distinct lines $\ell$ and $m$ have either one point
in common or none. 
In the first case they are \index{intersecting lines}\emph{intersecting} (briefly \index{36@\hskip.4mm$\nparallel$\hskip.4mm}$\ell\nparallel m$); 
in the second case, $\ell$ and $m$ are said to be \index{parallel lines}\emph{parallel} (briefly \index{36@\hskip.4mm$\parallel$\hskip.4mm}$\ell\parallel m$);
in addition, a line is always regarded as parallel to itself.

}

To emphasize that two lines on a diagram are parallel we will mark them with arrows of the same type.



\begin{thm}[\abs]{Proposition}\label{prop:perp-perp} Let $\ell$, $m$, and $n$ be three lines.
Assume that $n\perp m$ and $m\perp \ell$.
Then $\ell\parallel n$. 
\end{thm}

\parit{Proof.}
Assume the contrary; 
that is, $\ell\nparallel n$.
Then there is a point, say $Z$, of intersection of $\ell$ and~$n$.
Then by Theorem~\ref{perp:ex+un},
$\ell=n$.
Since any line is parallel to itself, we have that $\ell\parallel n$ --- a contradiction.
\qeds

\begin{thm}{Theorem}\label{thm:parallel}
For any point $P$ and any line $\ell$
there is a unique line $m$
which passes thru $P$ and is parallel to~$\ell$.
\end{thm}

The above theorem has two parts, existence and uniqueness.
In the proof of uniqueness we will use the method of similar triangles.

\parit{Proof; existence.} 
Apply Theorem~\ref{perp:ex+un} two times,
first to construct the line $m$ thru $P$ which is perpendicular to $\ell$,
and second to construct the line $n$ thru $P$ which is perpendicular to~$m$.
Then apply Proposition~\ref{prop:perp-perp}.

\parit{Uniqueness.}
If $P\in\ell$, then $m=\ell$ by the definition of parallel lines.
Further we assume $P\notin\ell$.

Let us construct the lines $n\ni P$ and $m\ni P$ as in the proof of existence, so $m\parallel \ell$.

Assume there is yet another line $s\ni P$ which is distinct from $m$ and parallel to~$\ell$.
Choose a point $Q\in s$ which lies with $\ell$ on the same side from~$m$.
Let $R$ be the foot point of $Q$ on~$n$.

\begin{figure}[h!]
\centering
\includegraphics{mppics/pic-74}
\end{figure}

Let $D$ be the point of intersection of $n$ and~$\ell$.
According to Proposition~\ref{prop:perp-perp} $(QR)\parallel m$. 
Therefore, $Q$, $R$, and $\ell$ lie on the same side of~$m$. 
In particular, $R\in [P D)$.

Choose $Z\in [P Q)$ such that 
$$\frac{PZ}{PQ}=\frac{PD}{PR}.$$
By SAS similarity condition (or equivalently by Axiom~\ref{def:birkhoff-axioms:4})
we have that $\triangle RPQ\sim \triangle DPZ$;
therefore $(Z D)\perp(P D)$.
It follows that $Z$ lies on $\ell$ and $s$ --- a contradiction.\qeds

\begin{thm}{Corollary}\label{cor:parallel-1}
Assume $\ell$, $m$, and $n$ are lines
such that $\ell\parallel m$ and $m\parallel n$.
Then $\ell\parallel n$.
\end{thm}

\parit{Proof.}
Assume the contrary; that is, $\ell\nparallel n$.
Then there is a point $P\in \ell\cap n$.
By Theorem~\ref{thm:parallel},
$n=\ell$ --- a contradiction.
\qeds

Note that from the definition, we have that $\ell\parallel m$ if and only if $m\z\parallel \ell$.
Therefore, according to the above corollary, ``$\parallel$'' is an 
\index{equivalence relation}\emph{equivalence relation}.
That is, for any lines $\ell$, $m$, and $n$ the following conditions hold:
\begin{enumerate}[(i)]
\item $\ell\parallel \ell$;
\item if $\ell\parallel m$, then $m\parallel \ell$;
\item if $\ell\parallel m$ and $m\parallel n$, then 
$\ell\parallel n$.
\end{enumerate}

\begin{thm}{Exercise}\label{ex:perp-perp}
Let $k$, $\ell$, $m$, and $n$ be lines such that $k\perp \ell$, $\ell\perp m$, and $m\perp n$.
Show that $k\nparallel n$.
\end{thm}

\begin{thm}{Exercise}\label{ex:construction-parallel}
Make a ruler-and-compass construction of a line thru a given point which is parallel to a given line.
\end{thm}

\section*{Reflection across a point}
\addtocontents{toc}{Reflection across a point.}

Fix a point $O$.
If $O$ is the midpoint of a line segment $[XX']$, then we say that $X'$ is a reflection of $X$ across the point $O$.

Note that the map $X\mapsto X'$ is uniquely defined; it is called a \index{reflection!across a point}\emph{reflection} across $O$.
In this case $O$ is called the \index{center!of reflection}\emph{center of reflection}. 
We assume that $O'=O$; that is, $O$ is a reflection of itself across itself.
If the reflection across $O$ moves a set $S$ to itself, then we say that $S$ is \index{central symmetry}\emph{centrally symmetric} with respect to $O$. 

Recall that any motion is either direct or indirect;
that is, it either preserves or reverts signs angles (see page \pageref{direct motion}). 

\begin{thm}[\abs]{Proposition}\label{prop:point-reflection}
Any reflection across a point is a direct motion.
\end{thm}

%???+PIC

\parit{Proof.}
Observe that if $X'$ is a reflection of $X$ across $O$, 
then $X$ is a reflection of $X'$.
In other words, the composition of the reflection with itself is the identity map.
In particular any reflection across a point is a bijection.

\begin{wrapfigure}{o}{33mm}
\centering
\includegraphics{mppics/pic-76}
\end{wrapfigure}

Fix two points $X$ and $Y$;
let $X'$ and $Y'$ be their reflections across $O$.
To check that the reflection is distance preserving, we need to show that $X'Y'=XY$.

We may assume that $X$, $Y$ and $O$ are distinct; otherwise the statement is trivial.
By definition of reflection across $O$, we have that $OX=OX'$, $OY=OY'$, and the angles $XOY$ and $X'OY'$ are vertical;
in particular $\measuredangle XOY\z=\measuredangle X'OY'$.
By SAS, $\triangle XOY\cong\triangle X'OY'$;
therefore $X'Y'=XY$.

Finally, the reflection across $O$ cannot be indirect since $\measuredangle XOY\z=\measuredangle X'OY'$;
therefore it is a direct motion.
\qeds

\begin{thm}{Exercise}
Suppose $\angle AOB$ is right.
Show that the composition of reflections across the lines $(OA)$ and $(OB)$ is a reflection across $O$.

Use this statement and Corollary~\ref{cor:reflection+angle} to build another proof of Proposition~\ref{prop:point-reflection}.
\end{thm}

{

\begin{wrapfigure}{r}{23mm}
\vskip-6mm
\centering
\includegraphics{mppics/pic-78}
\end{wrapfigure}


\begin{thm}{Theorem}\label{thm:parallel-point-reflection}
Let $\ell$ be a line, $Q\in \ell$, and $P$ is arbitrary point.
Suppose $O$ is the midpoint of $[PQ]$.
Then a line $m$ passing thru $P$ is parallel to $\ell$ if and only if $m$ is a reflection of $\ell$ across $O$.
\end{thm}

}

\parit{Proof; ``if'' part.}
Assume $m$ is a reflection of $\ell$ across $O$.
Suppose $\ell\nparallel m$; that is $\ell$ and $m$ intersect at a single point $Z$.
Denote by $Z'$ be the reflection of $Z$ across $O$.

\begin{figure}[h!]
\centering
\includegraphics{mppics/pic-80}
\end{figure}

Note that $Z'$ lies on both lines $\ell$ and $m$.
It follows that $Z'=Z$ or equivalently $Z=O$.
In this case $O\in \ell$ and therefore the reflection of $\ell$ across $O$ is $\ell$ itself;
that is, $\ell=m$ and in particular $\ell\parallel m$ --- a contradiction. 

\parit{``Only-if'' part.}
Let $\ell'$ be the reflection of $\ell$ across $O$.
According to the ``if'' part of the theorem, $\ell'\parallel \ell$.
Note that both lines $\ell'$ and $m$ pass thru $P$.
By uniqueness of parallel line (\ref{thm:parallel}), if $m\parallel \ell$, then $\ell'=m$; whence the statement follows.
\qeds


{

\begin{wrapfigure}{r}{25mm}
\centering
\includegraphics{mppics/pic-82}
\end{wrapfigure}

\section*{Transversal property}
\addtocontents{toc}{Transversal property.}

If the line $t$ intersects each line $\ell$ and $m$ at one point, then we say that $t$ is a \index{transversal}\emph{transversal} to $\ell$ and~$m$.
For example, on the diagram, line $(CB)$ is a transversal 
to $(AB)$ and~$(CD)$.

}

\begin{thm}{Transversal property}\label{thm:parallel-2} 
$(AB)\parallel(C D)$ if and only if 
$$2\cdot(\measuredangle A B C+\measuredangle B C D)\equiv 0.
\eqlbl{A B C + B C D}$$ 
Equivalently 
$$\measuredangle A B C+\measuredangle B C D
\equiv 
0
\quad
\text{or}
\quad
\measuredangle A B C+\measuredangle B C D
\equiv
\pi.$$ 
Moreover, if $(AB)\ne(C D)$, then in the first case 
$A$ and $D$ lie on the opposite sides of $(BC)$,
in the second case 
$A$ and $D$ lie on the same sides of~$(BC)$.
\end{thm}



\parit{Proof; ``only-if'' part.}
Denote by $O$ the midpoint of $[BC]$. %???+PIC

Assume $(AB)\parallel(C D)$.
According to Theorem~\ref{thm:parallel-point-reflection},
$(CD)$ is a reflection of $(AB)$ across $O$.

\begin{wrapfigure}{r}{30mm}
\vskip-4mm
\centering
\includegraphics{mppics/pic-84}
\end{wrapfigure}

Let $A'$ be the reflection of $A$ across $O$.
Then $A'\in (CD)$ and by Proposition~\ref{prop:point-reflection} we have that
\[\measuredangle ABO=\measuredangle A'CO.\eqlbl{A B O = A' C O}\]
Note that 
\[\measuredangle ABO\equiv\measuredangle ABC,
\qquad
\measuredangle A'CO\equiv-\measuredangle BCA'.\eqlbl{eq:A B O= A B C}
\]
Since $A'$, $C$ and $D$ lie on one line, Exercise~\ref{ex:ABCO-line} implis that 
\[2\cdot \measuredangle BCD\equiv 2\cdot \measuredangle BCA'.\eqlbl{eq:2BCD= 2BCA'}\]
Finally note that \ref{A B O = A' C O}, \ref{eq:A B O= A B C}, and \ref{eq:2BCD= 2BCA'} imply \ref{A B C + B C D}.
\qeds

\parit{``If''-part.}
Given points $A$, $B$, and $C$ there is unique line $(CD)$ such that \ref{A B C + B C D} holds.
Indeed, suppose there are two such lines $(CD)$ and $(CD')$, then
$$2\cdot(\measuredangle A B C+\measuredangle B C D)\equiv 2\cdot(\measuredangle A B C+\measuredangle B C D')\equiv0.
$$ 
Therefore 
$2\cdot\measuredangle B C D\equiv 2\cdot\measuredangle B C D'$
and by Exercise~\ref{ex:ABCO-line},  $D'\in (CD)$, or equivalently the line $(CD)$ coincides with the line $(CD')$.

From the ``only-if'' part we know that if \ref{A B C + B C D} holds then $(CD)\parallel(AB)$.
By Theorem~\ref{thm:parallel} there is unique line thru $C$ that is parallel to $(AB)$.
Therefore it must be the same line $(CD)$ such that \ref{A B C + B C D} hold.

\medskip

Finally, if $(AB)\ne(C D)$ and $A$ and $D$ lie on the opposite sides of $(BC)$, then $\angle ABC$ and $\angle BCD$ have opposite signs.
Therefore
\[-\pi\z<\measuredangle A B C+\measuredangle B C D<\pi.\]
Applying \ref{A B C + B C D}, we get $\measuredangle A B C+\measuredangle B C D=0$.

Similarly if $A$ and $D$ lie on the same side of $(BC)$,
then $\angle ABC$ and $\angle BCD$ the same sign.
Therefore
\[0<|\measuredangle A B C+\measuredangle B C D|<2\cdot\pi\]
and \ref{A B C + B C D} implies that $\measuredangle A B C+\measuredangle B C D\z\equiv\pi$.
\qeds

%???)

\begin{thm}{Exercise}\label{ex:smililar+parallel}
Let $\triangle ABC$ be a nondegenerate triangle and $P$ lies between $A$ and $C$.
Suppose that a line $\ell$ passes thru $P$ and parallel to $(AB)$.
Then $\ell$ cross the side $[BC]$ at another point, say $Q$, and 
\[\triangle ABC\z\sim\triangle PQC.\]
In particular, 
\[\frac{PC}{AC}=\frac{QC}{BC}.\]

\end{thm} 

\begin{thm}{Exercise}\label{ex:trisection}
Trisect a given segment with a ruler and a compass.
\end{thm}

\section*{Angles of triangles}
\addtocontents{toc}{Angles of triangle.}



\begin{thm}{Theorem}\label{thm:3sum}
In any $\triangle A B C$, we have
$$\measuredangle A B C+ \measuredangle B C A + \measuredangle C A B \equiv \pi.$$

\end{thm}

%(???

\parit{Proof.} 
First note that 
if $\triangle A B C$ is degenerate, then the equality follows from Corollary~\ref{cor:degenerate=pi}.
Further we assume that $\triangle A B C$ is nondegenerate.

\begin{wrapfigure}{o}{33mm}
\centering
\includegraphics{mppics/pic-86}
\end{wrapfigure} 

Let $X$ be the reflection of $C$ across the midpoint $M$ of $[AB]$.
By Proposition~\ref{prop:point-reflection}
$\measuredangle BAX\z=\measuredangle ABC$.
Note that $(AX)$ is a reflection of $(CB)$ across $M$;
therefore by Theorem~\ref{thm:parallel-point-reflection}, $(AX)\z\parallel (CB)$.

Since $[BM]$ and $[MX]$ do not intersect $(CA)$,
the points $B$, $M$, and $X$ lie on the same side of $(CA)$.
Applying the transversal property for the transversal $(CA)$ to $(AX)$ and $(CB)$, we get that 
\[\measuredangle BCA+\measuredangle CAX\equiv \pi.\eqlbl{eq:ABC+CAB}\]

Since $\measuredangle BAX=\measuredangle ABC$,
we have 
\[\measuredangle CAX\equiv\measuredangle CAB+\measuredangle ABC\]
The latter identity and \ref{eq:ABC+CAB} imply the theorem.\qeds

%???)

{

\begin{wrapfigure}{r}{30mm}
\vskip-4mm
\centering
\includegraphics{mppics/pic-88}
\end{wrapfigure}

\begin{thm}{Exercise}\label{ex:pent}
Let $\triangle ABC$ be a nondegenerate triangle.
Assume there is a point $D\in [BC]$ 
such that 
\[\measuredangle BAD\equiv \measuredangle DAC,
\quad
BA=AD\z=DC.\]
Find the angles of $\triangle ABC$. 
\end{thm}

}

\begin{thm}{Exercise}\label{ex:|3sum|}
Show that 
$$|\measuredangle A B C|+ |\measuredangle B C A| + |\measuredangle C A B| = \pi$$
for any $\triangle ABC$.
\end{thm} 



{

\begin{wrapfigure}{r}{30mm}
\vskip-12mm
\centering
\includegraphics{mppics/pic-90}
\vskip4mm
\includegraphics{mppics/pic-92}
\end{wrapfigure}

\begin{thm}{Exercise}\label{ex:right-isos}
Let $\triangle ABC$ be an isosceles nondegenerate triangle with the base~$[AC]$.
Suppose $D$ is a reflection of $A$ across $B$.
Show that $\angle ACD$ is right.
\end{thm}



\begin{thm}{Exercise}\label{ex:pi/4-isos}
Let $\triangle ABC$ be an isosceles nondegenerate triangle with base~$[AC]$. 
Assume that a circle is passing thru $A$,
centered at a point on $[AB]$,
and tangent to $(BC)$ at the point~$X$.
Show that $\measuredangle CAX\z=\pm\tfrac\pi4$.
\end{thm}

}

\begin{thm}{Exercise}\label{ex:quadrilateral}
Show that for any quadrilateral $ABCD$, we have
$$\measuredangle ABC+\measuredangle BCD+\measuredangle CDA+\measuredangle DAB\equiv 0.$$

\end{thm}

{

\begin{wrapfigure}{r}{26mm}
\vskip2mm
\centering
\includegraphics{mppics/pic-94}
\end{wrapfigure}

\section*{Parallelograms}
\addtocontents{toc}{Parallelograms.}

A quadrilateral $ABCD$ in the Euclidean plane is called \index{quadrilateral!degenerate quadrilateral}\index{degenerate!quadrilateral}\emph{nondegenerate} if no three points from $A,B,C,D$ lie on one line.

}
A nondegenerate quadrilateral  is called a \index{parallelogram}\emph{parallelogram}
if its opposite sides are parallel.



\begin{thm}{Lemma}\label{lem:parallelogram}
Any parallelogram is centrally symmetric with respect to a midpoint of one of its diagolals.

In particular, if $\square A B C D$ is a parallelogram, then
\begin{enumerate}[(a)]
\item its diagonals $[AC]$ and $[BD]$ intersect each other at their midpoints;
\item $\measuredangle A B C= \measuredangle C D A$;
\item $AB=CD$.
\end{enumerate}
\end{thm}

{

\begin{wrapfigure}{r}{33mm}
\centering
\includegraphics{mppics/pic-96}
\end{wrapfigure}%???redo

\parit{Proof.} Let $\square A B C D$ be a parallelogram.
Denote by $M$ the midpoint of $[AC]$.

Since $(AB)\parallel (CD)$, Theorem~\ref{thm:parallel-point-reflection} implies that $(CD)$ is a reflection of $(AB)$ across $M$.
The same way $(BC)$ is a reflection of $(DA)$ across $M$.
Since $\square A B C D$ is nondegenerate, it follows that $D$ is a reflection of $B$ across $M$; in other words, $M$ is the midpoint of $[BD]$.

The reaming statements follow since reflection across $M$ is a direct motion of the plane (see \ref{prop:point-reflection}).
\qeds

}

\begin{thm}{Exercise}\label{ex:romb}
Assume $ABCD$ is a quadrilateral such that
\[AB=CD=BC=DA.\]
Show that $ABCD$ is a parallelogram.
\end{thm}

A quadrilateral as in the exercise above is called a \index{rhombus}\emph{rhombus}.

A quadraliteral $ABCD$ is called a \index{rectangle}\emph{rectangle} if the angles $ABC$, $BCD$, $CDA$, and $DAB$ are right.
Note that according to the transversal property (\ref{thm:parallel-2}),
any rectangle is a parallelogram.

A rectangle with equal sides is called a \index{square}\emph{square}.

\begin{thm}{Exercise}\label{ex:rectangle}
Show that the parallelogram $ABCD$ is a rectangle
if and only if $AC=BD$.
\end{thm}

\begin{thm}{Exercise}\label{ex:romb2}
Show that the parallelogram $ABCD$ is a rhombus
if and only if $(AC)\perp (BD)$.
\end{thm}

Assume $\ell\parallel m$, and $X,Y\in m$.
Let $X'$ and $Y'$ denote the foot points of $X$ and $Y$ on~$\ell$.
Note that $\square XYY'X'$ is a rectangle.
By Lemma~\ref{lem:parallelogram}, $XX'=YY'$.
That is, any point on $m$ lies on the same distance from $\ell$.
This distance is called the \index{distance!between parallel lines}\emph{distance between} $\ell$ and~$m$.


\section*{Method of coordinates}
\addtocontents{toc}{Method of coordinates.}

The following exercise is important;
it shows that our axiomatic definition agrees with the model definition described on page \pageref{def:d_2}.


\begin{thm}{Exercise}\label{ex:coordinates} 
Let $\ell$ and $m$ be perpendicular lines in the Euclidean plane.
Given a point $P$, let $P_\ell$ and $P_m$ denote the foot points of $P$ on $\ell$ and $m$ correspondingly.


\begin{enumerate}[(a)]
\item Show that for any $X\in \ell$ and $Y\in m$ there is a unique point $P$ such that $P_\ell=X$ and $P_m=Y$.
\end{enumerate}

\begin{enumerate}[(a)]\addtocounter{enumi}{1}
\item
Show that 
$PQ^2=P_\ell Q_\ell^2+P_mQ_m^2$
for any pair of points $P$ and~$Q$.
\end{enumerate}

\begin{enumerate}[(a)]\addtocounter{enumi}{2}
\item Conclude that the plane is isometric to $(\mathbb{R}^2,d_2)$; see page \pageref{def:d_2}.
\end{enumerate}

\end{thm}

\begin{wrapfigure}{r}{36mm}
\centering
\includegraphics{mppics/pic-98}
\end{wrapfigure}

Once this exercise is solved, we can apply 
the method of coordinates
to solve any problem in Euclidean plane geometry.
This method is powerful and universal;
it will be developed further in Chapter~\ref{chap:complex}.

\begin{thm}{Exercise}\label{ex:abc}
Use the Exercise~\ref{ex:coordinates}
to give an alternative proof of Theorem~\ref{thm:abc} in the Euclidean plane.

That is, prove that given the real numbers $a$, $b$, and $c$ such that 
 $$0<a\le b\le c\le a+b,$$
there is a triangle $ABC$
such that $a=BC$, $b=CA$, and $c=AB$.
\end{thm} 

\begin{thm}{Exercise}\label{ex:line-coord}
Consider two distinct points $A=(x_A,y_A)$ and $B\z=(x_B,y_B)$ on the coordinate plane.
Show that the perpendicular bisector to $[AB]$ is described by the equation
\[2\cdot (x_B-x_A)\cdot x+2\cdot (y_B-y_A)\cdot y=x_B^2+y_B^2-x_A^2-x_B^2.\]

Conclude that any line is described by an equation
\[a\cdot x+b\cdot y=c\]
for some constants $a$, $b$, and $c$ such that $a\ne 0$ or $b\ne0$.
\end{thm}


\begin{thm}{Exercise}\label{ex:circle-coord}
Show that for fixed real values $a$, $b$, and $c$ the equation 
\[x^2+y^2+a\cdot x+b\cdot y+c=0\]
describes a circle, one-point set or empty set.

Show that if it is a circle then it has center $(-\tfrac a2,-\tfrac b2)$ and the radius $r=\tfrac12\cdot \sqrt{a^2+b^2-4\cdot c}$.
\end{thm}

\begin{thm}{Exercise}\label{ex:apolonnius}
Use the previous exercise to show that given two distinct point $A$ and $B$ and positive real number $k\ne1$,
the locus of points $M$ such that $AM=k\cdot BM$ is a circle. 
\end{thm}

\begin{figure}[h!]
\centering
\includegraphics{mppics/pic-100}
\end{figure}

The circle in the exercise above is an example of the so called \index{Apollonian circle}\emph{Apollonian circle}.
Few of these circles for fixed pair $(A,B)$ and different $k$ are shown on the diagram; for $k=1$, it becomes the perpendicular bisector to $[AB]$.










\addtocontents{toc}{\protect\end{quote}}



The same holds in the sphere and the hyperbolic plane.


Now assume we want to extend area functional to all sets keeping all the 
conditions in Theorem~\ref{thm:area} except uniqueness.

It turns on the unit sphere $\Sigma$ one can find a set
$\mathcal{A}$ such that 5 conguent copies of $\mathcal{A}$
cover whole $\Sigma$ and there are 6 disjoint conguent copies of $\mathcal{A}$ in $\Sigma$.

This example was constructed by Hausdorff in \cite{hausdorff};
it is quite conterintutive 

It follows that one can not define area of $\mathcal{A}$.
Indeed if $s=\area\mathcal{A}$,
the conditions in Theorem~\ref{thm:area} would imply 
\[6\cdot s\le \area \Sigma\le 5\cdot s,\]
a contradiction. 


Similar examples show that one can not define area for all bounded sets in the hyperbolic plane.
A set in the plane is called \index{bounded set}\emph{bounded} if it lies inside a circle.

After these examples it might be surprising to learn that
in the Euclidean plane one can define area for all bounded sets
which satisfies all the conditions in the Theorem~\ref{thm:area} except uniqueness.
In case you are interested in these results, we suggest to read ...


















\section*{Area of disc}
\addtocontents{toc}{Area of disc.}

A \emph{disc} is the set 
formed points inside and on a given circle.

The area .

For example one can define the area of any disc;
that is the set formed by circle and the points inside it.
Let us try to make sense of it.

Fix a large positive integer $n$.
Inscrive in $\Gamma$ the regular $n$-gon $\bm{A}=A_1\dots A_n$.
Surcumscribe $\Gamma$ by a regular $n$-gon $\bm{B}=B_1\dots B_n$.
Note that 
\[\bar{\bm{A}}\subset D\subset\bar{\bm{B}}.\]

It is reasonable to assume that area of $D$ satisfies monotonicity property;
for polygons it is proved in ???.
In this case 
\[\area\bm{A}\le\area D\le \area\bm{B}.\]

One the other hand taking $n$ large enough, 
the difference $\area\bm{B}-\area\bm{A}$ can be made arbitrary small.
Indeed, straightforward computations show that
$\area\bm{A}=\cos\tfrac\pi n\cdot\area\bm{B}$. 
Since $\cos\tfrac\pi n\ge 1-\tfrac2n $ and $\area \bm{A}\le 4$,
we get
\[\area\bm{B}-\area\bm{A}\le \tfrac8n.\]










\section*{Quadrable sets}
\addtocontents{toc}{Quadrable  sets.}

As we showed in the previous section,
one can extend the notion of area to the sets in the plane which can be squized between two bodies of polygons with arbitrary close area.
The class of such sets is called quadrable sets;
in this section we define them formally.

Given a subset  $\mathcal{S}$ in the plane
set
\[\overline\area\,\mathcal{S}=\inf_{\bar{\bm{B}}\supset \mathcal{S}}\{\area\bm{B}\};\]
that is $\bar a=\overline\area\,\mathcal{S}$ is the maximal number such that $\bar a\le \area\bm{B}$ for any polygon $\bm{B}$ such that $\bar{\bm{B}}\supset \mathcal{S}$.

If $\mathcal{S}$ is unbounded
then there is no polygon $\bm{B}$ such that $\bar{\bm{B}}\supset \mathcal{S}$.
In this case we assume $\overline\area\,\mathcal{S}=\infty$.

In a similar way let us define lower area 
\[\underline\area\,\mathcal{S}=\sup_{\bar{\bm{A}}\subset \mathcal{S}}\{\area\bm{A}\};\]

Note that for any set 
\[\underline\area\,\mathcal{S}\le\overline\area\,\mathcal{S}.\]
Indeed, assume contrary. Then there is a real number $a$ such that 
\[\underline\area\,\mathcal{S}>a>\overline\area\,\mathcal{S}.\]
By definition of $\underline\area$ and $\overline\area$ there are 
polygonal sets $\mathcal{P}$ and $\mathcal{Q}$ 
such that $\area\mathcal{P}>a$, $\area \mathcal{Q}<a$
and $\mathcal{P}\subset \mathcal{S}\subset\mathcal{Q}$.
In particular $\mathcal{P}\subset \mathcal{Q}$ 
and $\area\mathcal{P}>\area \mathcal{Q}$;
the later contradicts ???

If
\[\underline\area\,\mathcal{S}=\overline\area\,\mathcal{S}\]
then the set $\mathcal{S}$ is called quadrable;
in this case we define 
\[\area\mathcal{S}\df\underline\area\,\mathcal{S}=\overline\area\,\mathcal{S}\]

The argument above shows that for any disc is a quadrable set.

Let $\mathcal S$ be a set in the plane.
We say that $\mathcal S$ is squareble if 
given $\epsilon>0$ there are collection of disjoint polygons $\bm{A}_1,\dots, \bm{A}_n$ and $\bm{B}_1,\dots, \bm{B}_k$
such that
\[\bar{\bm{A}}_1\cup\dots\cup\bar{\bm{A}}_n\subset\mathcal S\subset \bar{\bm{B}}_1\cup\dots\cup\bar{\bm{A}}_k\]
and 
\[\area\bm{A}_1+\dots+\area\bm{A}_n
>
\area\bm{B}_1+\dots+\area\bm{B}_k-\epsilon.\]

\begin{thm}{Exercse}
Show that any line segment is a quadrable set with vanishing area.
\end{thm}

\begin{thm}{Exercse}
Show that any circle is a quadrable set with vanishing area.
\end{thm}

A plane set $\mathcal{S}$
is called \emph{convex} if for any two points $X,Y\in \mathcal{S}$
we have $[XY]\subset \mathcal{S}$

\begin{thm}{Advanced exercse}
Show that any bounded convex set is quadrable.
\end{thm}



Informally squeezed


\section*{Area of triangle}
\addtocontents{toc}{Area  of triangle.}


A function $F$ which returns a real number 
for any solid triangle is called \emph{area functional} if it satisfies the followng conditions.

\begin{enumerate}
\item For any triangle $\triangle ABC$ we have $F(\triangle ABC)\ge 0$ and 
if  $\triangle ABC$ nondegenerate then $F(\triangle ABC)>0$.
\item If $\triangle ABC\cong\triangle A'B'C'$ 
 then $F(\triangle ABC)=F(\triangle A'B'C')$
\item For any triangle $ \triangle ABC$ and point $D$ between $B$ and $C$, we have
$$F(\triangle ABC)=F(\triangle ABD)+F(\triangle ADC).$$
\end{enumerate}

The proof of the following proposition is quite nontrivial.

\begin{thm}{Proposition}
There is an area functional. 

Moreover it is unique up to multiplication by a positive constant. 
That is, if $F$ and $F'$ are area functionals then there is a positive real number $k$ such that 
\[F'(\triangle ABC)=k\cdot F(\triangle ABC)\]
for any triangle $\triangle ABC$.
\end{thm}

Let $\triangle ABC$ be a right triagle with unit length
and $F$ be an area functional such that $F(\triangle ABC)=\tfrac12$.
Then for any triangle $\triangle XYZ$ the value $a=F(\triangle XYZ)$ is called \emph{area} of $\triangle XYZ$;
brifly $a=\area(\triangle XYZ)$.















\section*{Subdivisions and area}
\addtocontents{toc}{Subdivisions and area.}

\begin{thm}{Definition}
Let $\bm{A}$, $\bm{B}_1,\dots,\bm{B}_k$
be a finite collection of polygons.
We say tath $\bm{A}$ 
is subdivided into $\bm{B}_1,\dots,\bm{B}_k$
if 
\[\bar{\bm{A}}=\bar{\bm{B}}_1\cup\dots\cup\bar{\bm{B}}_k\]
and 
\[\mathring{\bm{B}}_i\cap\mathring{\bm{B}}_j=\emptyset\]
for any $i\ne j$.
\end{thm}

A function $F$ which returns a nondegenerate real value for any poygon 
is called area functional if it satisfies the following conditions.

\begin{enumerate}
\item If  two polygons $\bm{A}$ and $\bm{A}'$ have congruent bodies (that is $\bar{\bm{A}}\cong \bar{\bm{A}}'$)
then 
\[F(\bm{A})=F(\bm{A}')\]
\item If a polygon $\bm{A}$ is subdivided into polygons $\bm{B}_1, \dots,\bm{B}_n$
then 
\[F(\bm{A})=F(\bm{B}_1)+\dots+F(\bm{B}_n)\]
\item If $\bm{B}$ is a square with unit side then 
\[F(\bm{B})=1.\]
\end{enumerate}

The proof of the following theorem is long and not at all simple.

\begin{thm}{Proposition}
There is unique area functional.
\end{thm}

From now on we will denote the area of polygon $\bm{A}$ as $\area \bm{A}$.

Use the definition of area to prove the following two statements.


\begin{thm}{Exercise}
Assume $\square ABCD$ is a parallelogram.
Show that 
\[\area\triangle ABC=\area\triangle BCD.\]

\end{thm}

\begin{thm}{Exercise}
Let $A'$ be the medpoint of side $[BC]$ of the triangle $\triangle ABC$.
Show that 
\[\area\triangle AA'B=\area\triangle AA'C.\]

\end{thm}



















\section*{Two propositions}
\addtocontents{toc}{Two propositions.}

\begin{thm}{Proposition}\label{prop:monotonicity-of-area}
Assume $\bm{A}$ and $\bm{B}$ are two polygons such that $\bar{\bm{A}}\supset \bar{\bm{B}}$.
Then $\area \bm{A}\ge \area \bm{B}$.
\end{thm}

To prove the statement it is sufficient to subdivide $\bm{A}$ into collction of polygons which include $\bm{B}$, say $\bm{B}$, $R_1,\dots,R_n$.
Then by ???
\[\area \bm{A}=\area \bm{B}+\area R_1+\dots +\area R_n.\]
The statement follows since $\area R_i\ge 0$.

Two construct the nesesury subdivision, extend all the sides of $\bm{B}$.
These lines cut $\bm{A}$ into polygons.
Denote by $R_1,\dots,R_n$ those of these polygons which do not lie inside $\bar{\bm{B}}$.
Note that $\bm{B}$, $R_1,\dots,R_n$ form a subdivision of $\bm{A}$.


\begin{thm}{Proposition}\label{prop:1/n}
If $\bm{B}_n$ is a square with side $\tfrac1n$ then 
\[\area \bm{B}_n=\tfrac1{n^2}.\]

\end{thm}

\parit{Proof.}
Set $a_n=\area \bm{B}_n$.
Recall that according to ???, $a_1=1$.

Note that the unit square $\bm{B}_1$ can be divided into $n^2$ squires congruent to $\bm{B}_n$.
Therefore
\[n^2\cdot a_n=a_1=1.\]
Hence the result follows.
\qeds




\section*{Polygons}
\addtocontents{toc}{Polygons.}

A \emph{polygon} is a finite sequence of distrinct points in the plane, say
\[\bm{A} =A_1A_2\dots A_n.\]
The points $A_i$ in the sequence are called \emph{vertices} of the polygon $\bm{A}$.

A polygon with three vertices is called \emph{triangle}
polygon with four vertices is called \emph{quadrilateral}
and a polygon with $n$ vertices is called also \emph{$n$-gon}.

\begin{wrapfigure}{o}{28mm}
\begin{lpic}[t(-4mm),b(0mm),r(0mm),l(0mm)]{pics/polygon(1)}
\lbl[tr]{6,1;$A_1$}
\lbl[tl]{16,1;$A_2$}
\lbl[tl]{17,8;$A_3$}
\lbl[b]{25,17,-75;$\dots$}
\lbl[b]{2,22;$A_n$}
\lbl[r]{22.5,15.5;$\partial\bm{A}\rightarrow$}
\lbl[b]{15.5,17.5;$\uparrow$}
\lbl[rb]{20.5,10.5;$\searrow$}
\lbl[b]{11,22,10;side}
\lbl[b]{8,13,-50;diagonal}
\end{lpic}
\end{wrapfigure}

The segments $[A_nA_1],[A_1A_2],\dots,[A_{n-1} A_n]$
are called \emph{sides} and the remaining segments $[A_iA_j]$ are called \emph{diagonals} of the polygon $\bm{A}$.

The union of all the sides of a polygon $\bm{A}$ is called its contour,
and denoted as $\partial \bm{A}$.

If any two sides of the polygon 
intersect only at the common end 
then the polygon is called  \emph{simple}\index{simple polygon}.

\section*{Interior and body}
\addtocontents{toc}{Interior and body.}

It is intuitively clear that 
the complement to the contour of a simple polygon
has exactly two components 
one is \emph{bounded} 
and the other is \emph{unbounded}. 

Let us make this statement more presise.
First let us define bounded and unbounded sets.

\begin{thm}{Definition}
A set $\mathcal{S}$ in the plane 
is called \emph{bounded} 
if for some (and therefore any) point $Z$ 
there real number $r$ such that $ZX<r$ for any $X\in \mathcal{S}$.

If there is there is no such number $r$, 
then the set is called \emph{unbounded}.
\end{thm}

Now we need to explain the meaning of word \emph{component} above. 

Fix a polygon $\bm{A}$.
We say that two points $X,Y\notin \partial \bm{A}$,
lie
on the same side from $\partial\bm{A}$ (briefly $X\leftrightarrow Y$)
if there is a finite sequence of points
\[X=X_0, X_1,\dots, X_k=Y\]
such that non of the segments $[X_{i-1}X_i]$ intersects $\partial\bm{A}$.
(It means that one can get from $X$ to $Y$ without crossing $\partial\bm{A}$.)

Note that ``$\leftrightarrow$'' is an \emph{equivalent relation}.
That is, 
for any three points $X,Y,Z\notin\partial \bm{A}$,
we have 
(1) $X\leftrightarrow X$,
(2) if $X\leftrightarrow Y$
then  $Y\leftrightarrow X$
and 
(3) if $X\leftrightarrow Y$ 
and $Y\leftrightarrow Z$ 
then  $X\leftrightarrow Z$.

The set of points which lie one the same side with a given one (and therefore with each other) is called \emph{component}.

\begin{thm}{Proposition-Definition}
Let $\bm{A}$ be a simple polygon.
Then the complement of $\partial\bm{A}$
has exactly two components,
one bounded and the other is unbounded. 

\begin{center}
\begin{lpic}[t(-4mm),b(0mm),r(0mm),l(0mm)]{pics/polygons(1)}
\lbl[tr]{6,1;$A_1$}
\lbl[tl]{16,1;$A_2$}
\lbl[tl]{17,8;$A_3$}
\lbl[b]{25,17,-75;$\dots$}
\lbl[b]{2,22;$A_n$}
\lbl[r]{22.5,15.5;$\partial\bm{A}\rightarrow$}
\lbl[b]{15.5,17.5;$\uparrow$}
\lbl[rb]{20.5,10.5;$\searrow$}
\lbl[b]{11,22,10;side}
\lbl[b]{8,13,-50;diagonal}
\end{lpic}
\end{center}

The bounded component is called \emph{interior of polygon} and denoted as $\mathring{\bm{A}}$
and the unbounded component is called \emph{exterior of polygon}.
The union of interior and the contour is called \emph{body of the the polygon} it will be denoted as $\bar{\bm{A}}$, so 
\[\bar{\bm{A}}=\mathring{\bm{A}}\cup \partial\bm{A}.\]
\end{thm}

We omit the proof of the proposition; it is not hard but tedious.

\section*{Congruence}
\addtocontents{toc}{Congruence.}

By now we have three important sets
contour $\partial\bm{A}$,
interior $\mathring{\bm{A}}$
and the body $\bar{\bm{A}}$
associated with any simple polygon $\bm{A}$.
Note that these sets for different polygons might coincide.
For example polygon $\bm{A}=A_1A_2\dots A_n$ is different from
the polygon $\bm{A}'=A_2\dots A_nA_1$, 
so 
\[\bm{A}'\ne \bm{A},
\ \ \text{but}\ \ 
\partial\bm{A}'=\partial\bm{A},\ \  
\mathring{\bm{A}}'=\mathring{\bm{A}},\ \ 
\bar{\bm{A}}'=\bar{\bm{A}}.\]

Similarly polygons and their bodies behave with respect to congruence,
which we about to define.

\begin{thm}{Definition}
Two polygons $\bm{A}$ and $\bm{B}$
are called congruent (briefly $\bm{A}\cong\bm{B}$)
if they have the same number of vertices 
and there is a motion of the plane 
which sends every vertex of $\bm{A}$ to the correspondig vertex of $\bm{B}$.
That is, $\bm{A}=A_1\dots A_n$ and $\bm{B}=B_1\dots B_n$
for some positive integer $n$ and there is a motion $f$
such that $B_i=f(A_i)$ for every $i$.
\end{thm}

\begin{thm}{Definition}\label{def:cong-sets}
Two sets $\mathcal{S}$ and $\mathcal{T}$ in the plane  
are called congruent 
(briefly $\mathcal{S}\cong \mathcal{T}$)
if 
$\mathcal{T}=f(\mathcal{S})$ for some motion $f$ of the plane.
\end{thm}

Applying this definition with the identity map $f$,
we get 
\[\partial\bm{A}'\cong\partial\bm{A},\ \  
\mathring{\bm{A}}'\cong\mathring{\bm{A}},\ \ 
\bar{\bm{A}}'\cong\bar{\bm{A}}\]
where polygons $\bm{A}$ and $\bm{A}$ as above.

On the other hand $\bm{A}'\ncong\bm{A}$.















\section*{Scissors-congruence}
\addtocontents{toc}{Scissors-congruence.}

We say that two subsets $X$ and $Y$ 
of the plane are congruent (briefly $X\cong Y$)
if there is a motion of the plane $m$
such that $Y=m(X)$.

Recall that $\triangle ABC\cong\triangle A'B'C'$ means that there is a motion of the plane, say $m$, 
such that $A'=m(A)$, $B'=m(B)$ and $C'=m(C)$.
Now for the sets of their vertices,
$\{A,B,C\}\cong\{A',B',C'\}$ means that there is a motion $m$ such that $\{A',B',C'\}=m(\{A,B,C\})$.
For example it might happen that $A'=m(B)$, $B'=m(C)$ and $C'=m(A)$
in this case $\triangle ABC\cong\triangle C'A'B'$ and in general $\triangle ABC\not\cong\triangle A'B'C'$.

We say that polygon $P$ is subdivided into polygons $\bm{B}_1,\dots, \bm{B}_n$
if $P=\bm{B}_1\cup\dots\cup \bm{B}_n$ and no pairs of polygons $\bm{B}_i$, $\bm{B}_j$ have common interior point.

Two polygons $P$ and $P'$ are called scissors-congruent
if one can subdivide $P$ into polygons $\bm{B}_1,\dots, \bm{B}_n$
and $P'$ into polygons $\bm{B}_1',\dots, \bm{B}_n'$
in such a way that $\bm{B}_i\cong \bm{B}_j$.



The following theorem is qute nontrivial.

\begin{thm}{Theorem}
There is a function $S$ which returns a nonnegative real number for any polygon $P$ such that two polygons $P$ and $P'$ are scissors-congruent
if and only if $S(P)=S(P')$.
\end{thm}

Note that if $S$ is a function satisfying the theorem then so is any function $S'(P)=k\cdot S(P)$ with $k>0$.
Given a polygon $\bm{B}$ we can take $k=\tfrac{1}{S(\bm{B})}$,
in this case $S'(\bm{B})=1$;
in other words $\bm{B}$ will become the unit of area.

\parbf{Comment.}
Analogous theoerem holds for sphere and hyperbolic plane.
In the $3$-dimensional Euclidean space the theorem does not hold.


\section*{Defining properties of area}
\addtocontents{toc}{Defining properties of area.}

We say that a solid polygon $P$ is subdivided 
into two polygons $P_1$ and $P_2$ 
if $P=P_1\cup P_2$ 
and the polygons $P_1$ and $P_2$ do not have common interior points. 

A function $S$ 
which returns a real number for any solid polygon is called \emph{area}
if it is satisfies the following conditions.

\begin{enumerate}[(i)]
\item\label{area:positive} The area of any polygon is nonnegative.
\item\label{area:additive} If polygon $P$ is subdivided in two polygons $P_1$ and $P_2$ then 
\[S(P)=S(P_1)+S(P_2).\]
\item\label{area:equal} If two solid polygons $P_1$ and $P_2$ are congruent then $S(P_1)=S(P_2)$
\item\label{area:unit} Unit square has unit area; that is $S(\bm{B})=1$ if $\bm{B}$ a unit squre.
\end{enumerate}


\begin{thm}{Proposition}
There is unique are functional for polygons in the Euuclidean plane. 
\end{thm}

We omit the proof of this proposition;
it is quite technical and tedious. 

\parbf{Remark.}
Analogous statement hold for hyperbolic geometry, but the normalization property has to be formulated differently.
For example one can state it as 
\[\lim_{\epsilon\to0}\frac{S(T_\epsilon)}{\epsilon^2}=\tfrac12,\]
where 
$T_\epsilon$ is right the isosceles triangle with legs $\epsilon$.


\begin{thm}{Corollary}
Let $P$ and $\bm{B}$ be twopolygons and $P\subset \bm{B}$.
Then $\area P\le \area \bm{B}$.
\end{thm}

To prove the above corollary one has to subdivide $\bm{B}$ into into polygons
such that $P$ is one of them and apply additivity of area.
corrolary above 
follows from properties 
(\ref{area:positive}) and (\ref{area:additive}).






\section*{Area functional}
\addtocontents{toc}{Area functional.}

A function $F$ which returns a real number 
for any triangle is called \emph{area functional} if it satisfies the followng conditions.

\begin{enumerate}
\item For any triangle $\triangle ABC$ we have $F(\triangle ABC)\ge 0$ and 
if  $\triangle ABC$ nondegenerate then $F(\triangle ABC)>0$.
\item If $\triangle ABC\cong\triangle A'B'C'$ 
 then $F(\triangle ABC)=F(\triangle A'B'C')$
\item For any triangle $ \triangle ABC$ and $D\in [BC]$ we have
$$F(\triangle ABC)=F(\triangle ABD)+F(\triangle ADC).$$
\end{enumerate}

The proof of the following proposition is quite nontrivial.

\begin{thm}{Proposition}
There is an area functional. 

Moreover it is unique up to multiplication by a positive constant. 
That is, if $F$ and $F'$ are area functionals then there is a positive real number $k$ such that 
\[F'(\triangle ABC)=k\cdot F(\triangle ABC)\]
for any triangle $\triangle ABC$.
\end{thm}

Let $\triangle ABC$ be a right triagle with unit length
and $F$ be an area functional such that $F(\triangle ABC)=\tfrac12$.
Then for any triangle $\triangle XYZ$ the value $a=F(\triangle XYZ)$ is called \emph{area} of $\triangle XYZ$;
brifly $a=\area(\triangle XYZ)$.






\section*{Area of polygons}
\addtocontents{toc}{Area of polygons.}

We say that point $X$ belongs to the \emph{interior} of traingle $\triangle ABC$
if $A$ and $X$ lie on the same side from the line $(BC)$,
$B$ and $X$ lie on the same side from the line $(CA)$
and 
$C$ and $X$ lie on the same side from the line $(AB)$.

Note that according to the definition,
a degenerate triangle has emplty interior.

The union of interior of $\triangle ABC$ and its sides $[AB]$, $[BC]$, $[CA]$
will be called triangular set and denoted by $\bar\triangle ABC$.
Recall that $\triangle ABC$ is just an ordered triple of distinct points in the plane,
while $\bar\triangle ABC$ is a set.
In particular $\bar\triangle ABC=\bar\triangle BAC$, while $\triangle ABC\ne\triangle BAC$.

We say that two trianglar set do not \emph{overlap} if the interiors of the corresponding triangles do not interect.
The nonoverlaping trianglar set might intesect along s subsets of its sides, in particular they might share a vertex.

Assume that  $\bar\triangle_1,\dots,\bar\triangle_n$ a finite collection of pairwise nonvoerlapping triangles
and $S$ is the union of all their interiours and sides.
In this case we say that $S$ is subdivided into triangular sets $\bar\triangle_1,\dots,\bar\triangle_n$.
The set which admit a subdivision into triangles are called \emph{polygonal sets}.

\begin{thm}{Proposition-Definition}
Assume that polygonal set $S$ admits two subdivision into triangles
$\triangle_1,\dots,\triangle_n$
and $\triangle'_1,\dots,\triangle'_k$
then
\[\area\triangle_1+\dots+\area\triangle_n
=\area\triangle'_1+\dots+\area\triangle'_k.\]

In particular, we can define the area of $S$ using the following formula
\[\area S=\area\triangle_1+\dots+\area\triangle_n.\]

\end{thm}

Instead of giving the proof of the proposition above we will give one illustration s an exercis;
in fact this exercise leads to the proof of the proposition.

\begin{thm}{Exercise}
Assume that the diagonals of quadrilateral $ABCD$ intersect at the point $M$.
Show that
\[\area\triangle ABC+\area\triangle CDA=\area\triangle BCD+\area\triangle DAB.\]
\end{thm}

Assume that diagonals of quadrilateral $ABCD$ interect at a point $M$.
In this case the union of triangular sets $\bar\triangle ABC$ and 

Assume $S$ is a \emph{polygonal sets} subdivided into triangles sets $\bar\triangle_1,\dots,\bar\triangle_n$.
Set 
\[\area S=\area\bar\triangle_1+\dots+\area\triangle_n.\]
We claim that the right hand side does not depend on the choice of the subdivision that is, if $S$ admits an other subdivided into triangles sets $\bar\triangle_1',\dots,\bar\triangle_k'$
then 
\[\area\bar\triangle'_1+\dots+\area\bar\triangle'_k
=
\area\bar\triangle_1+\dots+\area\bar\triangle_n.\]


Assume the point $D$ lies the vertices $B$ and $C$ of the $\triangle ABC$.
Then trianglar set $\bar\triangle ABC$ admits a subdivision into  $\bar\triangle ABD$ and $\bar\triangle ACD$.

Further, given a subdivision of polygonal set into trianglar sets $\bar\triangle_1,\dots,\bar\triangle_n$,
we can subdivide as above 
any trianglar set, say $\bar\triangle_1$,
into two $\bar\triangle_1'$ and $\bar\triangle_1''$.
Then 
$\bar\triangle_1',\bar\triangle_1'',\bar\triangle_2\dots,\bar\triangle_n$
forms a new subdivision of $S$.
We also can revert this construction if the union of two triangular sets 
say $\bar\triangle_i\cup \bar\triangle_j$ forms a triangular set.
These two operations produce a new subdivision from the old one \emph{elmentary move}.

According to ???,
the elementary moves do not change the total sum of the areas of the triangular sets in the subdivision.
In particular if two one subdivision of polygonal set can be obtained from the other by applying a sequence of elementary moves then 

The following proposition 



The proposition above makes possible to define area of polgonal sets
as the sum of areas of all the tringle in its subdivision into triangles.


\begin{thm}{Exercise}
Assume $\blacktriangle ABC\supset \blacktriangle XYZ$.
Show that 
\[\area \triangle ABC\ge \area \triangle XYZ\]

\end{thm}


Any finite union of solid triangles segments and points is called polygonal set.

A point $X$ in a polygonal set $P$ is called interior point of $P$ 
if for some $\epsilon>0$ all the points $Y$ in the plane such that $XY<\epsilon$ lie in $P$.

We say that a polygonal sets $P$ 
is subdivided into polygonal sets $\bm{B}_1,\dots, \bm{B}_n$
if $P=\bm{B}_1\cup\dots\cup \bm{B}_n$
and no pair $\bm{B}_i$, $\bm{B}_j$ has common interior point.

We say that a polygonal set $P$ 
is smaller than a polygonal set $P'$
if $P$ admis a subdivision into $\bm{B}_1,\dots, \bm{B}_n$
and $P'$ contains 

Two polygonal sets called 













%%%%%%%%%%%%%%%

To prove the converse, let us argue by contradiction.
Assume that \ref{eq:inscribed-angle} holds for some $P\notin \Gamma$.
Note that $\measuredangle X O Y\ne 0$ and therefore $\measuredangle X P Y$ is distinct from $0$ and $\pi$;
that is, $\triangle PXY$ is nondegenerate.

\begin{center}
\begin{lpic}[t(0mm),b(0mm),r(0mm),l(0mm)]{pics/inscribed-angle-6(1)}
\lbl[rb]{11,41;$P'$}
\lbl[r]{10,29;$P$}
\lbl[l]{43,22;$X$}
\lbl[rt]{11,3;$Y$}
\lbl[rb]{21,23;$O$}
%%%%
\lbl[t]{49,10;$P$}
\lbl[l]{93,22;$X$}
\lbl[rt]{61,3;$Y$}
\lbl[rb]{71,23;$O$}
\end{lpic}
\end{center}

If the line $(PY)$ is secant to $\Gamma$, denote by $P'$ the point of intersection of $\Gamma$ and $(PY)$ which is distinct from $Y$.
From above we get 
$$2\cdot\measuredangle X P' Y
\equiv
\measuredangle X O Y.$$
In particular, 
$$2\cdot\measuredangle X P' Y
\equiv
2\cdot\measuredangle X P Y.$$
By Transversal property (\ref{thm:parallel-2}),
$(P'X)\parallel (PX)$.
Since $\triangle PXY$ is nondegenerate,
the later implies $P=P'$, which contradicts $P\notin \Gamma$.

In the remaining case, if $(PY)$ is tangent to $\Gamma$,
the proof goes along the same lines.
Namely, by Theorem~\ref{thm:tangent-angle},
$$2\cdot\measuredangle P Y X
\equiv
\measuredangle Y O X.$$
Therefore 
$$2\cdot(\measuredangle X P Y + \measuredangle P Y X)
\equiv0.$$

By Transversal property~\ref{thm:parallel-2},
$(PY)\z\parallel (XY)$;
therefore $(PY)\z= (XY)$.
That is, $\triangle PXY$ is degenerate,
a contradiction.













\parbf{Exercise~\ref{ex:center-proj}.}
Follows from Theorem \ref{thm:circle-center-proj} the same way as the exercises~\ref{ex:affine-perp} and \ref{ex:midpoint-proj}.








\parbf{Exercise~\ref{ex:affine+circles}.}
According to Exercise~\ref{ex:center-circ-affine}, the transformation has to send center of the circle to the center of its image.

Assume that the transformation is given as
\[\beta\:\left(\begin{smallmatrix}
x\\ y
\end{smallmatrix} \right)
  \mapsto
  \left(\begin{smallmatrix}
a&b\\ c&d
\end{smallmatrix} \right)
  \cdot
  \left(\begin{smallmatrix}
x\\ y
\end{smallmatrix} \right)
  +
\left(\begin{smallmatrix}
v\\ w
\end{smallmatrix} \right).
\]
Applying the parallel translation we may assume that the origin did not move; 
that is, $\left(\begin{smallmatrix}
v\\ w
\end{smallmatrix} \right)=0$.
Applying a homothety we may assume that unit circle centered at the origin is mapped to itself.
Since $\beta\left(\begin{smallmatrix}
1\\ 0
\end{smallmatrix} \right)
=
\left(\begin{smallmatrix}
a\\c 
\end{smallmatrix} \right)$,
$\beta\left(\begin{smallmatrix}
0\\ 1
\end{smallmatrix} \right)
=
\left(\begin{smallmatrix}
b\\d 
\end{smallmatrix} \right)$ 
and 
$\beta\left(\begin{smallmatrix}
1/\sqrt{2}\\ 1/\sqrt{2}
\end{smallmatrix} \right)
=
\left(\begin{smallmatrix}
(a+b)/\sqrt{2}\\(c+d)/\sqrt{2} 
\end{smallmatrix} \right)$
we get
\begin{align*}
a^2+c^2&=1\\
b^2+d^2&=1\\
(a+b)^2+(c+d)^2&=2.
\end{align*}
It follows that the matrix takes one of two forms $\bigl(\begin{smallmatrix}
a&b\\ -b&a
\end{smallmatrix} \bigr)$
or 
$\bigl(\begin{smallmatrix}
a&b\\ b&-a
\end{smallmatrix} \bigr)$ where $a^2+b^2=1$;
that is, the map is a motion of the plane.




\parbf{Exercise~\ref{ex:Moebius=inversive}.}
Follows from
Exercise~\ref{ex:invesion-Mob}.













\begin{thm}{Proposition}
The identity map in a group of transformations on set $\mathcal X$ can be distinguished
from all the other transformations as the unique transformation $\iota$ such that
\[\iota\circ\alpha=\alpha
\eqlbl{eps-circ-alpha}\] 
for any transformation $\alpha$.
\end{thm}

\parit{Proof.}
Assume $\iota$ is the identity map;
that is, $\iota(x)=x$ for any  $x$.
Then 
\[\iota\circ\alpha(x)=\iota(\alpha(x))=\alpha(x).\]
Hence \ref{eps-circ-alpha} follows.

Assume $\iota'$ is a transformation which satisfies \ref{eps-circ-alpha}.
Then 
\[\iota'(x)=\iota'(\iota(x))=\iota'\circ\iota(x)=\iota(x).\]
The later implies $\iota'=\iota$.
\qeds














The group motions of plane (Euclidean or hyperbolic or absolute) is equipped with natural binary operation, given two motions $\alpha$ and$\beta$ one can take its composition $\alpha\circ\beta$.
Also, for any motion $\alpha$,
its inverse $\alpha^{-1}$ is a motion of the plane.












\parit{Second proof.}
Consider such a configuration in the Euclidean space.
Assume in addition that the lines $(AA')$, $(BB')$ and $(CC')$ do not lie in one plane and intersect at point $O$.

Denote by $(PQR)$ the plane passing through three points.

Nothe that
\begin{align*}
(AB)&= (OAB)\cap (ABC),&
(A'B')&= (OAB)\cap (A'B'C'),\\
(BC)&= (OBC)\cap (ABC),&
(B'C')&= (OBC)\cap (A'B'C'),\\
(CA)&= (OCA)\cap (ABC),&
(C'A')&= (OCA)\cap (A'B'C').
\end{align*}
Further
\begin{align*}
X&=(BC)\cap(B'C')=(OBC)\cap (ABC)\cap (A'B'C'),\\
Y&=(CA)\cap(C'A')=(OCA)\cap (ABC)\cap (A'B'C'),\\
Z&=(AB)\cap(A'B')=(OAB)\cap (ABC)\cap (A'B'C').
\end{align*}
Note that $(ABC)\cap (A'B'C')$ is a line and
\[X,Y,Z\in (ABC)\cap (A'B'C');\]
that is, the points are collinear. 
Passing  to the orthogonal projection of this configuration to a plane proves proves the theorem.

Applying a perspective projection,
the case $(AA')\parallel(BB')\parallel(CC')$ 
can be reduced to the one discussed above.
\qeds

















The idealized ruler can be used only to draw a line through given two points. (The ruler is assumed to be infinite in length, and it has no markings on it and only one edge.)
The idealized ruler and compass can be used only to draw a circle with given center and radius.
We may always mark an additional point in the plane
as well as on the constructed sets, lines or circles or on their intersections.
Doing this we have no control on the choice the additional point. 

We can also look at the different set of instruments, 
for example
you may only use the ruler;
you may invent your own instruments, for example an instrument which produce midpoint for given two points, 
or produce a circle or line which pass through given three points.
Thanks to number of computer programs which were developed in the last few years, one can play with these these instruments as well.

Any construction problem starts with the \index{initial configuration}\emph{initial configuration},
it is finite collection of points, lines, circles
and asked for an algorithm of constructing some other configuration of points.

A construction is a finite sequence of the following five basic constructions which use the points, lines and circles that have already been constructed. 
These are:
\begin{enumerate}[(a)]
\item Draw the line through two marked points;
\item Draw the circle through one marked point with center at another marked point;
\item Mark the point which is the intersection of two constructed lines (if they intersect);
\item Mark the one or two points in the intersection of a line and a circle (if they intersect);
\item Mark the one or two points in the intersection of two circles (if they intersect);
\end{enumerate}

The rules above describe the classical compass-and-ruler constructions. 
But one can look at the different instruments, for example
you may only consider ruler,
you may invent your own instruments, for example an instrument which produce midpoint for given two points, or produce a circle or line which pass through given three points.
Thanks to number of computer programs which were developed in the last few years, one can play with these these instruments as well. 


Use the following links to see the following standard constructions.
\begin{itemize}
\item \href{run:./car/segment-bisector.html}{Bisection of a segment.}
\item \href{run:./car/angle-bisector.html}{Bisection of a angle.}
\item \href{run:./car/perpendicular-1.html}{Perpendicular to a line from an external point.}
\item \href{run:./car/perpendicular-2.html}{Perpendicular to a line at a point on the line.}
\item \href{run:./car/5-sect.html}{Dividing a line segment into 5 equal segments.}
\end{itemize}

\begin{thm}{Exercises}
Use the following links to java applets to do the following constructions. 
\begin{itemize}
\item Euclidean geometry: 
\href{run:./car/parallel.html}{parallel line}, 
\href{run:./car/trisection.html}{segment trisection}, 
\href{run:./car/incircle.html}{incircle of triangle},
\item Inversive geometry: 
\href{run:./car/inverse.html}{inverse of a point}, 
\href{run:./car/h-line.html}{perpedicular circle}, 
\href{run:./car/perpendicular-circles.html}{perpedicular circle 2}
\item Hyperbolic geometry: 
\href{run:./car/h-center.html}{h-center}, 
\href{run:./car/h-triangle.html}{perpendicular bisectors of h-triangle}, 
\href{run:./h-equal.html}{h-circle}.
\end{itemize}
\end{thm}
















\parbf{Exercise~\ref{ex:car-midpoint}.}
\begin{enumerate}[1.]
\item Draw a circle with center $A$ through $B$;
\item Draw a circle with center $B$ through $A$;
\item Draw a line $\ell$ through the points of interections of the constructed cuircles.
\item The intersection of $\ell$ and $[AB]$ is the midpoint of $[AB]$.
\end{enumerate}
\begin{center}
%\begin{wrapfigure}[11]{o}{28mm}
\begin{lpic}[t(-3mm),b(0mm),r(0mm),l(0mm)]{pics/ex-midpoint(1)}
\lbl[r]{22.5,25;$A$}
\lbl[l]{48,25;$B$}
\lbl[l]{36.5,34;$\ell$}
\end{lpic}
%\end{wrapfigure}
\end{center}










($\Rightarrow$).
If $A$, $B$, $C$ and $D$ lie on one line then both $\ang ABC$ and $\ang CDA$ are zero or straight. 
Hence \ref{eq:inscribed-4angle} follows.

If $A$, $B$, $C$ and $D$ lie on one circle,
denote by $O$ the center of the circle.
According to Theorem~\ref{thm:inscribed-angle},
\begin{align*}
2\cdot\measuredangle ABC
&\equiv\measuredangle AOB,
\\
2\cdot\measuredangle CDA
&\equiv\measuredangle BOA.
\end{align*}
Adding these two identities, we get \ref{eq:inscribed-4angle}.

\parit{}($\Leftarrow$).
Assume \ref{eq:inscribed-4angle} holds.


If $\triangle ABC$ is degenerate
then according to Corollary~\ref{cor:degenerate=pi}, $\measuredangle ABC=0$ or $\pi$.
From \ref{eq:inscribed-4angle}, $\measuredangle CDA=0$ or $\pi$.
Applying Corollary~\ref{cor:degenerate=pi} again,
we get $B,D\in (AC)$; 
that is, all points $A$, $B$, $C$ and $D$ lie on one line.

If $\triangle ABC$ is nondegenerate,
denote by $\Gamma$ the circumcircle of $\triangle ABC$.
Applying the ``only if'' part of Theorem~\ref{thm:inscribed-angle}, we get the result.












From \ref{eq:inscribed-4angle}, $\measuredangle ADC\not\equiv 0$ nor $\pi$;
therefore $\triangle CDA$ is nondegenerate and in particular

\begin{wrapfigure}{o}{45mm}
\begin{lpic}[t(-7mm),b(0mm),r(0mm),l(0mm)]{pics/inscribed-angle-4a(1)}
\lbl[r]{1,21;$A$}
\lbl[rb]{6,34;$B$}
\lbl[lb]{40,32;$C$}
\lbl[tl]{30,1.5;$D'$}
\lbl[l]{35,13;$D$}
\end{lpic}
\end{wrapfigure}

$$(DA)\ne (CD).\eqlbl{eq:DAnpCD}$$


Now we need to consider two cases. 

\parit{Case 1.} Assume $(CD)$ is secant of $\Gamma$.
Denote by $D'$ be the point of intersection $\Gamma$ and $(CD)$ which is distinct from $C$.
From \ref{eq:inscribed-4angle} and the ``only if''-part, we get
$$2\cdot \measuredangle ADC\equiv2\cdot \measuredangle AD'C.$$
According to Corollary~\ref{cor:parallel-2}, 

\begin{wrapfigure}{o}{45mm}
\begin{lpic}[t(-5mm),b(0mm),r(0mm),l(0mm)]{pics/inscribed-angle-4b(1)}
\lbl[r]{0,21;$A$}
\lbl[rb]{5,35;$B$}
\lbl[lb]{43,20;$C$}
\lbl[l]{41,8;$D$}
\end{lpic}
\end{wrapfigure}

$$(DA)\z\parallel(D'A).$$
Hence $(DA)\z=(D'A)$. 
From \ref{eq:DAnpCD} we get $D=D'$;
hence the result follows.

\parit{Case 2.} Now assume $(CD)$ is tangent of $\Gamma$.
Let $O$ denotes the center of $\Gamma$.
Then
by Theorem~\ref{thm:tangent-angle}
$$2\cdot\measuredangle ACD\equiv \measuredangle AOC.$$


According to Theorem~\ref{thm:inscribed-angle},
\begin{align*}
2\cdot\measuredangle ABC&\equiv\measuredangle AOC.
\end{align*}
Therefore
\begin{align*}
2\cdot\measuredangle DCA+2\cdot\measuredangle ABC\equiv0.
\end{align*}

According to Corollary~\ref{cor:parallel-2}, $(DC)\z\parallel(DA)$. 
Therefore $D\in (AC)$ which contradicts \ref{eq:DAnpCD}.
















First note that if $(AD)\perp(BC)$ then by angle-side-angle condition $\triangle A D B\cong\triangle ADC$.
In particular $AB=AC$ and $DB=DC$; hence \ref{bisect-ratio} follows.

In the remaining case we may assume that $\ang B D A$ is obtuse (otherwise switch labels of $B$ and $C$).

Choose a point $D'\not=D$ on $(AD)$ such that $B D\z=B D'$ (it exists due to Lemma~\ref{lem:tangent}).

Since $\measuredangle DAB\equiv-\measuredangle DAC$,
the point
$D$ lies between $C$ and $B$.
Since $\triangle BDD'$ is isosceles,
angle $\ang BDD'$ is acute.
On the other hand $\ang B D A$ is obtuse, 
which implies that $D$ lies between $A$ and $D'$.

In particular the angles $\ang C D A$ and $\ang B D D'$ are vertical.
Since $\triangle D B D'$ is isosceles, we have 
\begin{align*}
\measuredangle B D' A&\equiv \measuredangle B D' D\equiv
\\
&\equiv -\measuredangle B D D'\equiv
\\
&\equiv -\measuredangle C D A
\end{align*}
Since $(AD)$ bisects $\ang B A C$, we have $\measuredangle DAB\equiv -\measuredangle D A C$.
Applying the AA condition, we get $\triangle ADC\sim \triangle A D' B$.
Thus,
$$\frac{DB}{DC}=\frac{D' B}{DC}=\frac{AB}{AC}.$$
\qedsf









\section*{Area of spherical triangle\footnote{This is a slightly modified version of ``Area of a spherical triangle''  by Thomas Foregger, Igor Khavkine, John Smith. Freely available at \href{http://planetmath.org/AreaOfASphericalTriangle.html}{PlanetMath.org}.}}
\addtocontents{toc}{Area of spherical triangle.}

\begin{wrapfigure}[9]{o}{31mm}
\begin{lpic}[t(0mm),b(-0mm),r(0mm),l(0mm)]{pics/s-trig-my-1(1)}
\end{lpic}
\end{wrapfigure}

A spherical triangle is formed by connecting three points, its vertices, 
on the surface of a sphere with great arcs.
The spherical triangle with vertices $A$, $B$ and $C$ will be denoted by $\triangle_s ABC$.

The angle at each vertex is measured 
as the angle between the tangents to the sides.

The following theorem is analogous to Corollary~\ref{cor:area-h-trig}.

\begin{thm}{Theorem}\label{thm:s-trig-area}
Let $\triangle_s ABC$ be a spherical triangle on a unit sphere.
Denote by $\alpha$, $\beta$ and $\gamma$ the internal angles of $\triangle_s ABC$ 
at $A$, $B$ and $C$ correspondingly.
Then
$$\area (\triangle_s ABC) = \alpha+\beta+\gamma-\pi.$$

\end{thm}

\begin{thm}{Corollary}
The sum of the internal angles of a spherical
triangle is greater than or equal to $\pi$.
\end{thm}

\begin{thm}{Corollary}
Let $\mathcal{P}$ be a spherical $n$-gon 
with the internal angles $\alpha_1,\z\dots,\alpha_n$.
Then 
$$\area\mathcal{P}=\alpha_1+\dots+\alpha_n-\pi\cdot(n-2).$$

\end{thm}

In the following proof, we use that area of unit sphere is $4\cdot\pi$.
This fact follows from the formula for area of surface of revolution;
we do not present these calculations here.

On the picture above, you see a spherical triangle $\triangle_sABC$.
Note that by continuing the sides of the original triangle into
great circles, another spherical triangle is formed. The triangle $\triangle_sA'B'C'$
is antipodal to $\triangle_sABC$; it can be obtained by reflecting $\triangle_sABC$ through $O$. 
By symmetry, both triangles must
have the same area.

\parit{Proof.}
The \index{spherical diangle}\emph{spherical diangle} is a subset of the sphere which is bounded by two great arcs that intersect in two points, which must lie on a diameter. 

Six diangles
with vertices on the diameters $AA'$, $BB'$ and $CC'$ are shown on the picture.

\begin{figure}[h]
\includegraphics[scale=0.3]{pics/img4}\ \ 
\includegraphics[scale=0.3]{pics/img5}\ \ 
\includegraphics[scale=0.3]{pics/img6}
\end{figure}

Clearly, a diangle occupies an area that is proportional to the absolute value of the angle measure. 
Since the area of the sphere
is $4\cdot\pi$, 
the area of the diangles are $2\cdot\alpha$, $2\cdot\beta$ and $2\cdot\gamma$ correspondingly.

Note that these 6 diangles cover the entire sphere while overlapping
only on the triangles $\triangle_s ABC$ and $\triangle_s A'B'C'$;
the diangles cover these triangles three times and the remaining part of sphere only once.
Therefore 
$$
4\cdot\pi + 4\cdot\area(\triangle_s ABC)= 4\cdot \alpha+4\cdot \beta+4\cdot\gamma.
$$
Hence, the result follows.
\qeds













\begin{thm}{Theorem}\label{thm:sum-trig-h}
Given three positive real numbers $\alpha$, $\beta$ and $\gamma$ such that
$$\alpha+\beta+\gamma<\pi$$ 
there is a hyperbolic triangle $\triangle_hABC$
such that with
$$|\measuredangle_hCAB|=\alpha,\ \ |\measuredangle_hABC|=\beta\ \ \text{and}\ \ |\measuredangle_hBCA|=\gamma.\eqlbl{eq:abc}$$
 
Moreover this triangle is uniquely determined up to congruence. 
\end{thm}

%???PIC

\parit{Proof.}
Set 
$$\epsilon=(\pi-\alpha-\beta-\gamma)/2.$$
Note that there is a Euclidean triangle $\triangle ABC$ such that 
$$\measuredangle CAB=\alpha,\ \ \measuredangle ABC=\beta+\epsilon\ \ \text{and}\ \ \measuredangle BCA=\gamma+\epsilon.$$

Let us construct an arc $BQC$
such that if $[BX)$ and $[CY)$ are tangents half-lines to arc $BQC$
at $B$ and $C$ then $\measuredangle CBX=\measuredangle ACY=-\epsilon$.
Denote by $\Gamma$ the circle containing arc $BQC$.

Note that $O$ lies outside of $\Gamma$.
Draw a line $(AW)$ which is tangent to $\Gamma$ at $W$.
In this case the circle $\Omega$ centered at $A$  and passing through $W$
is perpendicular to $\Gamma$.
Note that $\triangle_h ABC$ in the model with absolute $\Omega$ satisfies \ref{eq:abc}. 

To prove the last statement, first note that we can assume that $A$ is the center of absolute.
In this case, if $\triangle_hABC$ satisfies \ref{eq:abc}
then $\triangle ABC$ has angles with measures 
$\pm \alpha$, $\pm \beta+\epsilon$ and $\gamma+\epsilon$
and the arc forming $[BC]_h$ is completely determined by  $B$, $C$ and $\epsilon$.
Therefore if $W$ is an ideal point of $(BC)_h$ then $\tfrac {AB}{AW}$ and $\tfrac {AC}{AW}$ are completely determined by $\alpha$, $\beta$ and $\gamma$.
Therefore so are the h-distances $AB_h$ and $AC_h$. 
It remains to apply SAS congruence condition in h-plane.
\qeds












\section*{Existence of area}

\begin{thm}{Proposition}
Given a finite collection of polygonal regions $\mathcal{R}_1,\mathcal{R}_2,\dots, 
\end{thm}


Fix small $\epsilon>0$.
Given a polygonal region $\mathcal R$,
consider the maximal integer number $n$ such that there are $n$ points $P_1,P_2,\dots P_n$ in $\mathcal R$ such that $P_iP_j>\epsilon$ for all $i\ne j$.
Set $\pack_\epsilon \mathcal R=n$.

\begin{thm}{Exercise}
Show that $\pack_\epsilon \mathcal R$ is well defined integer number for any polygonal region.
\end{thm}

\begin{thm}{Exercise}\label{ex:bounded}
For any polygonal region $\mathcal R$ there is a constant $C$ such that
$$\epsilon^2\cdot\pack_\epsilon \mathcal R\le C$$
for any $\epsilon>0$.

Moreover if $\mathcal R$ is nondegenerate then there is a constrant $c>0$ such that 
$$\epsilon^2\cdot\pack_\epsilon \mathcal R> c$$
for any $\epsilon>0$.
\end{thm}

There is a set-theoretical monster-construction
which named \emph{ultralimit}.
Given a bounded sequence of real numbers $(x_n)=(x_1,x_2,\dots)$,
it gives a real number $\mathfrak{F}(x_n)$ such that
\begin{itemize}
\item $x=\mathfrak{F}(x_n)$ is a partial limit of the sequence $x_1,x_2,\dots$; 
that is, $x$ is a limit of a subsequence of $x_1,x_2,\dots,x_n$.
\item $\mathfrak{F}(x_n+y_n)=\mathfrak{F}(x_n)+\mathfrak{F}(y_n)$;
that is, given two bounded sequences of real numbers $(x_n)$ and $(y_n)$, consider sequence $(z_n)$ such that $z_n=x_n+y_n$ for each $n$ then $\mathfrak{F}(z_n)=\mathfrak{F}(x_n)+\mathfrak{F}(y_n)$.
\item $\mathfrak{F}(c\cdot x_n)=c\cdot \mathfrak{F}(x_n)$;
that is, given two bounded sequence of real numbers $(x_n)$ and a real number $c$, consider sequence $(z_n)$ such that $z_n=c\cdot x_n$ for each $n$ then $\mathfrak{F}(z_n)=c\cdot \mathfrak{F}(x_n)$.
\end{itemize}

The construction is not unique. 
For the sequence $x_n=(-1)^n$,
depending on the construction we might have $\mathfrak{F}(x_n)=1$ or $\mathfrak{F}(x_n)=-1$.

We will need only existence of one such construction
and we are not going to prove. %REF???

Note that if $x_n$ converges then
$$\mathfrak{F}(x_n)=\lim_{n\to\infty} x_n.$$
The ultralimits behave unpredictably when you pass to a subsequence.
 while
$\mathfrak{F}(x_n)=1$.
In fact it is true that if $\mathfrak{F}(x_n)$ does not depend on the choice of  $\mathfrak{F}$ then the sequence $(x_n)$ is converging.

Let $\mathcal R$ be a polygonal region.
Consider sequence $(x_n)$ with elements
$$x_n=\frac{\pack_{1/n} \mathcal R}{n^2}.$$
Note that according to the Exercise~\ref{ex:bounded} the sequence $x_n$ is bounded.
Define 
$$s(\mathcal R)=\mathfrak{F}(x_n).$$

Note that $x_n\ge 0$ for any $n$, therefore 
$$s(\mathcal R)\ge 0$$
for any polygonal region $\mathcal R$.
According to Exercise~\ref{ex:bounded},
$$s(\mathcal R)> 0$$
for any nongedenerate polygonal regions.


\begin{thm}{Exercise}
Let $\mathcal R_1$ and $\mathcal R_2$ be  two polygonal regions.
Show that there is a constant $C$ such tat
\begin{align*}
\pack_\epsilon \mathcal R_1 +\pack_\epsilon\mathcal R_2 &-C\cdot\epsilon\le
\\
\le
\pack_\epsilon (\mathcal R_1&\cup\mathcal R_2) +\pack_\epsilon (\mathcal R_1\cap\mathcal R_2)
\le 
\\
&\le
\pack_\epsilon \mathcal R_1 +\pack_\epsilon\mathcal R_2 +C\cdot\epsilon.
\end{align*}

Conclude that 
$$s(\mathcal{R}_1)+s (\mathcal{R}_2)=s(\mathcal{R}_1\cup\mathcal{R}_2)+s(\mathcal{R}_1\cap\mathcal{R}_2)$$
for any two polygonal regions $\mathcal R_1$ and $\mathcal R_2$.
\end{thm}

Denote by $\mathcal{Q}_1$ the unit square.
From the last exercise it follows that if one sets
$$\area\mathcal{R}=s(\mathcal{R})/s(\mathcal{Q}_1),$$
then the obtained functional satisfies all the properties of area.

\section*{Uniqueness of area}

Once the existence of area functional is proved we can use the properties of area alone to find the area of different polygonal regions.

\begin{thm}{Theorem}
Any degenerate polygonal region has zero area.
In particular, the are of segment is zero.
\end{thm}


\begin{thm}{Theorem}
The area of rectangular region with sides $a$ and $b$ is equal to $a\cdot b$. 
\end{thm}


\begin{thm}{Theorem}
Area of triangular region is equal to ...
\end{thm}

Finally note that any polygonal region can be presented as a union of finite number of triangles in such a way that distinct triangle do not have common points inside.
Applying the 

???

Consider triangle $\triangle ABC$.
Recall that altitude at $A$ is the distance from $A$ to its foot point on $(BC)$
Denote the altitude at $A$, $B$ and $C$ by $h_a$, $h_b$ and $h_c$ correspondingly.
Set $a=BC$, $b=CA$ and $c=AB$.

\begin{thm}{Exercise}
Show that 
$$h_a\cdot a=h_b\cdot b=h_c\cdot c.$$
\end{thm}


Let us define area of triangle $\triangle ABC$ the following way.
$$\area \triangle ABC= \tfrac12\cdot h_a\cdot a.
$$
From the above exercise we get that 

$$\area \triangle ABC=\area \triangle BAC=\area ACB=\dots$$


\begin{thm}{Exercise}
Assume $D$ lies between $B$ and $C$.
Show that 
$$\area \triangle ABC=\area \triangle ABD+\area \triangle ADC$$
\end{thm}



Given triangle $\triangle ABC$ denote by
$h_a$ to be the distance from $A$ to its foot point on $(BC)$;
the value $h_a$ sometimes called altitude 

Let us first define area of triangle as 


\section*{Existence of area in Euclidean plane}

\section*{Existence of area in hyperbolic plane}




\section*{The sum of angles of polygon} 
\addtocontents{toc}{The sum of angles of polygon.}

In this section we work in an absolute plane;
that is, we do not want to specify is it Euclidean or hyperbolic geometry.

A \index{broken line}\emph{broken line} $A_1A_2\dots A_n$
is the union of segments $[A_1A_2]$, $[A_2A_3]$, ..., $[A_{n-1}A_n]$
and  \index{broken line!closed broken line}\emph{closed broken line} $A_1A_2\dots A_n$
is the union of segments $[A_1A_2]$, $[A_2A_3]$, ..., $[A_{n-1}A_n]$, $[A_nA_1]$.
A closed broken line  $A_1A_2\dots A_n$ is called \index{broken line!simple closed broken line}\emph{simple} if the segments  $[A_1A_2]$, $[A_2A_3], \z\dots,[A_{n-1}A_n]$ intersect each other only at the common ends.

The set $\mathcal{P}$ bounded by simple closed broken line $A_1A_2\dots A_n$ is called
\index{polygon}\emph{polygon} or \index{$n$-gon}\emph{$n$-gon}.

Here is a formal definition, $X\in\mathcal{P}$ if $X$ lie on the closed broken line $A_1A_2\dots A_n$ or one (and therefore any) half-line $[XY)$ which does not pass through any of $A_i$, intersects the broken line $A_1A_2\dots A_n$ at odd number of points.

In this case
\begin{itemize}
\item the points $A_1$, $A_2$, ..., $A_n$ are called \index{vertex of polygon}\emph{vertices} of $\mathcal{P}$;
\item the segments $[A_1A_2]$, $[A_2A_3]$, ..., $[A_{n}A_1]$ 
are called 
\emph{side of polygon}\emph{sides} of $\mathcal{P}$;
\item the remaining segments of the form $[A_iA_j]$ are called  
\index{diagonal of polygon}\emph{diagonals} of $\mathcal{P}$.
\end{itemize}

The  $n$-gon $\mathcal{P}$ can be always divided by diagonals into $(n-2)$ triangles $\triangle A_iA_jA_k$.
This fact is not completely obvious, but it is not hard to prove.

Without loss of generality, we may assume that all the angles $\ang A_iA_jA_k$,
$\ang A_jA_kA_i$ and $\ang A_kA_iA_j$ in these triangles are positive.

Denote by $\alpha_i$ the sum of angles of all these triangles at vertex $A_i$.
Note that $\alpha_i$ takes value in $(0,2\cdot\pi)$ and it does not depend on choice of subdivision into triangles.
The angle $\alpha_i$ is called 
\index{angle!internal angle}\emph{internal angle} of $\mathcal{P}$ at $A_i$.

By Theorem~\ref{thm:sum-trig-h}, the sum of angles of each nondegenerate triangle in the subdivision is less than $\pi$.
Since the $n$-gon is divided into $n-2$ triangles, 
it follows that if $\mathcal{P}$ is a polygon in h-plane then
$$\alpha_1+\alpha_2+\dots+\alpha_n< (n-2)\cdot\pi.$$
In Eulcidean plane, according to Theorem~\ref{thm:3sum}, we have 
$$\alpha_1+\alpha_2+\dots+\alpha_n=(n-2)\cdot\pi.$$
Therefore the value
$$S(\mathcal{P})=\pi\cdot(n-2)-\alpha_1-\alpha_2-\dots-\alpha_n$$
is positive for any $n$-gon in the hyperbolic plane and zreo for any $n$-gon $\mathcal{P}$ in Euclidean plane .

\begin{thm}{Lemma}\label{lem:n-sum-angle}
The functional $S$ defined above satisfies additivity; 
that is, if a polygon $\mathcal{P}$ is cut by a broken line into two polygons 
$\mathcal{Q}_1$ and $\mathcal{Q}_2$ then
$$S(\mathcal{P})=
S(\mathcal{Q}_1)+S(\mathcal{Q}_2).$$

\end{thm}

\begin{wrapfigure}{o}{52mm}
\begin{lpic}[t(-10mm),b(0mm),r(0mm),l(-6mm)]{pics/two-poly(1)}
\lbl[]{20,25;$\mathcal{Q}_1$}
\lbl[]{35,33;$\mathcal{Q}_2$}
\end{lpic}
\end{wrapfigure}

\parit{Proof.}
Assume $\mathcal{P}$ be an $n$-gon
and the cutting the broken line has $m$ vertices inside $\mathcal{P}$.
Assume polygon $\mathcal{Q}_1$ has $n_1$ vertex shared with of $\mathcal{P}$ 
and $\mathcal{Q}_2$ has $n_2$ vertex shared with of $\mathcal{P}$.

Thus $\mathcal{Q}_1$ is an $(n_1+m)$-gon and $\mathcal{Q}_2$ is an $(n_2+m)$-gon
and 
$$n_1+n_2=n+2.\eqlbl{n1+n2}$$

Denote by $\phi$ the sum of internal angles of $\mathcal{P}$ 
and by $\psi_i$  the sum of internal angles of $\mathcal{Q}_i$.
Note that each internal angle of $\mathcal{Q}_i$ either contributes to $\phi$ or it is an angle of the broken line.
Further, each of vertex of the broken line inside of $\mathcal{P}$ contributes $2\cdot\pi$ to $\psi_1+\psi_2$.
Hence 
$$\phi+2\cdot m\cdot\pi=\psi_1+\psi_2.\eqlbl{phi1+phi2}$$
Further,
$$\begin{aligned}
S(\mathcal{P})&=(n-2)\cdot\pi-\phi
&
S(\mathcal{Q}_i)&=(n_i+m-2)\cdot\pi-\psi_i.
\end{aligned}\eqlbl{S(P)}$$
The identities \ref{n1+n2}, \ref{phi1+phi2} and \ref{S(P)} imply the result.
\qeds



\section*{Area}\label{def:area}
\addtocontents{toc}{Area.}

Area is a function of the set of all polygons in the plane (Euclidean or hyperbolic).
Which satisfies the following conditions:

\begin{enumerate}[(a)]
\item\label{area:0} $\area\mathcal P>0$ for any polygon $\mathcal P$.
\item\label{area:1} If $\mathcal P$ and $\mathcal P'$ are congruent polygons then
$$\area\mathcal P=\area\mathcal P'.$$
\item\label{area:2}  Additivity:
 if a polygon $\mathcal{P}$ is cut by a broken line into two polygons 
$\mathcal{Q}_1$ and $\mathcal{Q}_2$ then
$$\area\mathcal{P}=
\area\mathcal{Q}_1+\area\mathcal{Q}_2.$$
\item\label{area:3}  Normalzation\footnote{In Euclidean geometry this condition is equivalent to the fact that the unit square has area $1$. In hyperbolic plane there are no squares, so we can not normalize the area this way.}: if $a(\epsilon)$ is the area of right triangle with legs equal $\epsilon$ then
$$\lim_{\epsilon\to0} \tfrac{a(\epsilon)}{\epsilon^2}=\tfrac12.$$
\end{enumerate}

In Euclidean geometry, existence of area has a  long and tedious proof.
By that reason generic textbooks 
take the existence of area as an ``obvious fact''.
That is also the reason why we did not mention area up to now in this course.

Once the existence of area is proved, 
one can derive formulas for area of triangles and use it to show uniqueness of area functional.
One could also extend the definition of area to a bigger class of sets,
in particular one can talk about area of set bounded by a circle.

In hyperbolic geometry,
proving the existence of area functional is easy;
one can take 
$$\area\mathcal{P}= S(\mathcal{P})$$
where $S(\mathcal{P})$ is defined in the previous section.

Let us check that the conditions (\ref{area:0})--(\ref{area:3}).
The condition (\ref{area:0}) is proved in the previous section.
The condition (\ref{area:1}) is obvious.
The condition (\ref{area:2}) follows from Lemma~\ref{lem:n-sum-angle}.
The condition (\ref{area:3}) can be checked by direct calculations.
\qeds


The area functional is unique (we do not prove it but it is true).
Therefore further we can always assume that
$$\area\mathcal{P}= S(\mathcal{P})$$
for any polygon $\mathcal{P}$ in h-plane.


\begin{thm}{Corollary}\label{cor:area-h-trig}
Let $\Delta$ be a triangle in h-plane with absolute values of the internal angles $\alpha$, $\beta$ and $\gamma$.
Then 
$$\area\Delta=\pi-\alpha-\beta-\gamma.$$

In particular, by Theorem~\ref{thm:3sum-h},
area of any triangle in h-plane is less than $\pi$ and it is equal to $\pi$ for the ideal triangle.
\end{thm}

\begin{thm}{Exercise}\label{ex:area-sup}
Let $ABCD$ be a quadrilateral in h-plane such that the h-angles at $A$, $B$ and $C$ are right and $AB_h=BC_h$.
 Find the optimal upper for $\area(ABCD)$
\end{thm}

















\parbf{Exercise~\ref{ex:area-sup}.}
Note that 
\begin{align*}
\area ABCD
&=
2\cdot\pi-|\measuredangle ABC|-|\measuredangle BCD|-|\measuredangle CDA|-|\measuredangle DAB|=
\\
&=\tfrac{\pi}2-|\measuredangle CDA|.
\end{align*}
The value $|\measuredangle CDA|$ can be arbitrary close to $0$.
Therefore $\tfrac{\pi}2$ is the optimal upper bound.









;
it is
Consider the two lines $m$ and $n$ passing through $P$ and tangent to the circle containing $\ell$.
Note that the intersections of $m$ and $n$ with the h-plane are h-lines, which do not intersect $\ell$.
Hence Axiom~h-$\!$\ref{def:birkhoff-axioms:4} follows.










\parit{Proof.} 
Follows from Exercise~\ref{ex:centers-of-perp-circles}.
Note that given a point $Z$ outside of $\Omega$,
there is unique circle $\Gamma$ centered at $Z$ and perpendicular to $\Omega$.

Indeed, it is sufficient to show that the radius of $\Gamma$ is uniquely determined by $Z$ and $\Omega$.
By Lemma~\ref{lem:tangent}, 
if $T$ is a point of intersection of the circles 
$\Gamma$ and $\Omega$ then $\ang OTZ$ is right.
Clearly $ZT$ is the radius of $\Gamma$.
By Pythagorean theorem, 
\begin{align*}
ZT^2&=ZO^2-OT^2=
\\
&=ZO^2-r^2,
\end{align*}
where $r$ is the radius of $\Omega$.

\begin{wrapfigure}{o}{43mm}
\begin{lpic}[t(-5mm),b(0mm),r(0mm),l(0mm)]{pics/PO-Gamma(1)}
\lbl[br]{6,31.5;$\Omega$}
\lbl[b]{40,31.5;$\Gamma$}
\lbl[t]{18,16;$O$}
\lbl[t]{33,16;$P$}
\lbl[t]{40,16;$Z$}
\lbl[b]{33,30;$T$}
\end{lpic}
\end{wrapfigure}

If in addition $O$ is the inversion of $P$ in $\Gamma$ then 
$Z\in (OP)$ and $OZ\cdot PZ=ZT^2$.
In particular, $\triangle OTZ\sim \triangle TPZ$.

Since $\ang OTZ$ is right, 
so are $\ang ZPT$ and $\ang OPT$.
It follows that $\triangle OPT\sim OTZ$ 
since $\ang POT$ is shared.
In particular,
$$OP\cdot OZ=OT^2;$$
that is, $Z$ is the inversion of $P$ in $\Omega$

Summarizing, we get that a circle $\Gamma$ satisfies the condition of the lemma
if and only if $\Gamma\perp\Omega$
and
its center  $Z$ is the inversion of $P$ in $\Omega$.
Hence the result follows.
\qeds













\parbf{Exercise~\ref{ex:hat-d-and-+}.} 
Notice that for a fixed $\alpha$,
 the map $\beta\mapsto f(\alpha,\beta)$ is a motion of $((-\pi,\pi],\hat d)$.
The same way, for fixed $\beta$ the map $\alpha\mapsto f(\alpha,\beta)$ is a motion of $((-\pi,\pi],\hat d)$.

Applying triangle inequality, we get
\begin{align*}
\hat d(f(\alpha_n,\beta_n),f(\alpha,\beta))
&\le\hat d(f(\alpha_n,\beta_n),f(\alpha_n,\beta)
+
\hat d(f(\alpha_n,\beta),f(\alpha,\beta)
=
\\
&=
\hat d(\beta_n,\beta)+\hat d(\alpha_n,\alpha).
\end{align*}
In particular, $\alpha_n\to\alpha$ and $\beta_n\to \beta$
implies that $f(\alpha_n,\beta_n)\to f(\alpha,\beta)$
in $((-\pi,\pi],\hat d)$ as $n\to\infty$.














\section*{Congruent sets}
\addtocontents{toc}{Congruent sets.}



(In a more abstract language this means that  the set of motions $\mathcal X\to\mathcal X$ forms a {}\emph{group}.
This group is called \index{group of motions}\emph{group of motions} of $\mathcal X$.
The identity map $\mathcal X\to\mathcal X$ serves as the unit element  and composition is the group operation.)

\begin{thm}{Definition}
Two subsets $V,W\subset \mathcal X$ are called \index{congruent subsets}\emph{congruent} (briefly \index{$\cong$}$V\z\cong W$)
if there is a motion $f\:\mathcal X\to\mathcal X$ which sends $V$ to $W$.
\end{thm}

It is evident from the above that ``$\cong$'' is an  \index{equivalence relation}\emph{equivalence relation};
that is, 
\begin{itemize}
\item $V\cong W$ if and only if $W\cong V$ and 
\item if $U\cong V$ and $V\cong W$ then $U\cong W$.
\end{itemize}


Note that congruent sets have to be isometric to each other.
The following example shows that on the Manhattan plane $(\mathbb{R}^2,d_1)$ isometric subsets might be not congruent.

On the other hand,
any pair isometric sets of Euclidean plane $(\mathbb{R}^2,d_2)$ are congruent.
We will not use, but it might be useful to know.



therefore all the sets $\{A,B\}$, $\{A,C\}$ and $\{B,C\}$ are isometric to each other.
Note that the map $f\:(x,y)\mapsto (-x,y)$ is a motion of $(\mathbb{R}^2,d_1)$; 
$f(A)=A$ and $f(B)=C$.
Thus the set $\{A,B\}$ is congruent to the set $\{A,C\}$.









\section*{Metric on $\bm{(-\pi,\pi]}$}
\addtocontents{toc}{Metric on $(-\pi,\pi]$.}

Note that for any real number $\alpha$ there is unique value $\beta\in(-\pi,\pi]$ such that
$\beta\equiv \alpha$.
By that reason we may assume that angle measure takes value in the real interval $(-\pi,\pi]$

We need to introduce a metric $\hat{d}$ on $(-\pi,\pi]$
different from the metric induced from $\mathbb{R}$.
Further we will use this metric in the formulation 
of continuity of angle measure.
Note that angles with measures $\pm(\pi-\epsilon)$ are ``geometrically close'' for small $\epsilon>0$;
so, in this metric $\hat{d}$, the values $\pm(\pi-\epsilon)$ have to be close to each other.

Given two real numbers 
$\alpha,\beta\in (-\pi,\pi]$,
set \label{def:max-dist}\index{$\hat{d}$}
$$\hat{d}(\alpha,\beta)
=
\min\{|\alpha-\beta|,2\cdot\pi-|\alpha-\beta|\}.
\eqlbl{eq:def-hat-d}
$$

Next we will show that $\hat{d}$ is a metric.

\begin{thm}{Proposition}
The function $\hat{d}(\alpha,\beta)$ is a metric on the set $(-\pi,\pi]$.
\end{thm}

\parit{Proof.}
Among the conditions in the Definition~\ref{def:metric-space}, we need to prove only triangle inequality
$$\hat{d}(\alpha,\gamma)\le \hat{d}(\alpha,\beta)+\hat{d}(\beta,\gamma);
\eqlbl{eq:trig-inq-hat-d}$$
the rest of conditions are  evident.

Note that if $\alpha-\beta\equiv \alpha'-\beta'$
for some values 
$\alpha,\beta,\alpha',\beta'\in (-\pi,\pi]$
then 
$$\hat{d}(\alpha,\beta)
=
\hat{d}(\alpha',\beta').
\eqlbl{eq:hat-d+shift}$$
Indeed, without loss of generality we may assume that $\delta=\alpha-\beta>0$;
since $\alpha-\beta\equiv \alpha'-\beta'$,
we have $\alpha'-\beta'$ may only take  values $\delta$, or $\delta-2\cdot\pi $.
In both cases \ref{eq:hat-d+shift} follows from \ref{eq:def-hat-d}.

Therefore we can assume in \ref{eq:trig-inq-hat-d}
that $\beta=0$;
that is, we need to prove that
$$\hat{d}(\alpha,\gamma)
\le 
\hat{d}(\alpha,0)+\hat{d}(0,\gamma);
$$
Or equivalently,
$$\min\{|\alpha+\gamma|,2\cdot\pi-|\alpha+\gamma|\}\le |\alpha|+|\gamma|.$$
The later is true since 
$|\alpha+\gamma|\le |\alpha|+|\gamma|$.
\qeds

\begin{thm}{Exercise}\label{ex:hat-d-and-+}
Consider the map 
$f(\alpha,\beta)$ which returns a number in $(-\pi,\pi]$ for any pair of numbers $\alpha,\beta\in (-\pi,\pi]$
such that 
$$f(\alpha,\beta)\equiv \alpha+\beta.$$
(By now, it should be evident that $f$ is uniquely defined.)

Show that $f$ is continuous with respect to the metric $\hat d$.
\end{thm}











\parit{Proof.}
Recall that $\hat d$ denotes the metric on the interval $(-\pi,\pi]$
introduced on page \pageref{def:max-dist}.
Consider function $s\:(-\pi,\pi]\to \mathbb{R}$
defined as 
$$s(\alpha)=\left[
\begin{aligned}
&\pi-\alpha&&\text{if}&&\alpha\ge\tfrac\pi2
\\
&\alpha&&\text{if}&&-\tfrac\pi2\le\alpha\le\tfrac\pi2
\\
-\,&\pi-\alpha&&\text{if}&&\alpha\le-\tfrac\pi2
\end{aligned}
\right.
.$$
Note that for any $\alpha\in (-\pi,\pi]$ we have
$s(\alpha)=\tfrac\pi2-\hat d(\alpha,\tfrac\pi2)$.
Therefore by triangle inequality
$|s(\alpha)-s(\beta)|\le d(\alpha,\beta)$. 
In particular 
$$s\:((-\pi,\pi],\hat{d})\to\mathbb{R}$$ 
is a continuous function.
Further note that
\begin{itemize}
\item $s(\alpha)=0$ if and only if $\alpha=0$ or $\pi$;
\item $s(\alpha)>0$ if and only if $0<\alpha<\pi$;
\item $s(\alpha)<0$ if and only if $\alpha<0$.
\end{itemize}
In particular, if $\measuredangle AOB\ne 0,\pi$ 
then the angle $\ang AOB$ and the real number $s(\measuredangle AOB)$ have the same sign.

Consider the function 
$f(t)=\measuredangle A_tO_tB_t$.

Since 
the points $O_t$, $A_t$ and $B_t$ do not lie on one line,
Theorem~\ref{thm:straight-angle} implies that $f(t)=\measuredangle A_tO_tB_t\ne 0$ or $\pi$ for any $t\in[0,1]$.
Therefore by Axiom~\ref{def:birkhoff-axioms:2c},
$f$ is a continuous function.
Further,
by Intermediate value theorem, $f(0)$ and $f(1)$ have the same sign;
hence the result follows.
\qeds








\item\label{def:birkhoff-axioms:2c} 
The function $\measuredangle\:(A,O,B)\mapsto\measuredangle A O B$
is defined at the set $W$ of all triples 
$(A,O,B)$ such that $O\ne A$ and $O\ne B$.
Moreover if $\hat{d}$ denotes the metric on $(-\pi,\pi]$ introduced above then
$$\measuredangle\: W\to ((-\pi,\pi],\hat{d})$$ 
is continuous.









\begin{thm}{Exercise}\label{ex:angle-signs-in-trig}
Assume $\triangle ABC\cong\triangle A'B'C'$. 
Note that by Axiom~\ref{def:birkhoff-axioms:3},
\begin{align*}
\measuredangle C'A'B'&\equiv\pm\measuredangle CAB,
&
\measuredangle A'B'C'&\equiv\pm\measuredangle ABC,
\\
&\text{and}&\measuredangle B'C'A'&\equiv\pm\measuredangle BCA,
\end{align*}
Show that the signs in the above three identities can be chosen
to be the same. 
\end{thm}









\parit{Proof.}
Assume first that $C_1=C_2$, so we can set 
\begin{align*}
C&=C_1=C_2,
&
C'&=C_1'=C_2.
\end{align*}


Applying to Proposition~\ref{prop:arc(angle=tan)},
\begin{align*}
\measuredangle X_1AX_2&\equiv\measuredangle X_1AC+\measuredangle CAX_2\equiv
\\
&\equiv(\pi-\measuredangle CB_1A)+(\pi-\measuredangle AB_2 C)\equiv
\\
&
\equiv -(\measuredangle CB_1A+\measuredangle AB_2 C)
\end{align*}

Analogously,
\begin{align*}
\measuredangle Y_1A'Y_2&
\equiv -(\measuredangle C'B_1'A'+\measuredangle A'B_2' C').
\end{align*}

By Theorem~\ref{lem:inverse-4-angle}(\ref{lem:inverse-4-angle:angle}), 
\[\measuredangle CB_1A+\measuredangle AB_2 C
\equiv
-(\measuredangle C'B_1'A'+\measuredangle A'B_2' C')
\]
Hence the result follows.

Now let us come back to the general case.

Let $\Gamma_1$ and $\Gamma_2$ be the circlines containing the arcs $AB_1C_1$ and $AB_2C_2$.
Assume $\Gamma_1$ and $\Gamma_2$ intersect at two points at $A$ and say at $C$.
In this case we can extend or shorten the arcs  $AB_iC_i$ in $\Gamma_i$ to make the $C_1=C_2=C$;
note that it will not change the angles between tangent half-lines at $A$.
This way we reduce the problem to the case which was already considered.

The remaining case when $\Gamma_1$ and $\Gamma_2$ are tangent at $A$ is left to the reader;
it can be proved applying the case above twice.
\qeds













\begin{thm}{Proposition}\label{prop:parallel-2}
$(AB)\parallel(C D)$ if and only if 
$$\measuredangle A B C+\measuredangle B C D\equiv 0\ \text{or}\ \pi.\eqlbl{A B C + B C D}$$ 
Moreover, if $\measuredangle A B C+\measuredangle B C D\equiv 0$
then the points $A$ and $D$ lie on the opposite sides from $(BC)$ 
and if $\measuredangle A B C+\measuredangle B C D\equiv \pi$
 the points $A$ and $D$ lie on the same side from $(BC)$.
\end{thm}

\parit{Proof.}
We may assume that $\measuredangle ABC\ne 0$ or $\pi$;
otherwise the statement is evident. 

\parit{}($\Leftarrow$).
Arguing by contradiction, 
assume \ref{A B C + B C D} holds, but $(AB)\z\nparallel(C D)$.
Let $Z$ be the point of intersection of $(AB)$ and $(CD)$.

Denote by $M$ be the midpoint of $[BC]$; note that $M\ne Z$.
Choose $Z'\in (ZM)$ distinct from $Z$ such that $MZ'=MZ$.

Note that two pairs of angles 
$\ang ZMB$, $\ang Z'MC$
and
$\ang ZMC$, $\ang Z'MB$
are vertical.
It follows that
\begin{align*}
\triangle MZB&\cong \triangle MZ'C,
& 
\triangle MZC&\cong \triangle MZ'B,
\\
\measuredangle MBZ&=\measuredangle MCZ',
&
\measuredangle MCZ&=\measuredangle MBZ'.
\end{align*}

Since $Z$ lies on $(AB)$ and $(CD)$,
we have
\begin{align*}
2\cdot\measuredangle ABC&\equiv 2\cdot\measuredangle ZBC
&
2\cdot\measuredangle BCD &\equiv 2\cdot\measuredangle BCZ
\end{align*}
Therefore
\begin{align*}
2\cdot\measuredangle ZBZ'
&\equiv 2\cdot \measuredangle ZBC+2\cdot \measuredangle CBZ'\equiv
\\
&\equiv 2\cdot \measuredangle ZBC+2\cdot \measuredangle BCZ\equiv
\\
&\equiv 0
\end{align*}
That is, $Z'\in (AB)$.

The same way we get that $Z'\in (CD)$.
By Axiom~\ref{def:birkhoff-axioms:2}, $(CD)=(AB)$,
in particular $(CD)=(AB)$.

\parit{}($\Rightarrow$). Assume $(AB)\parallel(C D)$.
Choose $D'$ so that 
$$2\cdot \measuredangle A B C+2\cdot \measuredangle B C D'\equiv 0.$$
Then according to ``if''-part, $(AB)\parallel (CD')$.

From the uniqueness of parallel line (Theorem~\ref{thm:parallel})
we get $(CD)=(CD')$.
In particular, $2\cdot\measuredangle BCD\equiv 2\cdot\measuredangle BCD'$.
Hence the result follows.
\qeds








%??? REDO with P in the center of absolute.

\begin{wrapfigure}{o}{43mm}
\begin{lpic}[t(-5mm),b(0mm),r(0mm),l(0mm)]{pics/angle-par(0.85)}
\lbl[tr]{21,18;$Q$}
\lbl[tr]{21,29;$P$}
\lbl[br]{22,60;$P'$}
\lbl[bl]{42,39;$H$}
\lbl[bl]{24,46;$M$}
\lbl[tl]{43,18;$A$}
\lbl[tr]{3,18;$B$}
\lbl[tr]{37,43.5;\small{$\phi$}}
\end{lpic}
\end{wrapfigure}

\parit{Proof.}
Denote by $A$ and $B$ the ideal points of $\ell$.

Applying an isometry of h-plane if necessary,
we may assume $Q$ is the center of absolute, 
so $[AB]$ is a diameter of absolute.

Draw the h-line $(PZ)_h$
so
the circle $\Gamma$ containing $(PZ)_h$ is tangent to $(AB)$ at $A$.
Clearly $|\measuredangle_h Q P Z|=\phi$; without loss of generality 
we may assume that $\measuredangle_h Q P Z=\phi$.

Set $a=QP$; by Lemma~\ref{lem:O-h-dist},
$$a=\frac{e^h-1}{e^h+1}.$$

Let $P'$ be the inverse of $P$ in the absolute
and $M$ be the midpoint of $[PP']$.
Clearly 
$$PM=\tfrac12\cdot(QP'-QP)=\frac{\frac1a-a}{2}.$$

Note that $\phi=\measuredangle MHP$.
Since $HM=AQ=1$, we get
$$\tan\phi=\frac{MP}{HM}=\frac{\frac1a-a}{2}$$


Therefore
$$\tan \phi
=\frac{\frac{e^h+1}{e^h-1}-\frac{e^h-1}{e^h+1}}2
=\frac{4\cdot e^{h}}{2\cdot( e^{2\cdot h}-1)}
=\frac{2}{e^{h}-e^{-h}}=\frac1{\sinh h}.$$
\qedsf












\parit{Proof.} %???CHANGE THE PROOF???
Further we assume that both arcs are nondegenerate.
The degenerate cases are left to the reader.

According to Theorem~\ref{thm:tangent-angle},
\begin{align*}
2\cdot \measuredangle X_1AC
&\equiv -2\cdot \measuredangle Y_1CA,
\\
2\cdot \measuredangle CAX_2
&\equiv -2\cdot \measuredangle ACY_2.
\end{align*}
Since the pairs of points $(X_1, Y_1)$ and $(X_2, Y_2)$
lie on the same side from $(AC)$,
we get that the  angles in the pairs of angles 
$\ang X_1AC$, $\ang Y_1CA$ 
and $\ang CAX_2$, $\ang ACY_2$
have opposite signs.
Therefore 
\begin{align*}
\measuredangle X_1AC
&\equiv - \measuredangle Y_1CA,
\\
 \measuredangle CAX_2
&\equiv - \measuredangle ACY_2.
\end{align*}
Adding these two identities, we get the result.
\qeds








\begin{thm}{Exercise}\label{ex:3-inverions}
Consider the three inversions
in the circles centered at $O$ and radii $r_1$, $r_2$ and $r_3$.
Show that the composition of these three inversions is an inversion in a circle centered at $O$.
Find the radius of this circle.

Try to use this problem to finish the proof of Theorem \ref{thm:angle-inversion} in case $A\in\Omega$.
\end{thm}







Two arcs $AB_1C_1$ and $AB_2C_2$ are called \index{tangent!tangent arcs}\emph{tangent} at $A$ if their tangent half-lines at $A$ coincide.

\begin{thm}{Lemma}\label{lem:tangent-to-tangent}
Let $AB_1C_1$ and $AB_2C_2$ be arcs tangent at $A$.
Assume $A'B_1'C_1'$ and $A'B_2'C_2'$ be their inversions in a circle.
Then the arcs $A'B_1'C_1'$ and $A'B_2'C_2'$ are tangent at $A'$.
\end{thm}

This lemma might look evident, 
but the formal proof is not short and tedious.

\parit{Proof.}
Denote by $\Gamma_1$, $\Gamma_2$, $\Gamma_1'$ and $\Gamma_2'$ the circles containing arcs $AB_1C_1$, $AB_2C_2$, $A'B_1'C_1'$ and $A'B_2'C_2'$ correspondingly.

Note that $\Gamma_1$ and $\Gamma_2$ are tangent at $A$; 
that is, $A$ is the only point of intersection of 
$\Gamma_1$ and $\Gamma_2$.
Therefore $\Gamma_1'$ and $\Gamma_2'$ are tangent at $A'$.
In particular the tangent lines to $\Gamma_1'$ and $\Gamma_2'$ at $A'$ coincide.

It remains to show that the half lines tangent to $A'B_1'C_1'$ and $A'B_2'C_2'$ at $A'$ can not be opposite. 

Let $[AX)$ be the tangent half-line to arcs $AB_1C_1$ and $AB_2C_2$  at $A$ 
and $[AY_1)$ be the tangent half-lines to arcs $A'B_1'C_1'$   at $A'$.
Denote by $O$ the center of inversion.

Note that
 the angles $\ang OAX$ and $\ang OAY_1$ have the same signs.
Further $\ang OAX$ is acute (obtuse)
if and only if $\ang OAY_1$  is obtuse (correspondingly acute).

Repeating the same argument for half-line $[AY_2)$ which is tangent  to the arcs $A'B_2'C_2'$   at $A'$,
we get that angles $\ang OAY_1$ and $\ang OAY_2$ have the same sign which implies that they can not be opposite unless the angles are $0$ and $\pi$.
In the remaining case these angle have to be both  acute or  both obtuse;
which means they both $0$ or both $\pi$.
\qeds







\begin{wrapfigure}{o}{58mm}
\begin{lpic}[t(-10mm),b(0mm),r(0mm),l(0mm)]{pics/angle-inversion(0.8)}
\lbl[rb]{47,33;$A$}
\lbl[lt]{39,31;$A'$}
\lbl[rb]{57,11;$B_1$}
\lbl[lt]{34,22;$B_1'$}
\lbl[r]{36,3;$C_1$}
\lbl[rt]{31,12;$C_1'$}
\lbl[t]{48,15;$Z_1$}
\lbl[b]{69,28;$X_1$}
\lbl[br]{22,15;$Y_1$}

\lbl[lb]{49,43;$B_2$}
\lbl[b]{41,43;$C_2$}
\lbl[rw]{35,35;$B_2'$}
\lbl[rb]{37,39;$C_2'$}
\lbl[br]{36,55;$Z_2$}
\lbl[tl]{65,44.5;$X_2$}
\lbl[t]{10,34;$Y_2$}
\end{lpic}
\end{wrapfigure}

 
\parit{Proof.}
Assume $A\ne A'$; 
that is, $A$ does not lie on the circle of inversion.

According to Exercise~\ref{ex:tangent-arc},
there are arcs $AZ_1A'$ and $AZ_2A'$ such that the half-lines $[AX_1)$ and $[AX_2)$
are tangent to these arcs at $A$.

From Corollary~\ref{cor:perp-inverse}, it follows that
the arcs $AZ_1A'$ and $AZ_2A'$ are inverted to themselves.
By Lemma~\ref{lem:tangent-to-tangent}, 
arc $AZ_1A'$ is tangent to arc $A'B_1'C_1'$
and  $AZ_2A'$ is tangent to arc $A'B_2'C_2'$ at $A'$.

From Exercise~\ref{ex:two-arcs}, the result follows.

The remaining special case $A\in\Omega$ is left to the reader;
see Exercise~\ref{ex:3-inverions} for a hint.
\qeds









 \begin{wrapfigure}{o}{50mm}
\begin{lpic}[t(-5mm),b(-1mm),r(0mm),l(0mm)]{pics/P-hat(1)}
\lbl[rt]{31.5,31.5;$O$}
\lbl[t]{43,37;$P$}
\lbl[b]{87,71;$P'$}
\lbl[l]{55,42;$Q$}
\lbl[t]{63,22;$A$}
\lbl[b]{45,62;$B$}
\end{lpic}
\caption*{$\hat P=\hat P'=Q$}
\end{wrapfigure}


\parit{Proof.} Denote by $Q$ the intersection of $[OP)$ and $[AB]$.
We need to show that $Q=\hat P$.

Let $P'$ be inverse of $P$ in $\Omega$.
According to ???, $P'\in \Gamma$.
Set $x=OP$, so $OP'=\tfrac1x$.

By ??? $$\triangle QPA\sim\triangle QBP'\ \ \text{and}\ \ \triangle QPB\sim\triangle QAP'.$$
Further by ???
$$\triangle OPA\sim\triangle OAP'\ \ \text{and}\ \ \triangle OPB\sim\triangle OBP'.$$
Therefore
\begin{align*}
\frac{P'Q}{PQ}&=\frac{AQ}{PQ}\cdot\frac{P'Q}{AQ}=
\\
&=\frac{P'A}{BP}\cdot\frac{P'B}{PA}=
\\
&=\frac{P'B}{BP}\cdot\frac{P'A}{PA}=
\\
&=\frac{OP'}{OB}\cdot\frac{OP'}{OA}=\frac1{x^2}.
\end{align*}
It follows that
$$PQ=$$


Without loss of generality, we may assume that $P$ lies inside $\Omega$;
otherwise sqitch $P$ with $P'$.
Set $a=OP$, so

Then 
\begin{align*}
OQ&=OP+PQ=
\\
&=OP+PP'\cdot \frac{PQ}{P'Q}
\end{align*}


\qeds



\begin{thm}{Lemma}
Let $P$ and $Q$ be distinct points in h-plane and the points $A$ and $B$ are the ideal points of h-line $(PQ)_h$.
Then the points $A$, $\hat P$, $\hat Q$ and $B$ lie on one line.
More over 
$$PQ_h=\tfrac12\cdot\ln\frac{A\hat P\cdot B\hat Q}{\hat PB\cdot \hat QA}.$$
\end{thm}

 
 
 
 
 
 
 
 
 
 In hyperbolic geometry the definition of parallel lines is slightly different.
Given a point $P$ off the h-line $(AB)_h$ 
there are infinite number of h-lines passing through $P$ and not intersecting $(AB)_h$. 
Among them there are only two parallel h-lines to $(AB)_h$. 
An h-line $(PQ)_h$ is called parallel to $(AB)_h$
in direction from $A$ to $B$ if the points $Q$ and $B$ lie on one side from $(AP)_h$, but any line from $P$ which leaves $Q$ and $A$ on the opposite sides is intersecting $(AB)_h$.

If $Q$ as above and $F$ is the foot point of $P$ on $(AB)_h$ then $\phi=|\measuredangle_h FPQ|$ is called angle of parallelism.
Clearly $\phi$ is completely determined by $h=PF_h$.

The same way one can define an h-line parallel to $(AB)_h$
in direction from $B$ to $A$.
The remaining lines through $P$ which do not intesect $(AB)_h$ are called ultra parallel.












\section{}

Let $P$ and $Q$ be two points of h-plane.
Define h-distance  as 
$$PQ_h=-\ln\delta(P,Q).$$
\parbf{Comments.}
Since $\delta(P,Q)\le 1$, we need to use negative constant in front of $\ln\delta(P,Q)$ to make $PQ_h$ nonnegative.
We might use any negative constant instead of $-1$, all of them will give a metric; the choice of the constant is made to have some formulas simpler.


\begin{thm}{Theorem} \label{lem:h-dist}
The function $(P,Q)\mapsto PQ_h$ is a distance function on h-plane.
Moreover, if we equip h-plane with h-distance function then
\begin{enumerate}[(a)]
\item Any inverse in extended line is an isometry of h-plane.
\item  h-lines and only they form a lines in h-plane.
\end{enumerate}
\end{thm}



\parit{Proof.}
From Lemma~\ref{lem:0-1}, $PQ_h\ge0$ and $PQ_h=0$ iff $P=Q$.

From Lemma~\ref{lem:delta-inverse}, it follows that inverse in extended line 
preserves h-distance. 
Further, from Lemma~\ref{lem:perp-inverse}, any such inverse is a bijection of h-plane. 

It remains to prove triangle inequality;
that is,
$$PQ_h\le PZ_h+ZQ_h\eqlbl{eq:h-tringle}$$
for any $Z$ in h-plane.

\begin{wrapfigure}{o}{35mm}
\begin{lpic}[t(0mm),b(0mm),r(0mm),l(0mm)]{pics/h-triangle-in(0.8)}
\lbl[t]{8,26;$P$}
\lbl[t]{29,38;${\leftarrow} Q$}
\lbl[t]{21,30;$Z$}
\end{lpic}
\end{wrapfigure}

From Lemma~\ref{lem:delta-dot}, it follows that \ref{eq:h-tringle} becomes equality for any $Z\in[PQ]_h$.
Thus, if \ref{eq:h-tringle} does not hold, then for some $Z\in[PQ]_h$ there is $Z'$ in the h-plane such that $PZ'_h<PZ_h$ and $QZ'_h<QZ_h$.

Apply Lemma~\ref{lem:delta-inverse}, for $P$ and $\lambda=\delta(P,Z)$ and for $Q$ and $\lambda=\delta(Q,Z)$.
We get two circles $\Gamma_P$ and $\Gamma_Q$,
both of them are passing through $Z$ and perpendicular to $(PQ)_h$.
It follows that circles $\Gamma_P$ and $\Gamma_Q$ are tangent at $Z$
and there is no point $Z'$ which lie inside $\Gamma_P$ and $\Gamma_Q$ at the same time.
Moreover $Z$ is the only point of intersection of $\Gamma_P$ and $\Gamma_Q$.

Finally note that since \ref{eq:h-tringle} becomes equality for any $Z\in[PQ]_h$, 
any h-line forms a line in h-plane with h-distance;
that is, any h-line equipped with h-distance is isometric to $\mathbb{R}$. 
Since $Z$ is the only point of intersection of $\Gamma_P$ and $\Gamma_Q$, it follows that there no other lines in this metric space.
\qeds
 
\section*{Axiom~IV}

Further we always assume that h-plane is equipped with h-distance function.

Above we proved (Theorem~\ref{lem:h-dist}) that h-plane is a metric space, obliviously it contains at least two points.
Further, theorems~\ref{lem:h-dist} and \ref{thm:h-line} show that that Axiom~I holds for h-plane.
The fact that Axiom~II holds for h-plane follows directly from the definition of h-angle.

To show that model Poincar\'e satisfies Definition~\ref{def:birkhoff-axioms-absolute},
it remains to prove that Axiom~IV holds in h-plane.
That is, we need to prove the following theorem:

\begin{thm}{Theorem}
Assume that for h-triangles $\triangle_h A B C$ and $\triangle_h A' B' C'$ in the h-plane, we have
$$A' B'_h= A B_h,
\ A' C'_h= A C_h$$
$$\ \text{and}\ 
\measuredangle_h C' A' B'=\pm\measuredangle_h C A B.\eqlbl{birkhoff-signs}$$
Then 
$$ B' C'_h= B C_h,
\ \measuredangle_h A' B' C'=\pm\measuredangle_h A B C,
\ \measuredangle_h B' C' A'=\pm\measuredangle_h B  C A $$
With signs as in \ref{birkhoff-signs}.

In particular $\triangle_h A B C\cong\triangle_h A' B' C'$.
\end{thm}


\parit{Proof.}
Note that there is an h-line $\ell$ such that the inverse in the extension of $\ell$ sends $A$ to the center of absolute.
Recall that inverse in $\ell$ preserves h-distance only changes the sign of h-angles.%???
Thus, we may assume that $A$ is the center of absolute.
The same way we may assume that $A'$ is also the center of absolute (so $A'=A$).

It follows that the sides $[AB]_h$, $[AC]_h$, $[AB']_h$ and $[AC']_h$ are also segments in ordinary Euclidean sense.
In particular 
$$\measuredangle B'AC'\equiv\measuredangle_h B'AC'\equiv\pm\measuredangle_h BAC\equiv\pm\measuredangle BAC.$$
Direct calculation show that 
$$AB_h=-\ln\frac{r-AB}{r+AB},$$
where $r$ is the radius of absolute.
Therefore $A B'=AB$ and $A C'=AC$.
It follows that there is a rotation or reflection of euclidean plane which does not move $A$ and sends $B\mapsto B'$, $C\mapsto C'$.
Note that the rotation or reflection described above is also an isometry of h-plane,
moreover it preserves h-angles.
Hence   
$$ B' C'_h= B C_h,
\ \measuredangle_h A' B' C'=\pm\measuredangle_h A B C,
\ \measuredangle_h B' C' A'=\pm\measuredangle_h B  C A.$$
\qedsf













\section{Absolute geometry}






 


\section*{Larger angle---longer side}

The next corollary states that in a nondegenerate triangle 
the larger side corresponds to a larger angle.

\begin{thm}{Corollary}\label{cor:larger-side=>larger-angle}
Assume that in a triangle $\triangle ABC$ the angles $\ang CAB$ and $\ang ABC$ are positive.
Then then 
$CA>CB$ if and only if $\measuredangle CAB>\measuredangle ABC$.
\end{thm}

\parit{Proof.}
Note that for degenerate triangle the statement is trivially holds.

Set $\alpha=\measuredangle CAB$ and $\beta=\measuredangle ABC$.
If $\alpha=\beta$ then by Theorem~\ref{thm:isos},
$CA=CB$.

Therefore,  $\pi>\alpha> \beta>0$ or $\pi> \beta>\alpha>0$.
We will consider the first case; 
the second is completely analogous.

\begin{wrapfigure}[13]{o}{32mm}
\begin{lpic}[t(0mm),b(0mm),r(0mm),l(2mm)]{pics/ABCXZ(0.7)}
\lbl[rt]{2,1;$A$}
\lbl[lt]{38,1;$B$}
\lbl[rb]{10,42;$C$}
\lbl[tl]{31,35;$X$}
\lbl[l]{22,24;$Z$}
\lbl[r]{31,5;$\beta$}
\lbl[lw]{13,10;$\alpha$}
\end{lpic}
\end{wrapfigure}

Use Axiom \ref{def:birkhoff-axioms:2a}
to construct the half-line $[AX)$ 
such that $\measuredangle XAB=\beta$.

Note that the angle
$\measuredangle XAC= \beta-\alpha$ is negative.
By Proposition~\ref{prop:half-plane}, 
$C$ and $B$ lie on the opposite sides of $(AX)$.
Hence $[CB]$ intersects $(AX)$ at a point, say at $Z$.

Note that $\measuredangle ZAB$ is either $\beta$ or $\beta-\pi$.
On the other hand $Z$ and $C$ lie on the same side from $(AB)$ and therefore the $\measuredangle ZAB>0$.
Hence 
$$\measuredangle ZAB=\beta.$$

By ASA consition (\ref{thm:ASA}), 
$$ZA=ZB.$$
By Corollary \ref{cor:degenerate-trig},
$$AC>AZ+CZ$$.
Therefore
$$AC>AZ+CZ=BZ+ZC=BZ.$$
\qedsf
\chapter*{Introduction}
\addcontentsline{toc}{chapter}{Introduction}
\addtocontents{toc}{\protect\begin{quote}}

This book is meant to be 
rigorous, 
conservative, 
elementary and
minimalist.
At the same time it includes about the maximum what students can absorb in one semester.

Approximately one-third of the material used to be covered in high school, but not any more.

The present book is based 
on the courses given by the author 
at the Pennsylvania State University
as an introduction to the foundations of geometry.
The lectures were oriented to sophomore and senior university students.  
These students already had a calculus course.
In particular, they are familiar with the real numbers and continuity.
It makes it possible to cover the material faster 
and  in a more rigorous way
than it could be done in high school.

\section*{Prerequisite}
\addtocontents{toc}{Prerequisite.}

The students should be familiar 
with the following topics.
\begin{itemize}
\item Elementary set theory: 
$\in$,
$\cup$, 
$\cap$,
$\backslash$,
$\subset$,~$\times$.
\item Real numbers: intervals, inequalities, algebraic identities.
\item Limits, continuous functions and the intermediate value theorem.
\item Standard functions: 
absolute value, 
natural logarithm,
exponential function. 
Occasionally, trigonometric functions  are used, 
but these parts can be ignored.
\item  Chapter~\ref{chap:trans} uses matrix algebra of $2{\times}2$-matrices.
\item To read Chapter~\ref{chap:sphere}, it is better to have some previous experience with the {}\emph{scalar product}, also known as {}\emph{dot product}.
\item To read Chapter~\ref{chap:complex}, it is better to have some previous experience with complex numbers.
\end{itemize} 

\section*{Overview}
\addtocontents{toc}{Overview.}

We use the so called {}\emph{metric approach} introduced by Birkhoff.
It means that we define the Euclidean plane as a {}\emph{metric space} which satisfies a list of properties ({}\emph{axioms}).
This way we minimize the tedious parts
which are unavoidable in the more classical Hilbert's approach.
At the same time the students have a chance to learn basic geometry of metric spaces.

Here is a dependency graph of the chapters.

\smallskip
\begin{center}
\begin{tikzpicture}[->,>=stealth',shorten >=1pt,auto,scale=1.4,
  thick,main node/.style={circle,draw,font=\sffamily\bfseries,minimum size=8mm}]

  \node[main node] (1) at (0,15/6) {1};
  \node[main node] (2) at (1,15/6){2};
  \node[main node] (3) at (2,15/6) {3};
  \node[main node] (4) at (3,15/6) {4};
  \node[main node] (5) at (3.5,10/6) {5};
  \node[main node] (6) at (4,5/6) {6};
  \node[main node] (7) at (5,5/6) {7};
  \node[main node] (8) at (3,5/6){8};
  \node[main node] (9) at (2,5/6) {9};
  \node[main node] (10) at (2.5,10/6) {10};
  \node[main node] (11) at (1.5,10/6){11};
  \node[main node] (12) at (.5,10/6) {12};
  \node[main node] (13) at (1.5,0) {13};
  \node[main node] (14) at (0.5,0) {14};
  \node[main node] (15) at (1,5/6) {15};
  \node[main node] (16) at (0,5/6) {16};
  \node[main node] (17) at (2.5,0) {17};
  \node[main node] (18) at (3.5,0) {18};
  \node[main node] (19) at (4.5,0) {19};

  \path[every node/.style={font=\sffamily\small}]
   (1) edge node[right]{}(2)
   (2) edge node[right]{}(3)
   (3) edge node[right]{}(4)
   (4) edge node[right]{}(5)
   (5) edge node[right]{}(6)
   (6) edge node[right]{}(7)
   (6) edge node[right]{}(19)
   (9) edge node[right]{}(13)
   (13) edge node[right]{}(14)
   (6) edge node[right]{}(8)
   (6) edge node[right]{}(18)
   (8) edge node[right]{}(9)
   (9) edge node[right]{}(11)
   (9) edge node[right]{}(15)
   (9) edge node[right]{}(17)
   (5) edge node[right]{}(10)
   (10) edge node[right]{}(11)
   (11) edge node[right]{}(12)
   (14) edge node[right]{}(16)
   (15) edge node[right]{}(16)
   (12) edge node[dashed,right]{}(16);
\end{tikzpicture}
\end{center}

\smallskip

In (1) we give all the definitions necessary to formulate the axioms;
it includes metric space, 
lines, 
angle measure, 
continuous maps and congruent triangles.

%???extra chapter???

Further we do Euclidean geometry:
(\ref{chap:axioms}) Axioms and immediate corollaries,
(\ref{chap:half-planes}) Half-planes and continuity,
(\ref{chap:cong}) Congruent triangles,
(\ref{chap:perp}) Circles, motions, perpendicular lines,
(\ref{chap:parallel}) Parallel lines and similar triangles
--- this is the first chapter where we use Axiom \ref{def:birkhoff-axioms:4}, an equivalent of Euclid's parallel postulate.
In (\ref{chap:triangle}) we give the most classical theorem of triangle geometry;
this chapter is included mainly as an illustration.


In the following two chapters we discuss geometry of circles on the Euclidean plane:
(\ref{chap:inscribed-angle}) Inscribed angles, (\ref{chap:inversion}) Inversion.
It  will be used to construct the model of the hyperbolic plane.

Further 
we discuss non-Euclidean geometry:
(\ref{chap:non-euclid})
Neutral geometry --- geometry without the parallel postulate,
(\ref{chap:poincare})
Conformal disc model ---
this is a construction of the hyperbolic plane,
an example of a neutral plane which is not Euclidean.
In (\ref{chap:h-plane}) we discuss geometry of the constructed hyperbolic plane --- this is the highest point in the book.

In the reamining chapters we discuss some additional topics:
(\ref{chap:trans}) Affine geometry,
(\ref{chap:proj}) Projective geometry,
(\ref{chap:sphere}) Spherical geometry, 
(\ref{chap:klein}) Projective model of the hyperbolic plane,
(\ref{chap:complex}) Complex coordinates,
(\ref{chap:car}) Geometric constructions,
(\ref{chap:area}) Area.
The proofs in these chapters are not completely rigorous.

We encourage to use the visual assignments available at the author's website.

\section*{Disclaimer}

It is  impossible to find the original reference to most of the theorems discussed here, so I do not even try to.
Most of the proofs discussed in the book 
already appeared in the Euclid's Elements.

\section*{Recommended books}

\begin{itemize}
\item Kiselev's textbook \cite{kiselev} ---
a classical book for school students.
Should help if you have trouble following this book.

\item Moise's book, \cite{moise} ---
should be good for further study.

\item Greenberg's book \cite{greenberg}  --- a historical tour in the axiomatic systems of various geometries.

\item Prasolov's book \cite{prasolov} is perfect to master your problem-solving skills.

\item Akopyan's book \cite{akopyan} --- a collection of problems formulated in figures.

\item Methodologically my lectures
were very close to Sharygin's  textbook \cite{sharygin}.
This is the greatest textbook in geometry for school students,
I recommend it to anyone who can read Russian.


\end{itemize}

\section*{Acknowlegments}

Let me thank  
Matthew Chao, 
Alexander Lytchak,
Alexei Novi\-kov
and Lukeria Petrunina
for useful suggestions and correcting the misprints.






\addtocontents{toc}{\protect\end{quote}}
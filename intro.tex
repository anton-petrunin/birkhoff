\chapter*{Introduction}
\addcontentsline{toc}{chapter}{Introduction}

This book is meant to be 
rigorous, 
conservative, 
elementary, 
and minimalist.
At the same time, it includes about the maximum that students can absorb in one semester.

Approximately one-third of the material used to be covered in high school, but not anymore.

The present book is based 
on the courses given by the author 
at the Pennsylvania State University
as an introduction to the foundations of geometry.
The lectures were oriented to sophomore and senior university students.  
These students already had a calculus course.
In particular, they are familiar with real numbers and continuity.
It makes it possible to cover the material faster 
and  in a more rigorous way
than it could be done in high school.

\section{Prerequisite}


The students should be familiar 
with the following topics:
\begin{itemize}
\item Elementary set theory: 
$\in$,
$\cup$, 
$\cap$,
$\backslash$,
$\subset$,~$\times$.
\item Real numbers: intervals, inequalities, algebraic identities.
\item Limits, continuous functions, and the intermediate value theorem.
\item Standard functions: 
absolute value, 
natural logarithm,
exponential function. 
Occasionally, trigonometric functions  are used, 
but these parts can be ignored.
\item  Chapter~\ref{chap:trans} uses basic vector algebra.
\item To read Chapter~\ref{chap:sphere}, it is better to have some previous experience with the {}\emph{scalar product}, also known as the {}\emph{dot product}.
\item To read Chapter~\ref{chap:complex}, it is better to have some previous experience with complex numbers.
\end{itemize} 

\section{Overview}

We use the so-called {}\emph{metric approach} introduced by Birkhoff.
It means that we define the Euclidean plane as a {}\emph{metric space} that satisfies a list of properties ({}\emph{axioms}).
This way we minimize the tedious parts
which are unavoidable in the more classical Hilbert's approach.
At the same time, the students have a chance to learn the basic geometry of metric spaces.

Here is a dependency graph of the chapters.

\begin{figure}[!ht]
\centering

\begin{tikzpicture}[->,>=stealth',shorten >=1pt,auto,scale=1.4,
  thick,main node/.style={circle,draw,font=\sffamily\bfseries,minimum size=8mm}]

  \node[main node] (1) at (1,15/6) {\ref{chap:metr}};
  \node[main node] (2) at (2,15/6){\ref{chap:axioms}};
  \node[main node] (3) at (3,15/6) {\ref{chap:half-planes}};
  \node[main node] (4) at (4,15/6) {\ref{chap:cong}};
  \node[main node] (5) at (3.5,10/6) {\ref{chap:perp}};
  \node[main node] (6) at (4.5,10/6) {\ref{chap:parallel}};
  \node[main node] (61) at (4,5/6) {\ref{chap:angle-sum}};
  \node[main node] (7) at (5,5/6) {\ref{chap:triangle}};
  \node[main node] (8) at (3,5/6){\ref{chap:inscribed-angle}};
  \node[main node] (9) at (2,5/6) {\ref{chap:inversion}};
  \node[main node] (10) at (2.5,10/6) {\ref{chap:non-euclid}};
  \node[main node] (11) at (1.5,10/6){\ref{chap:poincare}};
  \node[main node] (12) at (.5,10/6) {\ref{chap:h-plane}};
  \node[main node] (13) at (1.5,0) {\ref{chap:trans}};
  \node[main node] (14) at (0.5,0) {\ref{chap:proj}};
  \node[main node] (15) at (1,5/6) {\ref{chap:sphere}};
  \node[main node] (16) at (0,5/6) {\ref{chap:klein}};
  \node[main node] (17) at (2.5,0) {\ref{chap:complex}};
  \node[main node] (18) at (3.5,0) {\ref{chap:car}};
  \node[main node] (19) at (4.5,0) {\ref{chap:area}};

  \path[every node/.style={font=\sffamily\small}]
   (1) edge node{}(2)
   (2) edge node{}(3)
   (3) edge node{}(4)
   (4) edge node{}(5)
   (5) edge node{}(6)
   (6) edge node{}(61)
   (61) edge node{}(8)
   (61) edge node{}(7)
   (61) edge node{}(19)
   (9) edge node{}(13)
   (13) edge node{}(14)
   (61) edge node{}(18)
   (8) edge node{}(9)
   (9) edge node{}(11)
   (9) edge node{}(15)
   (9) edge node{}(17)
   (5) edge node{}(10)
   (10) edge node{}(11)
   (11) edge node{}(12)
   (14) edge node{}(16)
   (15) edge[dashed] node{}(16)
   (12) edge node{}(16);
\end{tikzpicture}
\end{figure}

In (\ref{chap:metr}) we give all the definitions necessary to formulate the axioms;
it includes metric space, 
lines, 
angle measure, 
continuous maps,
and congruent triangles.

Further, we do Euclidean geometry:
(\ref{chap:axioms}) Axioms and immediate corollaries;
(\ref{chap:half-planes}) Half-planes and continuity;
(\ref{chap:cong}) Congruent triangles;
(\ref{chap:perp}) Circles, motions, and perpendicular lines;
(\ref{chap:parallel}) Similar triangles and (\ref{chap:angle-sum}) Parallel lines  
--- these are the first two chapters where we use Axiom~\ref{def:birkhoff-axioms:4}, an equivalent of Euclid's parallel postulate.
In (\ref{chap:triangle}) we give the most classical theorems of triangle geometry;
this chapter is included mainly as an illustration.

In the following two chapters, we discuss the geometry of circles on the Euclidean plane:
(\ref{chap:inscribed-angle}) Inscribed angles; (\ref{chap:inversion}) Inversion.
It  will be used to construct the model of the hyperbolic plane.

Further, we discuss non-Euclidean geometry:
(\ref{chap:non-euclid})
Neutral geometry --- geometry without the parallel postulate;
(\ref{chap:poincare})
Conformal disc model ---
this is a construction of the hyperbolic plane,
an example of a neutral plane that is not Euclidean.
In (\ref{chap:h-plane}) we discuss geometry of the constructed hyperbolic plane --- this is the highest point in the book.

In the remaining chapters, we discuss some additional topics:
(\ref{chap:trans}) Affine geometry;
(\ref{chap:proj}) Projective geometry;
(\ref{chap:sphere}) Spherical geometry;
(\ref{chap:klein}) Projective model of the hyperbolic plane;
(\ref{chap:complex}) Complex coordinates;
(\ref{chap:car}) Geometric constructions;
(\ref{chap:area}) Area.
The proofs in these chapters are not completely rigorous.

We encourage the use of visual assignments on the author's website.

\section{Disclaimer}

It is  impossible to find the original reference to most of the theorems discussed here, so I do not even try to.
Most of the proofs discussed in the book 
already appeared in Euclid's Elements.

\section{Recommended resources}

\begin{itemize}
\item \emph{Byrne's Euclid} \cite{byrne} --- a colored version of the first six books of Euclid's Elements edited by Oliver Byrne. 

\item \emph{Kiselyov's geometry} \cite{kiselev} ---
a classical textbook for school students written by Andrey Kiselyov; it should help if you have trouble following this book.

\item \emph{Lessons in Geometry} by Jacques Hadamard \cite{hadamard} --- an encyclopedia of elementary geometry originally written for school teachers.

%\item Moise's book, \cite{moise} --- should be good for further study.

%\item Greenberg's book \cite{greenberg}  --- a historical tour in the axiomatic systems of various geometries.

\item \emph{Problems in geometry} by Victor Prasolov\cite{prasolov}  is perfect to master your problem-solving skills.

\item \emph{Geometry in figures} by Arseniy Akopyan \cite{akopyan} --- an encyclopedia of Euclidean geometry with barely any words.

\item Euclidea \cite{euclidea} --- a fun and challenging way to learn Euclidian constructions.
 
\item \emph{Geometry} by Igor Sharygin \cite{sharygin} --- the greatest textbook in geometry for school students, I recommend it to anyone who can read Russian.

\end{itemize}

\section{Acknowledgments}

{\sloppy

Let me thank 
Thomas Barthelme,
Matthew Chao, 
Quinn Culver,
Svetlana Katok, 
Alexander Lytchak,
Alexei Novikov,
and Lukeria Petrunina
for useful suggestions and correcting misprints.

This work was partially supported by
NSF grants
DMS-0103957,
DMS-0406482,
DMS-0905138,
DMS-1309340,
DMS-2005279,
and Simons Foundation grants 
245094, 584781.

}

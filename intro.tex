\chapter*{Introduction}
\addcontentsline{toc}{chapter}{Introduction}
\addtocontents{toc}{\protect\begin{quote}}

The book is meant to be 
rigorous, 
conservative, 
elementary and
minimalistic.
At the same time it includes about the maximum what students can absorb in one semester.

Approximately third of the material used to be covered in high school, not any more.

The present book is based 
on the courses given by the author 
at the Penn State 
as an introduction into Foundations of Geometry.
The lectures were oriented to sophomore and senior university students.  
These students already had a calculus course.
In particular they are familiar with the real numbers and continuity.
It makes possible to cover the material faster 
and  in a more rigorous way
than it could be done in high school.



\section*{Prerequisite}
\addtocontents{toc}{Prerequisite.}

The students has to be familiar 
with the following topics.
\begin{itemize}
\item Elementary set theory: 
$\in$,
$\cup$, 
$\cap$,
$\backslash$,
$\subset$,
$\times$.
\item Real numbers: intervals, inequalities, algebraic identities.
\item Limits, continuous functions and  Intermediate value theorem.
\item Standard functions: 
absolute value, 
natural logarithm,
exponent. 
Occasionally, trigonometric functions  are used, 
but these parts can be ignored.
\item  Chapter~\ref{chap:trans} 
use matrix algebra of $2{\times}2$-matrices.
\item To read Chapter~\ref{chap:sphere}, it is better to have some previous experience with {}\emph{scalar product}, also known as {}\emph{dot product}.
\item To read Chapter~\ref{chap:complex}, it is better to have some previous experience with complex numbers.
\end{itemize} 

\section*{Overview}
\addtocontents{toc}{Overview.}

We use so called {}\emph{metric approach} introduced by Birkhoff.
It means that we define Euclidean plane as a {}\emph{metric space} which satisfies a list of properties.
This way we minimize the tedious parts
which are unavoidable in the more classical Hilbert's approach.
At the same time the students have chance to learn basic geometry of metric spaces.

In the Chapter~\ref{chap:metr} we give all definitions necessary to formulate the axioms;
it includes metric space, lines, angle measure, continuous maps and congruent triangles.

Euclidean geometry is discussed in the chapters \ref{chap:axioms}--\ref{chap:triangle}.
In the  Chapter~\ref{chap:axioms}, we formulate the axioms and prove immediate corollaries.
In the chapters \ref{chap:half-planes}--\ref{chap:parallel} 
we develop Euclidean geometry to a dissent level.
In Chapter~\ref{chap:triangle} we give the most classical theorem of triangle geometry;
this chapter included mainly as an illustration.

\medskip
\begin{center}
\begin{tikzpicture}[->,>=stealth',shorten >=1pt,auto,scale=1.2,
  thick,main node/.style={circle,draw,font=\sffamily\bfseries,minimum size=8mm}]

  \node[main node] (1) at (0,3) {1};
  \node[main node] (2) at (1,3){2};
  \node[main node] (3) at (2,3) {3};
  \node[main node] (4) at (3,3) {4};
  \node[main node] (5) at (3.5,2) {5};
  \node[main node] (6) at (4,1) {6};
  \node[main node] (7) at (5,1) {7};
  \node[main node] (8) at (3,1){8};
  \node[main node] (9) at (2,1) {9};
  \node[main node] (10) at (2.5,2) {10};
  \node[main node] (11) at (1.5,2){11};
  \node[main node] (12) at (.5,2) {12};
  \node[main node] (13) at (1.5,0) {13};
  \node[main node] (14) at (0.5,0) {14};
  \node[main node] (15) at (1,1) {15};
  \node[main node] (16) at (0,1) {16};
  \node[main node] (17) at (2.5,0) {17};
  \node[main node] (18) at (3.5,0) {18};
  \node[main node] (19) at (4.5,0) {19};

  \path[every node/.style={font=\sffamily\small}]
   (1) edge node[right]{}(2)
   (2) edge node[right]{}(3)
   (3) edge node[right]{}(4)
   (4) edge node[right]{}(5)
   (5) edge node[right]{}(6)
   (6) edge node[right]{}(7)
   (6) edge node[right]{}(19)
   (9) edge node[right]{}(13)
   (13) edge node[right]{}(14)
   (6) edge node[right]{}(8)
   (6) edge node[right]{}(18)
   (8) edge node[right]{}(9)
   (9) edge node[right]{}(11)
   (9) edge node[right]{}(15)
   (9) edge node[right]{}(17)
   (5) edge node[right]{}(10)
   (10) edge node[right]{}(11)
   (11) edge node[right]{}(12)
   (14) edge node[right]{}(16)
   (15) edge node[right]{}(16)
   (12) edge node[dashed,right]{}(16);
\end{tikzpicture}
\end{center}

\medskip

In the chapters \ref{chap:inscribed-angle}--\ref{chap:inversion} we discuss geometry of circles on the Euclidean plane. 
These two chapters 
will be used in the construction of the model of hyperbolic plane.

In the chapters \ref{chap:non-euclid}--\ref{chap:h-plane}
we discuss non-Euclidean geometry.
In Chapter~\ref{chap:non-euclid},
we introduce the axioms of absolute geometry.
In Chapter~\ref{chap:poincare}
we describe so called conformal disc model.
This is a construction of hyperbolic plane,
an example of absolute plane which is not Euclidean.
In the Chapter \ref{chap:h-plane} we discuss geometry of the constructed hyperbolic plane --- this is the highest point in the book.

The chapters \ref{chap:trans}--\ref{chap:area} contain additional topics:
Affine geometry,
Projective geometry,
Spherical geometry, 
Projective model,
Complex coordinates,
Geometric constructions
and Area correspondingly.
The proofs in these chapters are not completely rigorous.


\section*{Disclaimer}

I am not doing history.
It is  impossible to find the original reference to most of the theorems discussed here, so I do not even try.
Most of the proofs discussed in the book 
appeared already in the Euclid's Elements.

\section*{Recommended books}

\begin{itemize}
\item Kiselev's textbook \cite{kiselev} ---
a classical book for school students.
Should help if you have trouble to follow my book.

\item Moise's book, \cite{moise} ---
should be good for further study.

\item Greenberg's book \cite{greenberg}  --- a historical tour in the axiomatic systems of various geometries.

\item Prasolov's book \cite{prasolov} is perfect to master your problem-solving skills. 

\item Methodologically my lectures
were very close to Sharygin's  textbook \cite{sharygin}.
This is the greatest textbook in geometry for school students,
I recommend it to anyone who can read Russian.


\end{itemize}

\section*{Acknowlegments.}
I would like to thank  
Matthew Chao, 
Alexander Lytchak
and Alexei Novikov
for useful suggestions and correcting dozens of misprints.
%Andrew Baxter baxter@math.psu.edu 006 McAllister Building +1 814 865 5002





\addtocontents{toc}{\protect\end{quote}}
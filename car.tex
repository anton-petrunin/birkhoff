\chapter{Geometric constructions}
\label{chap:car}


The geometric constructions were introduced at the end of Chapter~\ref{chap:perp} and used everywhere since then.
They have a great pedagogical value as an introduction to mathematical proofs.

In this chapter, we discuss briefly the theory behind geometric constructions.


%\input{brahmagupta.tex}
%\input{car-kiselyov.tex}

\section{Classical problems}

The solutions to the following two problems are quite nontrivial.

\begin{thm}{Problem of Brahmagupta} 
Construct an inscribed quadrangle with given sides.
\end{thm}

{

\begin{wrapfigure}[7]{r}{22mm}
\vskip-6mm
\centering
\includegraphics{mppics/pic-284}
\end{wrapfigure}

Several solutions to the following problem are presented in \cite{hadamard}.
 
\begin{thm}{Problem of Apollonius} Construct a circle that is tangent to three given circles.
\end{thm}

There is an addictive way to become an expert in classical geometric constructions ---
play Euclidea \cite{euclidea}; it is an interactive geometry game.

}

\section{Impossible constructions}

Impossible construction problems cannot be solved in principle; 
that is, the required compass-and-ruler construction does not exist.
The following problems have existed for about two thousand years;
their impossibility was proved only in the $19^\text{th}$ century.
The method used in the proofs is indicated in the next section.


\begin{thm*}{Doubling the cube}
Construct the side of a new cube 
that has a volume twice as big as the volume of a given cube. 
\end{thm*}

In other words, 
given a segment of the length $a$,
one needs to construct a segment of length~$\sqrt[3]{2}\cdot a$.

\begin{thm*}{Squaring the circle}
Construct a square with the same area as a given circle.
\end{thm*}

If $r$ is the radius of the given circle, we need to construct a segment of length~$\sqrt{\pi}\cdot r$. 

\begin{thm*}{Angle trisection} 
Divide the given angle into three equal angles.
\end{thm*}

Moreover, a compass-and-ruler construction cannot trisect angle with measure~$\tfrac\pi3$. 
The existence of such a construction would imply the constructability of a regular 9-gon which is prohibited by the following famous result:

A \index{regular $n$-gon}\emph{regular $n$-gon} inscribed in a circle with center $O$ is a sequence of points $A_1\dots A_n$ on the circle such that 
\[\measuredangle A_nOA_1=\measuredangle A_1OA_2=\dots=\measuredangle A_{n-1}OA_n=\pm\tfrac2n\cdot \pi.\]
The points $A_1,\dots, A_n$ are \index{vertex!of a regular $n$-gon}\emph{vertices},
the segments $[A_1A_2], \z\dots, [A_nA_1]$ are \index{side!of a regular $n$-gon}\emph{sides} 
and the remaining segments $[A_iA_j]$ are \index{diagonal!of a regular $n$-gon}\emph{diagonals} of the $n$-gon.

The construction of a regular $n$-gon, therefore, is reduced to the construction of an angle with size $\tfrac2n\cdot \pi$.

\begin{thm}{Gauss--Wantzel theorem}
A regular $n$-gon can be constructed with a ruler and a compass 
if and only if 
$n$ is the product of a power of $2$ and any number of distinct Fermat primes.
\end{thm}

A \index{Fermat prime}\emph{Fermat prime} is a prime number of the form $2^k+1$ for an integer~$k$.
Only five Fermat primes are known  today: $3$, $5$, $17$, $257$, and $65537$.
For example, 
\begin{itemize}
\item one can construct a regular 340-gon since $340=2^2\cdot 5\cdot 17$ and $5$, as well as $17$, are Fermat primes;
\item one cannot construct a regular 7-gon since $7$ is not a Fermat prime;
\item one cannot construct a regular 9-gon; 
altho $9=3\cdot 3$ is a product of two Fermat primes, 
these primes are not distinct.
\end{itemize}

%\input{car-kiselyov.tex}

\section{Constructible numbers}

Let us give an intuitive definition of compass-and-ruler constructions; a formal definition is subtler than one might think \cite{engeler}.

In the classical compass-and-ruler constructions initial configuration can be completely described by a finite number of points;
each line is defined by two points on it and each circle is described by its center and a point on it (equivalently, you may describe a circle by three points on it).

In the same way, the result of construction can be described by a finite collection of points.

We may always assume that the initial configuration has at least two points;
if not, we could add one or two points to the configuration.
Moreover, by scaling the whole plane, we can assume that the first two points in the initial configuration lie at distance 1 from each other.

In this case, we can choose a  coordinate system, 
so that the points $(0,0)$ and $(1,0)$ are among the initial points;
so the initial configuration of $n$ points is described by 
$2\cdot n-4$ numbers --- their coordinates $x_3,y_3,\z\dots,x_n,y_n$. 

\medskip

It turns out that the coordinates of any point constructed with a compass and ruler
can be written thru the numbers $x_3,y_3,\z\dots,x_n,y_n$ using the four arithmetic operations ``$+$'', ``$-$'', ``$\cdot$'', ``$/\,$''
and the square root ``$\sqrt{\phantom{a}}\,$''.

For example, assume we want to find the points $X_1=(x_1,y_1)$ and $X_2\z=(x_2,y_2)$ of the intersections of 
a line passing thru $A=(x_A,y_A)$ and $B\z=(x_B,y_B)$ and
the circle with center $O=(x_O,y_O)$ that passes thru the point $W=(x_W,y_W)$.
Let us write the equations of the circle and the line in the coordinates $(x,y)$:
$$
\left\{
\begin{aligned}
(x-x_O)^2+(y-y_O)^2&=(x_W-x_O)^2+(y_W-y_O)^2,
\\
(x-x_A)\cdot(y_B-y_A)&=(y-y_A)\cdot(x_B-x_A).
\end{aligned}
\right.
$$
Note that coordinates $(x_1,y_1)$ and $(x_2,y_2)$ of the points $X_1$ and $X_2$ are solutions of this system.
Expressing $y$ from the second equation and substituting the result in the first one, gives us a quadratic equation in $x$, 
which can be solved using ``$+$'', ``$-$'', ``$\cdot$'', ``$/\,$''
and  ``$\sqrt{\phantom{a}}\,$'' only.


The same can be performed for the intersection of two circles. 
The intersection of two lines is even simpler; 
it is described as a solution of two linear equations and can be expressed using only four arithmetic operations;
the square root ``$\sqrt{\phantom{a}}\,$'' is not needed.

\medskip

On the other hand, it is easy to make compass-and-ruler constructions that produce segments of the lengths $a+b$ and $a-b$ from two given segments of lengths $a>b$.

\begin{wrapfigure}{r}{37mm}
\vskip-6mm
\centering
\includegraphics{mppics/pic-286}
\end{wrapfigure}

To perform ``$\cdot$'', ``$/\,$''
and ``$\sqrt{\phantom{a}}\,$'' consider the following diagram:
let $[AB]$ be a diameter of a circle; 
fix a point $C$ on the circle and let $D$ be the footpoint of $C$ on~$[AB]$.
By Corollary~\ref{cor:right-angle-diameter}, the angle $ACB$ is right.
Therefore 
$$\triangle ABC\sim\triangle ACD\sim \triangle CBD.$$
It follows that $AD\cdot BD=CD^2$. 

Using this diagram, one should guess a solution to the following exercise.

\begin{thm}{Exercise}\label{ex:a2/b}
Given two line segments with lengths $a$ and $b$, give a ruler-and-compass construction of segments with lengths $\tfrac {a^2}b$ and $\sqrt{a\cdot b}$.
\end{thm}


Taking $1$ for $a$ or $b$ above, we can produce 
$\sqrt a$, $a^2$, $\tfrac1b$.
Combining these constructions we can produce
$a\cdot b=(\sqrt{a\cdot b})^2$,
$\tfrac ab=a\cdot\tfrac 1b$.
In other words, we produced a \index{compass-and-ruler calculator}\emph{compass-and-ruler calculator} which can do ``$+$'', ``$-$'', ``$\cdot$'', ``$/\,$'', and the square root ``$\sqrt{\phantom{a}}\,$''.

The discussion above sketches a proof of the following theorem:
 
\begin{thm}{Theorem}\label{thm:constructible-numbers}
Assume that the initial configuration of geometric construction is given by the points $A_1=(0,0)$, $A_2=(1,0)$, $A_3\z=(x_3,y_3),\z\dots,A_n=(x_n,y_n)$.
Then a point $X=(x,y)$ can be constructed using a compass-and-ruler construction
if and only if both coordinates $x$ and $y$ can be expressed from the integer numbers and $x_3$, $y_3$, $x_4$, $y_4,\z\dots,x_n,y_n$ using the arithmetic operations ``$+$'', ``$-$'', ``$\cdot$'', ``$/\,$'', and ``$\sqrt{\phantom{a}}\,$''.
\end{thm}

The numbers that can be expressed from the given numbers using the arithmetic operations and the square root ``$\sqrt{\phantom{a}}\,$'' are called \index{constructible numbers}\emph{constructible};
if the list of given numbers is not given, then we can only use the integers.

{\sloppy
The theorem above translates any compass-and-ruler construction problem into a purely algebraic language.
Let us give some examples.
\begin{itemize}
\item The impossibility of the doubling-cube problem states that $\sqrt[3]{2}$ is not a constructible number.
That is, $\sqrt[3]{2}$ cannot be expressed thru integers using
``$+$'', ``$-$'', ``$\cdot$'', ``$/\,$'', and ``$\sqrt{\phantom{a}}\,$''.

\item The impossibility of squaring the circle states that 
$\sqrt{\pi}$, or equivalently $\pi$, is not a constructible number.

\item The impossibility of the angle trisection states that $\cos\tfrac\alpha3$ is not a constructible number from $\cos\alpha$.

\item The Gauss--Wantzel theorem says for which integers $n$ the number 
$\cos\tfrac{2\cdot\pi}n$ is constructible.
\end{itemize} 
Some of these statements might look evident, 
but rigorous proofs require some knowledge of abstract algebra (namely, field theory)
which is out of the scope of this book. 

}

In the next section, we discuss similar but simpler examples of impossible constructions with an unusual tool.

\begin{thm}{Exercise}\label{ex:5-gon}
\begin{enumerate}[(a)]
 \item\label{ex:5-gon:a} Show that the diagonal of a regular pentagon is $\tfrac{1+\sqrt5}2$ times larger than its side.
 \item\label{ex:5-gon:b} Use (\ref{ex:5-gon:a}) to make a compass-and-ruler construction of a regular pentagon.
\end{enumerate}
\end{thm}

\section{Set-square constructions}

{

\begin{wrapfigure}[5]{r}{26mm}
\vskip-12mm
\centering
\includegraphics{mppics/pic-288}
\end{wrapfigure}

A \index{set-square}\emph{set-square} (or 45°-\emph{set-square}) is a construction tool shown in the picture ---
it can produce a line thru a given point
that makes the angles
$\tfrac\pi2$ or $\pm\tfrac\pi4$ 
to a given line plus it can be used as a ruler.

}

\begin{thm}{Exercise}\label{ex:trisect-set-square}
Trisect a given segment with a set-square.
\end{thm}

The following theorem is an analog of Theorem~\ref{thm:constructible-numbers} for the set-square constructions.

\begin{thm}{Theorem}\label{thm:set-square-constructible-numbers}
Assume that the initial configuration of a geometric construction is given by the points $A_1=(0,0)$, $A_2=(1,0)$, $A_3\z=(x_3,y_3),\z\dots,A_n=(x_n,y_n)$.
Then a point $X=(x,y)$ can be constructed using a set-square construction
if and only if both coordinates $x$ and $y$ can be expressed from the integer numbers and $x_3$, $y_3$, $x_4$, $y_4,\z\dots,x_n$, $y_n$ using the arithmetic operations ``$+$'', ``$-$'', ``$\cdot$'', and ``$/$'' only. 
\end{thm}

The proof of this theorem is close to Theorem~\ref{thm:constructible-numbers}.
(The ``if'' part nearly follows from Exercise~\ref{ex:R-hom}.
The ``only-if'' part is proved by induction on the number of elementary constructions; one needs to write an equation for each line in a set-square construction and verify that an intersection point of such lines satisfies the theorem.)

Unlike Theorem~\ref{thm:constructible-numbers} it can be applied directly to show the impossibility of some constructions with a set-square --- no need to use the field theory.

Note that if all the coordinates $x_3,y_3,\dots,x_n,y_n$ are rational numbers, then the theorem above implies that with a set-square, one can only construct the points with rational coordinates.
A point with both rational coordinates is called \index{rational point}\emph{rational},
and if at least one of the coordinates is irrational, then the point is called \index{irrational point}\emph{irrational}.

\begin{thm}{Exercise}\label{ex:equilateral triangle}
Show that it is impossible to construct an equilateral triangle with a given base using
a set-square.
\end{thm}

{

\begin{wrapfigure}{r}{20mm}
\vskip-8mm
\centering
\includegraphics{mppics/pic-289}
\end{wrapfigure}


\begin{thm}{Exercise}\label{ex:set-square-bisect}
Show that it is impossible to bisect a given angle with a set-square only.
\end{thm}

\begin{thm}{Advanced exercise}\label{ex:90-60-30}
Consider another tool --- a 30°-set-square that can produce a line thru a given point
that makes the angles
$\tfrac\pi2$, $\pm\tfrac\pi3$, $\pm\tfrac\pi6$
to a given line and can be used as a ruler.

{Show that it is impossible to construct a square with a 30°-set-square.}
\end{thm}

}

\section{Verifications}
\label{sec:verification}

Suppose we need to verify that a given configuration is defined by a certain property. 
Is it possible to do this task by geometric construction with the given tools?
We assume that we can \emph{verify} that two constructed points coincide.

Evidently, if a configuration is constructible, then it is \index{verifiable construction}\emph{verifiable}%
\footnote{Adopting the terminology of computability theory, we may also say that such a construction is \index{decidable construction}\emph{decidable}.} --- simply repeat the construction and check if the result is the same.
Some nonconstuctable configurations are verifiable.
For example, it does not pose a problem to verify that the given angle is trisected while it is impossible to trisect a given angle with a ruler and compass.
A regular 7-gon provides another example of that type --- it is easy to verify, while Gauss--Wantzel theorem states that it is impossible to construct with a ruler and compass.

Since we did not prove the impossibility of angle trisection and the Gauss--Wantzel theorem, the following example might be more satisfactory.
It is based on Exercise~\ref{ex:equilateral triangle} which states that it is impossible to construct an equilateral with set-square only.

\begin{thm}{Exercise}\label{ex:equilateral triangle-verify}
Make a set-square construction verifying that 
\begin{enumerate}[(a)]
\item\label{ex:verify:triangle} a triangle is equilateral.
\item\label{ex:verify:bisector} a line bisects an angle.
\end{enumerate}
\end{thm}

This observation leads to a source of impossible constructions in a stronger sense --- those that are even not verifiable.

The following example is closely related to Exercise~\ref{ex:circumtool}.
Recall that a \index{circumtool}\emph{circumtool} produces a circle passing thru any given three points
or a line if all three points lie on one line;
the \index{inversor}\emph{inversor} --- a tool that constructs an inverse of a given point in a given circline.


\begin{thm}{Problem}\label{prob:center-inversor+circumtool}
Show that with a circumtool and inversor,
it is impossible to verify that a given point is the center of a given circle~$\Gamma$.
In particular, it is impossible to construct the center with a circumtool only.
\end{thm}

\parbf{Remark.}
In geometric constructions, we allow to choose \index{free points}\emph{free points},
say any point on the plane, or a point on a constructed line, or a point that does not lie on a constructed line, 
or a point on a given line that does not lie on a given circle, and so on.

In principle, when you make such a free choice it is possible to make the right construction by accident.
Nevertheless, we do not accept such a coincidence as true construction; 
we say that construction produces the center if it produces it for any free choices.


\parit{Solution.}\label{page:solution-for-ex:circumtool}
Arguing by contradiction, 
assume we have a verifying construction. 

Apply an inversion across a circle perpendicular to $\Gamma$ to the whole construction.
According to Corollary~\ref{cor:perp-inverse-clines},
the circle
$\Gamma$ maps to itself.
Recall that the inversion sends a circline to a circline (\ref{thm:inverse-cline}) and respects inversion (\ref{cor:invese-comp}).
Therefore we get that the whole  construction is mapped to an equivalent construction; 
that is, a constriction with a different choice of free points.

According to Exercise~\ref{ex:inv-center not=center-inv}, 
the inversion sends the center of $\Gamma$ to another point.
However, this construction claims that this new point is the center as well --- a contradiction.
\qeds

A similar example of impossible constructions for a ruler and a parallel tool
 is given in Exercise~\ref{ex:affine-perp}.
 
Let us discuss another example of a ruler-only construction.
Note that ruler-only constructions are invariant with respect to the projective transformations. 
In particular, to solve the following exercise, it is sufficient to construct a projective transformation that fixes two points and moves their midpoint.

\begin{thm}{Exercise}\label{ex:midpoint-proj}
Show that there is no ruler-only construction verifying that a given point is a  midpoint of a given segment.
In particular, it is impossible to construct the midpoint only with a ruler.
\end{thm}

The following theorem is a stronger version of the exercise above.

\begin{thm}{Theorem}\label{thm:circle-center-proj}
There is no ruler-only construction verifying that a given point is the center of a given circle.
In particular, it is impossible to construct the center only with a ruler.
\end{thm}

The proof uses the construction in Exercise~\ref{ex:cone}.

\parit{Sketch of the proof.}
The same argument as in the problem above shows that 
it is sufficient to construct a projective transformation 
that sends the given circle $\Gamma$ to a circle $\Gamma'$ such that the center of $\Gamma'$ is not the image of the center of~$\Gamma$.

Choose a circle $\Gamma$ that lies in the plane $\Pi$ in the Euclidean space.
By Theorem~\ref{thm:inversion-3d}, 
the inverse of a circle across a sphere is a circle or a line.
Fix a sphere $\Sigma$ with the center $O$ so that the inversion $\Gamma'$ of $\Gamma$
is a circle and the plane $\Pi'$ containing $\Gamma'$ is not parallel to $\Pi$;
any sphere $\Sigma$ in a general position will do.

Let $Z$ and $Z'$ denote the centers of $\Gamma$ and~$\Gamma'$.
Note that  $Z'\z\notin(OZ)$.
It follows that the perspective projection $\Pi\to \Pi'$ with center $O$ sends $\Gamma$ to $\Gamma'$, but $Z'$ is not the image of~$Z$.
\qeds

\section{Comparison of construction tools}

We say that one set of tools is \emph{stronger} than another if any configuration of points that can be constructed with the second set can be constructed with the first set as well.
If in addition, there is a configuration constructible with the first set, but not constructible with the second, then we say that the first set is {}\emph{strictly stronger} than the second.
Otherwise (that is, if any configuration that can be constructed with the first set can be constructed with the second), we say that the sets of tools are \emph{equivalent}. 
Two sets of tools might be also \emph{not comparable};
that is, there are constructions possible with the first set of tools and not possible with the second, and the other way around.

As an example, consider the following classical result:

\begin{thm}{Mohr--Mascheroni theorem}
Compass alone is equivalent to compass and ruler.
\end{thm}

Note that the theorem does \emph{not} state that one can construct a whole line with a compass alone!
--- since we consider only configurations of points we do not have to.
One may think that a \emph{line is constructed} if we construct two points on it.

For sure compass and ruler form a stronger set than a compass alone.
Therefore Mohr--Mascheroni theorem will follow once we solve the following two construction problems:
\begin{enumerate}[(i)]
\item Given four points $X$, $Y$, $P$, and $Q$, construct the intersection of the lines $(XY)$ and $(PQ)$ with compass only.
\item Given two points $X$, $Y$, and a circle $\Gamma$, construct the intersection of the lines $(XY)$ and $\Gamma$ with a compass only.
\end{enumerate}
Indeed, once we have these two constructions, we can do every step of a compass-and-ruler construction using a compass alone.

If you wonder how such a theorem can be proved, read about the \emph{Peaucellier--Lipkin inversor};
it is a planar linkage capable of transforming rotary motion into perfect straight-line motion.
Another classical theorem that can be proved using this linkage is the so-called \emph{Poncelet--Steiner theorem};
it states that the set of compass and ruler is equivalent to the ruler alone, provided that a single circle and its center are given.

\begin{thm}{Exercise}\label{ex:comparison}
Compare the following set of tools:
(a) a ruler and compass, 
(b) a set-square, 
(c) a ruler and a parallel tool,
and
(d) a circumtool and an inversor.
\end{thm}



\chapter{Geometric constructions}
\label{chap:car}
\addtocontents{toc}{\protect\begin{quote}}


Geometric constructions has great pedagogical value 
as an introduction to mathematical proofs.
We were using construction problems 
everywhere starting from Chapter~\ref{chap:perp}.

In this chapter we briefly discuss the classical results in geometric constructions.

\section*{Classical problems}
\addtocontents{toc}{Classical problems.}

In this section we list couple of classical construction problems;
each known for more than thousand years. 
The solutions of the following two problems are quite nontrivial.

\begin{thm}{Problem of Brahmagupta} 
Construct an inscribed quadrilateral with the given sides.
\end{thm}


 
\begin{thm}{Problem of Apollonius} Construct a circle which is tangent to the three given circles.
\end{thm}

{
\begin{wrapfigure}{o}{29mm}
\begin{lpic}[t(-6mm),b(0mm),r(0mm),l(3mm)]{pics/apollonius(1)}
\end{lpic}
\end{wrapfigure}

The following exercise is a simplified version of the problem of Apollonius, which is still nontrivial.


\begin{thm}{Exercise}\label{ex:simple-apollonius}
Construct a circle which pass thru the given point and tangent to two intersecting lines.
\end{thm}
}





The following three problems cannot be solved in principle; 
that is, the needed compass-and-ruler construction does not exist.

\parbf{Doubling the cube.} {\it Construct the side of a new cube, 
which has volume twice bigger than the volume of the given cube.} 

\medskip

In other words, 
given a segment of length $a$
one needs to construct a segment of length~$\sqrt[3]{2}\cdot a$.

\parbf{Squaring the circle.} {\it Construct a square with the same area as the given circle.} 

\medskip

If $r$ is the radius of the given circle, we need to construct a segment of length~$\sqrt{\pi}\cdot r$. 

\parbf{Angle trisection.} 
{\it Divide the given angle into three equal angles.}

\medskip

In fact, there is no compass-and-ruler construction which trisects angle with measure~$\tfrac\pi3$. 
Existence of such construction would imply constructability of regular 9-gon which is prohibited by the following famous result.
%???unefined n-gon

\begin{thm}{Gauss--Wantzel theorem}
A regular $n$-gon is constructable with ruler and compass 
if and only if 
$n$ is the product of a power of $2$ and any number of distinct Fermat primes.
\end{thm}

The \index{Fermat prime}\emph{Fermat prime} is a prime numbers of the form $2^k+1$ for some integer~$k$.
Only five Fermat primes are known  today:
$$3, 5, 17, 257, 65537.$$
For example, 
\begin{itemize}
\item one can construct a regular 340-gon since $340=2^2\cdot 5\cdot 17$ and $5$ as well as $17$ are Fermat primes;
\item one cannot construct a regular 7-gon since $7$ is not a Fermat prime;
\item one cannot construct a regular 9-gon; 
altho $9=3\cdot 3$ is a product of two Fermat primes, 
these primes are not distinct.
\end{itemize}

\medskip

The impossibility of these constructions 
was proved only in 19th century.
The method used in the proofs is indicated in the next section.

\section*{Constructable numbers}
\addtocontents{toc}{Constructable numbers.}

In the classical compass-and-ruler constructions initial configuration can be completely described by a finite number of points;
each line is defined by two points on it and each circle is described by its center and a point on it (equivalently, you may describe a circle by tree points on it).

The same way the result of construction can be described by a finite collection of points.

Choose a  coordinate system, such that one of the initial points as the origin $(0,0)$ and yet another initial point has coordinates~$(1,0)$.
In this coordinate system,
the initial configuration of $n$ points is described by 
$2\cdot n-4$ numbers, their coordinates, say $(x_3,y_3,x_4,y_4,\z\dots,x_n,y_n)$.

\medskip

It turns out that the coordinates of any point constructed with compass and ruler
can be written thru the numbers $x_3,y_3,x_4,y_4,\z\dots,x_n,y_n$ using the four arithmetic operations ``$+$'', ``$-$'', ``$\cdot$'', ``$/$''
and the square root ``$\sqrt{\phantom{a}}$''.

For example, assume we want to find the points $X_1=(x_1,y_1)$ and $X_2\z=(x_2,y_2)$ of intersection of 
a line passing thru $A=(x_A,y_A)$ and $B\z=(x_B,y_B)$ and
the circle with center $O=(x_O,y_O)$ which pass thru a point $W=(x_W,y_W)$.
Let us write the equations of these circle and line in the coordinates $(x,y)$:
$$
\left[
\begin{matrix}
(x-x_O)^2+(y-y_O)^2=(x_W-x_O)^2+(y_W-y_O)^2    
\\
(x-x_A)\cdot(y_B-y_A)=(y-y_A)\cdot(x_B-x_A)  
\end{matrix}
\right.
$$
Expressing $y$ from the second equation and substituting the result in the first one, gives us a quadratic equation in $x$, 
which can be solved using ``$+$'', ``$-$'', ``$\cdot$'', ``$/$''
and  ``$\sqrt{\phantom{a}}$'' only.

The same can be performed for the intersection of two circles. 
The intersection of two lines is even simpler; 
it described as a solution of two linear equations and can be expressed using only four arithmetic operations;
the square root ``$\sqrt{\phantom{a}}$'' is not needed.

\medskip

On the other hand, it is easy to produce  compass-and-ruler constructions which produce a segment of length $a+b$ and $a-b$ from two given segments of lengths $a>b$.

\begin{wrapfigure}[11]{r}{47mm}
\begin{lpic}[t(-5mm),b(0mm),r(0mm),l(3mm)]{pics/geom-calc(1)}
\lbl[r]{0,22;$A$}
\lbl[l]{42,22;$B$}
\lbl[t]{9,20;$D$}
\lbl[rb]{9,38;$C$}
\end{lpic}
\end{wrapfigure}

To perform ``$\cdot$'', ``$/$''
and ``$\sqrt{\phantom{a}}$'' consider the following diagram.
Let $[AB]$ be diameter of a circle; 
Consider point $C$ on the circle and let $D$ be the foot point of $C$ on~$[AB]$.
Note that 
$$\triangle ABC\sim\triangle ACD\sim \triangle BDC.$$
It follows that $AD\cdot DC=BD^2$.  

Using this diagram, one should guess the compass-and-ruler constructions 
which produce a segments of lengths
$\sqrt{a\cdot b}$ and $\tfrac {a^2}b$.
Say to construct  $\sqrt{a\cdot b}$ one can do the following:
(1) construct points $A$, $B$ and $D\in [AB]$
such that $AD=a$ and $BD=b$
(2) construct a circle $\Gamma$ with diameter $[AB]$
(3) draw the perpendicular $\ell$ thru $D$ to $(AB)$ 
(4) let $C$ be an intersection of $\Gamma$ and~$\ell$.
Then $DC= \sqrt{a\cdot b}$.

Taking $1$ for $a$ or $b$ above we can produce 
$\sqrt a$, $a^2$, $\tfrac1b$.
Further combining these constructions we can produce
$a\cdot b=(\sqrt{a\cdot b})^2$,
$\tfrac ab=a\cdot\tfrac 1b$.
In other words we produced a {}\emph{compass-and-ruler calculator},
which can do ``$+$'', ``$-$'', ``$\cdot$'', ``$/$'' and ``$\sqrt{\phantom{a}}$''.

The discussion above gives a sketch of the proof of the following theorem:
 
\begin{thm}{Theorem}\label{thm:constructable-numbers}
Assume that initial configuration of geometric construction is given by points $A_1=(0,0)$, $A_2=(1,0)$, $A_3=(x_3,y_3),\z\dots,A_n=(x_n,y_n)$.
Then a point $X=(x,y)$ can be constructed using a compass-and-ruler construction
if and only if both coordinates $x$ and $y$ can be expressed from the integer numbers and $x_3,y_3,x_4,y_4,\dots,x_n,y_n$ using the arithmetic operations ``$+$'', ``$-$'', ``$\cdot$'', ``$/$'' and the square root ``$\sqrt{\phantom{a}}$''.
\end{thm}

The numbers which can be expressed from the given numbers using the arithmetic operations and the square root ``$\sqrt{\phantom{a}}$'' are called \index{constructable numbers}\emph{constructable};
if the list of given numbers is not given, then we can only use the integers.

The theorem above translates any compass-and-ruler construction problem 
into purely algebraic language.
For example:
\begin{itemize}
\item Impossibility of solution for doubling cube problem states that $\sqrt[3]{2}$ is not a constructable number.
That is $\sqrt[3]{2}$ cannot be expressed thru integers using
``$+$'', ``$-$'', ``$\cdot$'', ``$/$'' and ``$\sqrt{\phantom{a}}$''.
\item Impossibility of solution for squaring the circle states that 
$\sqrt{\pi}$, or equivalently $\pi$, is not a constructable number.
\item Gauss--Wantzel theorem tells for which integers $n$ the number 
$\cos\tfrac{2\cdot\pi}n$ is constructable.
\end{itemize} 
Some of these statements might look evident, 
but a rigorous proofs require some knowledge of abstract algebra (namely, field theory)
which is out of scope of this book. 

In the next section we discuss a similar but simpler examples of impossible constructions with an unusual tool.

\section*{Constructions with set square}
\addtocontents{toc}{Constructions with set square.}
{
\begin{wrapfigure}{o}{19mm}
\begin{lpic}[t(-9mm),b(0mm),r(0mm),l(0mm)]{pics/set-square(1)}
\end{lpic}
\end{wrapfigure}

Set square is a construction tool 
which can produce a line thru the given point
which makes angle
$\tfrac\pi2$ or $\pm\tfrac\pi4$ 
with the given line.

}
\begin{thm}{Exercise}\label{ex:trisect-set-square}
Trisect the given segment with ruler and set square.
\end{thm}


Consider ruler-and-set-square construction.
Using the same idea as in the previous section,
we can define {}\emph{ruler-and-set-square constructable numbers}
and prove the following analog of  Theorem~\ref{thm:constructable-numbers}.

\begin{thm}{Theorem}
Assume that initial configuration of geometric construction is given by points $A_1=(0,0)$, $A_2=(1,0)$, $A_3=(x_3,y_3),\z\dots,A_n=(x_n,y_n)$.
Then a point $X=(x,y)$ can be constructed using a ruler-and-set-square construction
if and only if both coordinates $x$ and $y$ can be expressed from the integer numbers and $x_3$, $y_3$, $x_4$, $y_4,\z\dots,x_n$, $y_n$ using the arithmetic operations ``$+$'', ``$-$'', ``$\cdot$'', ``$/$''. 
\end{thm}

We omit the proof of this theorem, but it can be build on the ideas described in the previous section. 
Let us show how to use this theorem to show impossibility of some constructions with ruler and set square.

Note that if all the coordinates $x_3,y_3,\dots,x_n,y_n$ are rational numbers, then the theorem above implies that with ruler and set square one can only construct the points with rational coordinates.
A point with both rational coordinates is called \index{rational point}\emph{rational},
and if at least one of the coordinates is irrational, then the point is called \index{irrational point}\emph{irrational}.

\begin{thm}{Exercise}\label{ex:equilateral triangle}
Show that an equilateral triangle in Euclidean plane has at least one irrational point.

Conclude that with ruler and set square one cannot construct an equilateral triangle.
\end{thm}


\begin{thm}{Exercise}\label{ex:equilateral triangle-verify}
Make a ruler-and-set-square construction which verifies if the given triangle is  equilateral.
(We assume that we can ``verify'' if two constructed points coincide.) 
\end{thm}


\section*{More impossible constructions}
\addtocontents{toc}{More impossible constructions.}

In this section we discuss yet another source of impossible constructions. 

Recall that a \index{circumtool}\emph{circumtool} produces a circle passing thru given three points
or a line if all three points lie on one line.
Let us restate Exercise~\ref{ex:circumtool}.

\parbf{Exercise.}
\textit{Show that with circumtool only,
it is impossible to construct the center of given circle~$\Gamma$.}
\medskip

\parbf{Remark.}
In geometric constructions we allow to choose some free points,
say any point on the plane, or a point on a constructed line, or a point which does not lie on a constructed line and so on.

In principle, when you make such a free choice it is possible to mark the center of $\Gamma$ by an accident.
Nevertheless, we do not accept such coincidence as true construction; 
we say that a construction produce the center if it produce it for any free choices.


\parit{Solution.}\label{page:solution-for-ex:circumtool}
Arguing by contradiction, 
assume we have a construction of the center. 

Apply to whole construction an inversion in a circle perpendicular to~$\Gamma$.
According to Corollary~\ref{cor:perp-inverse-clines},
the circle
$\Gamma$ maps to itself.
Since inversion sends circline to circline, we get that all construction is mapped to an equivalent construction; 
that is, a constriction with different choice of free points.

According to Exercise~\ref{ex:inv-center not=center-inv}, 
the inversion sends the center of $\Gamma$ to another point.
That is, following the same construction, we can end up at a different point, a contradiction.
\qeds

\begin{thm}{Exercise}\label{ex:center-verify}
Show that there is no circumtool-only construction which verifies if the given point is the center of given circle.
(We assume that we can only ``verify'' if two constructed points coincide.) 
\end{thm}

A similar example of impossible constructions for ruler and parallel tool
 is given in Exercise~\ref{ex:affine-perp}.
 
Let us discuss yet another example for ruler-only construction.
Note that ruler-only constructions are invariant with respect to the projective transformations. 
In particular, to solve the following exercise, it is sufficient to construct a projective transformation which fix two points $A$ and $B$ and moves its midpoint.

\begin{thm}{Exercise}\label{ex:midpoint-proj}
Show that  midpoint of given segment cannot be constructed with ruler only.
\end{thm}

The following theorem is a stronger version of the exercise above.

\begin{thm}{Theorem}\label{thm:circle-center-proj}
The center of given circle cannot be constructed with ruler only.
\end{thm}

\parit{Proof.}
To prove this theorem 
it is sufficient to construct a projective transformation 
which sends the given circle $\Gamma$ to a circle, say $\Gamma'$ but the center of $\Gamma'$ is not the image of center of~$\Gamma$.

Let $\Gamma$ be a circle which lies in the plane $\Pi$ in the Euclidean space.

By Theorem~\ref{thm:inverion-3d}, 
inverse of circle in a sphere is a circle or a line.
Fix a sphere $\Sigma$ with center $O$ so that the inversion $\Gamma'$ of $\Gamma$
is a circle and the plane $\Pi'$ containing $\Gamma'$ is not parallel to $\Pi$;
any sphere $\Sigma$ in general position will do.

Denote by $Z$ and $Z'$ the centers of $\Gamma$ and~$\Gamma'$.
Note that  $Z'\z\notin(OZ)$.
It follows that the perspective projection $\Pi\to \Pi'$ with center at $O$ sends $\Gamma$ to $\Gamma'$, but $Z'$ is not the image of~$Z$.
\qeds

\section*{Construction of polar}


Assume $\Gamma$ is a circle in the plane.
Given point $P$ consider two chords $[XX']$ and $[YY']$ of $\Gamma$
such that $(XX')\cap (YY')=P$.
Let $Z=(XY)\cap(X'Y')$ and $Z'=(XY')\cap(X'Y)$ and $p=(ZZ')$.
If $P\in \Gamma$ we assume that $p$ is the tangent to $\Gamma$ at~$P$.

\begin{thm}{Claim}\label{clm:polar}
The line $p=(ZZ')$ does not depend on the choice of chords.
Moreover, $P\mapsto p$ is a duality (see page~\pageref{page:duality}).
\end{thm}

\begin{wrapfigure}[7]{o}{30mm}
\begin{lpic}[t(-4mm),b(0mm),r(0mm),l(0mm)]{pics/polar(1)}
\lbl[t]{27,8;$P$}
\lbl[l]{15,0;$p$}
\lbl[rb]{1,6;$X$}
\lbl[ltw]{19.5,8;$X'$}
\lbl[b]{9,19.5;$Y$}
\lbl[b]{17.5,16.5;$Y'$}
\lbl[l]{14,24;$Z$}
\lbl[trw]{12,12;$Z'$}
\end{lpic}
\end{wrapfigure}

The line $p$ is called polar of $P$ with respect to~$\Gamma$.
The same way the point $P$ is called polar of the line $p$ with respect to~$\Gamma$.

We will not give a proof of this claim, but will try to use it in the constructions.

\begin{thm}{Exercise}\label{ex:tangent ruler}
Let $p$ be the polar line of point $P$ with respect to the circle~$\Gamma$.
Assume that $p$ intersects $\Gamma$ at points $V$ and~$W$.
Show that the lines $(PV)$ and $(PW)$ are tangent to~$\Gamma$.

Come up with a ruler-only construction of the tangent lines to the given circle thru the given point.
\end{thm}

\begin{thm}{Exercise}\label{ex:concentric-circ}
Assume two concentric circles $\Gamma$ and $\Gamma'$ are given.
Construct the common center of $\Gamma$ and $\Gamma'$ with ruler only.
\end{thm}

\addtocontents{toc}{\protect\end{quote}}

%%%???Constuctions 
%perp{ex:construction-perpendicular}{ex:center}{ex:tangent}{ex:tangent-circle}
%inscriberd-angle{ex:two-right}{ex:perpendicular-ruler}{ex:3x120}
%inversion{ex:consturuction-of-inversion}
%affine{ex:R-hom}
%car{ex:simple-apollonius}{ex:trisect-set-square}{ex:tangent ruler}{ex:concentric-circ}
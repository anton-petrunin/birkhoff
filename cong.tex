\chapter{Congruent triangles}\label{chap:cong}
\addtocontents{toc}{\protect\begin{quote}}

\section*{Side-angle-side condition}
\addtocontents{toc}{Side-angle-side condition.}

Our next goal is to give conditions which guarantee congruence of two triangles.

One of such conditions is given in Axiom~\ref{def:birkhoff-axioms:3}; it states that if two pairs of sides of two triangles are equal, and the included angles are equal up to sign, then the triangles are congruent.
This axiom is also called {}\emph{side-angle-side congruence condition}, or briefly, \index{SAS congruence condition}\emph{SAS congruence condition}.

\section*{Angle-side-angle condition}
\addtocontents{toc}{Angle-side-angle condition.}

\begin{thm}[\abs]{ASA condition}\label{thm:ASA}\index{ASA congruence condition}
Assume that 
\begin{align*}
AB&=A'B',
&
\measuredangle A B C &= \pm\measuredangle A' B' C', 
&
\measuredangle C A B&=\pm\measuredangle C' A' B'
\end{align*}
 and $\triangle A' B' C'$ is nondegenerate.
Then 
$$\triangle A B C\cong\triangle A' B' C'.$$

\end{thm}

\begin{wrapfigure}[7]{o}{26mm}
\centering
\begin{lpic}[t(-4mm),b(3mm),r(0mm),l(0mm)]{pics/AA(1)}
\lbl[t]{1,.5;$A'$}
\lbl[b]{10.5,29;$B'$}
\lbl[t]{16,.5;$C'$}
\lbl[lt]{21,.5;$C''$}
\end{lpic}
\end{wrapfigure}

Note that for degenerate triangles the statement does not hold.
For example, consider one triangle with sides $1$, $4$, $5$ 
and the other with sides $2$, $3$,~$5$.

\parit{Proof.} 
According to Theorem~\ref{thm:signs-of-triug},
either
$$\begin{aligned}
 \measuredangle A B C &= \measuredangle A' B' C',
\\
\measuredangle C A B&=\measuredangle C' A' B'
\end{aligned}\eqlbl{eq:+angles}$$
or
$$\begin{aligned}
\measuredangle A B C &= -\measuredangle A' B' C',
\\
\measuredangle C A B&=-\measuredangle C' A' B'.
\end{aligned}\eqlbl{eq:-angles}$$
Further we assume that \ref{eq:+angles} holds; 
the case \ref{eq:-angles} is analogous.



Let $C''$ be the point on the half-line $[A' C')$ such that $A' C''\z=A C$. 

By Axiom~\ref{def:birkhoff-axioms:3}, 
$\triangle A' B' C''\cong \triangle A B C$. 
Applying Axiom~\ref{def:birkhoff-axioms:3} again,
we get that
$$\measuredangle A' B' C'' = \measuredangle A B C=\measuredangle A' B' C'.$$
By Axiom~\ref{def:birkhoff-axioms:2a}, $[B'C')=[B C'')$. 
Hence
$C''$ lies on $(B' C')$ as well as on~$(A' C')$.

Since $\triangle A' B' C'$ is not degenerate, $(A' C')$ is distinct from~$(B' C')$.
Applying Axiom~\ref{def:birkhoff-axioms:1}, we get that $C''=C'$. 

Therefore, 
$\triangle A' B' C'=\triangle A' B' C''\cong\triangle A B C$.
\qeds

\section*{Isosceles triangles}
\addtocontents{toc}{Isosceles triangles.}

A triangle with two equal sides is called \index{isosceles triangle}\emph{isosceles};
the remaining side is called the \index{base}\emph{base}.


\begin{thm}[\abs]{Theorem}\label{thm:isos}
Assume $\triangle A B C$ is an isosceles triangle with the base $[A B]$. 
Then 
$$\measuredangle A B C\equiv -\measuredangle B A C.$$
Moreover, the converse holds if $\triangle A B C$ is nondegenerate.
\end{thm}

\begin{wrapfigure}{o}{29mm}
\centering
\begin{lpic}[t(0mm),b(0mm),r(0mm),l(2mm)]{pics/isos(1)}
\lbl[rt]{0,1;$A$}
\lbl[lt]{23,1;$B$}
\lbl[b]{11.5,27;$C$}
\end{lpic}
\end{wrapfigure}

The following proof is due to Pappus of Alexandria.

\parit{Proof.}
Note that
$$C A = C B,
\quad 
C B=C A,
\quad
\measuredangle A C B \equiv -\measuredangle B C A.$$
Therefore, by Axiom~\ref{def:birkhoff-axioms:3},
$$\triangle C A B\cong\triangle C B A.$$
Applying the theorem on the signs of angles of triangles (\ref{thm:signs-of-triug}) and Axiom~\ref{def:birkhoff-axioms:3} again,
we get that 
$$\measuredangle B A C
\equiv -\measuredangle A B C.$$

To prove the converse, we assume that
$\measuredangle C A B \z\equiv - \measuredangle C B A$.
By ASA condition \ref{thm:ASA}, $\triangle C A B\z\cong\triangle CBA$.
Therefore,~$C A\z=C B$.
\qeds

A triangle with three equal sides is called \index{equilateral triangle}\emph{equilateral}. 

\begin{thm}{Exercise}\label{ex:equilateral}
Let $\triangle ABC$ be an equilateral triangle.
Show that 
\[\measuredangle ABC=\measuredangle BCA=\measuredangle CAB.\]

\end{thm}


\section*{Side-side-side condition}
\addtocontents{toc}{Side-side-side condition.}

\begin{thm}[\abs]{SSS condition}\label{thm:SSS}\index{SSS congruence condition}
$\triangle A B C\cong\triangle A' B' C'$ if 
$$A' B'=A B,
\quad 
B' C'=B C
\quad 
\text{and}
\quad 
C' A'=C A.$$

\end{thm}

Note that this condition is valid for degenerate triangles as well.

\parit{Proof.} 
Choose $C''$ so that $A' C''= A' C'$ and $\measuredangle B' A' C''= \measuredangle B A C$.
According to Axiom~\ref{def:birkhoff-axioms:3},
$$\triangle A' B' C''\cong\triangle A B C.$$

It will suffice to
prove that 
$$\triangle A' B' C'\cong\triangle A' B' C''.\eqlbl{eq:A'B'C'simA'B'C''}$$
The condition \ref{eq:A'B'C'simA'B'C''} trivially holds if $C''\z=C'$.
Thus, it remains to consider the case $C''\z\ne C'$.

\begin{wrapfigure}{o}{40mm}
\centering
\begin{lpic}[t(0mm),b(0mm),r(0mm),l(2mm)]{pics/SSS(1)}
\lbl[r]{1.5,20.5;$A'$}
\lbl[l]{35.5,20.5;$B'$}
\lbl[b]{24,40.5;$C'$}
\lbl[t]{24,1;$C''$}
\end{lpic}
\end{wrapfigure}

Clearly, the corresponding sides of $\triangle A' B' C'$ and $\triangle A' B' C''$ are equal.
Hence the triangles
$\triangle C' A' C''$ and $\triangle C' B' C''$ are isosceles.
By Theorem~\ref{thm:isos}, we have 
\begin{align*}
 \measuredangle A' C'' C'&\equiv -\measuredangle A' C' C'',
\\
\measuredangle C' C'' B'&\equiv -\measuredangle C'' C' B'.
\end{align*}
Adding them, we get that
$$\measuredangle A' C'' B'
\equiv -\measuredangle A' C' B'.$$
Applying Axiom~\ref{def:birkhoff-axioms:3} again,
we get \ref{eq:A'B'C'simA'B'C''}.
\qeds

{

\begin{wrapfigure}{o}{34mm}
\centering
\begin{lpic}[t(-4mm),b(-0mm),r(0mm),l(1mm)]{pics/isos-2(1)}
\lbl[t]{30,1;$B$}
\lbl[t]{1,1;$A$}
\lbl[lb]{26,13;$A'$}
\lbl[rb]{7,13;$B'$}
\lbl[b]{16,28.5;$C$}
\end{lpic}
\end{wrapfigure}

\begin{thm}{Advanced exercise}\label{ex:SMS}
Let $M$ be the midpoint of the side $[A B]$ of $\triangle A B C$ and
$M'$ be the midpoint of the side $[A' B']$ of $\triangle A' B' C'$.
Assume $C' A'=C A$, $C' B'= C B$ and $C' M'\z= C M$.
Prove that 
\[\triangle A' B' C'\z\cong\triangle A B C.\]

\end{thm}

\begin{thm}{Exercise}\label{ex:isos-sides}
Let $\triangle A B C$ be an isosceles triangle with the base $[A B]$.
%???(
Suppose that the points $A'\z\in [B C]$ and $B'\z\in[A C]$ are such that $C A'\z=C B'$.
%???)
Show that
\end{thm}
}
\vskip-2mm
{\it
\begin{enumerate}[(a)]
\item $\triangle A A' C\cong \triangle B B' C$;
\item $\triangle A B B'\cong \triangle B A A'$.
\end{enumerate}
}

\begin{thm}{Exercise}\label{ex:degenerate-trig}
Show that if $AB+BC=AC$
 then $B\in [AC]$.
\end{thm}

\begin{thm}{Exercise}\label{ex:ABC-motion}
Let $\triangle ABC$ be a nondegenerate triangle and 
let $f$ be a motion of the plane 
such that 
$$f(A)=A,
\quad 
f(B)=B
\quad 
\text{and}
\quad
f(C)=C.$$

Show that $f$ is the identity;
that is, $f(X)=X$ for any point $X$ on the plane.
\end{thm}

%(???

\section*{On angle-side-side and side-angle-angle}
\addtocontents{toc}{On angle-side-side and side-angle-angle.}

In each of the conditions SAS, ASA, and SSS we specify three corresponding parts of the triangles.
Let us discuss other triples of corresponding parts.

The fist triple is called {}\emph{side-side-angle}, or briefly SSA;
it specifies two sides and a non-included angle.
This condition is not sufficient for congruence;
that is, there are two nondegenerate triangles $ABC$ and $A'B'C'$ such that
\[AB=A'B',\quad BC=B'C',\quad \measuredangle BAC\equiv\pm \measuredangle B'A'C',\]
but $\triangle ABC\not\cong\triangle A'B'C'$ and moreover $AC\ne A'C'$.

\begin{wrapfigure}{r}{55mm}
\centering
\begin{lpic}[t(-4mm),b(-0mm),r(0mm),l(1mm)]{pics/ASS(1)}
\lbl[t]{2,0;$A$}
\lbl[lb]{22.5,16;$B$}
\lbl[t]{17,0;$C$}
\lbl[t]{27,0;$A'$}
\lbl[lb]{48.5,16;$B'$}
\lbl[t]{52,0;$C'$}
\end{lpic}
\end{wrapfigure}

We will not use this negative statement in the sequel and therefore there is no need to prove it formally.
An example can be guessed from the diagram.

The second triple is {}\emph{side-angle-angle}, or briefly SAA;
it specifies one side and two angles one of which is opposite to the side.
This provides a congruence condition; 
that is, if one of the triangles $ABC$ and $A'B'C'$ is nondegenerate then
\[AB=A'B',\quad \measuredangle ABC\equiv\pm \measuredangle A'B'C',\quad\measuredangle BCA\z\equiv\pm \measuredangle B'C'A'\]
implies $\triangle ABC\cong\triangle A'B'C'$.

The SAA condition will not be used directly in the sequel.
One proof of this condition can be obtained from ASA and the theorem on sum of angles of triangle proved below (see~\ref{thm:3sum}). 
For a more direct proof, see Exercise~\ref{ex:SAA}.

Another triple is called {}\emph{angle-angle-angle}, or briefly AAA;
by Axiom~\ref{def:birkhoff-axioms:4}, it is not sufficient for congruence. However AAA turns out to be a congruence condition in the hyperbolic plane; see \ref{thm:AAA}.

%)???


\addtocontents{toc}{
{\sloppy

}
\protect\end{quote}}

%\part*{Incidence geometry}
\addtocontents{toc}{\protect\begin{center}}
\addtocontents{toc}{\large{\bf Incidence geometry}}
\addtocontents{toc}{\protect\end{center}}

\chapter{Affine geometry}\label{chap:trans}
\addtocontents{toc}{\protect\begin{quote}}

\section*{Affine transformations}
\addtocontents{toc}{Affine transformations.}

A bijection of Euclidean plane to itself 
is called \index{affine transformation}\emph{affine transformation}
if it maps any line to a line.

We say that three points are \index{collinear}\emph{collinear} if they lie on one line. 
Note that affine transformation sends collinear points to collinear; the following exercise gives a converse.

\begin{thm}{Exercise}\label{ex:collinear=affine}
Assume $f$ is a bijection from Euclidean plane to itself which sends collinear points to collinear points.
Show that $f$ is an affine transformation.
(In other words, show that $f$ maps noncollinear points to noncollinear.)
\end{thm}

\begin{thm}{Exercise}\label{ex:affine-par}
Show that affine transformation sends parallel lines to the parallel lines.
\end{thm}

\emph{Affine geometry} studies  so called \index{incidence structure}\emph{incidence structure} of Euclidean plane.
The incidence structure is the data about which points lie on which lines and nothing else;
we can not talk about distances, angles and so on.
One may also say that affine geometry studies
the properties of Euclidean plane which preserved under affine transformations.

\section*{Constructions with parallel tool and ruler}
\addtocontents{toc}{Constructions with parallel tool and ruler.}

Let us consider geometric constructions with ruler and \index{parallel tool}\emph{parallel tool};
the later makes possible to draw a line through the given point parallel to the given line.
By Exercisers~\ref{ex:affine-par}, any construction with these two tools are invariant with respect to affine transformation.
For example, 
to solve the following exercise,
it is sufficient to prove that midpoint of given segment can be constructed with ruler and parallel tool.

\begin{thm}{Exercise}\label{ex:midpoint-affine}
Let $M$ be the midpoint of segment $[AB]$ in the Euclidean plane.
Assume that an affine transformation sends points $A$, $B$ and $M$
to $A'$, $B'$ and $M'$ correspondingly.
Show that $M'$ is the midpoint of $[A'B']$.
\end{thm}

The following exercise will be used in the proof of Theorem~\ref{thm:affine=linear}.

\begin{thm}{Exercise}\label{ex:R-hom}
Assume that in Euclidean plane we have 4 points with coordinates 
$(0,0)$, $(1,0)$, $(a,0)$ and $(b,0)$.
Use ruler and parallel tool to construct the points with coordinates $(a\cdot b,0)$ and $(a+b,0)$.
\end{thm}

\begin{thm}{Exercise}\label{ex:center-circ-affine}
Use ruler and parallel tool to construct the center of the given circle.
\end{thm}

\section*{Matrix form}
\addtocontents{toc}{Matrix form.}

Since the lines are defined in terms of metric;
any motion of Euclidean plane is also an affine transformation.

On the other hand, 
there are affine transformations of Euclidean plane which are not motions.

Consider Euclidean plane with coordinate system;
let us use the column notation for the coordinates;
that is, we will write $\left(\begin{smallmatrix}
x\\y
\end{smallmatrix} \right)$ instead of $(x,y)$.

As it follows from the theorem below,
the so called {}\emph{shear mapping} $\left(\begin{smallmatrix}
x\\ y
\end{smallmatrix} \right)\mapsto \left(\begin{smallmatrix}
x+k\cdot y\\ y
\end{smallmatrix} \right)$ is an affine transformation.
The shear mapping can change the angle between vertical and horizontal lines almost arbitrary.
The later can be used to prove impossibility of some constructions with ruler and parallel tool;
here is one example.

\begin{thm}{Exercise}\label{ex:affine-perp}
Show that with ruler and parallel tool one can not construct
a line perpendicular to the given line.
\end{thm}

\begin{thm}{Theorem}\label{thm:affine=linear}
A map $\beta$ from the plane to itself
is an affine transformation if and only if 
\[\beta\:\left(\begin{smallmatrix}
x\\ y
\end{smallmatrix} \right)
  \mapsto
  \left(\begin{smallmatrix}
a&b\\ c&d
\end{smallmatrix} \right)
  \cdot
  \left(\begin{smallmatrix}
x\\ y
\end{smallmatrix} \right)
  +
\left(\begin{smallmatrix}
v\\ w
\end{smallmatrix} \right)
\eqlbl{eq:affine=linear}
\]
for some fixed invertible matrix $\bigl(\begin{smallmatrix}
a&b\\ c&d
\end{smallmatrix} \bigr)$ and vector $\bigl(\begin{smallmatrix}
v\\ w
\end{smallmatrix} \bigr)$.

In particular, any affine transformation of Euclidean plane is continuous.
\end{thm}

In the proof of ``only if'' part,
we will use the following algebraic lemma.

\begin{thm}{Algebraic lemma}\label{lem:R-auto}
Assume $f\:\mathbb{R}\to\mathbb{R}$ is a function such that
\begin{align*}
f(1)&=1,\\
f(x+y)&=f(x)+f(y),\\ 
f(x\cdot y)&=f(x)\cdot f(y) 
\end{align*}
for any $x,y\in\mathbb{R}$.
Then $f(x)=x$ for any $x\in \mathbb{R}$.
\end{thm}

Note that we do not assume that $f$ is continuous.

The function $f$ satisfying three conditions in the lemma
is called \index{field automorphism}\emph{field automorphism}.
Therefore the lemma states that the identity function is the only automorphism of the field of real numbers.

On the other hand, the conjugation $z\mapsto\bar z$
(see page \pageref{page:cojugation=authomorphism})
gives an example of nontrivial automorphism of complex numbers.

\parit{Proof.}
Since 
\[f(0)+f(1)=f(0+1),\]
we get 
\[f(0)+1=1;\]
that is,
\[f(0)=0.
\eqlbl{eq:0=0}\]

Further
\[0=f(0)=f(x)+f(-x).\]
Therefore 
\[f(-x)=-f(x)\ \text{for any}\ \ x\in \mathbb{R}.
\eqlbl{eq:f-x}\] 

Further
\begin{align*}
f(2)&=f(1)+f(1)=1+1=2\\
f(3)&=f(2)+f(1)=2+1=3\\
&\dots
\end{align*}
Together with \ref{eq:f-x},
the later implies that 
$$f(n)=n\ \ \text{for any integer}\ \ n.$$ 

Since
\[f(m)=f(\tfrac mn)\cdot f(n)\]
we get
$$f(\tfrac mn)=\tfrac mn$$
for any rational number $\tfrac mn$.

Assume $a\ge 0$.
Then the equation $x\cdot x=a$ has a real solution $x=\sqrt{a}$.
Therefore $[f(\sqrt{a})]^2=f(\sqrt{a})\cdot f(\sqrt{a})=f(a)$.
Whence $f(a)\ge 0$.
That is
\[a\ge 0\ \ \Longrightarrow\ \ f(a)\ge 0.\eqlbl{a>0=>b>0}\]

Applying \ref{eq:f-x}, 
we also get 
\[a\le 0\ \ \Longrightarrow\ \ f(a)\le 0.\eqlbl{a<0=>b<0}\]

Finally, assume $f(a)\ne a$ for some $a\in\mathbb{R}$.
Then there is a rational number $\tfrac{m}{n}$ which lies between $a$ and $f(a)$;
that is, 
the numbers $x\z=a-\tfrac{m}{n}$ and $y=f(a)-\tfrac{m}{n}$ have opposite signs.
Since $f(\tfrac{m}{n})=\tfrac{m}{n}$, we get $f(x)=y$.
The later contradicts \ref{a>0=>b>0} or \ref{a<0=>b<0}.
\qeds

\begin{thm}{Lemma}\label{lem:3-fix}
Assume $\gamma$ is an affine transformation which fix three points $\left(\begin{smallmatrix}
0\\ 0
\end{smallmatrix} \right)$, 
$\left(\begin{smallmatrix}
1\\ 0
\end{smallmatrix} \right)$ 
and $\left(\begin{smallmatrix}
0\\ 1
\end{smallmatrix} \right)$ on the coordinate plane.
Then $\gamma$ is the identity map; 
that is, $\gamma\left(\begin{smallmatrix}
x\\ y
\end{smallmatrix} \right)
=
\left(\begin{smallmatrix}
x\\ y
\end{smallmatrix} \right)$ for any point $\left(\begin{smallmatrix}
x\\ y
\end{smallmatrix} \right)$.
\end{thm}

\parit{Proof.}
Since affine transformation sends lines to lines, we get that each axes is mapped to itself.

According to Exercise~\ref{ex:affine-par}, 
parallel lines are mapped to parallel lines.
Therefore we get that horizontal lines mapped to horizontal lines 
and
vertical lines mapped to vertical.
In other words,
\[\gamma\left(\begin{smallmatrix}
x\\ y
\end{smallmatrix} \right)
=
\left(\begin{smallmatrix}
f(x)\\ h(y)
\end{smallmatrix} \right).\]
for some functions $f,h\:\mathbb{R}\to\mathbb{R}$.

Note that $f(1)=h(1)=1$ and according to 
Exercise \ref{ex:R-hom}, 
both $f$ and $h$ satisfies the other two conditions of Algebraic lemma~\ref{lem:R-auto}.
Applying the lemma, we get that $f$ and $h$ 
are identity functions
and so is $\gamma$.
\qeds

\parit{Proof of Theorem~\ref{thm:affine=linear}.}
Recall that matrix 
$\left(\begin{smallmatrix}
a&b\\ c&d
\end{smallmatrix} \right)$
is invertible if
$$\det\left(\begin{smallmatrix}
a&b\\ c&d
\end{smallmatrix} \right)
=a\cdot d-b\cdot c\ne 0;$$
in this case the matrix 
\[\tfrac{1}{a\cdot d-b\cdot c}
  \cdot\left(\begin{smallmatrix}
d&-b\\ -c&a
\end{smallmatrix} \right)\] 
is the inverse of 
$\left(\begin{smallmatrix}
a&b\\ c&d
\end{smallmatrix} \right)$.

Assume that the map $\beta$ is described by \ref{eq:affine=linear}.
Note that
\[\left(\begin{smallmatrix}
x\\ y
\end{smallmatrix} \right)
\mapsto
  \tfrac{1}{a\cdot d-b\cdot c}
  \cdot\left(\begin{smallmatrix}
d&-b\\ -c&a
\end{smallmatrix} \right)
\cdot
\left(\begin{smallmatrix}
x-v\\ y-w
\end{smallmatrix} \right).
\eqlbl{eq:beta-inv}
\] 
is inverse of $\beta$.
In particular $\beta$ is a bijection.

A line in the plane form the solutions of equations
\[p\cdot x+q\cdot y+r=0,\eqlbl{ax+by+c=0}\]
where $p\ne 0$ or $q\ne 0$.
Find $\left(\begin{smallmatrix}
x\\ y
\end{smallmatrix} \right)$ from its $\beta$-image by formula \ref{eq:beta-inv} and substitute the result in \ref{ax+by+c=0}.
You will get the equation of the image of the line.
The equation has the same type as \ref{ax+by+c=0}, 
with different constants, in particular it describes a line.
Therefore $\beta$ is an affine transformation.

To prove ``only if'' part,
fix an affine transformation $\alpha$.
Set 
\begin{align*}
\left(\begin{smallmatrix}
v\\ w
\end{smallmatrix} \right)
&= 
\alpha\left(\begin{smallmatrix}
0\\ 0
\end{smallmatrix} \right),
\\
\left(\begin{smallmatrix}
a\\ c
\end{smallmatrix} \right)
&=
\alpha\left(\begin{smallmatrix}
1\\ 0
\end{smallmatrix} \right)
-
\alpha\left(\begin{smallmatrix}
0\\ 0
\end{smallmatrix} \right),
\\
\left(\begin{smallmatrix}
b\\ d
\end{smallmatrix} \right)
&=
\alpha\left(\begin{smallmatrix}
1\\ 0
\end{smallmatrix} \right)
-
\alpha\left(\begin{smallmatrix}
0\\ 0
\end{smallmatrix} \right).
\end{align*}


Note that the points 
$\alpha\left(\begin{smallmatrix}
0\\ 0
\end{smallmatrix} \right)$, 
$\alpha\left(\begin{smallmatrix}
0\\ 1
\end{smallmatrix} \right)$, 
$\alpha\left(\begin{smallmatrix}
1\\ 0
\end{smallmatrix} \right)$ do not lie on one line. 
Therefore the matrix $\bigl(\begin{smallmatrix}
a&b\\ c&d
\end{smallmatrix} \bigr)$
is invertible.

For the affine transformation $\beta$ defined by \ref{eq:affine=linear}
we have 
\begin{align*}
\beta\left(\begin{smallmatrix}
0\\ 0
\end{smallmatrix} \right)
&=
\alpha\left(\begin{smallmatrix}
0\\ 0
\end{smallmatrix} \right),
\\
\beta\left(\begin{smallmatrix}
1\\ 0
\end{smallmatrix} \right)
&=
\alpha\left(\begin{smallmatrix}
1\\ 0
\end{smallmatrix} \right),
\\
\beta\left(\begin{smallmatrix}
0\\ 1
\end{smallmatrix} \right)
&=
\alpha\left(\begin{smallmatrix}
0\\ 1
\end{smallmatrix} \right).
\end{align*}


It remains to show that $\alpha=\beta$ or equivalently the composition $\gamma=\alpha\circ \beta^{-1}$ is the identity map.

Note that $\gamma$ is an affine transformation which fix points $\left(\begin{smallmatrix}
0\\ 0
\end{smallmatrix} \right)$, 
$\left(\begin{smallmatrix}
1\\ 0
\end{smallmatrix} \right)$ 
and $\left(\begin{smallmatrix}
0\\ 1
\end{smallmatrix} \right)$.
It remains to apply Lemma~\ref{lem:3-fix}.
\qeds

\section*{On inversive transformations}
\addtocontents{toc}{On inversive transformations.}


Recall that inversive plane is Euclidean plane with added a point at infinity, denoted as $\infty$.
We assume that any line pass through $\infty$.
The term {}\emph{circline} stays for {}\emph{circle or line};

The \index{inversive transformation}\emph{inversive transformation} is bijection from inversive plane to itself which sends circlines to circlines.
Inversive geometry can be defined as geometry which  {}\emph{circline incidence structure} of inversive plane;
that is we can say which points lie on which circlines.

\begin{thm}{Theorem}\label{thm:inversions-inversive}
A map from inversive plane to itself is an inversive transformation
if and only if it can be presented as a composition of inversions and reflections.  
\end{thm}

\parit{Proof.}
According to Theorem~\ref{thm:inverse-cline} any inversion is a inversive transformation.
Therefore the same holds for composition of inversions.

To prove converse, 
fix an inversive transformation $\alpha$.

Assume $\alpha(\infty)=\infty$.
Recall that any circline passing through $\infty$ is a line.
If follows that $\alpha$ maps lines to lines;
that is,
it is an affine transformation.

Further, $\alpha$ is not an arbitrary affine transformation,
it maps circles to circles.

Composing $\alpha$ with a reflection, say $\rho_1$, we can assume that $\alpha'=\rho_1\circ\alpha$ maps the unit circle with center at the origin to a concentric circle. 

Composing the obtained map $\alpha'$ with a {}\emph{homothety} 
\[\chi\:\left(\begin{smallmatrix}
x\\ y
\end{smallmatrix} \right)\mapsto \left(\begin{smallmatrix}
k\cdot x\\ k\cdot y
\end{smallmatrix} \right),\]
we can assume that $\alpha''=\chi\circ\alpha'$ sends the unit circle to itself.


Composing the obtained map $\alpha''$
 a reflection $\rho_2$ in a line through the origin,
we can assume that in addition $\alpha'''=\rho_2\circ\alpha''$ the point $(1,0)$ maps to itself.

By Exercise~\ref{ex:center-circ-affine},
$\alpha'''$ fixes the center of the circle;
that is, it fixes the origin.

The obtained map $\alpha'''$ is an affine transformation.
Applying Theorem~\ref{thm:affine=linear}, together with the properties of $\alpha''$ described above we get
\[\alpha'''\:\left(\begin{smallmatrix}
x\\ y
\end{smallmatrix} \right)
  \mapsto
  \left(\begin{smallmatrix}
1&b\\ 0&d
\end{smallmatrix} \right)
  \cdot
  \left(\begin{smallmatrix}
x\\ y
\end{smallmatrix} \right)
\]
for an invertible matrix $\left(\begin{smallmatrix}
1&b\\ 0&d
\end{smallmatrix} \right)$.
Since the point $(0,1)$ maps to the unit circle we get 
\[b^2+d^2=1.\]
Since the point $(\tfrac1{\sqrt{2}},\tfrac1{\sqrt{2}})$ maps to the unit circle we get 
\[(b+d)^2=1.\]
It follows 
\[\alpha'''\:\left(\begin{smallmatrix}
x\\ y
\end{smallmatrix} \right)
  \mapsto
  \left(\begin{smallmatrix}
1&0\\ 0&\pm1
\end{smallmatrix} \right)
  \cdot
  \left(\begin{smallmatrix}
x\\ y
\end{smallmatrix} \right);
\]
that is, either $\alpha'''$ is the identity map 
or reflection the $x$-axis.

Note that the homothety $\chi$ is a composition of two inversions in concentric circles.
Hence the result follows.

Now assume $P=\alpha(\infty)\ne \infty$.
Consider an inversion $\beta$ in a circle with center at $P$.
Note that $\beta(P)=\infty$; therefore $\beta\circ\alpha(\infty)=\infty$.
Since $\beta$ is inversive, so is $\beta\circ\alpha$.
From above we get that $\beta\circ\alpha$ is a composition of reflections and inversions therefore so is $\alpha$.
\qeds

\begin{thm}{Exercise}\label{ex:reflection/inversive}
Show that any reflection can be presented as a composition of three inverses. 
\end{thm}

Note that exercise above together with Theorem~\ref{thm:inversions-inversive},
implies that any inversive map is a composition of inversions,
no reflections are needed.

\addtocontents{toc}{\protect\end{quote}}
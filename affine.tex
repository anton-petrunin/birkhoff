%\part*{Additional topics}
\addtocontents{toc}{\protect\begin{center}}
\addtocontents{toc}{\large{\bf Additional topics}} %???pagebrake in toc
\addtocontents{toc}{\protect\end{center}}

\chapter{Affine geometry}\label{chap:trans}
\addtocontents{toc}{\protect\begin{quote}}

\section*{Affine transformations}
\addtocontents{toc}{Affine transformations.}

\emph{Affine geometry} studies the so called \index{incidence structure}\emph{incidence structure} of the Euclidean plane.
The incidence structure says which points lie on which lines and nothing else;
we cannot talk about distances, angle measures, and so on.

In other words, affine geometry studies
the properties of the Euclidean plane which preserved under {}\emph{affine transformations} defined below.

A bijection from Euclidean plane to it self is called \index{affine transformations}\emph{affine transformation} if it maps lines to lines;
that is, image of any line is a line.

\begin{thm}{Exercise}\label{ex:affine-par}
Show that affine transformation of the Euclidean plane sends a pair of parallel lines to a pair of parallel lines.
\end{thm}

Since the lines are defined using only metric, we get the following observation.

\begin{thm}{Observation}
Any motion of Euclidean plane is an affine transformation.
\end{thm}

The following exercise provides more general examples of affine transformations.

\begin{thm}{Exercise}\label{ex:afine-linear}
The following maps of coordinate plane to it self are affine transformations:
\begin{enumerate}[(a)]
\item\label{ex:afine-linear:shear} Shear map defined by $(x,y)\mapsto (x+k\cdot y,y)$ for a constant~$k$.
\item\label{ex:afine-linear:scaling} Scaling defined by $(x,y)\mapsto (a\cdot x,a\cdot y)$ for a constant $a\ne 0$.
\item $x$-scaling and $y$-scaling defined correspondingly by 
\[(x,y)\mapsto (a\cdot x,y),\quad\text{and}\quad(x,y)\mapsto (x,a\cdot y)\]
for a constant $a\ne 0$.
\item\label{affine-general-formula} A transformation defined by
\[(x,y)\mapsto(a\cdot x+b\cdot y+r,c\cdot x+d\cdot y+s)\]
for constants $a,b,c,d,r,z$ such that the matrix $(\begin{smallmatrix}a&b\\c&d\end{smallmatrix})$ is invertible. 
\end{enumerate}
\end{thm}

From Corollary~\ref{cor:affine-linear}, it will follows that any affine transformation can be written in the form \textit{(\ref{affine-general-formula})}.

Recall that three points are \index{collinear}\emph{collinear} if they lie on one line.

\begin{thm}{Exercise}\label{ex:collinear=affine}
Suppose $P\mapsto P'$ is a bijection of Euclidean plane that maps collinear triple of points to collinear triples.
Show that $P\mapsto P'$ maps noncollinear triples of points to noncollinear.

Conclude that $P\mapsto P'$ is an affine transformation.
\end{thm}


\section*{Constructions}
\addtocontents{toc}{Constructions.}

Let us consider geometric constructions with a ruler and a \index{parallel tool}\emph{parallel tool};
the latter makes possible to draw a line thru a given point parallel to a given line.
By Exercisers~\ref{ex:affine-par}, any construction with these two tools are invariant with respect to affine transformations.
For example, 
to solve the following exercise,
it is sufficient to prove that midpoint of a given segment can be constructed with a ruler and a parallel tool.

\begin{thm}{Exercise}\label{ex:midpoint-affine}
Let $M$ be the midpoint of segment $[AB]$ in the Euclidean plane.
Assume that an affine transformation sends the points $A$, $B$, and $M$
to $A'$, $B'$, and $M'$ correspondingly.
Show that $M'$ is the midpoint of~$[A'B']$.
\end{thm}

The following exercise will be used in the proof of Theorem~\ref{thm:affine=linear}.

\begin{thm}{Exercise}\label{ex:R-hom}
Assume that the points with the coordinates $(0,0)$, $(1,0)$, $(a,0)$, and $(b,0)$ are given.
Using a ruler and a parallel tool, construct the points with the coordinates $(a\cdot b,0)$ and $(a+b,0)$.
\end{thm}

\begin{thm}{Exercise}\label{ex:center-circ-affine}
Use ruler and parallel tool to construct the center of the given circle.
\end{thm}

Note that the shear map described in \ref{ex:afine-linear}\textit{\ref{ex:afine-linear:shear}} can change the angle between lines almost arbitrary.
The latter can be used to prove impossibility of some constructions with a ruler and a parallel tool;
here is one example.

\begin{thm}{Exercise}\label{ex:affine-perp}
Show that with a ruler and a parallel tool one cannot construct a line perpendicular to a given line.
\end{thm}

\section*{Vector identity}
\addtocontents{toc}{Vector identity.}

Further we assume knowledge of vector algebra.
In particular, we use that if $\square ABCD$ is a parallelogram, then $\overrightarrow{AB}=\overrightarrow{DC}$ and 
$\overrightarrow{AC}=\overrightarrow{AB}+\overrightarrow{AD}$.

\begin{thm}{Proposition}\label{prop:affine-linear}
Let $P\mapsto P'$ be an affine transformation of Euclidean plane.
Then, for any triple of points of points $O$, $X$, $P$, we have
\[\overrightarrow{OP}=t\cdot \overrightarrow{OX}
\quad\text{if and only if}\quad
\overrightarrow{O'P'}=t\cdot \overrightarrow{O'X'}
\eqlbl{eq:OP=tOX}\]

\end{thm}

\parit{Proof.}
Observe that the affine transformation described in Exercise~\ref{ex:afine-linear} as well as all motions satisfy the condition \ref{eq:OP=tOX}.
Therefore a given affine transformation $P\mapsto P'$ satisfies \ref{eq:OP=tOX} if and only if its composition with motions and rescalings satisfies \ref{eq:OP=tOX}.

Applying this observation we can reduce the problem to its partial case.
Namely, we may assume that $O'=O$, $X'=X$;
further we can assume that $O$ is the origin of a coordinate system and $X$ has coordinates $(1,0)$.

In particular the affine transformation $P\mapsto P'$ maps $x$-axis to it self.
Note that if $\overrightarrow{OP}=t\cdot \overrightarrow{OX}$, then $P=(t,0)$ for some $t\in \mathbb{R}$.
Since $P'$ lies on $x$-axis, we have $P'=(f(t),0)$ for some function $t\mapsto f(t)$.
It remains to show that 
\[f(t)=t
\eqlbl{eq:f(x)=x}\]
for any $t$.

Since $O'=O$ and $X'=X$, we get that $f(0)=0$ and $f(1)=1$.
Further, according to Exercise~\ref{ex:R-hom}, we have that 
$f(x\cdot y)=f(x)\cdot f(y)$ and $f(x+y)=f(x)+f(y)$ for any $x,y\in\mathbb{R}$.

By the algebraic lemma (see \ref{lem:R-auto}), these conditions imply \ref{eq:f(x)=x}.
\qeds

\begin{thm}{Corollary}\label{cor:affine-linear}
Suppose an affine transformations maps a nondegenerate triange $OXY$ to a triangle $O'X'Y'$.
If 
\[\overrightarrow{OP}=x\cdot\overrightarrow{OX}+y\cdot\overrightarrow{OY},\]
then 
\[\overrightarrow{O'P'}=x\cdot\overrightarrow{O'X'}+y\cdot\overrightarrow{O'Y'}.\]
\end{thm}

\parit{Proof.}
If $a=0$ or $b=0$, then the statement follow directly from the proposition.

Otherwise consider points $V$ and $W$ defined by
\[\overrightarrow{OV}=x\cdot\overrightarrow{OX},
\quad
\overrightarrow{OW}=y\cdot\overrightarrow{OY}.
\]
By the proposition,
\[\overrightarrow{O'V'}=x\cdot\overrightarrow{O'X'},
\quad
\overrightarrow{O'W'}=y\cdot\overrightarrow{O'Y'}.
\]

Note that 
\[\overrightarrow{OP}=\overrightarrow{OV}+\overrightarrow{OW},\]
or, equivalently, $\square OVPW$ is a parallelogram.
According to Exercise~\ref{ex:affine-par}, $\square O'V'P'W'$ is a parallelogram as well.
Therefore 
\begin{align*}\overrightarrow{O'P'}&=\overrightarrow{O'V'}+\overrightarrow{O'W'}=
\\
&=x\cdot\overrightarrow{O'X'}+y\cdot\overrightarrow{O'Y'}.
\end{align*}
\qedsf

\begin{thm}{Exercise}\label{ex:affine-continuous}
Show that any affine transformation is continuous.
\end{thm}

The following exercise provides converse to Exercise \ref{affine-general-formula}\textit{\ref{affine-general-formula}}.

\begin{thm}{Exercise}\label{ex:affine-coordinates}
Show that any affine transformation can be written in coordinates as 
\[(x,y)\mapsto(a\cdot x+b\cdot y+r,c\cdot x+d\cdot y+s)\]
for constants $a,b,c,d,r,z$ such that the matrix $(\begin{smallmatrix}a&b\\c&d\end{smallmatrix})$ is invertible. 
\end{thm}


\section*{Algebraic lemma}
\addtocontents{toc}{Algebraic lemma.}

The following lemma was used in the proof of Proposition~\ref{prop:affine-linear}.

\begin{thm}{Lemma}\label{lem:R-auto}
Assume $f\:\mathbb{R}\to\mathbb{R}$ is a function such that for any $x,y\in\mathbb{R}$ we have
\begin{enumerate}[(a)]
\item\label{lem:R-auto:a} $f(1)=1$,
\item\label{lem:R-auto:b} $f(x+y)=f(x)+f(y)$,
\item\label{lem:R-auto:c} $f(x\cdot y)=f(x)\cdot f(y)$.
\end{enumerate}

Then $f$ is the identity function; that is,
$f(x)=x$ for any $x\in \mathbb{R}$.
\end{thm}

Note that we do not assume that $f$ is continuous.

The function $f$ satisfying these three conditions
is called \index{field automorphism}\emph{field automorphism}.
Therefore, the lemma states that the identity function is the only automorphism of the field of real numbers.
For the field of complex numbers, the conjugation $z\mapsto\bar z$ (see page \pageref{page:cojugation=authomorphism}) gives an example of nontrivial automorphism.

\parit{Proof.}
By \textit{(\ref{lem:R-auto:b})} we have
\[f(0)+f(1)=f(0+1).\]
By \textit{(\ref{lem:R-auto:a})}
\[f(0)+1=1;\]
whence
\[f(0)=0.
\eqlbl{eq:0=0}\]

Applying \textit{(\ref{lem:R-auto:b})} again, we get that
\[0=f(0)=f(x)+f(-x).\]
Therefore, 
\[f(-x)=-f(x)
\quad
\text{for any}
\quad
x\in \mathbb{R}.
\eqlbl{eq:f-x}\] 

Applying \textit{(\ref{lem:R-auto:b})} recurrently, we get that
\begin{align*}
f(2)&=f(1)+f(1)=1+1=2;\\
f(3)&=f(2)+f(1)=2+1=3;\\
&\dots
\end{align*}
Together with \ref{eq:f-x},
the latter implies that 
$$f(n)=n
\quad
\text{for any integer}
\quad
n.$$ 

By \textit{(\ref{lem:R-auto:c})}
\[f(m)=f(\tfrac mn)\cdot f(n).\]
Therefore
$$f(\tfrac mn)=\tfrac mn \eqlbl{eq:m/n}$$
for any rational number~$\tfrac mn$.

Assume~$a\ge 0$.
Then the equation $x\cdot x=a$ has a real solution $x\z=\sqrt{a}$.
Therefore, $[f(\sqrt{a})]^2=f(\sqrt{a})\cdot f(\sqrt{a})=f(a)$.
Hence $f(a)\ge 0$.
That is,
\[a\ge 0\quad\Longrightarrow\quad f(a)\ge 0.\eqlbl{a>0=>b>0}\]

Applying \ref{eq:f-x}, 
we also get 
\[a\le 0\quad \Longrightarrow\quad f(a)\le 0.\eqlbl{a<0=>b<0}\]

Now we are ready to the final step in the proof.
Assume $f(a)\ne a$ for some $a\in\mathbb{R}$.
Then there is a rational number $\tfrac{m}{n}$ which lies between $a$ and $f(a)$;
that is, 
the numbers 
\[x\z=a-\tfrac{m}{n}\quad\text{and}\quad y=f(a)-\tfrac{m}{n}\]
have opposite signs.

By \ref{eq:m/n},
\begin{align*}
y+\tfrac{m}{n}&=f(a)=
\\
&=f(x+\tfrac{m}{n})=
\\
&=f(x)+f(\tfrac{m}{n})=
\\
&=f(x)+\tfrac{m}{n};
\end{align*}
that is, $f(x)=y$.
By \ref{a>0=>b>0} and \ref{a<0=>b<0} the values $x$ and $y$ cannot have opposite signs --- a contradiction.
\qeds

%\begin{thm}{Exercise}\label{ex:Q(sqrt(2)}
%Let $S$ denotes the set of all numbers $p+q\cdot\sqrt2$, where $p$ and $q$ are rational numbers.
%Observe that for the map $f\:p+q\cdot\sqrt2\mapsto p-q\cdot\sqrt2$, we have $f(1)=1$, $f(x+y)=f(x)+f(y)$, and $f(x\cdot y)=f(x)\cdot f(y)$ for any $x,y\in S$.
%Since $f(\sqrt2)=-\sqrt2$ the map is not identity, so the proof of algebraic lemma can not be ap
%\end{thm}



\section*{On inversive transformations}
\addtocontents{toc}{On inversive transformations.}


Recall that inversive plane is Euclidean plane with added a point at infinity, denoted by~$\infty$.
We assume that every line passes thru~$\infty$.
Recall that the term {}\emph{circline} stands for {}\emph{circle or line}.

An \index{inversive transformation}\emph{inversive transformation} is a bijection from inversive plane to itself that sends circlines to circlines.
\emph{Inversive geometry} studies the {}\emph{circline incidence structure} of inversive plane;
it says which points lie on which circlines.

\begin{thm}{Theorem}\label{thm:inversions-inversive}
A map from inversive plane to itself is an inversive transformation
if and only if it can be presented as a composition of inversions and reflections.  
\end{thm}

Exercise~\ref{ex:inversion-Mob} gives another description of inversive transformations which use complex coordinates.

\parit{Proof.}
According to Theorem~\ref{thm:inverse-cline} any inversion is an inversive transformation.
Therefore, the same holds for composition of inversions and reflection.

To prove the converse, 
fix an inversive transformation~$\alpha$.

Assume $\alpha(\infty)=\infty$.
Recall that any circline passing thru $\infty$ is a line.
If follows that $\alpha$ maps lines to lines;
that is,
$\alpha$ is an affine transformation that maps circles to circles.

Note that any motions and scalings (defined in Exercise~\ref{ex:afine-linear}\textit{\ref{ex:afine-linear:scaling}}) are affine transformations that map circles to circles.
Composing $\alpha$ with motions and scalings, we can obtain another affine transformation $\alpha'$ that maps circles to circle and one unit circle to it self.
By Exercise~\ref{ex:center-circ-affine},
$\alpha'$ fixes the center of the circle.

Applying Proposition~\ref{prop:affine-linear}, we get that $\alpha'$ is distance preserving;
that is, $\alpha'$ is a motion.

Summarizing, $\alpha$ is a composition of motions and rescalings.
Observe that any rescaling is a composition of two inversions in concentric circles.
Recall that any motion is a composition of reflections (see Exercise~\ref{ex:3-reflections}).
Whence $\alpha$ is a composition of inversions and reflections.


In the remaining case $\alpha(\infty)\ne \infty$, set $P=\alpha(\infty)$.
Consider an inversion $\beta$ in a circle with center at~$P$.
Note that $\beta(P)=\infty$; 
therefore, $\beta\circ\alpha(\infty)=\infty$.
Since $\beta$ is inversive, so is $\beta\circ\alpha$.
From above we get that $\beta\circ\alpha$ is a composition of reflections and inversions;
therefore, so is~$\alpha$.
\qeds

\begin{thm}{Exercise}\label{ex:inversive-angle}
Show that inversive transformations preserve the angle between arcs up to sign.

More precisely, assume $A'B_1'C_1'$, $A'B_2'C_2'$ are the images of two arcs $AB_1C_1$, $AB_2C_2$ under an inversive transformation.
Let $\alpha$ and $\alpha'$ denote the angle between the tangent half-lines to $AB_1C_1$ and $AB_2C_2$ at $A$
and the angle between the tangent half-lines to $A'B_1'C_1'$ and $A'B_2'C_2'$ at $A'$ correspondingly.
Then 
\[\alpha'=\pm \alpha.\]
\end{thm}


\begin{thm}{Exercise}\label{ex:reflection/inversive}
Show that any reflection can be presented as a composition of three inversions. 
\end{thm}

The exercise above imply a stronger version of Theorem~\ref{thm:inversions-inversive};
namely \emph{any inversive map is a composition of inversions} ---
no reflections needed.

\addtocontents{toc}{\protect\end{quote}}

\chapter[Similar triangles]{Similar triangles}\label{chap:parallel}




\section{Similar triangles}

Two triangles $A' B' C'$ and $A B C$ are called
\index{triangle!similar triangles}\index{similar triangles}\emph{similar} (briefly \index{30@$\sim$}$\triangle A' B' C'\z\sim\triangle A B C$) if (1) their sides are proportional; 
that is, 
$$A' B'
=
k\cdot A B,
\quad
B' C'=k\cdot B C
\quad
\text{and}
\quad
C' A'
=
k\cdot C A
\eqlbl{dist}
$$
for some $k>0$, and (2) the corresponding angles are equal up to sign:
$$
\begin{aligned}
\measuredangle A' B' C'&=\pm\measuredangle A B C,
\\
\measuredangle B' C' A'&=\pm\measuredangle B C A,
\\ 
\measuredangle C' A' B'&=\pm\measuredangle CAB.
\end{aligned}
\eqlbl{angles}
$$

\parbf{Remarks.}
\begin{itemize}
\item According to \ref{thm:signs-of-triug}, in the above three equalities, the signs can be assumed to be the same.

\item If $\triangle A' B' C'\sim\triangle A B C$ with $k=1$ in \ref{dist}, 
 then $\triangle A' B' C'\z\cong\triangle A B C$.

\item Note that ``$\sim$'' is an 
\index{equivalence relation}\emph{equivalence relation}.
That is, 
\begin{enumerate}[(i)]
\item $\triangle A B C\sim\triangle A B C$
for any $\triangle A B C$.
\item If $\triangle A' B' C'\sim\triangle A B C$, then
$$\triangle A B C\sim\triangle A' B' C'.$$
\item If $\triangle A'' B'' C''\sim\triangle A' B' C'$ and $\triangle A' B' C'\z\sim\triangle A B C$, then 
$$\triangle A'' B'' C''\sim\triangle A B C.$$
\end{enumerate}
\end{itemize}

Using the new notation ``$\sim$'', we can reformulate Axiom~\ref{def:birkhoff-axioms:4}:

\begin{thm}{Reformulation of Axiom~\ref{def:birkhoff-axioms:4}}
If for the two triangles 
$\triangle ABC$, 
$\triangle AB'C'$,
and $k>0$ we have
$B'\in [AB)$,
$C'\in [AC)$,
$AB'=k\cdot AB$ and
$AC'=k\cdot AC$,
then $\triangle ABC\sim\triangle AB'C'$.
\end{thm}

In other words, the Axiom~\ref{def:birkhoff-axioms:4} provides 
a condition which guarantees that two triangles are similar.
Let us formulate three more such {}\emph{similarity conditions}.

\begin{thm}{Similarity conditions}\label{prop:sim}
Two triangles 
$\triangle ABC$ and $\triangle A'B'C'$
are similar if one of the following conditions holds:

(SAS)\index{SAS similarity condition}
For some constant $k>0$ we have
$$A B=k\cdot A' B',
\quad 
A C=k\cdot A' C',$$
$$
\text{and}
\quad 
\measuredangle B A C=\pm\measuredangle B' A' C'.$$

(AA)\index{AA similarity condition} The triangle $A' B' C'$ is nondegenerate
and 
$$\measuredangle A B C
=
\pm\measuredangle A' B' C',
\quad 
\measuredangle B A C
=
\pm\measuredangle B' A' C'.$$

(SSS)\index{SSS similarity condition} For some constant $k>0$ we have
$$A B=k\cdot A' B',
\quad
A C=k\cdot A' C',
\quad
CB=k\cdot C'B'.$$

\end{thm}

Each of these conditions is proved by applying Axiom~\ref{def:birkhoff-axioms:4} with the SAS, ASA, and SSS congruence conditions respectively
(see Axiom~\ref{def:birkhoff-axioms:3} and the conditions \ref{thm:ASA}, \ref{thm:SSS}).


\parit{Proof.}
Set $k=\tfrac{AB}{A'B'}$.
Choose points $B''\in [A'B')$ and $C''\in [A'C')$,
so that $A'B''=k\cdot A'B'$ and $A'C''=k\cdot A'C'$.
By Axiom~\ref{def:birkhoff-axioms:4},
$\triangle A'B'C'\z\sim \triangle A'B''C''$.

Applying the SAS, ASA, or SSS congruence condition, depending on the case, 
we get that $\triangle A'B''C''\cong \triangle ABC$.
Hence the result.
\qeds



A bijection $X\leftrightarrow X'$ from a plane to itself is called \index{angle-preserving transformation}\emph{angle-preserving transformation} if 
\[\measuredangle ABC= \measuredangle A'B'C'\]
for any triangle $ABC$ and its image $\triangle A'B'C'$.

(The term \index{transformation}\emph{transformation} is used for a bijection of space to itself that preserves a specified geometric structure.
For example, {}\emph{motions} are {}\emph{distance-preserving transformations}.)



\begin{thm}{Exercise}\label{ex:angle-preserving-euclid}
Show that any angle-preserving transformation of the plane multiplies all distances by a fixed constant.
\end{thm}

\section{Pythagorean theorem}

A triangle is called \index{triangle!right triangle}\emph{right} if one of its angles is right.
The side opposite the right angle is called the \index{hypotenuse}\emph{hypotenuse}. 
The sides adjacent to the right angle are called \index{leg}\emph{legs}. 


\begin{thm}{Theorem}\label{thm:pyth}
Assume $\triangle ABC$ is a right triangle with the right angle at~$C$.
Then
$$AC^2+BC^2=AB^2.$$ 

\end{thm}

\parit{Proof.}
Let $D$ be the footpoint of $C$ on~$(AB)$.

\begin{wrapfigure}[4]{r}{40mm}
\vskip-4mm
\centering
\includegraphics{mppics/pic-66}
\end{wrapfigure}

According to Lemma~\ref{lem:perp<oblique},
\begin{align*}
AD&<AC<AB
\intertext{and}
BD&<BC<AB.
\end{align*}
Therefore, $D$ lies between $A$ and $B$;
in particular, 
$$AD+BD=AB.\eqlbl{AD+BD=AB}$$

Note that by the AA similarity condition, we have
$$\triangle ADC\sim\triangle ACB\sim \triangle CDB.$$
In particular,
$$
\frac{A D}{A C}=\frac{A C}{A B}
\quad
\text{and}
\quad
\frac{B D}{B C}=\frac{B C}{B A}.
\eqlbl{BCD}$$

Let us rewrite the two identities in \ref{BCD}:
\begin{align*}
AC^2=AB\cdot AD
\quad
\text{and}
\quad
BC^2=AB\cdot B D.
\end{align*}
Summing up these two identities and applying \ref{AD+BD=AB}, we get that
$$AC^2 +BC^2=AB\cdot (AD+ B D)=AB^2.$$
\qedsf

The idea in the proof above appears in the Elements \cite[X.33]{euclid},
but the proof given there \cite[I.47]{euclid} is different; 
it uses the area method discussed in Chapter~\ref{chap:area}.


\begin{thm}{Exercise}\label{ex:pyth}
Assume $A$, $B$, $C$, and $D$ are as in the proof above.
Show that 
$$CD^2=AD\cdot BD.$$

\end{thm}

The following exercise is the converse to the Pythagorean theorem.

\begin{thm}{Exercise}\label{ex:pyth-conv}
Assume that $ABC$ is a triangle such that
$$AC^2+BC^2=AB^2.$$ 
Prove that the angle at $C$ is right.
\end{thm}


\section{Method of similar triangles}

The proof of the Pythagorean theorem given above uses the {}\emph{method of similar triangles}.
To apply this method, one has to search for pairs of similar triangles and then use the proportionality of corresponding sides and/or equalities of corresponding angles.
Finding such pairs might be tricky at first. 

{

\begin{wrapfigure}{r}{25mm}
\vskip-6mm
\centering
\includegraphics{mppics/pic-68}
\end{wrapfigure}


\begin{thm}{Exercise}\label{ex:two-pairs-sim}
Let $ABC$ be a nondegenerate triangle and the points $X$, $Y$, and $Z$ as on the diagram.
Assume $\measuredangle CAY\z\equiv\measuredangle XBC$.
Find four pairs of similar triangles with these six points as the vertices
and prove their similarity.
\end{thm}

}

\section{Ptolemy's inequality}

A \index{quadrangle}\emph{quadrangle} is defined as an ordered quadruple of distinct points in the plane.
These 4 points are called \index{vertex!of quadrangle}\emph{vertices}.
The quadrangle $ABCD$ will be also denoted by \index{25@$\square$}$\square ABCD$.

Given a quadrangle $ABCD$,
the four segments $[AB]$, $[BC]$, $[CD]$, and $[DA]$ are called \index{side!of quadrangle}\emph{sides of $\square ABCD$};
the remaining two segments $[AC]$ and $[BD]$ are called \index{diagonal!of quadrangle}\emph{diagonals of $\square ABCD$}.

\begin{thm}{Ptolemy's inequality}\label{ptolemy-inq}
In any quadrangle, the product of diagonals cannot exceed the sum of the products of its opposite sides;
that~is, 
\[AC\cdot BD\le AB\cdot CD+ BC\cdot DA\]
for any $\square ABCD$.
\end{thm}

We will present a classical proof of this inequality using the method of similar triangles with additional construction.
This proof is given as an illustration --- it will not be used further in the sequel.

\begin{wrapfigure}{r}{33mm}
\centering
\includegraphics{mppics/pic-70}
\end{wrapfigure}

\parit{Proof.}
Consider the half-line $[AX)$ such that $\measuredangle BAX=\measuredangle CAD$.
In this case, $\measuredangle XAD\z=\measuredangle BAC$ since adding $\measuredangle BAX$ or $\measuredangle CAD$ to the corresponding sides produces $\measuredangle BAD$.
We can assume that
\[AX=\frac{AB}{AC}\cdot AD.\]
In this case, we have
\begin{align*}\frac{AX}{AD}&=\frac{AB}{AC},
&
\frac{AX}{AB}&=\frac{AD}{AC}.
\intertext{Hence}
\triangle BAX&\sim \triangle CAD,
&
\triangle XAD&\sim\triangle BAC.
\intertext{Therefore}
\frac{BX}{CD}&=\frac{AB}{AC},
&
\frac{XD}{BC}&=\frac{AD}{AC},
\intertext{or, equivalently}
AC\cdot BX&=AB\cdot CD,
&
AC\cdot XD&=BC\cdot AD.
\end{align*}
Adding these two equalities we get 
\[AC\cdot(BX+XD)=AB\cdot CD+BC\cdot AD.\]
It remains to apply the triangle inequality, $BD\le BX+XD$.
\qeds

Using the proof above together with \ref{prop:inscribed-quadrangle}, one can show that the equality holds only if the vertices $A$, $B$, $C$, and $D$ appear on a line or a circle in the same cyclic order;
see also \ref{ptolemy-id} for another proof of the equality case.
Exercise~\ref{ex:ptolemy} below suggests another proof of Ptolemy's inequality using complex coordinates.

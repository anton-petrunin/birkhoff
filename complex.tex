\chapter{Complex coordinates}\label{chap:complex}

In this chapter, we give an interpretation of inversive geometry using complex coordinates.
The results of this chapter lead to a deeper understanding of both concepts.

\section{Complex numbers}

Informally,
a complex number is a number that can be put in the form 
$$z=x+i\cdot y,
\eqlbl{eq:z=x+iy}$$ 
where $x$ and $y$ 
are real numbers and $i^2=-1$. 

The set of complex numbers 
will be further denoted by~$\mathbb{C}$.
If $x$, $y$, and $z$ are as in \ref{eq:z=x+iy}, 
then $x$ is called the \index{real!part}\emph{real part} and $y$ the \index{imaginary!part}\emph{imaginary part} of the complex number~$z$.
Briefly, it is written as 
\[x=\Re z
\quad
\text{and}
\quad 
y=\Im z.\]

On the more formal level, a complex number is a pair of real numbers $(x,y)$ with the addition and multiplication described below;
the expression $x + i\cdot y$ 
is only a convenient way 
to write the pair $(x,y)$.
\[
\begin{aligned}
(x_1+i\cdot y_1) + (x_2+i\cdot y_2) 
&\df (x_1+x_2) + i\cdot(y_1+y_2);
\\
(x_1+i\cdot y_1)\cdot(x_2+i\cdot y_2) 
&\df 
(x_1\cdot x_2-y_1\cdot y_2) + i\cdot(x_1\cdot y_2+y_1\cdot x_2).
\end{aligned}
\eqlbl{eq:comlex+x}
\] 

\section{Complex coordinates}

Recall that one can think of the Euclidean plane
as the set of all pairs of real numbers $(x,y)$ equipped with the metric 
$$AB=\sqrt{(x_A-x_B)^2+(y_A-y_B)^2},$$
where $A=(x_A,y_A)$ and $B=(x_B,y_B)$.

One can pack the coordinates $(x,y)$ of a point
in one complex number $z=x+i\cdot y$.
This way we get a one-to-one correspondence between points of the Euclidean plane and~$\mathbb{C}$.
Given a point $Z=(x,y)$, 
the complex number $z=x+ i\cdot y$ is called the \index{complex coordinate}\emph{complex coordinate} of~$Z$.

Note that if $O$, $E$, and $I$ are points in the plane 
with complex coordinates $0$, $1$, and $i$, then $\measuredangle EOI=\pm\tfrac\pi2$.
Further, we assume that $\measuredangle EOI=\tfrac\pi2$;
if not, one has to change the direction of the $y$-coordinate. 

\section{Conjugation and absolute value}
\label{sec:complex-conjugation}

Let $z=x+i\cdot y$; 
that is, $z$ is a complex number with real part $x$ and imaginary part $y$.
If $y=0$, we say that the complex number $z$ is \index{real!complex number}\emph{real} and if $x=0$ we say that $z$ is \index{imaginary!number}\emph{imaginary}.
The set of points 
with real (imaginary) complex coordinates is a line in the plane,
which is called \index{real!line}\emph{real} (respectively \index{imaginary!line}\emph{imaginary}) line. 
The real line will be denoted as $\mathbb{R}$.

The complex number
\[\bar z\df x-i\cdot y\] is called the \index{complex conjugate}\emph{complex conjugate} of $z=x+i\cdot y$.
Let $Z$ and $\bar Z$ be the points in the plane with the complex coordinates $z$ and $\bar z$ respectively.
Note that the point $\bar Z$ is the reflection of $Z$ across the real line.

It is straightforward to check that
$$\begin{aligned}
x&=\Re z=\frac{z+\bar z}2,
&
y&=\Im z=\frac{z-\bar z}{i\cdot2},
&
x^2+y^2&=z\cdot\bar z.
\end{aligned}\eqlbl{eq:conj-1}$$

The last formula in \ref{eq:conj-1} makes it possible to express the quotient $\tfrac{w}{z}$ of two complex numbers $w$ and $z=x+i\cdot y$:
$$\frac{w}{z}=\tfrac{1}{z\cdot\bar z}\cdot w\cdot\bar z=\tfrac{1}{x^2+y^2}\cdot w\cdot\bar z.$$

\label{page:cojugation=authomorphism}
Note that
\begin{align*}
\overline {z+ w}&=\bar z+\bar w,
&
\overline {z- w}&=\bar z-\bar w,
&
\overline {z\cdot w}&=\bar z\cdot\bar w,
&
\overline {z/w}&=\bar z/\bar w.
\end{align*}
That is, the complex conjugation
{}\emph{respects}
all the arithmetic operations.

The value 
\begin{align*}
|z|&\df\sqrt{x^2+y^2}=\sqrt{(x+i\cdot y)\cdot(x-i\cdot y)}
=
\sqrt{z\cdot\bar z}
\end{align*}
is called the
\index{absolute value}\emph{absolute value} of $z$.
If $|z|=1$, then $z$ is called a \index{unit complex number}\emph{unit complex number}.

\begin{thm}{Exercise}\label{ex:|zw|}
Show that $|v\cdot w|=|v|\cdot |w|$ for any $v,w\in\mathbb{C}$.
\end{thm}

Suppose that $Z$ and $W$ are points with complex coordinates $z$ and $w$.
Note that
$$ZW=|z-w|.\eqlbl{eq:C-dist}$$
The triangle inequality for the points with complex coordinates $0$, $v$, and $v+w$ implies that
\[|v+w|\le |v|+|w|\]
for any $v,w\in\mathbb{C}$;
this inequality is also called \index{triangle!inequality}\emph{triangle inequality}.

\begin{thm}{Exercise}\label{ex:ptolemy}
Use the identity 
\[u\cdot (v-w)+v\cdot (w-u)+w\cdot(u-v)=0\]
for $u,v,w\in\mathbb{C}$ and the triangle inequality
to prove Ptolemy's inequality (\ref{ptolemy-inq}).
\end{thm}

\section{Euler's formula}

Let $\alpha$ be a real number.
The following identity is called \index{Euler's formula}\emph{Euler's formula}:
$$e^{i\cdot\alpha}=\cos\alpha+i\cdot\sin\alpha.
\eqlbl{eq:euler}$$
In particular, $e^{i\cdot\pi}=-1$ and $e^{i\cdot\frac\pi2}=i$.

{

\begin{wrapfigure}{r}{36mm}
\vskip-15mm
\centering
\includegraphics{mppics/pic-272}
\end{wrapfigure}

Geometrically, Euler's formula means the following:
Assume that
$O$ and $E$ 
are the points with complex coordinates $0$ and $1$ respectively.
Assume 
\[OZ=1\quad \text{and}\quad \measuredangle EOZ\z\equiv \alpha,\]
then $e^{i\cdot\alpha}$ is the complex coordinate of $Z$.
In particular, the complex coordinate of any point on the unit circle centered at~$O$
can be uniquely expressed as $e^{i\cdot\alpha}$ for some $\alpha\in(-\pi,\pi]$.

}

\parbf{Why should you think that \ref{eq:euler} is true?}
The proof of Euler's identity depends on the way you define the exponential function.
If you never had to apply the exponential function to an imaginary number,
you may take the right-hand side in \ref{eq:euler} 
as the definition of $e^{i\cdot\alpha}$.

In this case, formally nothing has to be proved,
but it is better to check that $e^{i\cdot\alpha}$ satisfies familiar identities.
Mainly,
$$e^{i\cdot \alpha}\cdot e^{i\cdot \beta}= e^{i\cdot(\alpha+\beta)}.$$
The latter can be proved using \ref{eq:comlex+x} and the following trigonometric formulas,
which we assume to be known:
\begin{align*}
\cos(\alpha+\beta)&=\cos\alpha\cdot\cos\beta-\sin\alpha\cdot\sin\beta,
\\
\sin(\alpha+\beta)&=\sin\alpha\cdot\cos\beta+\cos\alpha\cdot\sin\beta.
\end{align*}

If you know the power series for the sine, cosine, and exponential function, then the following might convince that the identity \ref{eq:euler} holds:

\begin{align*}
 e^{i\cdot \alpha } &{}= 1 + i\cdot \alpha  + \frac{(i\cdot \alpha )^2}{2!} + \frac{(i\cdot \alpha  )^3}{3!} + \frac{(i\cdot \alpha )^4}{4!} + \frac{(i\cdot  \alpha )^5}{5!} +  \cdots =
 \\
&= 1 + i\cdot \alpha  - \frac{\alpha ^2}{2!} - i\cdot\frac{ \alpha ^3}{3!} + \frac{\alpha ^4}{4!} + i\cdot\frac{ \alpha ^5}{5!} -  \cdots =
\\
&= \left( 1 - \frac{\alpha ^2}{2!} + \frac{\alpha ^4}{4!}  - \cdots \right) +  i\cdot\left( \alpha  - \frac{\alpha ^3}{3!} + \frac{\alpha ^5}{5!} -  \cdots \right) =
\\
&= \cos \alpha  +  i\cdot\sin \alpha.
\end{align*}

\section{Argument and polar coordinates}

As before, we assume that $O$ and $E$ are the points with complex coordinates $0$ and $1$ respectively.

Let $Z$ be a point distinct from $O$.
Set $\rho=OZ$ and $\theta=\measuredangle EOZ$.
The pair $(\rho,\theta)$ is called the \index{polar!coordinates}\emph{polar coordinates} of~$Z$.

If $z$ is the complex coordinate of $Z$, then $\rho=|z|$. 
The value $\theta$ is called the \index{argument}\emph{argument} of $z$
(briefly, $\theta=\arg z$).
In this case, 
$$z=\rho\cdot e^{i\cdot\theta}=\rho\cdot(\cos\theta+i\cdot\sin\theta).$$

\begin{wrapfigure}[4]{o}{30mm}
\vskip-8mm
\centering
\includegraphics{mppics/pic-274}
\end{wrapfigure}

Note that 
\begin{align*}
\arg (z\cdot w)&\equiv \arg z+\arg w
\intertext{and}
\arg \tfrac z w&\equiv \arg z-\arg w
\end{align*}
if $z\ne0 $ and $w\ne0$.
In particular, if $Z$, $V$, $W$ are points with complex coordinates $z$, $v$, and $w$ respectively, then
$$
\begin{aligned}
\measuredangle VZW
&=\arg\left(\frac{w-z}{v-z}\right)\equiv
\\
&\equiv \arg(w-z)-\arg(v-z)
\end{aligned}
\eqlbl{eq:angle-arg}$$
if $\measuredangle VZW$ is defined.

\begin{thm}{Exercise}\label{ex:3-sum-C}
Use the formula \ref{eq:angle-arg} to show that  
$$\measuredangle ZVW+\measuredangle VWZ+\measuredangle WZV\equiv \pi$$
for any $\triangle ZVW$ in the Euclidean plane.
\end{thm}

\begin{thm}{Exercise}\label{ex:C-sim}
Suppose that points $O$, $E$, $V$, $W$, and $Z$ have complex coordinates $0$, $1$, $v$, $w$, and $z=v\cdot w$ respectively.
Show that 
\[\triangle OEV\sim \triangle OWZ.\]

\end{thm}

The following theorem is a translation of Corollary~\ref{cor:inscribed-quadrangle} to the complex-number language.

\begin{thm}{Theorem}\label{thm:inscribed-quadrangle-C}
Let $\square UVWZ$ be a quadrangle and $u$, $v$, $w$, and $z$ be the complex coordinates of its vertices. 
Then $\square UVWZ$ is inscribed 
if and only if the number
$$\frac{(v-u)\cdot(z-w)}{(v-w)\cdot(z-u)}$$ 
is real.
\end{thm}

The value $\frac{(v-u)\cdot(w-z)}{(v-w)\cdot(z-u)}$ is called the 
\index{cross-ratio!complex cross-ratio}\emph{complex cross-ratio} of $u$, $w$, $v$, and $z$; 
it will be denoted by \index{64@$(u,v;w,z)$}$(u,w;v,z)$.



\begin{thm}{Exercise}\label{ex:real-cross-ratio}
Observe that the complex number $z\ne 0$ is real if and only if $\arg z=0$ or $\pi$;
in other words, $2\cdot\arg z\equiv 0$.

Use this observation to show that Theorem~\ref{thm:inscribed-quadrangle-C}
is indeed a reformulation of Corollary~\ref{cor:inscribed-quadrangle}.
\end{thm}

\section{Method of complex coordinates}

The following problem illustrates the method of complex coordinates.

\begin{thm}{Problem}\label{prob:2right-tringles}
Let $\triangle OPV$ and $\triangle OQW$ be isosceles right triangles such that 
\[\measuredangle VPO=\measuredangle OQW=\tfrac\pi2\] 
and $M$ be the midpoint of $[VW]$.
Assume $P$, $Q$, and $M$ are distinct points.
Show that  $\triangle PMQ$ is an isosceles right triangle.
\end{thm}

\parit{Solution.}
Choose the complex coordinates so that $O$ is the origin;
denote by $v, w, p, q, m$ the complex coordinates of the remaining points respectively.

Since $\triangle OPV$ and $\triangle OQW$ are isosceles and $\measuredangle VPO=\measuredangle OQW=\tfrac\pi2$,
\ref{eq:C-dist} and \ref{eq:angle-arg} imply that
\begin{align*}
v-p&=i\cdot p,
&
q-w&=i\cdot q.
\end{align*}

\begin{wrapfigure}{r}{37mm}
\centering
\includegraphics{mppics/pic-276}
\end{wrapfigure}

Therefore
\begin{align*}
m&=\tfrac12\cdot(v+w)=
\\
&=\tfrac{1+i}2\cdot p+\tfrac{1-i}2\cdot q.
\end{align*}

By straightforward computations, we get that
\[p-m=i\cdot (q-m).\]
In particular, $|p-m|=|q-m|$ and  $\arg\frac{p-m}{q-m}=\tfrac\pi2$;
that is, $PM=QM$ and $\measuredangle QMP =\tfrac\pi2$.  
\qeds

{

\begin{wrapfigure}{r}{36mm}
\vskip-4mm
\centering
\includegraphics{mppics/pic-278}
\end{wrapfigure}

\begin{thm}{Exercise}\label{ex:3-squares}
Consider three squares with common sides as on the diagram.
Use the method of complex coordinates to show that 
\[\measuredangle EOA+\measuredangle EOB+\measuredangle EOC=\pm\tfrac\pi2.\]

\end{thm}

}

\begin{thm}{Exercise}\label{ex:6-circles}
Check the following identity with six complex cross-ratios:
\[(u,w;v,z)\cdot(u',w';v',z')=\frac{(v,w';v',w)\cdot(z,u';z',u)}{(u,v';u',v)\cdot(w,z';w',z)}.\]
Use it together with Theorem~\ref{thm:inscribed-quadrangle-C} to prove that if
$\square UVWZ$, $\square UVV'U'$, $\square VWW'V'$, $\square WZZ'W'$, and $\square ZUU'Z'$
are inscribed, then  $\square U'V'W'Z'$ is inscribed as well.

\end{thm}

\begin{minipage}{.47\textwidth}
\centering
\includegraphics{mppics/pic-280}
\end{minipage}
\hfill
\begin{minipage}{.47\textwidth}
\centering
\includegraphics{mppics/pic-282}
\end{minipage}

\medskip

\begin{thm}{Exercise}\label{ex:4-sim}
Suppose that points $U$, $V$ and $W$ lie on one side of line $(AB)$ and 
$\triangle UAB\sim \triangle BVA \sim \triangle ABW$.
Denote by $a$, $b$, $u$, $v$, and $w$ the complex coordinates of $A$, $B$, $U$, $V$, and $W$ respectively.
\begin{enumerate}[(a)]
 \item Show that $\tfrac{u-a}{b-a}=\tfrac{b-v}{a-v}=\tfrac{a-b}{w-b}=\tfrac{u-v}{w-v}$.
 \item Conclude that $\triangle UAB\sim \triangle BVA \sim \triangle ABW\sim \triangle UVW$.
\end{enumerate}
 
\end{thm}

\section{Fractional linear transformations}

\begin{thm}{Exercise}\label{ex:movie}
Watch the video ``Möbius transformations revealed'' by Douglas Arnold and Jonathan Rogness.
(It is available on \href{http://youtu.be/JX3VmDgiFnY}{YouTube}.)
\end{thm}


The complex plane $\mathbb{C}$ extended by one ideal number $\infty$ 
is called the \index{extended complex plane}\emph{extended complex plane}.
It is denoted by $\hat{\mathbb{C}}$, so $\hat{\mathbb{C}}=\mathbb{C}\cup\{\infty\}$

A \index{fractional linear transformation}\emph{fractional linear transformation} or \index{M\"obius transformation}\emph{M\"obius transformation} of  $\hat{\mathbb{C}}$ is a function of one complex variable $z$
that can be written as
$$f(z) = \frac{a\cdot z + b}{c\cdot z + d},$$
where the coefficients $a$, $b$, $c$, $d$ are complex numbers satisfying $a\cdot d \z- b\cdot c \not= 0$.
(If $a\cdot d - b\cdot c = 0$ the function defined above is a constant and is not considered to be a fractional linear transformation.) 

In case $c\not=0$, we assume that
$$f(-d/c) = \infty
\quad
\text{and}
\quad
f(\infty) = a/c;$$
and if $c=0$ we assume
$f(\infty) = \infty$.

\section{Elementary transformations}

The following three types of fractional linear transformations are called \index{elementary transformation}\emph{elementary}:
\begin{enumerate}
\item $z\mapsto z+w,$
\item $z\mapsto w\cdot z$ for $w\ne0,$
\item $z\mapsto \frac1z.$
\end{enumerate}
 
\parbf{The geometric interpretations.}
Suppose that $O$ denotes the point with the complex coordinate~$0$.

The first map $z\mapsto z+w,$ corresponds to the so-called 
\index{parallel!translation}\emph{parallel translation} 
of the Euclidean plane, its geometric meaning should be evident.

The second map is called the \index{rotational homothety}\emph{rotational homothety} with the center at~$O$.
That is, the point $O$ maps to itself
and any other point $Z$ maps to a point $Z'$ such that $OZ'=|w|\cdot OZ$ and $\measuredangle ZOZ'=\arg w$.

The third map can be described as a composition of the inversion across the unit circle centered at $O$ and the reflection across $\mathbb{R}$ 
(the composition can be taken in any order).
Indeed, $\arg z\equiv -\arg \tfrac1z$.
Therefore, 
$$\arg z=\arg (1/\bar z);$$
that is, if the points $Z$ and $Z'$ have complex coordinates $z$ and $1/\bar z$,
then $Z'\in[OZ)$.
Clearly, $OZ=|z|$ and $OZ'=|1/\bar z|=\tfrac{1}{|z|}$.
Therefore, $Z'$ is the inverse of $Z$ across the unit circle centered at~$O$.

Finally, $\tfrac1z\z=\overline{(1/\bar z)}$ is the complex coordinate of
the reflection of $Z'$ across $\mathbb{R}$.

\begin{thm}{Proposition}\label{prop:mob-comp}
The map $f\:\hat{\mathbb{C}}\to\hat{\mathbb{C}}$ is a fractional linear transformation if and only if it can be expressed as a composition of elementary transformations. 
\end{thm}

\parit{Proof; the ``only if'' part.}
Fix a fractional linear transformation
\begin{align*}
f(z) &= \frac{a\cdot z + b}{c\cdot z + d}.
\intertext{Assume $c\ne 0$. Then}
f(z) &= \frac ac-\frac{a\cdot d-b\cdot c}{c\cdot(c\cdot z + d)} =
\\
&= \frac ac-\frac{a\cdot d-b\cdot c}{c^2}\cdot \frac1{z + \frac dc}.
\end{align*}
That is, 
$$f(z)=f_4\circ f_3\circ f_2\circ f_1 (z),
\eqlbl{eq:moebius-compose}$$
where $f_1$, $f_2$, $f_3$, and $f_4$ are the following elementary transformations:
\begin{align*}
f_1(z)&= z+\tfrac dc,
&
f_2(z)&= \tfrac1z,
\\
f_3(z)&= - \tfrac{a\cdot d-b\cdot c}{c^2} \cdot z,
&
f_4(z)&= z+\tfrac ac.
\end{align*}

If $c=0$, then
\[f(z) = \frac{a\cdot z + b}{ d}.\]
In this case, $f(z)=f_2\circ f_1 (z)$,
where 
\begin{align*}
f_1(z)&= \tfrac ad\cdot z,
&
f_2(z)= z+\tfrac bd.
\end{align*}

\parit{``If'' part.}
We need to show that by composing elementary transformations,
we can only get fractional linear transformations.
Note that it is sufficient to check that the composition of a fractional linear transformation
$$f(z) = \frac{a\cdot z + b}{c\cdot z + d}.$$
with any elementary transformation $z\mapsto z+w$, $z\mapsto w\cdot z$, and $z\mapsto \tfrac1z$ is a fractional linear transformation.

The latter is done by means of direct calculations.
\begin{align*}
\frac{a\cdot (z+w) + b}{c\cdot (z+w) + d}
&=
\frac{a\cdot z + (b+a\cdot w)}{c\cdot z + (d+c\cdot w)},
\\
\frac{a\cdot (w\cdot z) + b}{c\cdot (w\cdot z) + d}
&=
\frac{(a\cdot w)\cdot z + b}{(c\cdot w)\cdot z + d},
\\
\frac{a\cdot \frac1z + b}{c\cdot \frac1z + d}
&=
\frac{b\cdot z + a}{d\cdot z + c}.
\end{align*}
\qedsf


\begin{thm}{Corollary}\label{cor:cline-Moeb}
The image of a circline under a fractional linear transformation 
is a circline.
\end{thm}

\parit{Proof.}
By Proposition~\ref{prop:mob-comp},
it is sufficient to check that each elementary transformation sends a circline to a circline.

For the first and second elementary transformation, the latter is evident.

As noted above,
the map $z\mapsto\tfrac1z$ is a composition of inversion and reflection.
By Theorem~\ref{thm:inverse}, the inversion sends a circline to a circline.
Hence the result.
\qeds

\begin{thm}{Exercise}\label{ex:inverse-Mob}
Show that the inverse of a fractional linear transformation is a fractional linear transformation.
\end{thm}


\begin{thm}{Exercise}\label{ex:3-point-Mob}
Given distinct values $z_0,z_1,z_\infty\in \hat{\mathbb{C}}$,
construct a fractional linear transformation $f$ such that 
$f(z_0)=0$,
$f(z_1)=1$,
and 
$f(z_\infty)\z=\infty$.
Show that such a transformation is unique.
\end{thm}

\begin{thm}{Exercise}\label{ex:inversion-Mob}
Show that any inversion is a composition of the complex conjugation and a fractional linear transformation.

Use Theorem~\ref{thm:inversions-inversive} to conclude that any inversive transformation is either fractional linear transformation or a complex conjugate to a fractional linear transformation.
\end{thm}

\section{Complex cross-ratio}

Let $u$, $v$, $w$, and $z$ be four distinct complex numbers.
Recall that 
the complex number
$$
\frac{(u-w)\cdot(v-z)}{(v-w)\cdot(u-z)}$$
is called the \index{cross-ratio!complex cross-ratio}\emph{complex cross-ratio} of $u$, $v$, $w$, and $z$; 
it is denoted by $(u,v;w,z)$.

If one of the numbers $u$, $v$, $w$, $z$ is $\infty$, 
then the complex cross-ratio has to be defined by taking the appropriate limit; in other words, we assume that $\frac\infty\infty=1$.
For example,
$$(u, v; w, \infty)=\frac{(u-w)}{(v-w)}.$$

Assume that $U$, $V$, $W$, and  $Z$ are the points with complex coordinates  
$u$, $v$, $w$, and $z$ respectively.
Note that 
\begin{align*}
\frac{UW\cdot VZ}{VW\cdot UZ}&=|(u,v;w,z)|,
\\
\measuredangle WUZ +\measuredangle ZVW&=\arg\frac{u-w}{u-z}+\arg\frac{v-z}{v-w}\equiv \arg(u,v;w,z).
\end{align*}
These equations make it possible to reformulate Theorem~\ref{lem:inverse-4-angle} using the complex coordinates
the following way:

\begin{thm}{Theorem}\label{lem:inverse-4-angle-C}
Let $UWVZ$ and $U'W'V'Z'$  be two quadrangles 
such that the points $U'$, $W'$, $V'$, and $Z'$ are inverses of $U$, $W$, $V$, and $Z$ respectively.
Assume $u$, $w$, $v$, $z$, $u'$, $w'$, $v'$, and $z'$ are the complex coordinates of $U$, $W$, $V$, $Z$, $U'$, $W'$, $V'$, and $Z'$ respectively.

Then 
$$(u',v';w',z')=\overline{(u,v;w,z)}.$$

\end{thm}

The following exercise is a generalization of the theorem above.
It has a short solution using Proposition~\ref{prop:mob-comp}.

\begin{thm}{Exercise}\label{ex:C-cross-ratio}
Show that 
complex cross-ratios are {}\emph{invariant} under fractional linear transformations. 

That is, if a fractional linear transformation maps four distinct complex numbers $u, v, w, z$ to complex numbers $u', v', w', z'$ respectively, then
$$
(u',v';w',z')
=
(u,v;w,z).
$$

\end{thm}

\section{Schwarz--Pick theorem}

The following theorem shows 
that the metric in the conformal disc model naturally appears in other branches of mathematics.
We do not give its proof, but it can be found in any textbook on geometric complex analysis.

Suppose that $\mathbb{D}$ denotes the unit disc in the complex plane centered at~$0$;
that is, a complex number $z$
belongs to $\mathbb{D}$ if and only if $|z|<1$.

Let us use the disc $\mathbb{D}$ as an h-plane in the conformal disc model;
the h-distance between $z, w\in\mathbb{D}$ will be denoted by $d_h(z,w)$;
that is,
\[d_h(z,w)\df ZW_h,\]
where $Z$ and $W$ are h-points with complex coordinates $z$ and $w$ respectively.

A function $f\:\mathbb{D}\to \mathbb{C}$ is called \index{holomorphic function}\emph{holomorphic} if for every $z\in \mathbb{D}$
there is a complex number $s$ such that
\[f(z+w)=f(z)+s\cdot w+o(|w|).\]
In other words, $f$ is {}\emph{complex-differentiable}
at any $z\in\mathbb{D}$.
The complex number $s$ is called the {}\emph{derivative} of $f$ at $z$, or briefly $s=f'(z)$.

\begin{thm}{Schwarz--Pick theorem}
Assume $f\: \mathbb{D}\to \mathbb{D}$ is a holomorphic function.
Then 
\[d_h(f(z),f(w))\le d_h(z,w)\]
for any $z,w\in \mathbb{D}$.

If the equality holds for one pair of distinct numbers $z,w\in \mathbb{D}$, then it holds for any pair. 
In this case, $f$ is a fractional linear transformation as well as a motion of the h-plane.
\end{thm}

\begin{thm}{Exercise}\label{ex:schwarz-moebius}
Show that if a fractional linear transformation $f$ appears in the equality case of Schwarz--Pick theorem, then it can be written as 
\[f(z)=\frac{v\cdot z+\bar w}{w\cdot z+\bar v}.\]
where $v$ and $w$ are complex constants such that $|v|>|w|$.
\end{thm}

Recall that hyperbolic tangent $\tanh$ is defined in Section~\ref{sec:hyp-trig}.

\begin{thm}{Exercise}\label{ex:schwarz-tanh}
Show that 
\[\tanh [\tfrac12\cdot d_h(z,w)]=\left|\frac{z-w}{1-z\cdot\bar w}\right|.\]
Conclude that the inequality in Schwarz--Pick theorem can be rewritten as
\[\left|\frac{z'-w'}{1-z'\cdot\bar w'}\right|\le\left|\frac{z-w}{1-z\cdot\bar w}\right|,\]
where
$z'=f(z)$ and $w'=f(w)$.
\end{thm}

\begin{thm}{Exercise}\label{ex:schwarz}
Show that the Schwarz lemma stated below 
follows from Schwarz--Pick theorem.
\end{thm}

\begin{thm}{Schwarz lemma}
Let $f\: \mathbb{D}\to \mathbb{D}$ be a holomorphic function
and $f(0)=0$.
Then 
$|f(z)|\le |z|$
for any $z\in \mathbb{D}$.

Moreover, if equality holds for some $z\ne 0$, then there is a unit complex number $u$ 
such that 
$f(z)=u\cdot z$
for any $z\in\mathbb{D}$.
\end{thm}

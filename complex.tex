\chapter{Complex coordinates}\label{chap:complex}
\addtocontents{toc}{\protect\begin{quote}}

In this chapter, we give an interpretation of inversive geometry using complex coordinates.
The results of this chapter will not be used in this book,
but they lead to deeper understanding of both concepts.

\section*{Complex numbers}
\addtocontents{toc}{Complex numbers.}

Informally,
a complex number is a number that can be put in the form 
$$z=x+i\cdot y,
\eqlbl{eq:z=x+iy}$$ 
where $x$ and $y$ 
are real numbers and $i^2=-1$. 

The set of complex numbers 
will be further denoted by~$\mathbb{C}$.
If $x$, $y$ and $z$ are as in \ref{eq:z=x+iy}, 
then $x$ is called the real part and $y$ the imaginary part of the complex number~$z$.
Briefly it is written as 
\[x=\Re z
\quad
\text{and}
\quad 
y=\Im z.\]

On the more formal level, a complex number is a pair of real numbers $(x,y)$ with the addition and multiplication described below.
The formula $x + i\cdot y$ 
is only a convenient way 
to write the pair $(x,y)$.
\begin{align*}
(x_1+i\cdot y_1) + (x_2+i\cdot y_2) 
&\df (x_1+x_2) + i\cdot(y_1+y_2);
\\
(x_1+i\cdot y_1)\cdot(x_2+i\cdot y_2) 
&\df 
(x_1\cdot x_2-y_1\cdot y_2) + i\cdot(x_1\cdot y_2+y_1\cdot x_2).
\end{align*}

\section*{Complex coordinates}
\addtocontents{toc}{Complex coordinates.}

Recall that one can think of the Euclidean plane
as the set of all pairs of real numbers $(x,y)$ equipped with the metric 
$$AB=\sqrt{(x_A-x_B)^2+(y_A-y_B)^2},$$
where $A=(x_A,y_A)$ and $B=(x_B,y_B)$.

One can pack the coordinates $(x,y)$ of a point in the Euclidean plane
in one complex number $z=x+i\cdot y$.
This way we get one-to-one correspondence between points of the Euclidean plane and~$\mathbb{C}$.
Given a point $Z=(x,y)$, 
the complex number $z=x+ i\cdot y$ is called the \index{complex coordinate}\emph{complex coordinate} of~$Z$.

Note that if $O$, $E$ and $I$ are points in the plane 
with complex coordinates $0$, $1$ and $i$, then $\measuredangle EOI=\pm\tfrac\pi2$.
Further, we assume that $\measuredangle EOI=\tfrac\pi2$;
if not, one has to change the direction of the $y$-coordinate. 


\section*{Conjugation and absolute value}
\addtocontents{toc}{Conjugation and absolute value.}

Let $z=x+i\cdot y$ and both $x$ and $y$ be real.
Denote by $Z$ the point in the plane with the complex coordinate~$z$.

If $y=0$, we say that the complex number $z$ is \index{real complex number}\emph{real} and if $x=0$ we say that $z$ is \index{imaginary complex number}\emph{imaginary}.
The set of points 
with real and imaginary complex coordinates form lines in the plane,
which are called \index{real line}\emph{real} and \index{imaginary line}\emph{imaginary} lines; 
they will be denoted as $\mathbb{R}$ and~$i\cdot\mathbb{R}$.

\medskip

The complex number $\bar z=x-iy$ is called the \index{complex conjugate}\emph{complex conjugate} of~$z$.

Note that the point $\bar Z$ with the complex coordinate $\bar z$ 
is the reflection of $Z$ in the real line.

It is straightforward to check that
$$\begin{aligned}
x&=\Re z=\frac{z+\bar z}2,
&
y&=\Im z=\frac{z-\bar z}{i\cdot2},
&
x^2+y^2&=z\cdot\bar z.
\end{aligned}\eqlbl{eq:conj-1}$$

The last formula in \ref{eq:conj-1} makes it possible to express the quotient $\tfrac{w}{z}$ of two complex numbers $w$ and $z=x+i\cdot y$:
$$\frac{w}{z}=\tfrac{1}{z\cdot\bar z}\cdot w\cdot\bar z=\tfrac{1}{x^2+y^2}\cdot w\cdot\bar z.$$

\label{page:cojugation=authomorphism}
Note that
\begin{align*}
\overline {z+ w}&=\bar z+\bar w,
&
\overline {z- w}&=\bar z-\bar w,
&
\overline {z\cdot w}&=\bar z\cdot\bar w,
&
\overline {z/w}&=\bar z/\bar w.
\end{align*}
That is, the complex conjugation
{}\emph{respects}
all the arithmetic operations.

The value 
\begin{align*}
|z|&=\sqrt{x^2+y^2}=
\\
&=\sqrt{z\cdot\bar z}
\end{align*}
is called the
\index{absolute value}\emph{absolute value} of $z$.
If $|z|=1$, then $z$ is called a \index{unit complex number}\emph{unit complex number}.

Note that 
if $Z$ and $W$ are points in the Euclidean plane, $z$ and $w$ are their complex coordinates, then
$$ZW=|z-w|.$$

\section*{Euler's formula}
\addtocontents{toc}{Euler's formula.}

Let $\alpha$ be a real number.
The following identity is called \index{Euler's formula}\emph{Euler's formula}.
$$e^{i\cdot\alpha}=\cos\alpha+i\cdot\sin\alpha.
\eqlbl{eq:euler}$$
In particular, $e^{i\cdot\pi}=-1$ and $e^{i\cdot\frac\pi2}=i$.

{

\begin{wrapfigure}[11]{o}{34mm}
\begin{lpic}[t(-4mm),b(0mm),r(0mm),l(0mm)]{pics/expix(1)}
\lbl[tr]{13.5,12;$0$}
\lbl[tl]{25,12;$1$}
\lbl[bl]{15.5,25;$i$}
\lbl[rb]{7,21;$e^{i\cdot\alpha}$}
\lbl[rb]{19,17;$\alpha$}
\end{lpic}
\end{wrapfigure}

Geometrically, Euler's formula means the following:
Assume that
$O$ and $E$ 
are the points with complex coordinates $0$ and $1$ correspondingly.
Assume $OZ=1$ and $\measuredangle EOZ\z\equiv \alpha$,
then $e^{i\cdot\alpha}$ is the complex coordinate of $Z$.
In particular, the complex coordinate of any point on the unit circle centered at~$O$
can be uniquely expressed as $e^{i\cdot\alpha}$ for some $\alpha\in(-\pi,\pi]$.

}

\parbf{Why should you think that \ref{eq:euler} is true?}
The proof of Euler's identity depends on the way you define the exponential function.
If you never had to take the exponential function of an imaginary number,
you may take the right hand side in \ref{eq:euler} 
as the definition of the $e^{i\cdot\alpha}$.

In this case, formally nothing has to be proved,
but it is better to check that $e^{i\cdot\alpha}$ satisfies familiar identities.
Mainly,
$$e^{i\cdot \alpha}\cdot e^{i\cdot \beta}= e^{i\cdot(\alpha+\beta)}.$$
The latter can be proved using the following trigonometric formulas,
which we assume to be known:
\begin{align*}
\cos(\alpha+\beta)&=\cos\alpha\cdot\cos\beta-\sin\alpha\cdot\sin\beta
\\
\sin(\alpha+\beta)&=\sin\alpha\cdot\cos\beta+\cos\alpha\cdot\sin\beta
\end{align*}

If you know the power series for the sine, cosine and exponential function, the following might convince that the identity \ref{eq:euler} holds:

\begin{align*}
 e^{i\cdot \alpha } &{}= 1 + i\cdot \alpha  + \frac{(i\cdot \alpha )^2}{2!} + \frac{(i\cdot \alpha  )^3}{3!} + \frac{(i\cdot \alpha )^4}{4!} + \frac{(i\cdot  \alpha )^5}{5!} +  \cdots =
 \\
&= 1 + i\cdot \alpha  - \frac{\alpha ^2}{2!} - i\cdot\frac{ \alpha ^3}{3!} + \frac{\alpha ^4}{4!} + i\cdot\frac{ \alpha ^5}{5!} -  \cdots =
\\
&= \left( 1 - \frac{\alpha ^2}{2!} + \frac{\alpha ^4}{4!}  - \cdots \right) +  i\cdot\left( \alpha  - \frac{\alpha ^3}{3!} + \frac{\alpha ^5}{5!} -  \cdots \right) =
\\
&= \cos \alpha  +  i\cdot\sin \alpha.
\end{align*}

\section*{Argument and polar coordinates}
\addtocontents{toc}{Argument and polar coordinates.}

As before, we assume that $O$ and $E$ are the points with complex coordinates $0$ and $1$ correspondingly.

Let $Z$ be the a point distinct form $O$.
Set $\rho=OZ$ and $\theta=\measuredangle EOZ$.
The pair $(\rho,\theta)$ is called the \index{polar coordinates}\emph{polar coordinates} of~$Z$.

\begin{wrapfigure}[10]{o}{38mm}
\begin{lpic}[t(0mm),b(4mm),r(0mm),l(0mm)]{pics/polar-coordinates(1)}
\lbl[tr]{6,7;$0$}
\lbl[tl]{29,7;$1$}
\lbl[br]{6,30;$i$}
\lbl[b]{23,35;$z$}
\lbl[l]{14,12.5,28;$\theta=\arg z$}
\lbl[b]{14,20,55;$\rho=|z|$}
\end{lpic}
\end{wrapfigure}

If $z$ is the complex coordinate of $Z$, then $\rho=|z|$. 
The value $\theta$ is called the argument of $z$
(briefly, $\theta=\arg z$).
In this case, 
$$z=\rho\cdot e^{i\cdot\theta}=\rho\cdot(\cos\theta+i\cdot\sin\theta).$$

Note that 
\begin{align*}
\arg (z\cdot w)&\equiv \arg z+\arg w
\intertext{and}
\arg \tfrac z w&\equiv \arg z-\arg w
\end{align*}
if $z,w\ne0$.
In particular, if $Z$, $V$, $W$ are points with complex coordinates $z$, $v$ and $w$ correspondingly, then
$$
\begin{aligned}
\measuredangle VZW
&=\arg\left(\frac{w-z}{v-z}\right)\equiv
\\
&\equiv \arg(w-z)-\arg(v-z)
\end{aligned}
\eqlbl{eq:angle-arg}$$
if $\measuredangle VZW$ is defined.

\begin{thm}{Exercise}\label{ex:3-sum-C}
Use the formula \ref{eq:angle-arg} to show that  
$$\measuredangle ZVW+\measuredangle VWZ+\measuredangle WZV\equiv \pi$$
for any $\triangle ZVW$ in the Euclidean plane.
\end{thm}

\begin{thm}{Exercise}\label{ex:C-sim}
Assume that points $V$, $W$ and $Z$ have complex coordinates $v$, $w$ and $z=v\cdot w$ correspondingly and the point $O$ and $E$ are as above.
Show that $$\triangle OEV\sim \triangle OWZ.$$

\end{thm}

\begin{thm}{Exercise}\label{ex:3-squares}
Let $ABXW$, $BCYX$ and $CDZY$ be three distinct squares as on the diagram.
Use the formula \ref{eq:angle-arg} to show that 
\[\measuredangle DAZ+\measuredangle DBZ+\measuredangle DCZ=\pm\tfrac\pi2.\]

\end{thm}

\begin{center}
\begin{lpic}[t(0mm),b(0mm),r(0mm),l(0mm)]{pics/3-squares(1)}
\lbl[t]{2.5,.5;$A$}
\lbl[t]{22.5,.5;$B$}
\lbl[t]{42.5,.5;$C$}
\lbl[t]{62.5,.5;$D$}
\lbl[b]{3,24;$W$}
\lbl[b]{23,24;$X$}
\lbl[b]{43,24;$Y$}
\lbl[b]{63,24;$Z$}
\end{lpic}
\end{center}

The following Theorem is a reformulation of Theorem~\ref{thm:inscribed-quadrilateral} which uses complex coordinates.


\begin{thm}{Theorem}\label{thm:inscribed-quadrilateral-C}
Let $\square UVWZ$ be a quadrilateral and $u$, $v$, $w$ and $z$ be the complex coordinates of its vertices. 
Then $\square UVWZ$ is inscribed 
if and only if the number
$$\frac{(v-u)\cdot(z-w)}{(v-w)\cdot(z-u)}$$ 
is real.
\end{thm}

The value $\frac{(v-u)\cdot(w-z)}{(v-w)\cdot(z-u)}$ is called the 
\index{cross-ratio!complex cross-ratio}\emph{complex cross-ratio}; 
it will be discussed in more details below.


\begin{thm}{Exercise}\label{ex:real-cross-ratio}
Observe that the complex number $z\ne 0$ is real if and only if $\arg z=0$ or $\pi$;
in other words, $2\cdot\arg z\equiv 0$.

Use this observation to show that Theorem~\ref{thm:inscribed-quadrilateral-C}
is indeed a reformulation of  Theorem~\ref{thm:inscribed-quadrilateral}.
\end{thm}



\section*{M\"obius transformations}
\addtocontents{toc}{M\"obius transformations.}

\begin{thm}{Exercise}\label{ex:movie}
Watch video ``M\"obius Transformations Revealed'' by Douglas Arnold and Jonathan Rogness.
(It is 3 minutes long and available on \href{http://youtu.be/0z1fIsUNhO4}{YouTube}.)
\end{thm}


The complex plane $\mathbb{C}$ extended by one ideal number $\infty$ 
is called the \index{extended complex plane}\emph{extended complex plane}.
It is denoted by $\hat{\mathbb{C}}$, so $\hat{\mathbb{C}}=\mathbb{C}\cup\{\infty\}$

A \index{M\"obius transformation}\emph{M\"obius transformation} of  $\hat{\mathbb{C}}$ is a function of one complex variable $z$
which can be written as
$$f(z) = \frac{a\cdot z + b}{c\cdot z + d},$$
where the coefficients $a$, $b$, $c$, $d$ are complex numbers satisfying $a\cdot d \z- b\cdot c \not= 0$.
(If $a\cdot d - b\cdot c = 0$ the function defined above is a constant and is not considered to be a M\"obius transformation.) 

In case $c\not=0$, we assume that
$$f(-d/c) = \infty
\quad
\text{and}
\quad
f(\infty) = a/c;$$
and if $c=0$ we assume
$$f(\infty) = \infty.$$





\section*{Elementary transformations}
\addtocontents{toc}{Elementary transformations.}

The following three types of M\"obius transformations are called \index{elementary transformation}\emph{elementary}.

\begin{enumerate}
\item $z\mapsto z+w,$
\item $z\mapsto w\cdot z$ for $w\ne0,$
\item $z\mapsto \frac1z.$
\end{enumerate}
 
\parbf{The geometric interpretations.}
As before we will denote by $O$ the point with the complex coordinate~$0$.

The first map $z\mapsto z+w,$ corresponds to the so called 
\index{parallel translation}\emph{parallel translation} 
of the Euclidean plane, its geometric meaning should be evident.

The second map is called the \index{rotational homothety}\emph{rotational homothety} with the center at~$O$.
That is, the point $O$ maps to itself
and any other point $Z$ maps to a point $Z'$ such that $OZ'=|w|\cdot OZ$ and $\measuredangle ZOZ'=\arg w$.

The third map can be described as a composition of the inversion in the unit circle centered at $O$ and the reflection in $\mathbb{R}$ 
(the composition can be taken in any order).
Indeed, $\arg z\equiv -\arg \tfrac1z$.
Therefore, 
$$\arg z=\arg (1/\bar z);$$
that is, if the points $Z$ and $Z'$ have complex coordinates $z$ and $1/\bar z$,
then $Z'\in[OZ)$.
Clearly, $OZ=|z|$ and $OZ'=|1/\bar z|=\tfrac{1}{|z|}$.
Therefore, $Z'$ is the inverse of $Z$ in the unit circle centered at~$O$.
Finally, the reflection of $Z'$ in $\mathbb{R}$, 
has complex coordinate $\tfrac1z=\overline{(1/\bar z)}$.

\begin{thm}{Proposition}\label{prop:mob-comp}
The map $f\:\hat{\mathbb{C}}\to\hat{\mathbb{C}}$ is a M\"obius transformation if and only if it can be expressed as a composition of elementary   M\"obius transformations. 
\end{thm}

\parit{Proof; the ``only if'' part.}
Fix a M\"obius transformation
\begin{align*}
f(z) &= \frac{a\cdot z + b}{c\cdot z + d}.
\intertext{Assume $c\ne 0$. Then}
f(z) &= \frac{a\cdot z + b}{c\cdot z + d}=
\\
&= \frac ac-\frac{a\cdot d-b\cdot c}{c\cdot(c\cdot z + d)} =
\\
&= \frac ac-\frac{a\cdot d-b\cdot c}{c^2}\cdot \frac1{z + \frac dc}.
\end{align*}
That is, 
$$f(z)=f_4\circ f_3\circ f_2\circ f_1 (z),
\eqlbl{eq:moebius-compose}$$
where $f_1$, $f_2$, $f_3$ and $f_4$ are elementary transformations of the following form:
\begin{itemize}
\item $f_1(z)= z+\tfrac dc$,
\item $f_2(z)= \tfrac1z$,
\item $f_3(z)= - \tfrac{a\cdot d-b\cdot c}{c^2} \cdot z$,
\item $f_4(z)= z+\tfrac ac$.
\end{itemize}

\medskip

If $c=0$, then
\[f(z) = \frac{a\cdot z + b}{ d}.\]
In this case
\[f(z)=f_2\circ f_1 (z),\]
where
\begin{itemize}
\item $f_1(z)= \tfrac ad\cdot z$,
\item $f_2(z)= z+\tfrac bd$.
\end{itemize}

\parit{``If'' part.}
We need to show that by composing elementary transformations,
we can only get M\"obius transformations.
Note that it is sufficient to check that the composition of a M\"obius transformations
$$f(z) = \frac{a\cdot z + b}{c\cdot z + d}.$$
with any elementary transformation $z\mapsto z+w$, $z\mapsto w\cdot z$ and $z\mapsto \tfrac1z$ is a M\"obius transformations.

The latter is done by means of direct calculations.
\begin{align*}
\frac{a\cdot (z+w) + b}{c\cdot (z+w) + d}
&=
\frac{a\cdot z + (b+a\cdot w)}{c\cdot z + (d+c\cdot w)},
\\
\frac{a\cdot (w\cdot z) + b}{c\cdot (w\cdot z) + d}
&=
\frac{(a\cdot w)\cdot z + b}{(c\cdot w)\cdot z + d},
\\
\frac{a\cdot \frac1z + b}{c\cdot \frac1z + d}
&=
\frac{b\cdot z + a}{d\cdot z + c}.
\end{align*}
\qedsf


\begin{thm}{Corollary}\label{cor:cline-Moeb}
The image of a circline under a M\"obius transformation 
is a circline.
\end{thm}

\parit{Proof.}
By Proposition~\ref{prop:mob-comp},
it is sufficient to check that each elementary transformation sends a circline to a circline.

For the first and second elementary transformation, the latter is evident.

As it was noted above,
the map $z\mapsto\tfrac1z$ is a composition of inversion and reflection.
By Theorem~\ref{thm:inverse}, the inversion sends a circline to a circline.
Hence the result follows.
\qeds

\begin{thm}{Exercise}\label{ex:inverse-Mob}
Show that the inverse of a M\"obius transformation is a M\"obius transformation.
\end{thm}


\begin{thm}{Exercise}\label{ex:3-point-Mob}
Given distinct values $z_0,z_1,z_\infty\in \hat{\mathbb{C}}$,
construct a M\"obius transformation $f$ such that $f(z_0)=0$, $f(z_1)=1$ and $f(z_\infty)\z=\infty$.
Show that such a transformation is unique.
\end{thm}

\begin{thm}{Exercise}\label{ex:invesion-Mob}
Show that any inversion is a composition of the complex conjugation and a M\"obius transformation.  
\end{thm}



\section*{Complex cross-ratio}
\addtocontents{toc}{Complex cross-ratio.}

Given four distinct complex numbers $u$, $v$, $w$ and $z$,
the complex number
$$
\frac{(u-w)\cdot(v-z)}{(v-w)\cdot(u-z)}$$
is called the \index{cross-ratio!complex cross-ratio}\emph{complex cross-ratio}; 
it will be denoted by \index{64@$(u,v;w,z)$}$(u,v;w,z)$.

If one of the numbers $u$, $v$, $w$, $z$ is $\infty$, 
then the complex cross-ratio has to be defined by taking the appropriate limit; in other words, we assume that $\frac\infty\infty=1$.
For example,
$$(u, v; w, \infty)=\frac{(u-w)}{(v-w)}.$$

Assume that $U$, $V$, $W$ and  $Z$ are the points with complex coordinates  
$u$, $v$, $w$ and $z$ correspondingly.
Note that 
\begin{align*}
\frac{UW\cdot VZ}{VW\cdot UZ}&=|(u,v;w,z)|,
\\
\measuredangle WUZ +\measuredangle ZVW&=\arg\frac{u-w}{u-z}+\arg\frac{v-z}{v-w}\equiv 
\\
&\equiv \arg(u,v;w,z).
\end{align*}

It makes it possible to reformulate Theorem~\ref{lem:inverse-4-angle} using the complex coordinates
the following way.

\begin{thm}{Theorem}\label{lem:inverse-4-angle-C}
Let $UWVZ$ and $U'W'V'Z'$  be two quadrilaterals 
such that the points $U'$, $W'$, $V'$ and $Z'$ are inversions of $U$, $W$, $V$, and $Z$ correspondingly.
Assume $u$, $w$, $v$, $z$, $u'$, $w'$, $v'$ and $z'$ are the complex coordinates of $U$, $W$, $V$, $Z$, $U'$, $W'$, $V'$ and $Z'$ correspondingly.

Then 
$$(u',v';w',z')=\overline{(u,v;w,z)}.$$

\end{thm}

The following exercise is a generalization of the Theorem above.
It admits a short and simple solution which uses Proposition~\ref{prop:mob-comp}.

\begin{thm}{Exercise}\label{ex:C-cross-ratio}
Show that 
complex cross-ratios are \emph{invariant} under M\"obius transformations. 

That is, if a M\"obius transformation maps four distinct complex numbers $u, v, w, z$ to complex numbers $u', v', w', z'$ respectively, then
$$
(u',v';w',z')
=
(u,v;w,z).
$$

\end{thm}

\section*{Schwarz--Pick theorem}
\addtocontents{toc}{Schwarz--Pick theorem.}
The following theorem shows 
that the metric in the conformal disc model naturally appears in other branches of mathematics.
We do not give a proof, but it can be found in any textbook on geometric complex analysis.

Let us denote by $\mathbb{D}$
the unit disc in the complex plane centered at $0$;
that is, a complex number $z$
belongs to $\mathbb{D}$ if and only if $|z|<1$.

Let us use the disc $\mathbb{D}$ as a h-plane in the conformal disc model;
the h-distance between $z, w\in\mathbb{D}$ will be denoted by $d_h(z,w)$.

A function $f\:\mathbb{D}\to \mathbb{C}$ is called \index{holomorphic function}\emph{holomorphic} if for every $z\in \mathbb{D}$
there is a complex number $s$ such that
\[f(z+w)=f(z)+s\cdot w+o(|w|).\]
In other words, $f$ is {}\emph{complex-differentiable}
at any $z\in\mathbb{D}$.
The number $s$ above is called the derivative of $f$ at $z$ and is denoted by~$f'(z)$.

\begin{thm}{Schwarz--Pick theorem}
Assume $f\: \mathbb{D}\to \mathbb{D}$ is a holomorphic function.
Then 
\[d_h(f(z),f(w))\le d_h(z,w)\]
for any $z,w\in \mathbb{D}$.

If the equality holds for one pair of distinct numbers $z,w\in \mathbb{D}$, then it holds for any pair. 
Moreover, $f\: \mathbb{D}\to \mathbb{D}$ is a motion of the h-plane.
\end{thm}

\begin{thm}{Exercise}\label{ex:schwarz}
Show that the Schwarz lemma stated below 
follows from Schwarz--Pick theorem.
\end{thm}

\begin{thm}{Schwarz lemma}
Let $f\: \mathbb{D}\to \mathbb{D}$ be a holomorphic function
and $f(0)=0$.
Then 
$|f(z)|\le |z|$
for any $z\in \mathbb{D}$.

Moreover, if equality holds for some $z\ne 0$, then there is a unit complex number $u$ 
such that 
$f(z)=u\cdot z$
for any $z\in\mathbb{D}$.
\end{thm}






\addtocontents{toc}{\protect\end{quote}}
%\part*{Euclidean geometry}
\addtocontents{toc}{\protect\begin{center}}
\addtocontents{toc}{\large{\bf Euclidean geometry}}
\addtocontents{toc}{\protect\end{center}}
\chapter{The Axioms}
\label{chap:axioms}
\addtocontents{toc}{\protect\begin{quote}}


\section*{The Axioms}
\addtocontents{toc}{The Axioms.}\label{def:birkhoff-axioms} 

Summarizing the above discussion, let us give an axiomatic system of the Euclidean plane.


\bigskip 

\begin{enumerate}[I.]
\item\label{def:birkhoff-axioms:0} The \index{plane!Euclidean plane}\index{Euclidean plane}\emph{Euclidean plane} is a metric space with at least two points.


\item\label{def:birkhoff-axioms:1} There is one and only one line, 
that contains any two given distinct points $P$ and $Q$ in the Euclidean plane.

\item\label{def:birkhoff-axioms:2} 
Any angle $\angle AOB$ in the Euclidean plane 
defines a real number in the interval $(-\pi,\pi]$.
This number is called \index{angle measure}\emph{angle measure of $\angle AOB$}
and denoted by $\measuredangle A O B$.
It satisfies the following conditions:
\begin{enumerate}
\item\label{def:birkhoff-axioms:2a} Given a half-line $[O A)$ and $\alpha\in(-\pi,\pi]$ there is unique  half-line $[O B)$ such that $\measuredangle A O B= \alpha$
\item\label{def:birkhoff-axioms:2b} For any points $A$, $B$ and $C$ distinct from $O$ we have
$$\measuredangle A O B+\measuredangle B O C
\equiv\measuredangle A O C.$$
\item\label{def:birkhoff-axioms:2c} 
The function 
$$\measuredangle\:(A,O,B)\mapsto\measuredangle A O B$$
is continuous at any triple of points $(A,O,B)$
such that $O\ne A$ and $O\ne B$ and $\measuredangle A O B\ne\pi$.

\end{enumerate}

\item\label{def:birkhoff-axioms:3}  In the Euclidean plane, we have
$\triangle A B C\cong\triangle A' B' C'$
if and only if 
\begin{align*}
A' B'&=A B, & A' C'&= A C, &&\text{and}
&\measuredangle C' A' B'&=\pm\measuredangle C A B.
\end{align*}
\item\label{def:birkhoff-axioms:4}
If for two triangles 
$\triangle ABC$, 
$\triangle AB'C'$ in the Euclidean plane
and $k>0$ we have
\begin{align*}
B'&\in [AB),
& C'&\in [AC)
\\
AB'&=k\cdot AB,&
AC'&=k\cdot AC
\end{align*}
then
\begin{align*}
B'C'&=k\cdot BC,&
\measuredangle ABC&=\measuredangle AB'C'
&&\text{and}
&
\measuredangle ACB&=\measuredangle AC'B'.
\end{align*}
\end{enumerate}

\bigskip 


This set of axioms is very close to the one given by Birkhoff in \cite{birkhoff}.

From now on,  
we can use no information about Euclidean plane which does not follow from the five axioms above.

\begin{thm}{Exercise}\label{ex:infinite}
Show that the plane contains an infinite set of points.
\end{thm}


\section*{Lines and half-lines}
\addtocontents{toc}{Lines and half-lines.}

\begin{thm}{Proposition}\label{lem:line-line}
Any two distinct lines intersect at most at one point.
\end{thm}

\parit{Proof.}
Assume two lines $\ell$ and $m$ intersect at two distinct points $P$ and $Q$.
Applying Axiom~\ref{def:birkhoff-axioms:1}, we get $\ell=m$.
\qeds

\begin{thm}{Exercise}\label{ex:[OA)=[OA')}
Suppose $A'\in[OA)$ and $A'\not=O$ show that 
\[[O A)\z=[O A').\]

\end{thm}

\begin{thm}{Proposition}\label{prop:point-on-half-line}
Given $r\ge 0$ and a half-line $[O A)$ there is unique $A'\in [O A)$  such that $O A'=r$.
\end{thm}

\parit{Proof.}
According to definition of half-line, 
there is an isometry 
$$f\:[O A)\to [0,\infty),$$
such that $f(O)=0$.
By the definition of isometry, $O A'=f(A')$ for any $A'\z\in [O A)$.
Thus, $O A'=r$ if and only if $f(A')=r$.

Since isometry has to be bijective, the statement follows.
\qeds

\section*{Zero angle}
\addtocontents{toc}{Zero angle.}

\begin{thm}{Proposition}\label{lem:AOA=0}
$\measuredangle A O A= 0$ for any $A\not=O$.
\end{thm}

\parit{Proof.}
According to Axiom~\ref{def:birkhoff-axioms:2b},
$$\measuredangle A O A
+
\measuredangle A O A 
\equiv
\measuredangle A O A.$$
Subtract  $\measuredangle A O A$ from both sides, we get 
$\measuredangle A O A \equiv 0$.

Since $-\pi<\measuredangle A O A\le \pi$, 
we get $\measuredangle A O A \z= 0$.
\qeds

\begin{thm}{Exercise}\label{ex:2.4} 
Assume $\measuredangle A O B= 0$.
Show that $[OA)=[OB)$.
\end{thm}


\begin{thm}{Proposition}\label{lem:AOB+BOA=0}
For any $A$ and $B$ distinct from $O$,
we have 
$$\measuredangle A O B\equiv-\measuredangle B O A.$$

\end{thm}

\parit{Proof.}
According to Axiom~\ref{def:birkhoff-axioms:2b},
$$\measuredangle A O B+\measuredangle B O A \equiv\measuredangle A O A$$
By Proposition~\ref{lem:AOA=0}, $\measuredangle A O A=0$.
Hence the result follows.
\qeds

\section*{Straight angle}
\addtocontents{toc}{Straight angle.}

If $\measuredangle A O B=\pi$,
we say that $\angle A O B$ is a 
\index{angle!straight angle}\emph{straight angle}.
Note that by Proposition~\ref{lem:AOB+BOA=0}, 
if $\angle A O B$ is a straight angle 
then so is $\angle B O A$.

We say that point $O$ \index{between}\emph{lies between} points $A$ and $B$ if $O\not= A$, $O\not= B$ and $O\in[A B]$.

\begin{thm}{Theorem}\label{thm:straight-angle}
The angle $\angle A O B$ is straight 
if and only if $O$ 
\index{between}\emph{lies between} $A$ and $B$.
\end{thm}

\begin{wrapfigure}[3]{o}{40mm}
\begin{lpic}[t(-4mm),b(0mm),r(0mm),l(0mm)]{pics/AOB(1)}
\lbl[t]{6,0;$B$}
\lbl[t]{19,0;$O$}
\lbl[t]{31,0;$A$}
\end{lpic}
\end{wrapfigure}

\parit{Proof.}
By Proposition~\ref{prop:point-on-half-line},  we may assume that
$O A = O B = 1$.

\parit{}($\Leftarrow$).
Assume $O$  
lies between $A$ and $B$.
Set  $\alpha=\measuredangle A O B$.

Applying Axiom~\ref{def:birkhoff-axioms:2a},
we get a half-line $[OA')$ such that $\alpha\z=\measuredangle B O A'$.
By Proposition~\ref{prop:point-on-half-line}, we can assume that $OA'=1$.
According to Axiom~\ref{def:birkhoff-axioms:3},
\[\triangle AOB\z\cong\triangle BOA'.\]
Denote by $h$ the corresponding motion of the plane;
that is, $h$ is a motion such that $h(A)=B$, $h(O)=O$ and $h(B)=A'$. 

Then $(A'B)=h(AB)\ni h(O)=O$.
Therefore both lines $(AB)$ and $(A'B)$, 
 contain $B$ and $O$.
By Axiom~\ref{def:birkhoff-axioms:1}, $(AB)=(A'B)$.

By the definition of the line,
$(AB)$ contains exactly two points $A$ and $B$ on distance $1$ from $O$.
Since $OA'=1$ and $A'\ne B$, we get $A=A'$.

By Axiom \ref{def:birkhoff-axioms:2b} and Proposition~\ref{lem:AOA=0}, we get
\begin{align*}
2\cdot\alpha&=
\measuredangle AOB+\measuredangle BOA'=
\\
&=\measuredangle AOB+\measuredangle BOA\equiv
\\
&\equiv\measuredangle AOA=
\\
&= 0
\end{align*}
Therefore, by Exercise~\ref{ex:2a=0}, $\alpha$ is either $0$ or $\pi$.

Since $[OA)\ne [OB)$,  
we have $\alpha\ne 0$, see Exercise~\ref{ex:2.4}.
Therefore $\alpha=\pi$.


\parit{}($\Rightarrow$).
Suppose that $\measuredangle A O B= \pi$.
Consider line $(OA)$ and choose point $B'$ on $(OA)$ so that $O$ lies between $A$ and $B'$.

From above, we have $\measuredangle AOB'=\pi$.
Applying Axiom~\ref{def:birkhoff-axioms:2a}, 
we get $[O B)\z=[O B')$.
In particular, $O$ lies between $A$ and $B$.
\qeds 

A triangle $\triangle ABC$ is called 
\index{triangle!degenerate triangle}\index{degenerate! triangle}\emph{degenerate}
if $A$, $B$ and $C$ lie on one line.
The following corollary is just a reformulation of Theorem~\ref{thm:straight-angle}.

\begin{thm}{Corollary}\label{cor:degenerate=pi}
A triangle is degenerate if and only if one of its angles is equal to $\pi$ or $0$.
\end{thm}

\begin{thm}{Exercise}\label{ex:lineAOB}
Show that three distinct points $A$, $O$ and $B$ lie on one line if and only if 
$$2\cdot \measuredangle AOB\equiv 0.$$ 

\end{thm}

\begin{thm}{Exercise}\label{ex:ABCO-line}
Let $A$, $B$ and $C$ be three points distinct from $O$.
Show that $B$, $O$ and $C$ lie on one line if and only if
$$2\cdot \measuredangle AOB\equiv 2\cdot \measuredangle AOC.$$ 

\end{thm}



\section*{Vertical angles}
\addtocontents{toc}{Vertical angles.}

A pair of angles $\angle AOB$ and $\angle A'OB'$ 
is called \index{angle!vertical angles}\index{vertical angles}\emph{vertical}
if the point $O$ 
lies between $A$ and $A'$ 
and between $B$ and $B'$ at the same time.


\begin{thm}{Proposition}\label{prop:vert}
The vertical angles have equal measures.
\end{thm}

\parit{Proof.}
Assume that the angles $\angle AOB$ and $\angle A'OB'$ are vertical.

Note that the angles $\angle AOA'$ and $\angle BOB'$ are straight.
Therefore $\measuredangle AOA'=\measuredangle BOB'=\pi$.

{

\begin{wrapfigure}{o}{28mm}
\begin{lpic}[t(-2mm),b(0mm),r(0mm),l(0mm)]{pics/vertical(1)}
\lbl[b]{0,4;$A$}
\lbl[br]{22,22;$A'$}
\lbl[t]{13.5,11.5;$O$}
\lbl[tr]{4,20;$B$}
\lbl[bl]{24,6;$B'$}
\end{lpic}
\end{wrapfigure}

It follows that
\begin{align*}
0&=\measuredangle AOA'-\measuredangle BOB'\equiv
\\
&\equiv 
\measuredangle AOB+\measuredangle BOA'-\measuredangle BOA'-\measuredangle A'OB'
\equiv
\\
&\equiv\measuredangle AOB-\measuredangle A'OB'.
\end{align*}
Hence the result follows.
\qeds

}

\begin{thm}{Exercise}\label{ex:O-mid-AB+CD}
Assume $O$ 
is the midpoint for both segments 
$[A B]$ and $[C D]$.
Prove that $A C= B D$. 
\end{thm}




\addtocontents{toc}{\protect\end{quote}}
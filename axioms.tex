%\part*{Euclidean geometry}
\addtocontents{toc}{\protect\begin{center}}
\addtocontents{toc}{\large{\bf Euclidean geometry}}
\addtocontents{toc}{\protect\end{center}}
\chapter{Axioms}
\label{chap:axioms}
\addtocontents{toc}{\protect\begin{quote}}

%(???

\vfill

A system of axioms appears already in Euclid's ``Elements'' --- the most successful and influential textbook ever written.

The systematic study of geometries as axiomatic systems
 was
triggered by the discovery of non-Euclidean geometry.
The branch of mathematics, emerging this way, is called ``Foundations of geometry''.

The most popular system of axiom
was proposed in 1899 by David Hilbert.
This is also the first rigorous system by modern standards.
It contains twenty axioms in five groups, six ``primitive notions'' and three ``primitive terms'';
these are not defined in terms of previously defined concepts.

Later a number of different systems were proposed.
It is worth mentioning
the system of Alexandr Alexandrov \cite{alexandrov} which is very intuitive and elementary, 
the system of Friedrich Bachmann \cite{bachmann} which is based on the concept of symmetry,
and the system of Alfred Tarski \cite{tarski}, which was designed for analysis using mathematical logic.

We will use another system,
which is very close to the one proposed by George Birkhoff in \cite{birkhoff}.
This system is based on the {}\emph{key observations}  (\ref{preaxiomI})--(\ref{preaxiomV}) listed on page~\pageref{preaxiomI}.
The axioms use the notions of 
metric space, 
lines, 
angles,
triangles,
equalities modulo $2\cdot\pi$ ($\equiv$), 
the continuity of maps between metric spaces,
and the congruence of triangles ($\cong$).
All this discussed in the preliminaries.

Our system is build upon metric spaces.
In particular, we use the real numbers as a building block. 
By that reason our approach is not purely axiomatic --- we build the theory upon something else;
it resembles a model-based introduction to Euclidean geometry discussed on page~\pageref{page:model}.
We used this approach to minimize the tedious parts which are unavoidable in purely axiomatic foundations.

%???)

\newpage

\section*{The axioms}
\addtocontents{toc}{The axioms.}

\begin{framed}
\begin{enumerate}[I.]
\item\label{def:birkhoff-axioms:0} The \index{plane!Euclidean plane}\index{Euclidean plane}\emph{Euclidean plane} is a metric space with at least two points.


\item\label{def:birkhoff-axioms:1} 
There is one and only one line, that contains any two given distinct points $P$ and $Q$ in the Euclidean plane.

\item\label{def:birkhoff-axioms:2} 
Any angle $AOB$ in the Euclidean plane 
defines a real number in the interval $(-\pi,\pi]$.
This number is called \index{angle measure}\emph{angle measure of $\angle AOB$}
and denoted by $\measuredangle A O B$.
It satisfies the following conditions:
\begin{enumerate}[(a)]
\item\label{def:birkhoff-axioms:2a} 
Given a half-line $[O A)$ and $\alpha\in(-\pi,\pi]$, 
there is a unique  half-line $[O B)$, 
such that $\measuredangle A O B= \alpha$.
\item\label{def:birkhoff-axioms:2b} 
For any points $A$, $B$ and $C$, distinct from $O$ we have
$$\measuredangle A O B+\measuredangle B O C
\equiv\measuredangle A O C.$$
\item\label{def:birkhoff-axioms:2c} 
The function 
$$\measuredangle\:(A,O,B)\mapsto\measuredangle A O B$$
is continuous at any triple of points $(A,O,B)$,
such that $O\ne A$ and $O\ne B$ and $\measuredangle A O B\ne\pi$.

\end{enumerate}

\item\label{def:birkhoff-axioms:3}  
In the Euclidean plane, we have
$\triangle A B C\cong\triangle A' B' C'$
if and only if 
\begin{align*}
A' B'&=A B, & A' C'&= A C, &&\text{and}
&\measuredangle C' A' B'&=\pm\measuredangle C A B.
\end{align*}
\item\label{def:birkhoff-axioms:4}
If for two triangles $ABC$, $AB'C'$ in the Euclidean plane
and for $k>0$ we have
\begin{align*}
B'&\in [AB),
& C'&\in [AC),
\\
AB'&=k\cdot AB,&
AC'&=k\cdot AC,
\end{align*}
then
\begin{align*}
B'C'&=k\cdot BC,&
\measuredangle ABC&=\measuredangle AB'C',
&
\measuredangle ACB&=\measuredangle AC'B'.
\end{align*}
\end{enumerate}
\end{framed}

From now on,  
we can use no information about the Euclidean plane which does not follow from the five axioms above.

\begin{thm}{Exercise}\label{ex:infinite}
Show that there are (a) an infinite set of points,
(b) an infinite set of lines on the plane.
\end{thm}

\section*{Lines and half-lines}
\addtocontents{toc}{Lines and half-lines.}

\begin{thm}[\abs]{Proposition}\label{lem:line-line}
\let\thefootnote\relax\footnotetext{${}^\a$ A statement marked with ``$\a$'' if Axiom~\ref{def:birkhoff-axioms:4} was not used in its proof.
Ignore this mark for a while; it will be important in Chapter~\ref{chap:non-euclid}, see page \pageref{a-mark}.}
Any two distinct lines intersect at most at one point.
\end{thm}

\parit{Proof.}
Assume that two lines $\ell$ and $m$ intersect at two distinct points $P$ and~$Q$.
Applying Axiom~\ref{def:birkhoff-axioms:1}, we get that $\ell=m$.
\qeds

\begin{thm}{Exercise}\label{ex:[OA)=[OA')}
Suppose $A'\in[OA)$ and $A'\not=O$. 
Show that 
\[[O A)\z=[O A').\]

\end{thm}

\begin{thm}[\abs]{Proposition}\label{prop:point-on-half-line}
Given $r\ge 0$ and a half-line $[O A)$ there is a unique $A'\in [O A)$  such that $O A'=r$.
\end{thm}

\parit{Proof.}
According to definition of half-line, 
there is an isometry 
$$f\:[O A)\to [0,\infty),$$
such that $f(O)=0$.
By the definition of isometry, $O A'=f(A')$ for any $A'\z\in [O A)$.
Thus, $O A'=r$ if and only if $f(A')=r$.

Since isometry has to be bijective, the statement follows.
\qeds

\section*{Zero angle}
\addtocontents{toc}{Zero angle.}

\begin{thm}[\abs]{Proposition}\label{lem:AOA=0}
$\measuredangle A O A= 0$ for any $A\not=O$.
\end{thm}

\parit{Proof.}
According to Axiom~\ref{def:birkhoff-axioms:2b},
$$\measuredangle A O A
+
\measuredangle A O A 
\equiv
\measuredangle A O A.$$
Subtract  $\measuredangle A O A$ from both sides, we get that
$\measuredangle A O A \equiv 0$.

By Axiom~\ref{def:birkhoff-axioms:2}, $-\pi<\measuredangle A O A\le \pi$;
therefore $\measuredangle A O A \z= 0$.
\qeds

\begin{thm}{Exercise}\label{ex:2.4} 
Assume $\measuredangle A O B= 0$.
Show that $[OA)=[OB)$.
\end{thm}

\begin{thm}[\abs]{Proposition}\label{lem:AOB+BOA=0}
For any $A$ and $B$ distinct from $O$,
we have 
$$\measuredangle A O B\equiv-\measuredangle B O A.$$

\end{thm}

\parit{Proof.}
According to Axiom~\ref{def:birkhoff-axioms:2b},
$$\measuredangle A O B+\measuredangle B O A \equiv\measuredangle A O A$$
By Proposition~\ref{lem:AOA=0}, $\measuredangle A O A=0$.
Hence the result.
\qeds

\section*{Straight angle}
\addtocontents{toc}{Straight angle.}

If $\measuredangle A O B=\pi$,
we say that $\angle A O B$ is a 
\index{angle!straight angle}\emph{straight angle}.
Note that by Proposition~\ref{lem:AOB+BOA=0}, 
if $\angle A O B$ is a straight,
then so is $\angle B O A$.

We say that point $O$ \index{between}\emph{lies between} points $A$ and $B$, 
if $O\not= A$, $O\not= B$ and $O\in[A B]$.

\begin{thm}[\abs]{Theorem}\label{thm:straight-angle}
The angle $A O B$ is straight 
if and only if $O$ 
\index{between}\emph{lies between} $A$ and~$B$.
\end{thm}

\begin{wrapfigure}[2]{r}{40mm}
\begin{lpic}[t(-7mm),b(0mm),r(0mm),l(0mm)]{pics/AOB(1)}
\lbl[t]{6,0;$B$}
\lbl[t]{19,0;$O$}
\lbl[t]{31,0;$A$}
\end{lpic}
\end{wrapfigure}

\parit{Proof.}
By Proposition~\ref{prop:point-on-half-line},  we may assume that
$O A = O B = 1$.

\parit{``If'' part.}
Assume $O$  
lies between $A$ and~$B$.
Set  $\alpha=\measuredangle A O B$.

Applying Axiom~\ref{def:birkhoff-axioms:2a},
we get a half-line $[OA')$ such that $\alpha\z=\measuredangle B O A'$.
By Proposition~\ref{prop:point-on-half-line}, we can assume that $OA'=1$.
According to Axiom~\ref{def:birkhoff-axioms:3},
\[\triangle AOB\z\cong\triangle BOA'.\]
Let $f$ denotes the corresponding motion of the plane;
that is, $f$ is a motion such that $f(A)=B$, $f(O)=O$ and $f(B)=A'$. 

Then 
\[(A'B)=f(AB)\ni f(O)=O.\]
Therefore, both lines $(AB)$ and $(A'B)$ contain $B$ and~$O$.
By Axiom~\ref{def:birkhoff-axioms:1}, $(AB)=(A'B)$.

By the definition of the line,
$(AB)$ contains exactly two points $A$ and $B$ on distance $1$ from~$O$.
Since $OA'=1$ and $A'\ne B$, we get that $A=A'$.

By Axiom \ref{def:birkhoff-axioms:2b} and Proposition~\ref{lem:AOA=0}, we get that
\begin{align*}
2\cdot\alpha&=
\measuredangle AOB+\measuredangle BOA'=
\\
&=\measuredangle AOB+\measuredangle BOA\equiv
\\
&\equiv\measuredangle AOA=
\\
&= 0
\end{align*}
Therefore, by Exercise~\ref{ex:2a=0}, $\alpha$ is either $0$ or~$\pi$.

Since $[OA)\ne [OB)$,  
we have that $\alpha\ne 0$, see Exercise~\ref{ex:2.4}.
Therefore, $\alpha=\pi$.


\parit{``Only if'' part.}
Suppose that $\measuredangle A O B= \pi$.
Consider the line $(OA)$ and choose a point $B'$ on $(OA)$ so that $O$ lies between $A$ and~$B'$.

From above, we have that $\measuredangle AOB'=\pi$.
Applying Axiom~\ref{def:birkhoff-axioms:2a}, 
we get that $[O B)\z=[O B')$.
In particular, $O$ lies between $A$ and~$B$.
\qeds 

A triangle $ABC$ is called 
\index{triangle!degenerate triangle}\index{degenerate! triangle}\emph{degenerate}
if $A$, $B$ and $C$ lie on one line.
The following corollary is just a reformulation of Theorem~\ref{thm:straight-angle}.

\begin{thm}[\abs]{Corollary}\label{cor:degenerate=pi}
A triangle is degenerate if and only if one of its angles is equal to $\pi$ or~$0$. Moreover in a degenerate triangle the angle measures are $0$, $0$ and $\pi$.
\end{thm}

\begin{thm}{Exercise}\label{ex:lineAOB}
Show that three distinct points $A$, $O$ and $B$ lie on one line if and only if 
$$2\cdot \measuredangle AOB\equiv 0.$$ 

\end{thm}

\begin{thm}{Exercise}\label{ex:ABCO-line}
Let $A$, $B$ and $C$ be three points distinct from~$O$.
Show that $B$, $O$ and $C$ lie on one line if and only if
$$2\cdot \measuredangle AOB\equiv 2\cdot \measuredangle AOC.$$ 

\end{thm}

\begin{thm}{Exercise}\label{ex:infinite-number-of-lines} 
Show that there is a nondegenerate triangle.
\end{thm}

\section*{Vertical angles}
\addtocontents{toc}{Vertical angles.}

A pair of angles $AOB$ and $A'OB'$ 
is called \index{angle!vertical angles}\index{vertical angles}\emph{vertical}
if the point $O$ 
lies between $A$ and $A'$ 
and between $B$ and $B'$ at the same time.


\begin{thm}[\abs]{Proposition}\label{prop:vert}
The vertical angles have equal measures.
\end{thm}

\parit{Proof.}
Assume that the angles $AOB$ and $A'OB'$ are vertical.

Note that the angles $AOA'$ and $BOB'$ are straight.
Therefore, $\measuredangle AOA'=\measuredangle BOB'=\pi$.

{

\begin{wrapfigure}{o}{28mm}
\begin{lpic}[t(-2mm),b(0mm),r(0mm),l(0mm)]{pics/vertical(1)}
\lbl[b]{0,4;$A$}
\lbl[br]{22,22;$A'$}
\lbl[t]{13.5,11.5;$O$}
\lbl[tr]{4,20;$B$}
\lbl[bl]{24,6;$B'$}
\end{lpic}
\end{wrapfigure}

It follows that
\begin{align*}
0&=\measuredangle AOA'-\measuredangle BOB'\equiv
\\
&\equiv 
\measuredangle AOB+\measuredangle BOA'-\measuredangle BOA'-\measuredangle A'OB'
\equiv
\\
&\equiv\measuredangle AOB-\measuredangle A'OB'.
\end{align*}
Since $-\pi<\measuredangle AOB\le \pi$ and $-\pi<\measuredangle A'OB'\le \pi$, we get that $\equiv\measuredangle AOB=\measuredangle A'OB'$.
\qeds

}

\begin{thm}{Exercise}\label{ex:O-mid-AB+CD}
Assume $O$ 
is the midpoint for both segments 
$[A B]$ and $[C D]$.
Prove that $A C= B D$. 
\end{thm}




\addtocontents{toc}{\protect\end{quote}}

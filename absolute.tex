%\part*{Non-Euclidean geometry}
\addtocontents{toc}{\protect\contentsline{part}{\protect\numberline{}Non-Euclidean geometry}{}{}}

\chapter{Neutral plane}\label{chap:non-euclid}

Let us remove Axiom~\ref{def:birkhoff-axioms:4} from our axiomatic system (Section~\ref{sec:axioms}).
This way we define a new object called the 
\index{plane!neutral plane}\index{neutral plane}\emph{neutral plane} or \index{plane!absolute plane}\index{absolute plane}\emph{absolute plane}.
(In a neutral plane, Axiom~\ref{def:birkhoff-axioms:4} may or may not hold.)

Every theorem in neutral geometry holds in Euclidean geometry.
In other words, the Euclidean plane is an example of a neutral plane. 
In the next chapter, we will construct an example of a neutral plane that is not Euclidean.

In this book, 
Axiom~\ref{def:birkhoff-axioms:4} was used starting from Chapter~\ref{chap:parallel}.
Therefore all the statements before hold in neutral geometry.

This makes all the discussed results
about
half-planes,
signs of angles,
congruence conditions,
perpendicular lines,
and reflections 
true in neutral geometry.
Recall that a statement is marked with ``$\a$''\label{a-mark} (for example, ``\textbf{Theorem.\abs}'') if it holds in every neutral plane, and the same proof works.


Let us give an example of a theorem in neutral geometry that admits a simpler proof in Euclidean geometry. 

\begin{thm}{Hypotenuse-leg congruence condition}\label{thm:hypotenuse-leg}
Assume that triangles $ABC$ and $A'B'C'$
have right angles at $C$ and $C'$ respectively, 
$AB\z=A'B'$ and $AC\z=A'C'$.
Then $\triangle ABC\cong\triangle A'B'C'$.
\end{thm}


\parit{Euclidean proof.} 
By the Pythagorean theorem $BC=B'C'$.
Then the statement follows from the SSS congruence condition.
\qeds

The proof of the Pythagorean theorem used properties of similar triangles, which in turn used Axiom~\ref{def:birkhoff-axioms:4}. 
Therefore this proof does not work in a neutral plane.

\parit{Neutral proof.}
Suppose that $D$ denotes the reflection of $A$ across $(BC)$
and $D'$ denotes the reflection of $A'$ across $(B'C')$.
Note that 
$$
AD=2\cdot AC=2\cdot A'C'=A'D',\qquad
BD=BA=B'A'=B'D'.
$$

{

\begin{wrapfigure}{r}{28mm}
\vskip-0mm
\centering
\includegraphics{mppics/pic-180}
\end{wrapfigure}

By the SSS congruence condition (\ref{thm:SSS}), 
we get that $\triangle ABD\cong \triangle A'B'D'$.

The statement follows since $C$ is the midpoint of $[AD]$
and $C'$ is the midpoint of $[A'D']$.  
\qeds

\begin{thm}{Exercise}\label{ex:tangent-angle-neutral}
Read the proof of Theorem~\ref{thm:tangent-angle} and identify the first statement that does not work in the netral plane.
\end{thm}

}


\begin{thm}{Exercise}\label{ex:abs-bisect=median}
Give a proof of Exercise~\ref{ex:bisect=median}
that works in the neutral plane. 
\end{thm}



\begin{thm}{Exercise}\label{ex:abs-inscibed}
Let $ABCD$ be an inscribed quadrangle in the neutral plane.
Show that
$$\measuredangle ABC+\measuredangle CDA\equiv \measuredangle BCD+\measuredangle DAB.$$

\end{thm}

One cannot use Corollary~\ref{cor:inscribed-quadrangle} to solve the exercise above since it uses Theorems~\ref{thm:tangent-angle} and \ref{thm:inscribed-angle},
which in turn uses Theorem~\ref{thm:3sum}.


\section{Two angles of a triangle}

In this section, we will prove a weaker form of Theorem~\ref{thm:3sum}
which holds in every neutral plane.

\begin{thm}{Proposition}\label{prop:2sum}
Let $\triangle ABC$ be a nondegenerate triangle in the neutral plane.
Then 
$$|\measuredangle CAB|+|\measuredangle ABC|< \pi.$$

\end{thm}

According to \ref{thm:signs-of-triug}, the angles $ABC$, $BCA$, and $CAB$
have the same sign.
Therefore, in the Euclidean plane, the theorem follows immediately from Theorem~\ref{thm:3sum}.

\begin{wrapfigure}{o}{23mm}
\vskip-1mm
\centering
\includegraphics{mppics/pic-182}
\end{wrapfigure}


\parit{Proof.}
Let $X$ be the reflection of $C$ across the midpoint $M$ of $[AB]$.
Applying \ref{ex:between}, we get
\[|\measuredangle CAX|=|\measuredangle CAB|+|\measuredangle ABC|.\eqlbl{eq:CAX=CAB+ABC}\]

By Proposition~\ref{prop:point-reflection+}
$\measuredangle BAX\z=\measuredangle ABC$.

Note that $\measuredangle CAX\z{\not\equiv} \pi$; otherwise, $X$ would lie on $(AC)$.
Therefore the identity \ref{eq:CAX=CAB+ABC} implies that
\[|\measuredangle CAB|+|\measuredangle ABC|=|\measuredangle CAX|<\pi.\]
\qedsf

\begin{thm}{Exercise}\label{ex:parallel-abs}
Assume $A$, $B$, $C$, and $D$ are points in a neutral plane
such that 
$$2\cdot \measuredangle ABC+2\cdot\measuredangle BCD\equiv 0.$$
Show that $(AB)\parallel (CD)$.
\end{thm}

Note that one cannot apply the transversal property (\ref{thm:parallel-2}) here.


\begin{thm}{Exercise}\label{ex:SAA}
Prove the \index{SAA congruence condition}\emph{side-angle-angle} congruence condition in neutral geometry.


In other words, let $ABC$ and $A'B'C'$ be two triangles in a neutral plane;
suppose that $\triangle A'B'C'$ is nondegenerate.
Show that $\triangle ABC\z\cong \triangle A'B'C'$
if 
$$AB=A'B',
\quad  
\measuredangle ABC=\pm\measuredangle A'B'C',
\quad 
\text{and}
\quad
\measuredangle BCA=\pm\measuredangle B'C'A'.$$

\end{thm}

In the Euclidean plane, the above exercise follows from ASA and the theorem on the sum of angles of a triangle (\ref{thm:3sum}).
However, Theorem~\ref{thm:3sum} cannot be used here, since its proof uses Axiom~\ref{def:birkhoff-axioms:4}.
Later (Theorem~\ref{thm:3sum-h}) 
we will show that Theorem~\ref{thm:3sum} does not hold in a neutral plane.

\begin{thm}{Exercise}\label{ex:chev<side}
Assume that point $D$ lies between vertices $A$ and $B$ of $\triangle ABC$ in the neutral plane.
Show that 
$$CD<CA
\quad
\text{or}
\quad
CD<CB.$$

\end{thm}

\section{Three angles of a triangle}

\begin{thm}{Proposition}\label{prop:angle-side}
Let $\triangle ABC$ and $\triangle A'B'C'$ be two triangles in the neutral plane
such that $AC=A'C'$ and $BC=B'C'$.
Then 
$$AB\z<A'B'
\quad
\text{if and only if}
\quad 
|\measuredangle ACB|<|\measuredangle A'C'B'|.$$

\end{thm}

\begin{wrapfigure}{o}{38mm}
\vskip-6mm
\centering
\includegraphics{mppics/pic-184}
\end{wrapfigure}

\parit{Proof.}
Without loss of generality, we may assume that $A=A'$, $C=C'$, and $\measuredangle ACB,\measuredangle ACB'\ge 0$.
In this case, we need to show that 
$$AB<AB'
\ 
\iff
\  
\measuredangle ACB<\measuredangle ACB'.$$

Choose a point $X$ so that 
$$\measuredangle ACX=\tfrac12\cdot(\measuredangle ACB+\measuredangle ACB').$$
Note that 
\begin{itemize}
\item $(CX)$ bisects $\angle BCB'$.
\item $(CX)$ is the perpendicular bisector of $[BB']$.
\item $A$ and $B$ lie on the same side of $(CX)$ if and only if $$\measuredangle ACB<\measuredangle ACB'.$$
\end{itemize}
From Exercise~\ref{ex:pbisec-side}, $A$ and $B$ lie on the same side of $(CX)$ if and only if $AB<AB'$.
Hence the result.
\qeds

\begin{thm}{Theorem}\label{thm:3sum-a}
Let $\triangle ABC$ be a triangle in the neutral plane.
Then 
$$|\measuredangle ABC|+|\measuredangle BCA|+|\measuredangle CAB|\le \pi.$$

\end{thm}

The following proof is due to Adrien-Marie Legendre \cite{legendre}, 
earlier proofs were given by Giovanni Saccheri \cite{saccheri}
and Johann Lambert \cite{lambert}.

\parit{Proof.} 
Set 
\begin{align*}
a&=BC,
&
b&=CA,
&
c&=AB,
\\
\alpha&=\measuredangle CAB,
&
\beta&=\measuredangle ABC,
&
\gamma&=\measuredangle BCA.
\end{align*}
Without loss of generality, we may assume that $\alpha,\beta,\gamma\ge 0$.

\begin{figure}[!ht]
\centering
\includegraphics{mppics/pic-186}
\end{figure}

Fix a positive integer~$n$.
Consider points $A_0$, $A_1,\dots,A_n$ on the half-line
$[BA)$, such that $BA_i=i\cdot c$ for each~$i$.
(In particular, $A_0=B$ and $A_1=A$.)
Construct points $C_1$, $C_2,\dots,C_n$,
so that
$\measuredangle A_iA_{i-1}C_i\z=\beta$ and $A_{i-1}C_i=a$ for each~$i$.

By SAS, we have constructed $n$ congruent triangles 
\begin{align*}
\triangle ABC&=\triangle A_{1}A_0C_1\cong\triangle A_2A_{1}C_2\cong
\dots
\cong\triangle A_nA_{n-1}C_n.
\end{align*}


Set $d=C_1C_2$ and $\delta=\measuredangle C_2A_1C_1$.
Note that 
$$\alpha+\beta+\delta=\pi.
\eqlbl{eq:gamma'}$$
By Proposition~\ref{prop:2sum}, we get that $\delta\ge 0$.

By construction
\begin{align*}
\triangle A_1C_1C_2&\cong\triangle A_{2}C_2C_3\cong\dots
\cong\triangle A_{n-1}C_{n-1}C_n.
\end{align*}
In particular, $C_iC_{i+1}=d$ 
for each~$i$.

Applying the triangle inequality several times, we get that
\begin{align*}
n\cdot c&=A_0A_n\le 
\\
&\le A_0C_1+C_1C_2+\dots+C_{n-1}C_n+C_nA_n=
\\
&=a+(n-1)\cdot d+b.
\end{align*}

In particular, 
\[c\le  d+\tfrac1n\cdot (a+b-d).\]
Since  $n$ is an arbitrary positive integer,
the latter implies
$c\le d$.
By Proposition~\ref{prop:angle-side}, 
it is equivalent to 
\[\gamma\le \delta.\] 
From \ref{eq:gamma'}, 
the theorem follows.
\qeds

\begin{thm}{Exercise}\label{ex:neutral-quadrangle}
Let $ABCD$ be a quadrangle in the neutral plane.
Suppose that $\angle DAB$ and $\angle ABC$ are right.
Show that $AB\le CD$.
\end{thm}

\section{The defect}\label{The defect}

The \index{defect of triangle}\emph{defect of triangle} $\triangle ABC$ is defined as 
$$\defect(\triangle ABC)
\df 
\pi-|\measuredangle ABC|-|\measuredangle BCA|-|\measuredangle CAB|.$$

Theorem~\ref{thm:3sum-a} states that \textit{the defect of each triangle in a neutral plane has to be nonnegative}.
According to Theorem~\ref{thm:3sum}, every triangle in
the Euclidean plane has zero defect.

{

\begin{wrapfigure}{o}{24mm}
\vskip-6mm
\centering
\includegraphics{mppics/pic-188}
\vskip4mm
\includegraphics{mppics/pic-189}
\end{wrapfigure}

\begin{thm}{Classroom exercise}\label{ex:defect}
Let $\triangle ABC$ be a nondegenerate triangle in the neutral plane.
Assume $D$ lies between $A$ and~$B$.
Show that 
$$\defect(\triangle ABC)=\defect(\triangle ADC)+\defect(\triangle DBC).$$

\end{thm}


\begin{thm}{Exercise}\label{ex:defect=} Let $ABC$ be a nondegenerate triangle in the neutral plane.
Suppose $X$ is a reflection of $C$ across a midpoint $M$ of $[AB]$.
Show that 
$$\defect(\triangle ABC)=\defect(\triangle AXC).$$
\end{thm}

}

\vskip-2mm

\begin{thm}{Exercise}\label{ex:neutral-rectangle}
Let $ABCD$ be a \index{rectangle}\emph{rectangle} in the neutral plane;
that is, $ABCD$ is a quadrangle with all right angles.
Show that $AB\z=CD$. 
\end{thm}

\begin{thm}{Advanced exercise}\label{ex:neutral-rectangle+}
Show that if the neutral plane has a rectangle, then all its triangles have zero defect.
\end{thm}

\section{Proving that something cannot be proved}
\label{sec:unprovable}

Many attempts were made to prove that every theorem in Euclidean geometry holds in neutral geometry.
The latter is equivalent to the statement that Axiom~\ref{def:birkhoff-axioms:4} is a \textit{theorem} in neutral geometry.

Some of these attempts were accepted as proof for long periods until a mistake was found.

Many statements in neutral geometry are equivalent to Axiom~\ref{def:birkhoff-axioms:4}.
It means that if we exchange Axiom~\ref{def:birkhoff-axioms:4} for each of these statements, then we will obtain an equivalent axiomatic system.

The following theorem provides a short list of such statements.
We are not going to prove it in the book.

\begin{thm}{Theorem}\label{thm:=IV}
The neutral plane is Euclidean if and only if one of the following equivalent conditions holds:
\begin{enumerate}[(a)]
\item\label{thm:=IV:main} 
There is a line $\ell$ 
and a point $P\notin\ell$ 
such that there is only one line passing thru $P$ 
and parallel to~$\ell$.
\item 
Every nondegenerate triangle can be circumscribed.
\item
There exists a pair of distinct lines that lie at a bounded distance from each other.
\item
There is a triangle with an arbitrarily large inradius.
\item\label{thm:=IV:defect}
There is a nondegenerate triangle with zero defect.
\item\label{thm:=IV:rectangle}
There exists a \index{rectangle}\emph{rectangle}; that is, a quadrangle with all right angles.
\end{enumerate}
\end{thm}

It is hard to imagine a neutral plane that does not satisfy some of the properties above.
That is partly the reason for a large number of false proofs;
each used one of such statements by accident.

Let us formulate the negation of \textit{(\ref{thm:=IV:main})} above as a new axiom;
we label it h-$\!$\ref{def:birkhoff-axioms:4} as a \textit{hyperbolic version} of Axiom~\ref{def:birkhoff-axioms:4}.

\begin{framed}
\begin{description}
\item[{\rm h-$\!$\ref{def:birkhoff-axioms:4}.}]\label{def:hyperbolic-4a}  
For every line $\ell$ and every point $P\notin\ell$
there are at least two lines that pass thru $P$ and parallel to~$\ell$.
\end{description}
\end{framed}

By Theorem~\ref{thm:parallel}, a neutral plane that satisfies Axiom~h-$\!$\ref{def:birkhoff-axioms:4} is not Euclidean. 
Moreover, according to Theorem~\ref{thm:=IV} (which we do not prove) 
in every non-Euclidean neutral plane, Axiom~h-$\!$\ref{def:birkhoff-axioms:4} holds.

It opens a way to look for a proof by contradiction.
Simply exchange  Axiom~\ref{def:birkhoff-axioms:4} to Axiom~h-$\!$\ref{def:birkhoff-axioms:4}
 and start to prove theorems in the obtained axiomatic system.
In the case if we arrive at a contradiction, 
we prove Axiom~\ref{def:birkhoff-axioms:4} in a neutral plane.
This idea was growing since the $5^\text{th}$ century;
the most notable results were obtained by Giovanni Saccheri \cite{saccheri}.

The system of axioms \ref{def:birkhoff-axioms:0}--\ref{def:birkhoff-axioms:3} and h-$\!$\ref{def:birkhoff-axioms:4} defines a new geometry which is now called \index{hyperbolic!geometry}\emph{hyperbolic} or \index{Lobachevsky  geometry}\emph{Lobachevsky  geometry}.
The more this geometry was developed,
it became more and more believable that there is no contradiction;
that is, the system of axioms \ref{def:birkhoff-axioms:0}--\ref{def:birkhoff-axioms:3}, and h-$\!$\ref{def:birkhoff-axioms:4} is \index{consistent}\emph{consistent}.
In fact, the following theorem holds true:


\begin{thm}{Theorem}\label{thm:consistent}
Hyperbolic geometry is consistent if and only if so is Euclidean geometry.
\end{thm}

The claims
that hyperbolic geometry has no contradiction can be found in the private letters of
Carl Friedrich Gauss, 
Ferdinand  Schweikart, 
and Franz Taurinus.%
\footnote{The oldest surviving letters were the Gauss letter to Christian Gerling in 1816 
and the yet more convincing letter dated 1818 
of Schweikart sent to Gauss via Gerling.}
They all seem to be afraid to state it in public.
For instance, in 1818 Gauss writes to Gerling:

\smallskip

\begin{quotation}{\it
\dots I am happy that you have the courage to express yourself as if you recognized the possibility that our parallels theory along with our entire geometry could be false.
But the wasps whose nest you disturb will fly around your head.}
\end{quotation}

\smallskip

Nikolai Lobachevsky came to the same conclusion independently.
Unlike the others, he dared to state it in public and in print \cite{lobachevsky}.
That cost him serious trouble.
A couple of years later, also independently, János Bolyai published his work \cite{bolyai}.

It seems that Lobachevsky was the first who had a proof of Theorem~\ref{thm:consistent} altho its formulation required rigorous axiomatics which was not developed at his time.
Later, Beltrami gave a cleaner proof of the ``if'' part of the theorem.
It was done by modeling points, lines, distances, and angle measures of one geometry using some other objects in another geometry.
The same idea was used earlier by Lobachevsky \cite[\S34]{lobachevsky-1840}; 
he modeled the Euclidean plane in the hyperbolic space.

The proof of Beltrami is the subject of the next chapter. 

%\medskip

%Arguably, the discovery of hyperbolic geometry was the second main scientific discovery of the $19^\text{th}$ century, trailing only Mendel's laws. %Law of multiple proportions
%The discovery of hyperbolic geometry was one of the main scientific discoveries of the $19^\text{th}$ century, on the same level are Mendel's laws and the law of multiple proportions.

\section{Curvature}

In a letter from 1824 Gauss writes: 

\begin{quotation}{\it
The assumption that the sum of the three angles is less than $\pi$ leads to a curious geometry, 
quite different from ours but completely consistent, 
which I have developed to my entire satisfaction, 
so that I can solve every problem in it with the exception of a determination of a constant, which cannot be designated a priori. 
The greater one takes this constant, the nearer one comes to Euclidean geometry, 
and when it is chosen indefinitely large the two coincide.
The theorems of this geometry appear to be paradoxical and, 
to the uninitiated, absurd; but calm, steady reflection reveals that they contain nothing at all impossible. 
For example, the three angles of a triangle become as small as one wishes, if only the sides are taken sufficiently large; 
yet the area of the triangle can never exceed a definite limit, regardless how great the sides are taken, 
nor indeed can it ever reach it.}
\end{quotation} 

In modern terminology, the constant that Gauss mentions 
can be expressed as $1/\sqrt{-k}$, 
where $k\le 0$, is the so-called \index{curvature}\emph{curvature} of the neutral plane, which we are about to introduce.

The identity in Exercise~\ref{ex:defect} suggests that the defect of a triangle should be proportional to its area.%
\footnote{The area in the neutral plane is discussed briefly at the end of Chapter~\ref{chap:area},
but the reader could also refer to an intuitive understanding of area measurement.}

In fact, for every neutral plane, there is a nonpositive real number $k$
such that 
$$k\cdot\area(\triangle ABC)+\defect(\triangle ABC)=0$$
for every $\triangle ABC$.
This number $k$ is called the \index{curvature}\emph{curvature} of the plane.

For example, by Theorem~\ref{thm:3sum}, the Euclidean plane has zero curvature.
By Theorem~\ref{thm:3sum-a}, the curvature of every neutral plane is nonpositive.

It turns out that up to isometry, the neutral plane is characterized by its curvature;
that is, two neutral planes are isometric if and only if they have the same curvature. 

In the next chapter, we will construct a {}\emph{hyperbolic plane};
this is, an example of a neutral plane with curvature $k=-1$.

Any neutral plane, distinct from Euclidean,
can be obtained by rescaling the metric on the hyperbolic plane.
Indeed,
if we rescale the metric by a positive factor $c$,
the area changes by factor $c^2$, while the defect stays the same.
Therefore, taking $c=\sqrt{-k}$,
we can get the neutral plane of the given curvature $k<0$.
In other words, all the non-Euclidean neutral planes become identical
if we use $r=1/\sqrt{-k}$ as the unit of length.

\medskip

In Chapter~\ref{chap:sphere}, we discuss spherical geometry.
Altho spheres are not neutral planes,
the spherical geometry is a close relative of Euclidean and hyperbolic geometries.

Nondegenerate spherical triangles have negative defects.
Moreover, 
if $r$ is the radius of the sphere, then
$$\tfrac1{r^2}\cdot\area(\triangle ABC)+\defect(\triangle ABC)=0$$
for every spherical triangle $ABC$; compare to Lemma~\ref{lem:area-spher-triangle}.
In other words, 
the sphere of radius $r$ has the curvature $k=\tfrac1{r^2}$.

%\part*{Non-Euclidean geometry}
\addtocontents{toc}{\protect\begin{center}}
\addtocontents{toc}{\large{\bf Non-Euclidean geometry}}
\addtocontents{toc}{\protect\end{center}}
\chapter{Absolute plane}\label{chap:non-euclid}
\addtocontents{toc}{\protect\begin{quote}}

Let us remove Axiom \ref{def:birkhoff-axioms:4} from our axiomatic system, see pages \pageref{def:birkhoff-axioms:0}--\pageref{def:birkhoff-axioms:4}.
This way we define a new object called 
\index{plane!absolute plane}\index{absolute plane}\emph{absolute plane} or \index{plane!neutral plane}\index{neutral plane}\emph{neutral plane}.
(In the absolute plane, 
the Axiom \ref{def:birkhoff-axioms:4} may or may not hold.)

Clearly any theorem in absolute geometry holds in Euclidean geometry.
In other words, Euclidean plane is an example of absolute plane. 
In the next chapter we will construct an example of absolute plane which is not Euclidean.

In this book, 
the Axiom \ref{def:birkhoff-axioms:4} was used
for the first time in the proof of uniqueness of parallel line in Theorem~\ref{thm:parallel}.
Therefore all the statements before Theorem~\ref{thm:parallel} also hold in absolute geometry.

It makes all the discussed results
about
half-planes,
signs of angles,
congruence conditions,
perpendicular lines and reflections 
true in absolute plane.
In fact \textbf{if you do not see words ``Euclidean plane'' or ``inversive plane'' in a formulation,
then it holds in absolute plane} and the same proof works.

Let us give an example of theorem in absolute geometry,
which admits a simpler proof in Euclidean geometry. 

\begin{thm}{Theorem}
Assume that triangles $\triangle ABC$ and $\triangle A'B'C'$
have right angles at $C$ and $C'$ correspondingly, 
$AB=A'B'$ and $AC\z=A'C'$.
Then $\triangle ABC\cong\triangle A'B'C'$.
\end{thm}


\parit{Euclidean proof.} 
By Pythagorean theorem $BC=B'C'$.
Then the statement follows from SSS congruence condition.
\qeds

Note that the proof of Pythagorean theorem used properties of similar triangles, which in turn used Axiom~\ref{def:birkhoff-axioms:4}. 
Whence the above proof is not working in absolute plane.

\begin{wrapfigure}[10]{o}{28mm}
\begin{lpic}[t(2mm),b(8mm),r(0mm),l(3mm)]{pics/ABC-D(1)}
\lbl[l]{23,2;$A$}
\lbl[b]{12,24.5;$B$}
\lbl[t]{12,1;$C$}
\lbl[r]{0.5,2;$D$}
\end{lpic}
\end{wrapfigure}

\parit{Absolute proof.}
Denote by $D$ the reflection of $A$ thru $(BC)$
and by $D'$ the reflection of $A'$ thru $(B'C')$.
Note that 
$$
AD=2\cdot AC=2\cdot A'C'=A'D',
$$
$$
BD=BA=B'A'=B'D'.
$$

By SSS, 
we get  $\triangle ABD\cong \triangle A'B'D'$.

The theorem follows since $C$ is the midpoint of $[AD]$
and $C'$ is the midpoint of $[A'D']$.  
\qeds

\begin{thm}{Exercise}\label{ex:abs-bisect=median}
Give a proof of Exercise~\ref{ex:bisect=median}
which works in the absolute plane. 
\end{thm}

\begin{thm}{Exercise}\label{ex:abs-inscibed}
Let $\square ABCD$ be an inscribed quadrilateral in the absolute plane.
Show that
$$\measuredangle ABC+\measuredangle CDA\equiv \measuredangle BCD+\measuredangle DAB.$$

\end{thm}


Note that the Theorem~\ref{thm:inscribed-quadrilateral} cannot be applied in the above exercise;
it use Theorems~\ref{thm:tangent-angle} and \ref{thm:inscribed-angle}; which in turns use Theorem~\ref{thm:3sum}.


\section*{Two angles of triangle}
\addtocontents{toc}{Two angles of triangle.}

In this section we will prove a weaker form of Theorem~\ref{thm:3sum}
which holds in absolute plane.

\begin{thm}{Proposition}\label{prop:2sum}
Let $\triangle ABC$ be nondegenerate triangle in the absolute plane.
Then 
$$|\measuredangle CAB|+|\measuredangle ABC|< \pi.$$

\end{thm}

Note according to \ref{thm:signs-of-triug}, the angles 
$\angle ABC$, 
$\angle BCA$ and $\angle CAB$
have the same sign.
Therefore in Euclidean plane the theorem follows immediately from Theorem~\ref{thm:3sum}.
In absolute geometry we need to work more.

\parit{Proof.}
By \ref{thm:signs-of-triug}, 
we may assume that $\angle CAB$
and $\angle ABC$ are positive.

Let $M$ be the midpoint of $[AB]$.
Chose $C'\in (CM)$ distinct from $C$ so that $C'M=CM$.


Note that the angles $\angle AMC$ and $\angle BMC'$
are vertical;
in particular 
$$\measuredangle AMC=\measuredangle BMC'.$$

By construction $AM\z=BM$ and $CM\z=C'M$.
Therefore $\triangle AMC\cong \triangle BMC'$; 
in particular 
$$\measuredangle CAB=\pm\measuredangle C'BA.$$

\begin{wrapfigure}[10]{o}{37mm}
\begin{lpic}[t(4mm),b(0mm),r(0mm),l(2mm)]{pics/2sum(1)}
\lbl[lt]{22.5,1;$B$}
\lbl[br]{10.5,22;$A$}
\lbl[tr]{0,1;$C$}
\lbl[lb]{33,22;$C'$}
\lbl[b]{18,14;$M$}
\end{lpic}
\end{wrapfigure}

According to \ref{thm:signs-of-triug}, 
the angles $\angle CAB$ and $\angle C'BA$ have the same sign as $\angle AMC$ and $\angle BMC'$.
Therefore
$$\measuredangle CAB=\measuredangle C'BA.$$

In particular,
\begin{align*}
\measuredangle C'BC&\equiv \measuredangle C'BA+\measuredangle ABC\equiv
\\
&\equiv \measuredangle CAB+\measuredangle ABC.
\end{align*}

Finally note that $C'$ and $A$ lie on the same side from $(CB)$.
Therefore the angles $\angle CAB$, $\angle ABC$ and $\angle C'BC$ are positive.
By Exercise~\ref{ex:PP(PN)}, the result follows.
\qeds

\begin{thm}{Exercise}\label{ex:parallel-abs}
Assume $A$, $B$, $C$ and $D$ be points in absolute plane
such that 
$$2\cdot \measuredangle ABC+2\cdot\measuredangle BCD\equiv 0.$$
Show that $(AB)\parallel (CD)$.
\end{thm}

Note that one cannot extract the solution of the above exercise from the proof of Transversal property (\ref{thm:parallel-2})


\begin{thm}{Exercise}\label{ex:SAA}
Prove \index{SAA congruence condition}\emph{side-angle-angle congruence condition} in absolute plane.

In other words, let $\triangle ABC$ and $\triangle A'B'C'$ be two triangles in absolute plane.
Show that $\triangle ABC\cong \triangle A'B'C'$
if 
$$AB=A'B',
\quad  
\measuredangle ABC=\pm\measuredangle A'B'C'
\quad 
\text{and}
\quad
\measuredangle BCA=\pm\measuredangle B'C'A'.$$

\end{thm}

Note that in the Euclidean plane, the above exercise follows from ASA and the theorem on sum of angles of triangle (\ref{thm:3sum}).
However, Theorem~\ref{thm:3sum} cannot be used here since its proof use Axiom~\ref{def:birkhoff-axioms:4}.
Later, in theorem Theorem~\ref{thm:3sum-h}, 
we will show that Theorem~\ref{thm:3sum} does not hold in absolute plane.

\begin{thm}{Exercise}\label{ex:chev<side}
Assume that point $D$ lies between the vertices $A$ and $B$ of triangle $\triangle ABC$ in the absolute plane.
Show that 
$$CD<CA
\quad
\text{or}
\quad
CD<CB.$$

\end{thm}

\section*{Three angles of triangle}
\addtocontents{toc}{Three angles of triangle.}

\begin{thm}{Proposition}\label{prop:angle-side}
Let $\triangle ABC$ and $\triangle A'B'C'$ be two triangles in the absolute plane
such that $AC=A'C'$ and $BC=B'C'$.
Then 
$$AB\z<A'B'
\quad
\text{if and only if}
\quad 
|\measuredangle ACB|<|\measuredangle A'C'B'|.$$

\end{thm}

\begin{wrapfigure}{o}{41mm}
\begin{lpic}[t(-0mm),b(0mm),r(0mm),l(2mm)]{pics/ABBCX(1)}
\lbl[l]{37,1;$A$}
\lbl[r]{4,1;$C$}
\lbl[lb]{17,20;$B$}
\lbl[b]{2,23;$B'$}
\lbl[l]{11.5,26;$X$}
\end{lpic}
\end{wrapfigure}

\parit{Proof.}
Without loss of generality, we may assume that $A=A'$ and $C=C'$ and $\measuredangle ACB,\measuredangle ACB'\ge 0$.
In this case we need to show that 
$$AB<AB'
\quad
\iff
\quad 
\measuredangle ACB<\measuredangle ACB'.$$

Choose a point $X$ so that 
$$\measuredangle ACX=\tfrac12\cdot(\measuredangle ACB+\measuredangle ACB').$$
Note that 
\begin{itemize}
\item $(CX)$ bisects $\angle BCB'$
\item $(CX)$ is the perpendicular bisector of $[BB']$.
\item $A$ and $B$ lie on the same side from $(CX)$ if and only if $$\measuredangle ACB<\measuredangle ACB'.$$
\end{itemize}
From Exercise~\ref{ex:pbisec-side}, $A$ and $B$ lie on the same side from $(CX)$ if and only if $AB<AB'$.
Hence the result follows.
\qeds

\begin{thm}{Theorem}\label{thm:3sum-a}
Let $\triangle ABC$ be a triangle in the absolute plane.
Then 
$$|\measuredangle ABC|+|\measuredangle BCA|+|\measuredangle CAB|\le \pi.$$

\end{thm}

The following proof is due to Legendre \cite{legendre}, 
earlier proofs were due to Saccheri \cite{saccheri}
and Lambert \cite{lambert}.

\parit{Proof.} 
Let $\triangle ABC$ be the given triangle.
Set 
\begin{align*}
a&=BC,
&
b&=CA,
&
c&=AB,
\\
\alpha&=\measuredangle CAB,
&
\beta&=\measuredangle ABC,
&
\gamma&=\measuredangle BCA.
\end{align*}
Without loss of generality, we may assume that $\alpha,\beta,\gamma\ge 0$.

\begin{center}
\begin{lpic}[t(1mm),b(0mm),r(0mm),l(0mm)]{pics/legendre(1)}
\lbl[t]{2,0;$A_0$}
\lbl[t]{22,0;$A_1$}
\lbl[t]{42,0;$A_2$}
\lbl[]{63,-1;$\dots$}
\lbl[t]{82,0;$A_n$}
\lbl[b]{10,24;$C_1$}
\lbl[b]{30,24;$C_2$}
\lbl[b]{50,25;$\dots$}
\lbl[b]{70,24;$C_n$}
\lbl[w]{12,2.5;$\,c\,$}
\lbl[w]{32,2.5;$\,c\,$}
\lbl[w]{52,2.5;$\,c\,$}
\lbl[w]{72,2.5;$\,c\,$}
\lbl[W]{7,12;$\,a$}
\lbl[W]{15,12;$\,b$}
\lbl[w]{21,22;$\,d$}
\lbl[W]{27,12;$\,a$}
\lbl[W]{35,12;$\,b$}
\lbl[w]{41,22;$\,d$}
\lbl[W]{47,12;$\,a$}
\lbl[W]{55,12;$\,b$}
\lbl[w]{61,22;$\,d$}
\lbl[W]{67,12;$\,a$}
\lbl[W]{75,12;$\,b$}
\lbl[rb]{16,5;$\alpha$}
\lbl[lb]{8,5;$\beta$}
\lbl[t]{10,16;$\gamma$}
\lbl[b]{22,9;$\delta$}
\lbl[rb]{36,5;$\alpha$}
\lbl[lb]{28,5;$\beta$}
\lbl[t]{30,16;$\gamma$}
\end{lpic}
\end{center}

Fix a positive integer $n$.
Consider points $A_0$, $A_1,\dots,A_n$ on the half-line
$[BA)$ so that $BA_i=i\cdot c$ for each $i$.
(In particular, $A_0=B$ and $A_1=A$.)
Let us construct the points $C_1$, $C_2,\dots,C_n$,
so that
$\measuredangle A_iA_{i-1}C_i\z=\beta$ and $A_{i-1}C_i=a$ for each $i$.

This way we construct $n$ congruent triangles 
\begin{align*}
\triangle ABC&=\triangle A_{1}A_0C_1\cong
\\
&\cong\triangle A_2A_{1}C_2\cong
\\
&\phantom{\cong\triangle A}\dots
\\
&\cong\triangle A_nA_{n-1}C_n.
\end{align*}


Set $d=C_1C_2$ and $\delta=\measuredangle C_2A_1C_1$.
Note that 
$$\alpha+\beta+\delta=\pi.
\eqlbl{eq:gamma'}$$
By Proposition~\ref{prop:2sum}, $\delta\ge 0$.

By construction
\begin{align*}
\triangle A_1C_1C_2&\cong\triangle A_{2}C_2C_3\cong\dots
\cong\triangle A_{n-1}C_{n-1}C_n.
\end{align*}
In particular, $C_iC_{i+1}=d$ 
for each $i$.


By repeated application
of the triangle inequality, we get 
that
\begin{align*}
n\cdot c&=A_0A_n\le 
\\
&\le A_0C_1+C_1C_2+\dots+C_{n-1}C_n+C_nA_n=
\\
&=a+(n-1)\cdot d+b.
\end{align*}

In particular, 
$$c\le  d+\tfrac1n\cdot (a+b-d).$$
Since  $n$ is arbitrary positive integer,
the latter implies
$$c\le d.$$

From Proposition~\ref{prop:angle-side} and SAS, 
the latter is equivalent to 
$$\gamma\le \delta.$$ 
From \ref{eq:gamma'}, 
the theorem follows.
\qeds

The \index{defect of triangle}\emph{defect of triangle} $\triangle ABC$ is defined as 
$$\defect(\triangle ABC)
\df 
\pi-|\measuredangle ABC|+|\measuredangle BCA|+|\measuredangle CAB|.$$

Note that Theorem~\ref{thm:3sum-a} sates that, defect of any triangle in absolute plane has to be nonnegative.
According to Theorem~\ref{thm:3sum}, any triangle in
Euclidean plane has zero defect.

\begin{wrapfigure}{o}{19mm}
\begin{lpic}[t(-0mm),b(0mm),r(0mm),l(2mm)]{pics/defect(1)}
\lbl[tr]{1,1;$A$}
\lbl[r]{6,19;$C$}
\lbl[lt]{15,1;$B$}
\lbl[t]{9,1.5;$D$}
\end{lpic}
\end{wrapfigure}

\begin{thm}{Exercise}\label{ex:defect}
Let $\triangle ABC$ be nondegenerate triangle in the absolute plane.
Assume $D$ lies between $A$ and $B$.
Show that 
$$\defect(\triangle ABC)=\defect(\triangle ADC)+\defect(\triangle DBC).$$

\end{thm}



\section*{How to prove that something\\ 
cannot be proved?}
\addtocontents{toc}{How to prove that something cannot be proved?}

Many attempts were made to prove that any theorem in Euclidean geometry holds in absolute geometry.
The latter is equivalent to the statement that Axiom \ref{def:birkhoff-axioms:4} is a {}\emph{theorem} in absolute geometry.

Many these attempts being accepted as proofs for long periods of time until the mistake was found.

There is a number of statements in the geometry of absolute plane which are equivalent to the Axiom~\ref{def:birkhoff-axioms:4}.
It means that if we exchange the Axiom~\ref{def:birkhoff-axioms:4}  to any of these statements, 
then we will obtain an equivalent axiomatic system.

Here we give a short list of such statements.
We are not going to prove the equivalence in the book.

\begin{thm}{Theorem}\label{thm:=IV}
An absolute plane is Euclidean if and only if one of the following equivalent conditions hold.
\begin{enumerate}[(a)]
\item\label{thm:=IV:main} 
There is a line $\ell$ 
and a point $P\notin\ell$ 
such that there is only one line passing thru $P$ 
and parallel to $\ell$.
\item 
Every nondegenerate triangle can be circumscribed.
\item
There exists a pair of distinct lines which lie on a bounded distance from each other.
\item
There is a triangle with arbitrary large inradius.
\item
There is a nondegenerate triangle with zero defect.
\item
There exists a quadrilateral in which all angles are right.
\end{enumerate}
\end{thm}

It is hard to imagine an absolute plane, which does not satisfy some of the properties above.
That is partly the reason why for the large number of false proofs;
each used one of such statements by accident.

Let us formulate the negation of (\ref{thm:=IV:main}) above as a new axiom.


\begin{framed}
\begin{description}
\item[{\rm h-$\!$\ref{def:birkhoff-axioms:4}.}]\label{def:hyperbolic-4a}  
For any line $\ell$ and any point $P\notin\ell$ 
there are at least two lines which pass thru $P$ 
and parallel to $\ell$.
\end{description}
\end{framed}

By Theorem~\ref{thm:parallel}, an absolute plane which satisfies Axiom~h-$\!$\ref{def:birkhoff-axioms:4} is not Euclidean. 
Moreover, according to the Theorem~\ref{thm:=IV} (which we do not prove) 
any non-Euclidean absolute plane Axiom~h-$\!$\ref{def:birkhoff-axioms:4} holds.

It opens a way to look for a proof by contradiction.
Simply exchange  Axiom~\ref{def:birkhoff-axioms:4} to Axiom~h-$\!$\ref{def:birkhoff-axioms:4}
 and start to prove theorems in the obtained axiomatic system.
In the case if we arrive to a contradiction, 
we prove the Axiom~\ref{def:birkhoff-axioms:4} in absolute plane.  

This idea was growing since 5th century;
the most notable result were obtained by Saccheri in \cite{saccheri}.
The more this new geometry was developed,
it became more and more believable that there will be no contradiction;
that is, the system of axioms \ref{def:birkhoff-axioms:0}--\ref{def:birkhoff-axioms:3} and h-$\!$\ref{def:birkhoff-axioms:4} is \index{consistent}\emph{consistent}.

This new type of geometry is now called \index{hyperbolic geometry}\emph{hyperbolic} or  \index{Lobachevskian geometry}\emph{Lobachevskian geometry}.
In fact the following theorem holds.

\begin{thm}{Theorem}\label{thm:consistent}
The hyperbolic geometry is consistent if and only if so is the Euclidean geometry.
\end{thm}

The statement
that hyperbolic geometry has no contradiction appears first in private letters of
Bolyai,  Gauss, Schweikart and Taurinus.%
\footnote{The oldest surviving letters were the Gauss letter to Gerling 1816 
and yet more convincing letter dated by 1818 
of Schweikart sent to Gauss via Gerling.}
They all seem to be afraid to state it in public.
Say, in 1818  
Gauss writes to Gerling

\begin{quotation}{\it
\dots I am happy that you have the courage to express yourself as if you recognized the possibility
that our parallels theory along with our entire geometry could be false. But the wasps whose
nest you disturb will fly around your head.\dots}
\end{quotation}

Lobachevsky came to the same conclusion independently.
Unlike the others he had courage to state it in public and in print
(see \cite{lobachevsky}).
That cost him serious troubles.

It seems that Lobachevsky was also the first who had a proof of Theorem \ref{thm:consistent}.
Later Beltrami gave a cleaner proof of ``if'' part of the theorem.
It was done by modeling points, lines, distances and angle measures of one geometry using some other objects in the other geometry.
The same idea was used originally by Lobachevsky in the proof of the ``only if'' part of the theorem, see \cite[34]{lobachevsky-1840}.

The proof of Beltrami is the subject of the next chapter. 

\medskip

Arguably, the discovery of hyperbolic geometry was the second main discoveries of 19th century, 
trailing only the Mendel's laws.

\section*{Curvature}
\addtocontents{toc}{Curvature.}
In a letter from 1824 Gauss writes: 

\begin{quotation}{\it
The assumption that the sum of the three angles is less than $\pi$ leads to a curious geometry, 
quite different from ours but thoroughly consistent, 
which I have developed to my entire satisfaction, 
so that I can solve every problem in it with the exception of a determination of a constant, which cannot be designated a priori. 
The greater one takes this constant, the nearer one comes to Euclidean geometry, 
and when it is chosen indefinitely large the two coincide.
The theorems of this geometry appear to be paradoxical and, 
to the uninitiated, absurd; but calm, steady reflection reveals that they contain nothing at all impossible. 
For example, the three angles of a triangle become as small as one wishes, if only the sides are taken large enough; 
yet the area of the triangle can never exceed a definite limit, regardless how great the sides are taken, 
nor indeed can it ever reach it.}
\end{quotation} 

In the modern terminology the constant which Gauss mentions, 
can be expressed as $1/\sqrt{-k}$, 
where $k\le 0$ is so called \index{curvature}\emph{curvature} of the absolute plane which we are about to introduce.

The identity in the Exercise~\ref{ex:defect}
suggests that defect of triangle 
should be proportional to its area.%
\footnote{The area in the absolute plane discussed briefly in the end of Chapter~\ref{chap:area},
but the reader could also refer to intuitive understanding of area.}

In fact for any absolute plane there is a nonpositive real number $k$
such that 
$$k\cdot\area(\triangle ABC)+\defect(\triangle ABC)=0$$
for any triangle $\triangle ABC$.
This number $k$ is called \index{curvature}\emph{curvature} of the plane.

For example, by Theorem~\ref{thm:3sum}, the Euclidean plane has zero curvature.
By Theorem~\ref{thm:3sum-a}, curvature of any absolute plane is nonpositive.

It turns out that up to isometry, the absolute plane is characterized by its curvature;
that is, two absolute planes are isometric if and only if they have the same curvature. 



In the next chapter we will construct hyperbolic plane,
this is an example of absolute plane with curvature $k=-1$.

Any absolute planes, distinct from Euclidean,
can be obtained by rescaling metric on the hyperbolic plane.
Indeed,
if we rescale the metric by a positive factor $c$,
the area changes by factor $c^2$, 
while defect stays the same.
Therefore taking $c=\sqrt{-k}$,
we can get the absolute plane given curvature $k<0$.
In other words, all the non-Euclidean absolute planes become identical
if we use $r=1/\sqrt{-k}$ as the unit of length.

\medskip

In the Chapter~\ref{chap:sphere},
we briefly discuss the geometry of the unit sphere.
Altho spheres are not absolute planes,
the spherical geometry is a close relative of Euclidean and hyperbolic geometries.

The nondegenerate spherical triangles have negative defect.
Moreover, 
if $R$ is the radius of the sphere, 
then 
$$\tfrac1{R^2}\cdot\area(\triangle ABC)+\defect(\triangle ABC)=0$$
for any spherical triangle $\triangle ABC$.
In other words, 
the sphere of radius $R$ has positive curvature $k=\tfrac1{R^2}$.


\addtocontents{toc}{\protect\end{quote}}
\chapter{Projective geometry}\label{chap:proj}
\addtocontents{toc}{\protect\begin{quote}}

\section*{Real projective plane}
\addtocontents{toc}{Real projective plane.}

In the Euclidean plane, two distinct lines might have one or zero points of intersection 
(in the latter case the lines are called parallel).
Our aim is to extend Euclidean plane by ideal points so that any two distinct lines will have exactly one point of intersection.

\begin{wrapfigure}{o}{31mm}
\begin{lpic}[t(-3mm),b(0mm),r(0mm),l(0mm)]{pics/pensil-tip(1)}
\end{lpic}
\begin{lpic}[t(5mm),b(-5mm),r(0mm),l(0mm)]{pics/pensil-par(1)}
\end{lpic}
\end{wrapfigure}

A collection of lines in the Euclidean plane is called \index{concurrent}\emph{concurrent} if they all intersect at a single point or all of them pairwise parallel.
A maximal set of concurrent lines in the plane is called \index{pencil}\emph{pencil}.
There are two types of pencils: 
\emph{central pencils} contain all lines passing thru a fixed point called the \index{center!center of the pencil}\emph{center of the pencil}
and  
\emph{parallel pencil} contain pairwise parallel lines.

Each point in the Euclidean plane is uniquely defines a central pencil with the center in it.
Note that any two lines completely determine the pencil containing both.

Let us add one \index{ideal point}\emph{ideal point} for each parallel pencil,
and assume that all these ideal points lie on one \index{ideal line}\emph{ideal line}.
We also assume that the ideal line belongs to each parallel pencil.

We obtain the so called \index{real projective plane}\emph{real projective plane}.
It comes with an incidence structure --- we say that three points lie on one line if the corresponding pencils contain a common line. 

Projective geometry studies this incidence structure on the projective plane.
Loosely speaking, any statement in projective geometry can be formulated using only terms {}\emph{collinear points},
\emph{concurrent lines}.



\section*{Euclidean space}
\addtocontents{toc}{Euclidean space.}

Let us repeat the construction of metric $d_2$ (page 
\pageref{def:d_2}) in the space.

Let $\mathbb{R}^3$ denotes the set of all triples $(x,y,z)$ of real numbers.
Assume $A=(x_A,y_A,z_A)$ and $B=(x_B,y_B,z_B)$ are arbitrary points.
Define the metric on $\mathbb{R}^3$ the following way:
$$AB
\df
\sqrt{(x_A-x_B)^2+(y_A-y_B)^2+(z_A-z_B)^2}.$$
The obtained metric space is called \index{Euclidean space}\emph{Euclidean space}.

Assume at least one of the real numbers $a$, $b$ or $c$ is distinct from zero.
Then the subset of points $(x,y,z)\in\mathbb{R}^3$ 
described by equation
$$a\cdot x+b\cdot y+c\cdot z+d=0$$ 
is called \index{plane!plane in the space}\emph{plane};
here $d$ is a real number.

It is straightforward to show the following:
\begin{itemize}
 \item Any plane in the Euclidean space is isometric to the Euclidean plane.
 \item Any three points in the space lie on a plane.
 \item An intersection of two distinct planes (if it is nonempty) is a line in each of these planes.
\end{itemize}

These statements make possible to generalize many notions and results from Euclidean plane geometry to the Euclidean space
by applying plane geometry in the planes of the space.

\section*{Perspective projection}
\addtocontents{toc}{Perspective projection.}

Consider two planes $\Pi$ and $\Pi'$ 
in the Euclidean space. 
Let $O$ be a point which does not belong neither to $\Pi$ nor~$\Pi'$.

Let us define the \index{perspective projection}\emph{perspective projection from $\Pi$ to $\Pi'$ with center at $O$}.
The projection of a point $P\in \Pi$
is defined as the intersection point $P'=\Pi'\cap (OP)$.

\begin{center}
\begin{lpic}[t(0mm),b(0mm),r(0mm),l(0mm)]{pics/perspective(1)}
\lbl{35,35;$\Pi$}
\lbl{10,40;$\Pi'$}
\lbl[bl]{65,11;$O$}
\end{lpic} 
\end{center}

Note that the perspective projection sends collinear points to collinear.
Indeed, assume three points $P$, $Q$, $R$ lie on one line $\ell$ in $\Pi$
and $P'$, $Q'$, $R'$ are their images in~$\Pi'$.
Let $\Theta$ be the plane containing $O$ and $\ell$.
Then all the points $P$, $Q$, $R$, $P'$, $Q'$, $R'$ lie on $\Theta$.
Therefore, the points $P',Q',R'$ lie on the  intersection line $\ell'=\Theta\cap \Pi'$.

The perspective projection is not a bijection between the planes.
Indeed, if the line $(OP)$ is parallel to $\Pi'$ 
(that is, if $(OP)\cap\Pi'=\emptyset$)
then the perspective projection is not defined.
Also, if $(OP')\parallel \Pi$ 
for $P'\in \Pi'$,
then the point $P'$ is not an image of the perspective projection.

Let us remind that a similar story happened with inversion.
An inversion is not defined at its center;
moreover, the center is not an inverse of any point.
To deal with this problem we passed to inversive plane 
which is Euclidean plane extended by one ideal point.

A similar strategy works for perspective projection $\Pi\to\Pi'$, but this time real projective plane is the right choice of extension.

Let $\hat \Pi$ and $\hat \Pi'$ denote the corresponding real projective planes.
Let us define a bijection between points in the real projective plane $\hat \Pi$ and the set $\Lambda$ of all the lines passing thru~$O$.
If $P\in \Pi$, then take the line $(OP)$;
if $P$ is an ideal point of $\hat \Pi$, so it is defined by a parallel pencil of lines, then take the line thru $O$ parallel to the lines in this pencil. 

The same construction gives a bijection between $\Lambda$ and $\hat \Pi'$.
Composing these two bijections $\hat \Pi\leftrightarrow \Lambda\leftrightarrow \hat \Pi'$, we get a bijection between $\hat \Pi$ and $\hat \Pi'$ which coincides with the perspective projection $P\mapsto P'$
where it is defined.

Note that the ideal line of $\hat\Pi$ maps to the intersection line of $\Pi'$ and the plane thru $O$ parallel to~$\Pi$.
Similarly the ideal line of $\hat\Pi'$
is the image of the intersection line of $\Pi$ and the plane thru $O$ parallel to~$\Pi'$.

Strictly speaking we described a transformation from one real projective plane to another, 
but if we identify the two planes, say by fixing a coordinate system in each, 
we get a projective transformation from the plane to itself. 

\begin{thm}{Exercise}\label{ex:persect}
Let $O$ be the origin of $(x,y,z)$-coordinate space
and the planes $\Pi$ and $\Pi'$ are given by the equations
$x=1$ and $y=1$ correspondingly.
The perspective projection from $\Pi$ to $\Pi'$ with center at $O$ sends $P$ to~$P'$.
Assume $P$ has coordinates $(1,y,z)$, find the coordinates of~$P'$.

For which points $P\in \Pi$ the perspective projection is undefined?
Which points $P'\in\Pi'$ are not images of points under perspective projection?
\end{thm}

\section*{Projective transformations}
\addtocontents{toc}{Projective transformations.}

A bijection from the real projective plane to itself 
which sends lines to lines 
is called \index{projective transformation}\emph{projective transformation}.

Projective geometry studies the properties of real projective plane which preserved under projective transformations.


Note that any affine transformation defines  a projective transformation on the corresponding real projective plane.
We will call such projective transformations \index{affine transformation}\emph{affine}; 
these are projective transformations which send the ideal line to itself.

The perspective projection discussed in the previous section 
gives an example of projective transformation which is not affine.

\begin{thm}{Theorem}\label{thm:moving}
Given a line $\ell$ in the real projective plane, there is a perspective projection which sends $\ell$ to the ideal line.

Moreover, any projective transformation can be obtained as a composition of an affine transformation and a perspective projection.
\end{thm}

\parit{Proof.}
Identify the projective plane with a plane $\Pi$ in the space.
Fix a point $O\notin \Pi$ and choose a plane $\Pi'$ which is
parallel to the plane containing $\ell$ and~$O$.
The corresponding perspective projection sends $\ell$ to the ideal line.

Assume $\alpha$ is a projective transformation.

If is $\alpha$ sends ideal line to itself,
then it has to be affine. 
It proves the theorem in this case.

If $\alpha$ sends the ideal line to the line $\ell$, choose a perspective projection $\beta$ which sends $\ell$ back to the ideal line.
The composition $\beta\circ\alpha$ sends ideal line to itself.
Therefore, $\beta\circ\alpha$ is affine. 
Hence the result.
\qeds

\section*{Moving points to infinity}
\addtocontents{toc}{Moving points to infinity.}

{


Theorem~\ref{thm:moving} makes possible to take any line in the projective plane and declare it to be ideal.
In other words, we can choose preferred affine plane by removing one line from the projective plane.
This construction provides a method for solving problems in projective geometry 
which will be illustrated by the following classical example.

\begin{thm}{Desargues' theorem}\label{thm:desargues}
Consider three concurrent lines $(AA')$, $(BB')$ and $(CC')$ in the real projective plane.
Set
\begin{align*}
X&=(BC)\cap (B'C'),&
Y&=(CA)\cap (C'A'),&
Z&=(AB)\cap (A'B').
\end{align*}
Then the points $X$, $Y$ and $Z$ are collinear.
\end{thm}

}

\begin{wrapfigure}{r}{44mm}
\begin{lpic}[t(-0mm),b(0mm),r(0mm),l(0mm)]{pics/desargues(0.9)}
\lbl[r]{15,32;$A$}
\lbl[lw]{35,42;$A'$}
\lbl[rw]{13,24;$B$}
\lbl[bl]{44,28.5;$B'$}
\lbl[rw]{18,19;$C$}
\lbl[ltw]{29,20;$C'$}
\lbl[rw]{22,55;$Z$}
\lbl[rtw]{22.5,18;$X$}
\lbl[r]{23,4;$Y$}
\end{lpic}
\end{wrapfigure}

\parit{Proof.}
Without loss of generality, we may assume that the line $(XY)$ is ideal.
If not, apply a perspective projection which sends the line $(XY)$ to the ideal line.

That is, we can assume that 
\[(BC)\z\parallel (B'C')\quad\text{and}\quad(CA)\z\parallel (C'A')\]
and we need to show that 
\[(AB)\z\parallel(A'B').\]

Assume that the lines $(AA')$, $(BB')$ and $(CC')$ intersect at point~$O$.
Since $(BC)\parallel (B'C')$, 
the transversal property (\ref{thm:parallel-2}) implies that $\measuredangle OBC\z= \measuredangle OB'C'$ and $\measuredangle OCB\z= \measuredangle OC'B'$.
By the AA similarity condition, $\triangle OBC\z\sim\triangle OB'C'$.
In particular,
\[\frac{OB}{OB'}=\frac{OC}{OC'}.\]

\begin{wrapfigure}{i}{48mm}
\begin{lpic}[t(0mm),b(0mm),r(0mm),l(0mm)]{pics/desargues-par(1)}
\lbl[r]{15,32;$A$}
\lbl[lw]{35,42;$A'$}
\lbl[rw]{13,24;$B$}
\lbl[rbw]{29,27;$B'$}
\lbl[rw]{18,19;$C$}
\lbl[lw]{43,18;$C'$}
\end{lpic}
\end{wrapfigure}

The same way we get that $\triangle OAC\z\sim\triangle OA'C'$ and
\[\frac{OA}{OA'}=\frac{OC}{OC'}.\]
Therefore, 
\[\frac{OA}{OA'}=\frac{OB}{OB'}.\]
By the SAS similarity condition, 
we get that $\triangle OAB\sim\triangle OA'B'$;
in particular, $\measuredangle OAB=\pm\measuredangle OA'B'$.

Note that $\measuredangle AOB=\measuredangle A'OB'$.
Therefore, 
\[\measuredangle OAB=\measuredangle OA'B'.\]
By the transversal property~\ref{thm:parallel-2},
$(AB)\parallel (A'B')$.

The case $(AA')\parallel(BB')\parallel(CC')$ is done similarly.
In this case the quadrilaterals $B'BCC'$ and $A'ACC'$ are parallelograms.
Therefore, 
\[BB'=CC'=AA'.\]
Hence $\square B'BAA'$ is a parallelogram and $(AB)\parallel (A'B')$.
\qeds



Here is another classical theorem of projective geometry.

\begin{thm}{Pappus's theorem}\label{thm:pappus}
Assume that two triples of points $A$, $B$, $C$,
and $A'$, $B'$, $C'$ are collinear.
Set 
\begin{align*}
X&=(BC')\cap(B'C),
&
Y&=(CA')\cap(C'A),
&
Z&=(AB')\cap(A'B).
\end{align*}
Then the points $X$, $Y$, $Z$ are collinear.
\end{thm}


\begin{center}
\begin{lpic}[t(0mm),b(0mm),r(0mm),l(0mm)]{pics/pappus(1)}
\lbl[lb]{3,23;$A$}
\lbl[tw]{3,2;$A'$}
\lbl[bw]{12,25;$B$}
\lbl[tw]{12,2;$B'$}
\lbl[bw]{21,28;$C$}
\lbl[tw]{23,2;$C'$}

\lbl[bw]{69,14.5;$A$}
\lbl[tw]{95,1;$A'$}
\lbl[bw]{77,21;$B$}
\lbl[tw]{78,1;$B'$}
\lbl[bw]{84,27;$C$}
\lbl[tw]{71,1;$C'$}
\lbl[lw]{18,16;$X$}
\lbl[bw]{11,16;$Y$}
\lbl[rw]{5.5,14;$Z$}
\end{lpic}
\end{center}

Pappus's theorem can be proved the same way as Desargues' theorem.

\parit{Idea of the proof.}
Applying a perspective projection, we can assume that $X$ and $Y$ lie on the ideal line.
It remains to show that $Z$ lies on the ideal line.

In other words, assuming that $(AB')\parallel (A'B)$ and $(AC')\parallel (A'C)$, we need to show that $(BC')\parallel(B'C)$.


\begin{thm}{Exercise}\label{ex:pappus}
Finish the proof of Pappus's theorem using the idea described above.
\end{thm}


\section*{Duality}
\addtocontents{toc}{Duality.}



Assume that a bijection $P\leftrightarrow p$ between the set of lines and the set of points of a plane is given.
That is,
given a point $P$, we denote by $p$ the corresponding line;
and the other way around, 
given a line $s$ we denote by $S$ the corresponding point. 

The bijection between points and lines is called \index{duality}\emph{duality}\label{page:duality}%
\footnote{Usual definition of duality is more general; we consider a special case which is also called \index{polarity}\emph{polarity}.}
if 
\[P\in s
\quad
\iff
\quad 
p\ni S.\]
for any point $P$ and line~$s$.

\begin{center}
\begin{lpic}[t(0mm),b(5mm),r(0mm),l(0mm)]{pics/dual(1)}
\lbl[tr]{2,17.5;$p$}
\lbl[tr]{11,25;$q$}
\lbl[lt]{20,25;$r$}
\lbl[tlw]{24,13;$s$}
\lbl[tr]{21,5;$A$}
\lbl[r]{13.5,19;$D$}
\lbl[t]{14,8.5;$C$}
\lbl[rw]{9,12;$B$}
\lbl[b]{19.5,13;$E$}
\lbl[tl]{6.5,6;$F$}


\lbl[bl]{53,18;$P$}
\lbl[tlw]{63.5,19;$Q$}
\lbl[tr]{62.5,3.5;$R$}
\lbl[br]{51.5,8.5;$S$}
\lbl[t]{45,18;$a$}
\lbl[lb]{48,22;$b$}
\lbl[rt]{52,1;$c$}
\lbl[l]{63.4,10;$d$}
\lbl[tr]{47,3;$e$}
\lbl[b]{45,7;$f$}
\lbl[t]{38,-2;Dual configurations.}
\end{lpic}
\end{center}

Existence of duality in a plane 
says that the lines and the points in this plane have the same rights in terms of incidence.

\begin{thm}{Exercise}\label{ex:dual-euclid}
Show that Euclidean plane does not admit a duality. 
\end{thm}

\begin{thm}{Theorem}\label{thm:dual}
The real projective plane admits a duality.
\end{thm}

\parit{Proof.}
Consider a plane $\Pi$ and a point $O\notin\Pi$ in the space;
let $\hat \Pi$ denotes the corresponding real projective plane.

Recall that there is a natural bijection $\hat \Pi\leftrightarrow\Lambda$ between $\hat \Pi$ 
and the set $\Lambda$ of all the lines passing thru~$O$.
Denote it by $P\leftrightarrow\dot P$;
that is,
\begin{itemize}
\item if $P\in \Pi$, then $\dot P=(OP)$;
\item if $P$ is an ideal point of $\hat \Pi$, 
so $P$ is defined as a parallel pencil of lines,
set $\dot P$ to be the line thru $O$ which is parallel to each lines in this pencil. 
\end{itemize}

Similarly there is a natural bijection $s\leftrightarrow\dot s$ between lines in  $\hat \Pi$ and all the planes passing thru~$O$.
If $s$ is a line in $\Pi$, 
then $\dot s$ is the plane containing $O$ and $s$;
if $s$ is the ideal line of $\hat \Pi$,
take $\dot s$ is the plane thru $O$ parallel to~$\Pi$. 

It is straightforward to check that $\dot P\subset\dot s$
if and only if $P\in s$;
that is, the bijections $P\leftrightarrow \dot P$ and $s\leftrightarrow \dot s$
remember all the incidence structure of the real projective plane~$\hat \Pi$.

It remains to construct a bijection $\dot s \leftrightarrow \dot S$
between the set of planes and 
the set of lines passing thru $O$ 
such that 
\[\dot r\subset \dot S
\quad
\iff
\quad
\dot R\supset \dot s
\eqlbl{iff-dual}\]
for any two lines line $\dot r$ and $\dot s$ passing thru~$O$.

Set $\dot S$ to be the plane thru $O$ 
which is perpendicular to~$\dot s$.
Note that both conditions \ref{iff-dual} are equivalent to $\dot r\perp \dot s$;
hence the result follows.
\qeds

\begin{thm}{Exercise}\label{ex:dula-coordinates}
Consider the Euclidean plane with $(x,y)$-coordinates; let $O$ denotes the origin.
Given a point $P\ne O$ with coordinates $(a,b)$ consider the line $p$ 
given by the equation 
$a\cdot x+b\cdot y=1$.

Show that the correspondence $P$ to $p$ extends to a duality of the real projective plane.

Which line corresponds to $O$?

Which point of the real projective plane corresponds to the line  $a\cdot x\z+b\cdot y=0$?
\end{thm}

The existence of duality in the real projective planes makes possible to formulate an equivalent dual statement to any statement in projective geometry.
For example, the dual statement for ``the points $X$, $Y$ and $Z$ lie on one line $\ell$''
would be the ``lines $x$, $y$ and $z$ intersect at one point $L$''.
Let us formulate the dual statement for Desargues' theorem~\ref{thm:desargues}.


\begin{thm}{Dual Desargues' theorem}\label{thm:dual-desargues}
Consider the collinear points $X$, $Y$ and~$Z$.
Assume that 
\[X=(BC)\cap (B'C'),\quad Y=(CA)\cap (C'A'),\quad Z=(AB)\cap (A'B').\]
Then the lines  $(AA')$, $(BB')$ and $(CC')$ are concurrent.
\end{thm}

In this theorem the points $X$, $Y$ and $Z$ 
are dual to the lines $(AA')$, $(BB')$ and $(CC')$ in the original formulation, and the other way around.

Once Desargues' theorem is proved, applying duality (\ref{thm:dual})
we get the dual Desargues' theorem.
Note that the dual Desargues' theorem is the converse to the original Desargues' theorem~\ref{thm:desargues}.

\begin{thm}{Exercise}\label{ex:dual-pappus}
Formulate the dual Pappus's theorem (see \ref{thm:pappus}).
\end{thm}

\begin{thm}{Exercise}\label{ex:dual-desargues-construction}
\begin{enumerate}[(a)]
\item\label{ex:dual-desargues-construction:a}
Given two parallel lines, construct with a ruler only a third parallel line thru a given point.
\item \label{ex:dual-desargues-construction:b}
Given a parallelogram, construct with a ruler only a line parallel to a given line thru a given point.
\end{enumerate}

\end{thm}

\section*{Axioms}
\addtocontents{toc}{Axioms.}

Note that the real projective plane described above satisfies the following set of axioms.

\begin{framed}

\begin{enumerate}[p-I.]
\item\label{def:proj-axioms:1} Any two distinct points lie on a unique line.
\item\label{def:proj-axioms:2} Any two distinct lines pass thru a unique point.
\item\label{def:proj-axioms:3} There exist at least four points of which no three are collinear.
\end{enumerate}

\end{framed}

Let us take these three axioms as a definition of the \index{projective plane}\emph{projective plane};
so the real projective plane discussed above becomes a particular example of projective plane.

\begin{wrapfigure}{o}{20mm}
\begin{lpic}[t(-0mm),b(0mm),r(0mm),l(0mm)]{pics/fano(1)}
\end{lpic}
\end{wrapfigure}

There is an example of projective plane which contains exactly 3 points on each line.
This is the so called \index{Fano plane}\emph{Fano plane} which you can see on the diagram;
it contains $7$ points and $7$ lines.
This is an example of \index{finite projective plane}\emph{finite projective plane};
that is, projective plane with finitely many points.

\begin{thm}{Exercise}\label{ex:finite-pp}
Show that any line in projective plane contains at least three points.
\end{thm}

Consider the following analog of Axiom~p-\ref{def:proj-axioms:3}.

\begin{framed}
\noindent {\rm p-III$'\!$.} 
There exist at least four lines of which no three are concurrent.
\end{framed}

\begin{thm}{Exercise}\label{ex:3=3'}
Show that Axiom \emph{p-\ref{def:proj-axioms:3}$'$} is equivalent to Axiom \emph{p-\ref{def:proj-axioms:3}}.
That is, 
\begin{center}
\emph{p-\ref{def:proj-axioms:1}, p-\ref{def:proj-axioms:2} and p-\ref{def:proj-axioms:3} imply  p-\ref{def:proj-axioms:3}$'$},
\end{center}
and 
\begin{center}
\emph{p-\ref{def:proj-axioms:1}, p-\ref{def:proj-axioms:2} and p-\ref{def:proj-axioms:3}$'$ imply p-\ref{def:proj-axioms:3}}.
\end{center}

\end{thm}

The exercise above shows that in the axiomatic system of projective plane,
lines and points have the same rights.
In fact, one can switch everywhere words ``point'' with ``line'', ``pass thru'' with ``lies on'', ``collinear'' with ``concurrent'' and we get an equivalent set of axioms ---  Axioms p-\ref{def:proj-axioms:1} and p-\ref{def:proj-axioms:2} convert into each other,
and the same happens wit the pair p-\ref{def:proj-axioms:3} and p-\ref{def:proj-axioms:3}$'$.

\begin{thm}{Exercise}\label{ex:oder}
Assume that one of the lines in a finite projective plane contains exactly $n+1$ points.
\begin{enumerate}[(a)]
\item\label{ex:oder:a} Show that each line contains exactly $n+1$ points.
\item\label{ex:oder:b} Show that the number of the points in the plane has to be 
\[n^2+n+1.\]
\item\label{ex:oder:c} Show that there is no projective plane with exactly 10 points.
\item\label{ex:oder:d} Show that in any finite projective plane the number of points coincides with the number of lines.
\end{enumerate}
\end{thm}

The number $n$ in the above exercise is called \index{order of finite projective plane}\emph{order} of finite projective plane.
For example Fano plane has order $2$.
Here is one of the most famous open problem in finite geometry.

\begin{thm}{Conjecture}
The order of any finite projective plane is a power of a prime number.
\end{thm}

\addtocontents{toc}{\protect\end{quote}}
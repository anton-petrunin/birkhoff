\chapter{Projective geometry}\label{chap:proj}

\section{Projective completion}

In the Euclidean plane, two distinct lines might have one or zero points of intersection 
(in the latter case the lines are parallel).
We aim to extend the Euclidean plane by ideal points so that any two distinct lines will have exactly one point of intersection.

\begin{wrapfigure}{o}{31mm}
\centering
\includegraphics{mppics/pic-226}
\vskip4mm
\includegraphics{mppics/pic-228}
\end{wrapfigure}

A collection of lines in the Euclidean plane is called \index{concurrent}\emph{concurrent} if they all intersect at a single point or all of them are pairwise parallel.
A maximal set of concurrent lines in the plane is called a \index{pencil}\emph{pencil}.
There are two types of pencils: 
\emph{central pencils} contain all lines passing thru a fixed point called the \index{center!center of the pencil}\emph{center of the pencil}
and  
\emph{parallel pencils} contain pairwise parallel lines.

Note that any two lines completely determine the pencil containing both.

Each point in the Euclidean plane uniquely defines a central pencil with the center in it.
Let us add one \index{point!ideal point}\index{ideal!point}\emph{ideal point} for each parallel pencil,
and assume that all these ideal points lie on one \index{line!ideal line}\index{ideal!line}\emph{ideal line}.
We also assume that the ideal line belongs to each parallel pencil.

We obtain the so-called \index{projective!plane}\index{real!projective plane}\emph{real projective plane} (or \index{projective!completion}\emph{projective completion} of the original plane). 
It comes with an incidence structure --- we say that three points lie on one line if the corresponding pencils contain a common line.
Projective geometry studies this incidence structure.

A parallel pencil contains the ideal line and the lines $y=m\cdot x+b$ with fixed slope $m$;
if $m=\infty$, we assume that the lines are given by equations $x=a$.
Therefore the projective completion contains all the points in the $(x,y)$-plane plus the ideal line containing one ideal point $P_m$ for every slope $m\in\mathbb{R}\cup\{\infty\}$.

\section{Euclidean space}

Let us repeat the construction of metric $d_2$ (Exercise~\ref{ex:dist-square}) in space.

Suppose that $\mathbb{R}^3$ denotes the set of all triples $(x,y,z)$ of real numbers.
Assume $A=(x_A,y_A,z_A)$ and $B=(x_B,y_B,z_B)$ are arbitrary points in $\mathbb{R}^3$.
Define the metric on $\mathbb{R}^3$ the following way:
$$AB
\df
\sqrt{(x_A-x_B)^2+(y_A-y_B)^2+(z_A-z_B)^2}.$$
The obtained metric space is called \index{Euclidean!space}\emph{Euclidean space}.

The subset of points in $\mathbb{R}^3$ is called \index{plane!plane in the space}\emph{plane} if it can be
described by an equation
$$a\cdot x+b\cdot y+c\cdot z+d=0$$ 
for constants $a$, $b$, $c$, and $d$ such that at least one of the values $a$, $b$ or $c$ is distinct from zero.

It is straightforward to show the following:
\begin{itemize}
 \item Any plane in the Euclidean space is isometric to the Euclidean plane.
 \item Any three points in the space lie on a plane.
 \item An intersection of two distinct planes (if it is nonempty) is a line in each of these planes.
\end{itemize}

These statements make it possible to generalize many notions and results from Euclidean plane geometry to the Euclidean space
by applying plane geometry in the planes of the space.

\section{Model of space}

Let us identify the Euclidean plane with a plane $\Pi$ in the Euclidean space $\mathbb{R}^3$ that does not pass thru the origin $O$.
Denote by $\hat\Pi$ the projective completion of $\Pi$.

Denote by $\Phi$ the set of all lines in the space thru $O$.
Let us define a bijection $P\leftrightarrow \dot P$ between $\hat \Pi$ and $\Phi$.
If $P\in \Pi$, then take the line $\dot P=(OP)$;
if $P$ is an ideal point of $\hat \Pi$, so it is defined by a parallel pencil of lines, then take the line $\dot P$ thru $O$ parallel to the lines in this pencil.

Further, denote by $\Psi$ the set of all planes in the space thru $O$.
In a similar fashion, we can define a bijection $\ell\leftrightarrow \dot \ell$ between lines in $\hat \Pi$ and $\Psi$.
If a line $\ell$ is not ideal, then take the plane  $\dot \ell$ that contains $\ell$ and $O$;
if the line $\ell$ is ideal, then take $\dot \ell$ to be the plane thru $O$ that is parallel to $\Pi$ (that is, $\dot\ell\cap\Pi=\emptyset$).

\begin{thm}{Observation}\label{obs:bijections}
Let $P$ and $\ell$ be a point and a line in the real projective plane.
Then $P\in \ell$ if and only if $\dot P\subset \dot \ell$,
where $\dot P$ and $\dot \ell$ denote the line and plane defined by the constructed bijections.
\end{thm}

\section{Perspective projection}
\label{sec:perspective-projection}

Consider two planes $\Pi$ and $\Pi'$ 
in the Euclidean space. 
Let $O$ be a point that belongs neither to $\Pi$ nor~$\Pi'$.

\begin{wrapfigure}{o}{43mm}
\centering
\includegraphics{mppics/pic-230}
\end{wrapfigure}

A \index{perspective projection}\emph{perspective projection from $\Pi$ to $\Pi'$ with center $O$} maps a point $P\in \Pi$
to the intersection point $P'=\Pi'\cap (OP)$.

In general, perspective projection is not a bijection between the planes.
Indeed, if the line $(OP)$ is parallel to $\Pi'$ 
(that is, if $(OP)\cap\Pi'=\emptyset$)
then the perspective projection of $P\in \Pi$ is undefined.
Also, if $(OP')\parallel \Pi$ 
for $P'\in \Pi'$,
then the point $P'$ is not an image of the perspective projection.

Denote by $\hat \Pi$ and $\hat \Pi'$ the projective completions of $\Pi$ and $\Pi'$ respectively. 
Note that the perspective projection is a restriction of the composition of two bijections $\hat \Pi\leftrightarrow\Phi \leftrightarrow\hat \Pi'$ constructed in the previous section.
By Observation~\ref{obs:bijections}, the perspective projection can be extended to a bijection $\hat \Pi\leftrightarrow\hat \Pi'$ that sends lines to lines.%
\footnote{A similar story happened with inversion.
An inversion is not defined at its center;
moreover, the center is not an inverse of any point.
To deal with this problem we passed to the inversive plane 
which is the Euclidean plane extended by one ideal point.
The same strategy worked for perspective projection $\Pi\to\Pi'$, but this time we need to add an ideal line.}

For example, suppose $O$ is the origin of $(x,y,z)$-coordinate space,
and the planes $\Pi$ and $\Pi'$ are given by the equations
$z=1$ and $x=1$ respectively.
Then the perspective projection from $\Pi$ to $\Pi'$
can be written in the coordinates as
\[(x,y,1)\mapsto (1,\tfrac yx,\tfrac 1x).\]
Indeed the coordinates have to be proportional;
points on $\Pi$ have unit $z$-coordinate, 
and points on $\Pi'$ have unit $z$-coordinate.

The perspective projection maps one plane to another.
However, we can identify the two planes by fixing a coordinate system in each.
In this case, we get a partially defined map from the plane to itself.
We will keep the name {}\emph{perspective transformation} for such maps.

For the described perspective projection; we may get the map 
\[\beta\:(x,y)\mapsto (\tfrac 1x,\tfrac yx).
\eqlbl{eq:(x,y)-perspective}\]
This map is undefined on the line $x=0$.
Also, points on this line are not images of points under perspective projection.

For example, to define an extension of the perspective projection $\beta$ in \ref{eq:(x,y)-perspective},
we have to observe that 
\begin{itemize}
\item The pencil of vertical lines $x=a$ is mapped to itself.
\item The ideal points defined by pencils of lines $y=m\cdot x+ b$ are mapped to the point $(0,m)$, and the other way around --- point $(0,m)$ is mapped to the ideal point defined by the  pencil of lines $y=m\cdot x+ b$.
\end{itemize}

\section{Projective transformations}

A bijection from the real projective plane to itself 
that sends lines to lines 
is called \index{projective!transformation}\emph{projective transformation}.

Note that any affine transformation defines  a projective transformation on the corresponding real projective plane.
We will call such projective transformations \index{affine transformation}\emph{affine}; 
these are projective transformations that send the ideal line to itself.

The extended perspective projection discussed in the previous section 
provides another source of examples of projective transformations.

\begin{thm}{Theorem}\label{thm:moving}
Given a line $\ell$ in the real projective plane, there is a perspective projection that sends $\ell$ to the ideal line.

Moreover, a perspective transformation is either affine or, in a suitable coordinate system, it can be written as a composition of the extension of perspective projection 
\[\beta\:(x,y)\mapsto (\tfrac xy,\tfrac 1y)\]
and an affine transformation.
\end{thm}

\parit{Proof.}
We may choose an $(x,y)$-coordinate system such that the line $\ell$ is defined by equation $y=0$.
Then the extension of $\beta$ gives the needed transformation.

Fix a projective transformation $\gamma$.
If $\gamma$ sends the ideal line to itself,
then it has to be affine. 
It proves the theorem in this case.

Suppose $\gamma$ sends the ideal line to a line $\ell$.
Choose a perspective projection $\beta$ as above.
The composition $\beta\circ\gamma$ sends the ideal line to itself.
That is, $\alpha=\beta\circ\gamma$ is affine.
Note that $\beta$ is self-inverse; therefore 
\[\gamma=\beta\circ\beta\circ\gamma=\beta\circ\alpha\]
--- hence the result.
\qeds

\begin{thm}{Exercise}\label{ex:proj-cross-ratio}
Let $P\mapsto P'$ be (a) an affine transformation, (b) the perspective projection defined by $(x,y)\mapsto (\tfrac xy,\tfrac 1y)$, or (c) an arbitrary projective transformation.
Suppose $P_1,P_2,P_3,P_4$ lie on one line.
Show that 
\[\frac{P_1P_2\cdot P_3P_4}{P_2P_3\cdot P_4P_1}=\frac{P'_1P'_2\cdot P'_3P'_4}{P'_2P'_3\cdot P'_4P'_1};\]
that is, each of these maps preserves cross-ratio for quadruples of points on one line.

\end{thm}



\section{Moving points to infinity}

{

\begin{wrapfigure}{r}{37mm}
\vskip0mm
\centering
\includegraphics{mppics/pic-234}
\end{wrapfigure}

Theorem~\ref{thm:moving} makes it possible to take any line in the projective plane and declare it to be ideal.
In other words, we can choose a preferred affine plane by removing one line from the projective plane.
This construction provides a method for solving problems in projective geometry 
which will be illustrated by the following classical example:


\begin{thm}{Desargues' theorem}\label{thm:desargues}\index{Desargues' theorem}
Consider three concurrent lines $(AA')$, $(BB')$, and $(CC')$ in the real projective plane.
Set
\begin{align*}
X&=(BC)\cap (B'C'),\\
Y&=(CA)\cap (C'A'),\\
Z&=(AB)\cap (A'B').
\end{align*}
Then the points $X$, $Y$, and $Z$ are collinear.
\end{thm}

}

\parit{Proof.}
Without loss of generality, we may assume that the line $(XY)$ is ideal.
If not, apply a perspective projection that sends the line $(XY)$ to the ideal line.

\begin{wrapfigure}{o}{40mm}
\vskip-8mm
\centering
\includegraphics{mppics/pic-236}
\end{wrapfigure}

That is, we can assume that 
\[(BC)\z\parallel (B'C')\quad\text{and}\quad(CA)\z\parallel (C'A'),\]
and we need to show that 
\[(AB)\z\parallel(A'B').\]

Assume that the lines $(AA')$, $(BB')$, and $(CC')$ intersect at point~$O$.
Since $(BC)\z\parallel (B'C')$, 
the transversal property (\ref{thm:parallel-2}) implies that $\measuredangle OBC\z= \measuredangle OB'C'$ and $\measuredangle OCB\z= \measuredangle OC'B'$.
By the AA similarity condition, $\triangle OBC\z\sim\triangle OB'C'$.
In particular,
\[\frac{OB}{OB'}=\frac{OC}{OC'}.\]

In the same way, we get that $\triangle OAC\z\sim\triangle OA'C'$ and
\[\frac{OA}{OA'}=\frac{OC}{OC'}.\]
Therefore, 
\[\frac{OA}{OA'}=\frac{OB}{OB'}.\]
By the SAS similarity condition, 
we get that $\triangle OAB\sim\triangle OA'B'$;
in particular, $\measuredangle OAB=\pm\measuredangle OA'B'$.

Note that $\measuredangle AOB=\measuredangle A'OB'$.
Therefore, 
\[\measuredangle OAB=\measuredangle OA'B'.\]
By the transversal property (\ref{thm:parallel-2}), we have
$(AB)\parallel (A'B')$.

The case $(AA')\parallel(BB')\parallel(CC')$ is done similarly.
In this case, the quadrangles $B'BCC'$ and $A'ACC'$ are parallelograms.
Therefore, 
\[BB'=CC'=AA'.\]
Hence $\square B'BAA'$ is a parallelogram and $(AB)\parallel (A'B')$.
\qeds

Here is another classical theorem of projective geometry.

\begin{thm}{Pappus' theorem}\label{thm:pappus}\index{Pappus' theorem}
Assume that two triples of points $A$, $B$, $C$,
and $A'$, $B'$, $C'$ are collinear.
Suppose that points $X$, $Y$, $Z$ are uniquely defined by
\begin{align*}
X&=(BC')\cap(B'C),
&
Y&=(CA')\cap(C'A),
&
Z&=(AB')\cap(A'B).
\end{align*}
Then the points $X$, $Y$, $Z$ are collinear.
\end{thm}

Pappus' theorem can be proved the same way as Desargues' theorem.

\parit{Idea of the proof.}
Applying a perspective projection, we can assume that $Y$ and $Z$ lie on the ideal line.
It remains to show that $X$ lies on the ideal line.

In other words, assuming that $(AB')\parallel (A'B)$ and $(AC')\parallel (A'C)$, we need to show that $(BC')\parallel(B'C)$.

\begin{figure}[!ht]
\centering
\includegraphics{mppics/pic-238}
\hskip15mm
\includegraphics{mppics/pic-240}
\end{figure}


\begin{thm}{Exercise}\label{ex:pappus}
Finish the proof of Pappus' theorem using the idea described above.
\end{thm}

The following exercise gives a partial converse to Pappus' theorem.

\begin{thm}{Exercise}\label{ex:pappus-converse}
Given two triples of points $A$, $B$, $C$,
and $A'$, $B'$, $C'$,
suppose distinct points $X$, $Y$, $Z$ are uniquely defined by
\begin{align*}
X&=(BC')\cap(B'C),
&
Y&=(CA')\cap(C'A),
&
Z&=(AB')\cap(A'B).
\end{align*}
Assume that the triples $A$, $B$, $C$,
and $X$, $Y$, $Z$ are collinear.
Show that the triple $A'$, $B'$, $C'$ is collinear.
\end{thm}

\begin{thm}{Exercise}\label{ex:desargues-construction}
Solve the following construction problem
\begin{enumerate}[(a)]
\item\label{ex:desargues-construction:desargues} using Desargues' theorem;
\item\label{ex:desargues-construction:pappus} using Pappus' theorem.
\end{enumerate}
\parbf{Problem.}
Suppose a parallelogram and a line $\ell$ are given.
Assume the line $\ell$ crosses all sides (or their extensions) of the parallelogram at different points. 
Construct another line parallel to $\ell$ with a ruler only.
\end{thm}


\section{Duality}

Assume that a bijection $P\leftrightarrow p$ between the set of lines and the set of points of a plane is given.
\begin{figure}[!ht]
\centering
\includegraphics{mppics/pic-242}
\hskip15mm
\includegraphics{mppics/pic-244}
\caption*{Dual configurations.}
\end{figure}
That is,
given a point $P$, we denote by $p$ the corresponding line;
and the other way around, 
given a line $\ell$ we denote by $L$ the corresponding point. 

The bijection between points and lines is called \index{duality}\emph{duality}\label{page:duality}%
\footnote{The standard definition of duality is more general; we consider a special case which is also called \index{polarity}\emph{polarity}.}
if 
\[P\in \ell
\quad
\iff
\quad 
p\ni L.\]
for any point $P$ and line~$\ell$.

{

\begin{wrapfigure}{r}{40mm}
\vskip-4mm
\centering
\includegraphics{mppics/pic-245}
\end{wrapfigure}

\begin{thm}{Exercise}\label{ex:dual-configurations}
Consider the configuration of lines and points on the diagram.

Start with a generic quadrangle $KLMN$ and extend it to a dual diagram; label the lines and points using the convention described above.
\end{thm}

\begin{thm}{Exercise}\label{ex:dual-euclid}
Show that the Euclidean plane does not admit a duality. 
\end{thm}

}

\begin{thm}{Theorem}\label{thm:dual}
The real projective plane admits a duality.
\end{thm}



\parit{Proof.}
Consider a plane $\Pi$ and a point $O\notin\Pi$ in the space;
suppose that $\hat \Pi$ denotes the corresponding real projective plane.

Recall that $\Phi$ and $\Psi$ denote the set of all lines and planes passing thru $O$.
According to Observation~\ref{obs:bijections}, there are bijections $P\leftrightarrow\dot P$  between points of $\hat\Pi$ and $\Phi$ and $\ell\leftrightarrow\dot\ell$ between lines in $\hat\Pi$ and $\Psi$ such that 
$P\in\ell$ if and only if $\dot P\subset \dot \ell$.

It remains to construct a bijection $\dot \ell \leftrightarrow \dot L$
between $\Phi$ and $\Psi$ 
such that 
\[\dot P\subset \dot \ell
\quad
\iff
\quad
\dot p\supset \dot L
\eqlbl{iff-dual}\]
for any two lines $\dot P$ and $\dot L$ passing thru~$O$.

Set $\dot \ell$ to be the plane thru $O$ 
that is perpendicular to~$\dot L$.
Note that both conditions \ref{iff-dual} are equivalent to $\dot P\perp \dot L$;
hence the result follows.
\qeds

\begin{thm}{Exercise}\label{ex:dula-coordinates}
Consider the Euclidean plane with $(x,y)$-coordinates; suppose that $O$ denotes the origin.
Given a point $P\ne O$ with coordinates $(a,b)$ consider the line $p$ 
given by the equation 
$a\cdot x+b\cdot y=1$.

Show that the correspondence $P$ to $p$ extends to a duality of the real projective plane.

Which line corresponds to $O$?

Which point corresponds to the line  $a\cdot x\z+b\cdot y=0$?
\end{thm}

Duality says that lines and points have the same rights in terms of incidence.
It makes it possible to formulate an equivalent dual statement to any statement in projective geometry.
For example, the dual statement for ``the points $X$, $Y$, and $Z$ lie on one line $\ell$''
would be the ``lines $x$, $y$, and $z$ intersect at one point $L$''.
Let us formulate the dual statement for Desargues' theorem~\ref{thm:desargues}.


\begin{thm}{Dual Desargues' theorem}\label{thm:dual-desargues}\index{Dual Desargues' theorem}
Consider the collinear points $X$, $Y$, and~$Z$.
Assume that 
\[X=(BC)\cap (B'C'),\quad Y=(CA)\cap (C'A'),\quad Z=(AB)\cap (A'B').\]
Then the lines  $(AA')$, $(BB')$, and $(CC')$ are concurrent.
\end{thm}

In this theorem, the points $X$, $Y$, and $Z$ 
are dual to the lines $(AA')$, $(BB')$, and $(CC')$ in the original formulation, and the other way around.

Once Desargues' theorem is proved, applying duality (\ref{thm:dual})
we get the dual Desargues' theorem.
Note that the dual Desargues' theorem is the converse to the original Desargues' theorem~\ref{thm:desargues}.

\begin{thm}{Exercise}\label{ex:dual-pappus}
Formulate the dual Pappus' theorem (see \ref{thm:pappus}).
\end{thm}

\begin{thm}{Exercise}\label{ex:dual-desargues-construction} 
Solve the following construction problem
\begin{enumerate}[(a)]
\item\label{ex:dual-desargues-construction:desargues} using dual Desargues' theorem;
\item\label{ex:dual-desargues-construction:pappus} using Pappus' theorem or its dual.
\end{enumerate}
\parbf{Problem.}
Given two parallel lines, construct a third parallel line thru a given point with a ruler only.
\end{thm}

\section{Construction of a polar}

In this section, we describe a powerful trick that can be used in the constructions with a ruler.

Assume $\Gamma$ is a circle in the plane and $P\notin \Gamma$.
\begin{figure}[!ht]
\centering
\includegraphics{mppics/pic-290}
\end{figure}
Draw two lines $x$ and $y$ thru $P$ that intersect $\Gamma$ at two pairs of points $X$, $X'$ and $Y$, $Y'$.
Let $Z=(XY)\cap(X'Y')$ and $Z'=(XY')\cap(X'Y)$.
Consider the line $p=(ZZ')$.

\begin{thm}{Claim}\label{clm:polar}
The constructed line $p=(ZZ')$ does not depend on the choice of the lines $x$ and $y$.

Moreover, $P\leftrightarrow p$ can be extended to a duality such that any point $P$ on the circle $\Gamma$ corresponds to a line $p$ tangent to $\Gamma$ at~$P$. 
\end{thm}

We will not prove this claim, but the proof is not hard.
If $P$ lies outside of $\Gamma$, it can be done by moving $P$ to infinity keeping $\Gamma$ fixed as a set.
If $P$ lies inside of $\Gamma$, it can be done by moving $P$ to the center of $\Gamma$.
The existence of corresponding projective transformations follows from the idea in Exercise~\ref{ex:cone}.

The line $p$ is called the \index{polar}\emph{polar} of the point $P$ with respect to~$\Gamma$.

The point $P$ is called the \index{pole}\emph{pole} of the line $p$ with respect to~$\Gamma$.

\begin{thm}{Exercise}\label{ex:revert}
Revert the described construction.
That is, given a circle $\Gamma$ and a line $p$ that is not tangent to $\Gamma$, construct a point $P$ such that the described construction for $P$ and $\Gamma$ produces the line $p$.
\end{thm}

\begin{thm}{Exercise}\label{ex:tangent ruler}
Let $p$ be the polar line of point $P$ with respect to the circle~$\Gamma$.
Assume that $p$ intersects $\Gamma$ at points $V$ and~$W$.
Show that the lines $(PV)$ and $(PW)$ are tangent to~$\Gamma$.

Come up with a ruler-only construction of the tangent lines to the given circle $\Gamma$ thru the given point $P\notin\Gamma$.
\end{thm}

\begin{thm}{Exercise}\label{ex:concentric-circ}
Assume two concentric circles $\Gamma$ and $\Gamma'$ are given.
Construct the common center of $\Gamma$ and $\Gamma'$ with a ruler only.
\end{thm}

\begin{thm}{Exercise}\label{ex:proj-perp}
Assume a line $\ell$ and a circle $\Gamma$ with its center $O$ are given.
Suppose $O\notin \ell$.
Construct a perpendicular from $O$ on $\ell$ with a ruler only.
\end{thm}

\section{Axioms}

Note that the real projective plane described above satisfies the following set of axioms:

\begin{framed}

\begin{enumerate}[p-I.]
\item\label{def:proj-axioms:1} Any two distinct points lie on a unique line.
\item\label{def:proj-axioms:2} Any two distinct lines pass thru a unique point.
\item\label{def:proj-axioms:3} There exist at least four points of which no three are collinear.
\end{enumerate}

\end{framed}

Let us take these three axioms as a definition of the \index{projective!plane}\emph{projective plane};
so the real projective plane discussed above becomes its particular example.

{

\begin{wrapfigure}{r}{26mm}
\vskip-0mm
\centering
\includegraphics{mppics/pic-246}
\end{wrapfigure}

There is an example of a projective plane that contains exactly 3 points on each line.
This is the so-called \index{Fano plane}\emph{Fano plane} which you can see on the diagram;
it contains $7$ points and $7$ lines.
This is an example of a \index{finite projective plane}\emph{finite projective plane};
that is, a projective plane with finitely many points.

}

\begin{thm}{Exercise}\label{ex:finite-pp}
Show that any line in a projective plane contains at least three points.
\end{thm}


Consider the following dual analog of Axiom~p-\ref{def:proj-axioms:3}:

\begin{framed}
\noindent {\rm p-III$'\!$.} 
There exist at least four lines of which no three are concurrent.
\end{framed}

\begin{thm}{Exercise}\label{ex:3=3'}
Show that Axiom \textit{p-\ref{def:proj-axioms:3}$'$} is equivalent to Axiom \textit{p-\ref{def:proj-axioms:3}}.
That is, 
\begin{center}
\textit{p-\ref{def:proj-axioms:1}, p-\ref{def:proj-axioms:2}, and p-\ref{def:proj-axioms:3} imply  p-\ref{def:proj-axioms:3}$'$},
\end{center}
and 
\begin{center}
\textit{p-\ref{def:proj-axioms:1}, p-\ref{def:proj-axioms:2}, and p-\ref{def:proj-axioms:3}$'$ imply p-\ref{def:proj-axioms:3}}.
\end{center}

\end{thm}

The exercise above shows that in the given axiomatic system,
lines and points have the same rights.
One can switch everywhere words ``point'' with ``line'', ``pass thru'' with ``lies on'', ``collinear'' with ``concurrent'' and we get an equivalent set of axioms ---  Axioms p-\ref{def:proj-axioms:1} and p-\ref{def:proj-axioms:2} convert into each other,
and the same happens with the pair p-\ref{def:proj-axioms:3} and p-\ref{def:proj-axioms:3}$'$.

\begin{thm}{Exercise}\label{ex:oder}
Assume that one of the lines in a finite projective plane contains exactly $n+1$ points.
\begin{enumerate}[(a)]
\item\label{ex:oder:a} Show that each line contains exactly $n+1$ points.
\item\label{ex:oder:b} Show that the plane contains  exactly $n^2+n+1$ points.
\item\label{ex:oder:c} Show that there is no projective plane with exactly 10 points.
\item\label{ex:oder:d} Show that in any finite projective plane, the number of points coincides with the number of lines.
\end{enumerate}
\end{thm}

The number $n$ in the above exercise is called \index{order of finite projective plane}\emph{order} of finite projective plane.
For example, the Fano plane has order $2$.
Let us finish by stating a famous open problem in finite geometry.

\begin{thm}{Conjecture}
The order of any finite projective plane is a power of a prime number.
\end{thm}

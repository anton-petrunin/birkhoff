\chapter{Hyperbolic plane}\label{chap:poincare}
\addtocontents{toc}{\protect\begin{quote}}

\begin{center}
\begin{lpic}[t(-4mm),b(0mm),r(0mm),l(0mm)]{pics/H2checkers_334(0.25)}
\end{lpic}          
\end{center}


In this chapter, we use inversive geometry 
to construct the model of hyperbolic plane --- a neutral plane which is not Euclidean.

Namely, we construct the so called \index{conformal disk model}\emph{conformal disk model} of the hyperbolic plane.
This model was discovered by Beltrami in  \cite{beltrami} 
and often called the {}\emph{Poincar\'e disc model}. 

The figure above shows the conformal disk model of the hyperbolic plane which is cut into congruent triangles with angles $\tfrac\pi3$, $\tfrac\pi3$ and~$\tfrac\pi4$.

\section*{Conformal disk model}
\addtocontents{toc}{Conformal disk model.}

In this section, we give new names for some objects in the Euclidean plane
which will represent lines, angle measures and distances in the  hyperbolic plane.

\parbf{Hyperbolic plane.}
Let us fix a circle on the Euclidean plane 
and call it \index{absolute}\emph{absolute}.
The set of points inside the absolute will be called the \index{hyperbolic plane}\index{plane!hyperbolic plane}\emph{hyperbolic plane} (or \index{plane!h-plane}\index{h-plane}\emph{h-plane}).

Note that the points on the absolute do {}\emph{not} belong to the h-plane.
The points in the h-plane will be also called \index{h-point}\emph{h-points}.

Often we will assume that the absolute is a unit circle.



\parbf{Hyperbolic lines.}
The intersections of the h-plane with circlines perpendicular to the absolute are called {}\emph{hyperbolic lines} or \index{h-line}\emph{h-lines}.

\begin{wrapfigure}{o}{59mm}
\begin{lpic}[t(-4mm),b(0mm),r(0mm),l(-2mm)]{pics/APQB(1)}
\lbl[rt]{34,16;$P$}
\lbl[r]{32,29;$Q$}
\lbl[t]{39,10;$A$}
\lbl[b]{38.5,36.5;$B$}
\lbl[rb]{55,13;$\Gamma$}
\lbl[b]{20,15;h-plane}
\end{lpic}
\end{wrapfigure}

By Corollary~\ref{cor:h-line}, there is a unique h-line which passes thru the given two distinct h-points $P$ and~$Q$.
This h-line will be denoted by~\index{62@$(PQ)_h$, $[PQ)_h$,$[PQ]_h$}$(PQ)_h$.

The arcs of hyperbolic lines will be called {}\emph{hyperbolic segments} or \index{h-segment}\emph{h-segments}.
An h-segment with endpoints $P$ and $Q$ will be denoted by~$[PQ]_h$.

The subset of an h-line on one side from a point will be called a {}\emph{hyperbolic half-line} (or \index{h-half-line}\emph{h-half-line}).
More precisely, an h-half-line is an intersection of the h-plane with arc which perpendicular to the absolute with only one endpoint in the h-plane.
An h-half-line starting at $P$ and passing thru $Q$ will be denoted by~$[PQ)_h$.

If $\Gamma$ is the circle containing the h-line $(PQ)_h$, then the points of intersection of $\Gamma$ with the absolute are called 
\index{point!ideal point}\index{ideal point}\emph{ideal points} of~$(PQ)_h$.
(Note that the ideal points of an h-line do not belong to the h-line.)

An ordered triple of h-points, say $(P,Q,R)$ will be called \emph{h-triangle $PQR$} and denoted by \index{21@$\triangle_h$}$\triangle_h P Q R$.

So far, $(PQ)_h$ is just a subset of the h-plane;
below we will introduce h-distance 
and later we will show that  $(PQ)_h$ is a line for the h-distance in the sense of the Definition~\ref{def:line}. 

\begin{thm}{Exercise}\label{ex:ideal-line-unique}
Show that an h-line is uniquely determined by its ideal points.
\end{thm}

\begin{thm}{Exercise}\label{ex:1ideal-line-unique}
Show that an h-line is uniquely determined by one of its ideal points and one h-point on it.
\end{thm}

\begin{thm}{Exercise}\label{ex:line/h-line}
Show that the h-segment $[PQ]_h$ coincides with the Euclidean segment $[PQ]$
if and only if the line $(PQ)$ pass thru the center of the absolute.
\end{thm}





\parbf{Hyperbolic distance.}\label{h-dist}
Let $P$ and $Q$ be distinct h-points.
Denote by $A$ and $B$ be the ideal points of $(PQ)_h$.
Without loss of generality, we may assume that on the Euclidean circle containing the h-line $(PQ)_h$, the points $A,P,Q,B$ appear in the same order.

Consider the function 
$$\delta(P,Q)\df\frac{AQ\cdot BP}{QB\cdot PA}.$$
Note that the right hand side is the cross-ratio, 
which appeared in Theorem~\ref{lem:inverse-4-angle}.
Set $\delta(P,P)=1$ for any h-point~$P$.
Set 
$$PQ_h\df\ln[\delta(P,Q)].$$

The proof that $PQ_h$ is a metric on the h-plane will be given below.
For now it is just a function which returns a real value $PQ_h$ for any pair of h-points $P$ and~$Q$.

\begin{thm}{Exercise}\label{ex:h-dist-eq}
Let $O$ be the center of the absolute and the h-points $O$, $X$ and $Y$ lie on one h-line in the same order.
Assume $OX\z=XY$.
Prove that $OX_h<XY_h$.
\end{thm}


\parbf{Hyperbolic angles.}\label{h-angle measure}
Consider three h-points $P$, $Q$ and $R$
such that $P\ne Q$ and $R\ne Q$.
The \emph{hyperbolic angle $PQR$} (briefly $\angle_h PQR$)\index{11@$\angle_h$} is an ordered pair of h-half-lines $[QP)_h$ and $[QR)_h$.

Let $[QX)$ and $[QY)$ be (Euclidean) half-lines 
which are tangent to $[QP]_h$ and $[QR]_h$ 
at~$Q$.
Then the \index{angle measure!hyperbolic angle measure}\index{hyperbolic angle measure}\emph{hyperbolic angle measure} (or \index{h-angle measure}\emph{h-angle measure}) \index{16@$\measuredangle_h$} of $\angle_h PQR$ denoted by
$\measuredangle_h PQR$ and defined as
$\measuredangle XQY$.

\begin{thm}{Exercise}\label{ex:h-perp-unique}
Let $\ell$ be an h-line and $P$ be an h-point which does not lie on~$\ell$.
Show that there is a unique h-line passing thru $P$ 
and perpendicular to~$\ell$.
\end{thm}

\section*{Plan of the proof}
\addtocontents{toc}{Plan of the proof.}

We defined all the {}\emph{h-notions} needed in the formulation of the axioms \ref{def:birkhoff-axioms:0}--\ref{def:birkhoff-axioms:3} and h-\ref{def:birkhoff-axioms:4}.
It remains to show that all these axioms hold; 
this will be done by the end of this chapter.

Once we are done with the proofs, 
we get that the model provides an example of a neutral plane; 
in particular, Exercise~\ref{ex:h-perp-unique} can be proved the same way as Theorem~\ref{perp:ex+un}.

Most importantly we will prove the ``if''-part of Theorem~\ref{thm:consistent}.

Indeed, any statement in hyperbolic geometry can be restated in the Euclidean plane using the introduced h-notions.
Therefore, if the system of axioms \ref{def:birkhoff-axioms:0}--\ref{def:birkhoff-axioms:3} and h-\ref{def:birkhoff-axioms:4} leads to a contradiction, then so does  the system  axioms \ref{def:birkhoff-axioms:0}--\ref{def:birkhoff-axioms:4}.

\section*{Auxiliary statements}
\addtocontents{toc}{Auxiliary statements.}

\begin{thm}{Lemma}\label{lem:P-->O} 
Consider an h-plane with a unit circle as the absolute.
Let $O$ be the center of the absolute and $P$ be another h-point.
Denote by $P'$ the inverse of $P$ in the absolute.

Then the circle $\Gamma$ with the center $P'$ and radius 
$\tfrac{\sqrt{1-OP^2}}{OP}$
is perpendicular to the absolute.
Moreover, $O$ is the inverse of $P$ in~$\Gamma$. 
\end{thm}

\begin{wrapfigure}[8]{o}{43mm}
\begin{lpic}[t(-8mm),b(0mm),r(0mm),l(0mm)]{pics/PO-Gamma(1)}
%\lbl[br]{6,31.5;$\Omega$}
\lbl[b]{40,31.5;$\Gamma$}
\lbl[t]{18,16;$O$}
\lbl[b]{33,14;$P$}
\lbl[b]{40,14;$P'$}
\lbl[b]{33,30;$T$}
\end{lpic}
\end{wrapfigure}

\parit{Proof.} 
Follows from Exercise~\ref{ex:centers-of-perp-circles}.
\qeds

Assume $\Gamma$ is a circline which is perpendicular to the absolute.
Consider the inversion $X\mapsto X'$
in $\Gamma$, 
or if $\Gamma$ is a line,
set $X\mapsto X'$ to be the reflection in~$\Gamma$.

The following observation says that the map $X\mapsto X'$ respects all the notions introduced in the previous section.
Together with the lemma above, it implies that in any problem which is formulated entirely {}\emph{in h-terms}  we can assume that a given h-point lies in the center of the absolute.

\begin{thm}{Main observation}\label{thm:main-observ}
The map $X\mapsto X'$ described above is a bijection of the h-plane to itself. 
Moreover, for any h-points $P$, $Q$, $R$ such that $P\ne Q$ and $Q\ne R$, the following conditions hold:
\begin{enumerate}[(a)]
\item\label{h-line-to-hline} The h-line $(PQ)_h$, h-half-line $[PQ)_h$ and h-segment $[PQ]_h$ are mapped to  $(P'Q')_h$, $[P'Q')_h$ and $[P'Q']_h$ correspondingly.
\item\label{h-reflect} $\delta(P',Q')=\delta(P,Q)$ and $P'Q'_h=PQ_h$.
\item\label{h-angle-mes} 
$\measuredangle_h P'Q'R'\equiv-\measuredangle_h PQR$.
\end{enumerate}

\end{thm}

It is instructive to compare this observation with Proposition \ref{prop:reflection}.

\parit{Proof.}
According to Theorem~\ref{thm:perp-inverse}, the map sends the absolute to itself. 
Note that the points on $\Gamma$ do not move, it follows that points inside of the absolute remain inside after the mapping and the other way around. 


Part~\textit{(\ref{h-line-to-hline})} follows from \ref{thm:inverse-cline} and \ref{thm:angle-inversion}.

Part~\textit{(\ref{h-reflect})} follows from Theorem~\ref{lem:inverse-4-angle}.

Part~\textit{(\ref{h-angle-mes})}  follows from Theorem~\ref{thm:angle-inversion}.
\qeds


\begin{thm}{Exercise}\label{ex:h-reflection}
Let $\Gamma$ be a circle which is perpendicular to the absolute and let $Q$ be an h-point lying on~$\Gamma$.
Assume $P$ is an h-point and $P'$ is its inversion in ~$\Gamma$.
Show that $PQ_h=P'Q_h$.
\end{thm}

\begin{thm}{Lemma}\label{lem:O-h-dist}
Assume that the absolute is a unit circle centered at~$O$.
Given an h-point $P$, set $x=OP$ and $y=OP_h$.
Then
\begin{align*}
y&=\ln\frac{1+x}{1-x}
&
&\text{and}
&
x&=\frac{e^y-1}{e^y+1}.
\end{align*}
 
\end{thm}

\begin{wrapfigure}[8]{o}{41mm}
\begin{lpic}[t(-5mm),b(0mm),r(0mm),l(-4mm)]{pics/AOPB(1)}
\lbl[tr]{2,20;$A$}
\lbl[t]{22,20;$O$}
\lbl[t]{30,20;$P$}
\lbl[tl]{42,20;$B$}
\end{lpic}
\end{wrapfigure}

\parit{Proof.}
Note that the h-line $(OP)_h$ forms a diameter of the absolute.
If $A$ and $B$ are the ideal points as in the definition of h-distance, then
\begin{align*}
OA&=OB=1,
\\ 
PA&=1+x,
\\
PB&=1-x.\end{align*}
In particular,
\begin{align*}
y&=\ln \frac{AP\cdot BO}{PB\cdot OA}=\ln\frac{1+x}{1-x}.
\end{align*}

Taking the exponential function of the left and the right hand side and applying obvious algebra manipulations, we get that
$$x=\frac{e^y-1}{e^y+1}.$$
\qedsf


\begin{thm}{Lemma}\label{lem:h-tiangle=}
Assume the points $P$, $Q$ and $R$ appear on one h-line in the same order.
Then 
$$PQ_h+QR_h=PR_h.$$ 

\end{thm}

\parit{Proof.}
Note that
$$PQ_h+QR_h=PR_h$$
is equivalent to 
\[\delta(P,Q)\cdot\delta(Q,R)=\delta(P,R).\eqlbl{eq:deltaPQR}\]

Let $A$ and $B$ be the ideal points of~$(PQ)_h$. 
Without loss of generality, we can assume that the points $A$, $P$, $Q$, $R$ and $B$ appear in the same order on the circline containing $(PQ)_h$.
Then
\begin{align*}
\delta(P,Q)\cdot\delta(Q,R)
&=
\frac{AQ\cdot BP}{QB\cdot PA}\cdot\frac{AR\cdot BQ}{RB\cdot QA}=
\\
&=\frac{AR\cdot BP}{RB\cdot PA}=
\\
&=\delta(P,R)
\end{align*}
Hence \ref{eq:deltaPQR} follows.
\qeds

Let $P$ be an h-point and $\rho>0$.
The set of all h-points $Q$ such that $PQ_h=\rho$ is called an \index{h-circle}\emph{h-circle} with the center $P$ and the \index{h-radius}\emph{h-radius} $\rho$.

\begin{thm}{Lemma}\label{lem:h-circle=circle}
Any h-circle  is a Euclidean circle which lies completely in the h-plane.

More precisely for any h-point $P$ and $\rho\ge 0$
there is a $\hat\rho\ge 0$ and a point $\hat P$ such that 
$$PQ_h= \rho
\quad 
\iff
\quad
\hat PQ= \hat\rho$$
for any h-point~$Q$.

Moreover, if $O$ is the center of the absolute, then 
\begin{enumerate}
\item $\hat O=O$ for any $\rho$ and
\item $\hat P\in (OP)$ for any $P\ne O$.
\end{enumerate}

\end{thm}

\begin{wrapfigure}{o}{44mm}
\begin{lpic}[t(-8mm),b(-3mm),r(0mm),l(-0mm)]{pics/h-circle-3(1)}
\lbl[t]{22,20;$O$}
\lbl[tr]{28,15;$Q$}
\lbl[t]{37,20;$P$}
\lbl[b]{33,23.5;$\hat P$}
\lbl[br]{31,29;$\Delta'_\rho$}
\end{lpic}
\end{wrapfigure}

\parit{Proof.}
According to Lemma~\ref{lem:O-h-dist}, 
$OQ_h= \rho$ if and only if $$OQ= \hat\rho=\frac{e^\rho-1}{e^\rho+1}.$$
Therefore, the locus of h-points $Q$ such that $OQ_h= \rho$ is a Euclidean circle, 
denote it by $\Delta_\rho$.

If $P\ne O$, applying Lemma~\ref{lem:P-->O} and the main observation (\ref{thm:main-observ})
we get
a circle $\Gamma$ perpendicular to the absolute such that $P$ is the inverse of $O$ in~$\Gamma$.

Let $\Delta_\rho'$ be the inverse of $\Delta_\rho$ in~$\Gamma$.
Since the inversion in $\Gamma$ preserves the h-distance,
$PQ_h=\rho$ if and only if $Q\z\in\Delta_\rho'$.

According to Theorem~\ref{thm:inverse-cline}, 
$\Delta_\rho'$ is a Euclidean circle.
Denote by $\hat P$ the Euclidean center and by $\hat\rho$ the Euclidean radius of $\Delta_\rho'$.

Finally, note that $\Delta_\rho'$ reflects to itself in $(OP)$;
that is, the center $\hat P$ lies on~$(OP)$.
\qeds

\begin{thm}{Exercise}\label{ex:h-circle=circle}
Assume $P$, $\hat P$ and $O$ are as in the Lemma~\ref{lem:h-circle=circle} and $P\z\ne O$.
Show that  $\hat P\in [OP]$.
\end{thm}

\section*{Axiom \ref{def:birkhoff-axioms:0}}
\addtocontents{toc}{Axioms: \ref{def:birkhoff-axioms:0},}

Evidently, the h-plane contains at least two points.
Therefore, to show that Axiom \ref{def:birkhoff-axioms:0} holds in the h-plane, we need to show that the h-distance defined on page \pageref{h-dist} is a metric on h-plane;
that is, the conditions \textit{(\ref{def:metric-space:a})}--\textit{(\ref{def:metric-space:d})} 
in Definition~\ref{def:metric-space} hold for h-distance.


The following claim says that the h-distance meets the conditions \textit{(\ref{def:metric-space:a})} 
and \textit{(\ref{def:metric-space:b})}.

\begin{thm}{Claim}
Given the h-points $P$ and $Q$,  we have
$PQ_h\ge 0$
and $PQ_h=0$ if and only if $P=Q$.
\end{thm}


\parit{Proof.}
According to Lemma~\ref{lem:P-->O}
and the main observation (\ref{thm:main-observ}), 
we may assume that $Q$ is the center of the absolute.
In this case
$$
\delta(Q,P)=\frac{1+QP}{1-QP}\ge 1$$
and therefore
$$QP_h=\ln[\delta(Q,P)]\ge 0.$$
Moreover, the equalities holds if and only if $P=Q$.
\qeds

The following claim says that the h-distance meets 
the condition~\ref{def:metric-space}\textit{\ref{def:metric-space:c}}.

\begin{thm}{Claim}
For any h-points $P$ and $Q$, we have
$PQ_h=QP_h$.
\end{thm}

\parit{Proof.}
Let $A$ and $B$ be ideal points of $(PQ)_h$ and
$A,P,Q,B$ appear on the circline containing $(PQ)_h$ in the same order.

\begin{wrapfigure}[8]{o}{26mm}
\begin{lpic}[t(-3mm),b(-2mm),r(0mm),l(0mm)]{pics/APQB-1(1)}
\lbl[rt]{15,16;$P$}
\lbl[rb]{14,28;$Q$}
\lbl[tl]{21,11;$A$}
\lbl[bl]{21,35.5;$B$}
\end{lpic}
\end{wrapfigure}

Then
\begin{align*}
PQ_h
&=\ln\frac{AQ\cdot BP}{QB\cdot PA}
=
\\
&=\ln\frac{BP\cdot AQ}{PA\cdot QB}=
\\
&=QP_h.
\end{align*}
\qedsf

The following claim shows, in particular, that
the triangle inequality 
(which is condition \ref{def:metric-space}\textit{\ref{def:metric-space:d}})
holds  for $h$-distance.

\begin{thm}{Claim}\label{clm:h-dist+trig-inq}
Given a triple of h-points $P$, $Q$ and $R$,
we have
\[PQ_h+QR_h\ge PR_h.\]
Moreover, the equality holds if and only if $P$, $Q$ and $R$ lie on one h-line in the same order.
\end{thm}

\parit{Proof.}
Without loss of generality, we may assume that $P$ is the center of the absolute
and 
$PQ_h\z\ge QR_h\z>0$.

Denote by $\Delta$ the h-circle with the center $Q$ 
and h-radius $\rho=QR_h$.
Let $S$ and $T$ be the points of intersection of $(PQ)$ and~$\Delta$.

By Lemma~\ref{lem:h-tiangle=}, $PQ_h\z\ge QR_h$.
Therefore, we can assume that the points $P$, $S$, $Q$ and $T$ appear on the h-line in the same order.

According to Lemma~\ref{lem:h-circle=circle},
$\Delta$ is a Euclidean circle;
denote by $\hat Q$ its Euclidean center.
Note that $\hat Q$ is the (Euclidean) midpoint of~$[ST]$.

\begin{wrapfigure}[8]{o}{30mm}
\begin{lpic}[t(-4mm),b(-3mm),r(0mm),l(0mm)]{pics/h-circle-4(1)}
\lbl[tr]{6.5,19;$P$}
\lbl[t]{21,19;$Q$}
\lbl[b]{18,22.5;$\hat Q$}
\lbl[b]{17,29;$\Delta$}
\lbl[br]{10.5,22.5;$S$}
\lbl[bl]{26,22.5;$T$}
\lbl[tr]{12.5,14;$R$}
\end{lpic}
\end{wrapfigure}

By the Euclidean triangle inequality 
$$PT
=
P\hat Q+\hat Q R
\ge 
PR\eqlbl{RT>RQ}$$
and the equality holds if and only if $T=R$. 

By Lemma~\ref{lem:O-h-dist},
\begin{align*}
PT_h&=\ln\frac{1+PT}{1-PT},\\
PR_h&=\ln\frac{1+PR}{1-PR}.
\end{align*}
Since the function $f(x)=\ln\frac{1+x}{1-x}$ is increasing for $x\in[0,1)$, 
inequality \ref{RT>RQ} implies
$$PT_h\ge PR_h$$
and the equality holds if and only if $T=R$.

Finally, applying Lemma~\ref{lem:h-tiangle=} again, 
we get that
$$PT_h=PQ_h+QR_h.$$
Hence the claim follows.
\qeds

\section*{Axiom \ref{def:birkhoff-axioms:1}}
\addtocontents{toc}{\ref{def:birkhoff-axioms:1},}

Note that once the following claim is proved,
Axiom \ref{def:birkhoff-axioms:1} 
follows from Corollary~\ref{cor:h-line}.

\begin{thm}{Claim}
A subset of the h-plane is an h-line if and only if it forms a line for the h-distance in the sense of Definition~\ref{def:line}.
\end{thm}

\parit{Proof.}
Let $\ell$ be an h-line.
Applying the main observation (\ref{thm:main-observ}) we can assume that $\ell$ contains the center of the absolute.
In this case, $\ell$ is an intersection of a diameter of the absolute and the h-plane.
Let $A$ and $B$ be the endpoints of the diameter.

Consider the map $\iota\:\ell\to \mathbb{R}$ defined as
$$\iota(X)=\ln \frac{AX}{XB}.$$
Note that $\iota\:\ell\to \mathbb{R}$ is a bijection.

Further, if $X,Y\in \ell$ and the points $A$, $X$, $Y$ and $B$ appear on $[AB]$ in the same order, then
\[\iota(Y)-\iota(X)=\ln \frac{AY}{YB}-\ln \frac{AX}{XB}=\ln \frac{AY\cdot BX}{YB\cdot XB}=XY_h.\]

We proved that any h-line is a line for h-distance.
The converse follows from Claim~\ref{clm:h-dist+trig-inq}.
\qeds


\section*{Axiom \ref{def:birkhoff-axioms:2}}
\addtocontents{toc}{\ref{def:birkhoff-axioms:2},}

Note that the first part of Axiom \ref{def:birkhoff-axioms:2} follows directly from the definition of the h-angle measure defined on page~\pageref{h-angle measure}.
It remains to show that $\measuredangle_h$ satisfies the conditions \ref{def:birkhoff-axioms:2a}, \ref{def:birkhoff-axioms:2b} and \ref{def:birkhoff-axioms:2c} on page \pageref{def:birkhoff-axioms:2b}.

The following two claims say that
$\measuredangle_h$ satisfies
 \ref{def:birkhoff-axioms:2a} and \ref{def:birkhoff-axioms:2b}.

\begin{thm}{Claim}\label{clm:h2a}
Given an h-half-line $[O P)_h$ and $\alpha\in(-\pi,\pi]$, there is a unique h-half-line $[O Q)_h$ such that $\measuredangle_h P O Q= \alpha$.
\end{thm}

\begin{thm}{Claim}\label{clm:h2b}
For any h-points $P$, $Q$ and $R$ distinct from an h-point $O$, we have
$$\measuredangle_h P O Q+\measuredangle_h Q O R
\equiv\measuredangle_h P O R.$$

\end{thm}

\parit{Proof of \ref{clm:h2a} and \ref{clm:h2b}.}
Applying the main observation, 
we may assume that $O$ is the center of the absolute.
In this case, for any h-point $P\ne O$,
$[OP)_h$ is the intersection of $[OP)$ with h-plane.
Hence the claims \ref{clm:h2a} and \ref{clm:h2b} 
follow from the corresponding axioms of the Euclidean plane.
\qeds

The following claim says that
$\measuredangle_h$ satisfies
 \ref{def:birkhoff-axioms:2c}.

\begin{thm}{Claim}\label{clm:h2c}
The function 
$$\measuredangle_h\:(P,Q,R)\mapsto\measuredangle_h P Q R$$
is continuous at any triple of points $(P,Q,R)$
such that $Q\ne P$, $Q\ne R$ and $\measuredangle_h P Q R\ne\pi$.
\end{thm}

\parit{Proof.}
Denote by $O$ the center of the absolute.
We can assume that $Q$ is distinct from~$O$.

Denote by $Z$ the inverse of $Q$ in the absolute
and by $\Gamma$ the circle perpendicular to the absolute which is centered at~$Z$.
According to Lemma~\ref{lem:P-->O},
the point $O$ is the inverse of $Q$ in~$\Gamma$.

Denote by $P'$ and $R'$ the inversions in $\Gamma$ of the points $P$ and $R$ correspondingly.
Note that the point $P'$ is completely determined by the points $Q$ and $P$.
Moreover, the map $(Q,P)\mapsto P'$ is continuous at any pair of points $(Q,P)$ such that $Q\ne O$.
The same is true for the map $(Q,R)\mapsto R'$

According to the main observation 
$$\measuredangle_h P Q R\equiv -\measuredangle_h P' O R'.$$
Since $\measuredangle_h P' O R'=\measuredangle P' O R'$ and 
the maps $(Q,P)\mapsto P'$, $(Q,R)\mapsto R'$ are continuous,
the claim follows from the corresponding axiom of the Euclidean plane.
\qeds

\section*{Axiom \ref{def:birkhoff-axioms:3}}
\addtocontents{toc}{\ref{def:birkhoff-axioms:3},}

The following claim says that Axiom~\ref{def:birkhoff-axioms:3} holds in the h-plane.

\begin{thm}{Claim}
In the h-plane, we have
$\triangle_h P Q R 
\cong
\triangle_h P' Q' R'$
if and only if 
\begin{align*}
Q' P'_h&=Q P_h, & Q' R'_h&= Q R_h &&\text{and}
&\measuredangle_h P' Q' R'&=\pm\measuredangle P Q R.
\end{align*}
 
\end{thm}

\parit{Proof.}
Applying the main observation, 
we can assume that both $Q$ and $Q'$ coincide with the center of the absolute.
In this case 
$$\measuredangle P' Q R'=\measuredangle_h P' Q R'=\pm\measuredangle_h P Q R=\pm\measuredangle P Q R.$$
Since 
$$Q P_h=Q P'_h\quad \text{and}\quad Q R_h=Q R'_h,$$
Lemma \ref{lem:O-h-dist} implies that the same holds for the Euclidean distances;
that is,
$$Q P=Q P'
\quad
\text{and}
\quad
Q R=Q R'.$$
By SAS,
there is a motion of the Euclidean plane which sends $Q$ to itself, $P$ to $P'$ and $R$ to $R'$

Note that the center of the absolute is fixed by the corresponding motion.
It follows that this motion gives also a motion of the h-plane;
in particular, the h-triangles  
$\triangle_h P Q R$ and $\triangle_h P' Q R'$ are h-congruent.
\qeds

\section*{Axiom h-$\!$\ref{def:birkhoff-axioms:4}}
\addtocontents{toc}{h-\ref{def:birkhoff-axioms:4}.}


Finally, we need to check that the Axiom~h-$\!$\ref{def:birkhoff-axioms:4} on page \pageref{def:hyperbolic-4a} holds;
that is, we need to prove the following claim.

\begin{thm}{Claim}
For any h-line $\ell$ and any h-point $P\notin\ell$ there are at least two h-lines which pass thru $P$ 
and have no points of intersection with~$\ell$.
\end{thm}

{

\begin{wrapfigure}{o}{44mm}
\begin{lpic}[t(-4mm),b(0mm),r(0mm),l(0mm)]{pics/absolute-triangle-3(1)}
\lbl[tl]{32,3;$A$}
\lbl[bl]{32,39;$B$}
\lbl[r]{20.5,21;$P$}
\lbl[b]{15,10,60;$m$}
\lbl[b]{15,34,-60;$n$}
\lbl[l]{28,21;$\ell$}
\end{lpic}
\end{wrapfigure}

\parit{Instead of proof.}
Applying the main observation we can assume that $P$ is the center of the absolute.

The remaining part of the proof can be guessed from the picture
\qeds

\begin{thm}{Exercise}\label{ex:3-h-lines}
Show that in the h-plane 
there are 3 mutually parallel h-lines 
such that any pair of these three lines lies on one side from the remaining h-line.
\end{thm}

}
 
\section*{Hyperbolic trigonometry}
\addtocontents{toc}{Hyperbolic trigonometry}

In this section we give formulas for h-distance using \index{hyperbolic functions}\emph{hyperbolic functions}.
One of these formulas will be used in the proof of the hyperbolic Pythagorean theorem (\ref{thm:pyth-h-poincare}).

Recall that $\cosh$, $\sinh$ and $\tanh$ denote \index{ch@$\cosh$}\index{hyperbolic cosine}\emph{hyperbolic cosine}, \index{sh@$\sinh$}\index{hyperbolic sine}\emph{hyperbolic sine} and \index{th@$\tanh$}\index{hyperbolic tangent}\emph{hyperbolic tangent};
that is, the functions defined by
\[\cosh x\df \tfrac{e^x+e^{-x}}2,
 \quad
 \sinh x\df \tfrac{e^x-e^{-x}}2,
\]
\[\tanh x\df \tfrac{\sinh x}{\cosh x}.
\]

The following identities on hyperbolic functions are analogous to classical trigonometric identities for sine and cosine. 
They might help to solve the following exercise. 

\begin{thm}{Double-argument identities}\label{double-argument}
The identities
\begin{align*}
\cosh (2\cdot x)&=(\cosh x)^2+(\sinh x)^2 
&&\text{and}&
\sinh (2\cdot x)&=2\cdot\sinh x\cdot \cosh x
\end{align*}
hold for any real value $x$.
\end{thm}
\qedsf

\parit{Proof.}
\begin{align*}
\cosh (2\cdot x)
&=\tfrac{e^{2\cdot x}+e^{-2\cdot x}}2=
\\
&=\left(\tfrac{e^x-e^{-x}}2\right)^2+\left(\tfrac{e^x+e^{-x}}2\right)^2=
\\
&=(\sinh x)^2+(\cosh x)^2;
\\
\sinh (2\cdot x)
&=\tfrac{e^{2\cdot x}-e^{-2\cdot x}}2=
\\
&=2\cdot\left(\tfrac{e^x-e^{-x}}2\right)\cdot\left(\tfrac{e^x+e^{-x}}2\right)=
\\
&=2\cdot\sinh x\cdot \cosh x.
\end{align*}

\begin{thm}{Advanced exercise}\label{ex:cosh}
Let $P$ and $Q$ be two h-poins distinct from the center of absolute.
Denote by $P'$ and $Q'$ the inverses of $P$ and $Q$ in the absolute.

\begin{wrapfigure}[20]{o}{38mm}
\begin{lpic}[t(-4mm),b(0mm),r(0mm),l(0mm)]{pics/PQPQ(1)}
\lbl[t]{10,25;$P$}
\lbl[t]{25,23;$Q$}
\lbl[b]{3,36;$P'$}
\lbl[b]{36,37;$Q'$}
\end{lpic}
\end{wrapfigure}

Show that 
\begin{enumerate}[(a)]
\item\label{ex:cosh/2} 
\[\cosh[\tfrac12\cdot PQ_h]=\sqrt{\frac{PQ'\cdot  P'Q}{PP'\cdot QQ'}};\]
\item\label{ex:coshsinh} 
\[\sinh[\tfrac12\cdot PQ_h]=\sqrt{\frac{PQ\cdot  P'Q'}{PP'\cdot QQ'}};\]
\item\label{ex:coshtanh} 
\[\tanh[\tfrac12\cdot PQ_h]=\sqrt{\frac{PQ\cdot  P'Q'}{PQ'\cdot P'Q}};\]
\item\label{ex:coshcosh} 
\[\cosh PQ_h=\frac{PQ\cdot  P'Q'+PQ'\cdot  P'Q}{PP'\cdot QQ'}.\]
\end{enumerate}

\end{thm}

\addtocontents{toc}{\protect\end{quote}}
\chapter{Preliminaries}\label{chap:metr}


\section{What is the axiomatic approach?}
\label{preaxioms}

In the axiomatic approach, one defines the plane as anything that satisfies a given list of properties.
These properties are called {}\emph{axioms}.
The axiomatic system for the theory 
is like the rules for a game.
Once the axiom system is fixed, a statement is considered to be true if it follows from the axioms, and nothing else is considered to be true.

The formulations of the first axioms were not rigorous at all.
For example, Euclid described a {}\emph{line} as \textit{breadthless length}
and a {}\emph{straight line} as a line that \textit{lies evenly with the points on itself}.
On the other hand,
these formulations were sufficiently clear 
so that one mathematician could understand the other.

The best way to understand an axiomatic system
is to make one by yourself.
Assume for now that we have an intuitive understanding of such notions as {}\emph{line} and {}\emph{point}.
Imagine an infinite and perfect surface of a chalkboard. 
Let us try to collect the key observations about this model.

\begin{enumerate}[(i)]
 \item\label{preaxiomI} We can measure distances between points.
 \item\label{preaxiomII} We can draw a unique line 
 that passes thru two given points.
 \item\label{preaxiomIII} We can measure angles.
 \item\label{preaxiomIV} If we rotate or shift we will not see the difference.
 \item\label{preaxiomV} If we change the scale we will not see the difference.
\end{enumerate}
These observations are good to start with.
Further, we will develop the language to reformulate them rigorously.

\section{What is a model?}
\label{page:model}

The Euclidean plane can be defined rigorously the following way:

\textit{Define a {}\emph{point} in the Euclidean plane as a pair of real numbers $(x,y)$ and define the {}\emph{distance} between the two points $(x_1,y_1)$ and $(x_2,y_2)$ by the following formula:}
\[\sqrt{(x_1-x_2)^2+(y_1-y_2)^2}.\]

That is it!
We gave a {}\emph{numerical model} of the Euclidean plane;
it builds the Euclidean plane from real numbers
while the latter is assumed to be known.

Shortness is the main advantage of the model approach,
but it is not intuitively clear why we define points and distances this way.

On the other hand, the observations made in the previous section are intuitively obvious ---
this is the main advantage of the axiomatic approach.
Another advantage --- the axiomatic approach is easily adjustable. 
For example, we may remove one axiom from the list,
or exchange it for another axiom. 
We will do it in Chapter~\ref{chap:non-euclid}.

\section{Metric spaces}

The notion of metric space provides 
a rigorous way to say: \textit{``we can measure distances between points''}.
That is, instead of (\ref{preaxiomI}) in Section~\ref{preaxioms},
we can say \textit{``Euclidean plane is a metric space''}.

\begin{thm}{Definition}\label{def:metric-space}
Let $\mathcal X$ be a nonempty set and 
$d$ be a function
that returns a real number $d(A,B)$
for any pair $A,B\in\mathcal X$.
Then $d$
is called \index{metric}\emph{metric} on 
$\mathcal X$ if, for any
$A,B,C\in \mathcal X$, the following conditions are satisfied:
\begin{enumerate}[(a)]
\item\label{def:metric-space:a} Positiveness: 
$d(A,B)\ge 0.$
\item\label{def:metric-space:b} $A=B$ if and only if 
$d(A,B)=0.$
\item\label{def:metric-space:c} Symmetry: $d(A, B) = d(B, A).$
\item\label{def:metric-space:d} \index{triangle!inequality}Triangle inequality: 
$d(A, C) \le d(A, B) + d(B, C).$
\end{enumerate}
A \index{metric!space}\emph{metric space} is a set with a metric on it. 
More formally, a metric space is a pair $(\mathcal X, d)$ where $\mathcal X$ is a set and $d$ is a metric on~$\mathcal X$.

The elements of $\mathcal X$ are called \index{point}\emph{points} of the metric space.
Given two points $A,B\z\in \mathcal X$, 
the value $d(A, B)$ is called \index{distance}\emph{distance} from $A$ to~$B$.
\end{thm}

\pagebreak

\subsection*{Examples}

\begin{itemize}
\item {}\emph{Discrete metric.} Let $\mathcal X$ be an arbitrary set. 
For any $A,B\z\in\mathcal X$ set $d(A,B)\z=0$ if $A=B$ and $d(A,B)=1$ otherwise.
The metric $d$ is called the \index{discrete metric}\emph{discrete metric} on~$\mathcal X$.
\item\index{real!line}\emph{Real line.}
Set of all real numbers ($\mathbb{R}$) with metric $d$ defined by 
$$d(A,B)\df|A-B|.$$
\end{itemize}

\begin{thm}{Exercise}\label{ex:dist-square}
Show that $d(A,B)=|A-B|^2$ is \textit{not} a metric on $\mathbb{R}$.
\end{thm}

\begin{itemize}
\item \textit{Metrics on the plane.}
Suppose that $\mathbb{R}^2$ denotes the set of all pairs $(x,y)$ of real numbers.
Assume $A=(x_A,y_A)$ and $B=(x_B,y_B)$.
Consider the following metrics on $\mathbb{R}^2$:
\begin{itemize}
\item\index{Euclidean!metric}\emph{Euclidean metric,} denoted by \index{59@$d_1$, $d_2$, $d_\infty$}$d_2$, and defined as \label{def:d_2}
$$d_2(A,B)=\sqrt{(x_A-x_B)^2+(y_A-y_B)^2}.$$
\item\label{Manhattan plane}\index{Manhattan plane}\emph{Manhattan metric,} denoted by $d_1$ and defined as 
$$d_1(A,B)=|x_A-x_B|+|y_A-y_B|.$$
\item{}\emph{Maximum metric,} denoted by $d_\infty$ and defined as 
$$d_\infty(A,B)=\max\{|x_A-x_B|,|y_A-y_B|\}.$$
\end{itemize}
\end{itemize}

\begin{thm}{Exercise}\label{ex:d_1+d_2+d_infty}
Prove that the following functions are metrics on $\mathbb{R}^2$:
(a)~$d_1$; (b)~$d_2$; (c)~$d_\infty$.
\end{thm}


\section{Shortcut for distance}

Most of the time, 
we study only one metric on space.
Therefore, we will not need to name the metric each time.

Given a metric space $\mathcal X$,
the distance between points $A$ and $B$ will be further denoted by 
$$AB
\quad
\text{or}
\quad
d_{\mathcal X}(A,B);$$
the latter is used only if we need to emphasize that $A$ and $B$ are points of the metric space~$\mathcal X$.

For example, the triangle inequality can be written as 
$$AC\le AB+BC.$$

For multiplication, we will always use ``$\cdot$'',
so $AB$ could not be confused with $A\cdot B$.

\begin{thm}{Exercise}\label{ex:4-triangle}
Show that the inequality
\[AB+PQ\le AP+AQ+BP+BQ\]
holds for any four points $A$, $B$, $P$, $Q$ in a metric space.
\end{thm}


\section{Isometries, motions, and lines}

In this section, we define lines in a metric space.
Once it is done the sentence \textit{``We can draw a unique line that passes thru two given points.''} becomes rigorous; see (\ref{preaxiomII}) in Section~\ref{preaxioms}. 

Recall that a map $f\:\mathcal{X}\to\mathcal{Y}$
is a \index{bijection}\emph{bijection}
if it gives an exact pairing of the elements of two sets.
Equivalently, $f\:\mathcal{X}\to\mathcal{Y}$ is a bijection if it has an \index{inverse}\emph{inverse};
that is, a map $g\:\mathcal{Y}\to\mathcal{X}$
such that 
$g(f(A))\z=A$ for any $A\in\mathcal{X}$
and
$f(g(B))\z=B$ for any $B\in\mathcal{Y}$. 

Let $\mathcal X$ and $\mathcal Y$ be two metric spaces and $d_{\mathcal X}$, $d_{\mathcal Y}$ be their metrics. 
A~map 
$$f\:\mathcal X \z\to \mathcal Y$$ 
is
called \index{distance-preserving map}\emph{distance-preserving} if 
$$d_{\mathcal Y}(f(A), f(B))
 = d_{\mathcal X}(A,B)$$
for any $A,B\in {\mathcal X}$.

A bijective distance-preserving map is called an \index{isometry}\emph{isometry}. 

Two metric spaces are called \index{isometric spaces}\emph{isometric} if there exists an isometry from one to the other.

The isometry from a metric space to itself 
is also called a \index{motion}\emph{motion} of the space.

\begin{thm}{Exercise}\label{ex:dist-preserv=>injective}
Show that any distance-preserving map is \index{injective map}\emph{injective};
that is, if $f\:\mathcal X\to\mathcal Y$ is a distance-preserving map, 
then $f(A)\ne f(B)$ for any pair of distinct points $A, B\in \mathcal X$.
\end{thm}

\begin{thm}{Exercise}\label{ex:motion-of-R}
Show that if $f\:\mathbb{R}\to\mathbb{R}$ is a motion of the real line,
then either (a)
$f(x)=f(0)+x$ for any $x\in \mathbb{R}$, 
or (b)
$f(x)=f(0)-x$ for any $x\in \mathbb{R}$. 

\end{thm}

\begin{thm}{Exercise}\label{ex:d_1=d_infty}
Prove that $(\mathbb{R}^2,d_1)$ is isometric to $(\mathbb{R}^2,d_\infty)$.
\end{thm}

\begin{thm}{Advanced exercise}\label{ad-ex:motions of Manhattan plane}
Describe all the motions of the Manhattan plane, defined in \ref{ex:dist-square}.
\end{thm}

If $\mathcal X$ is a metric space and $\mathcal Y$ is a subset of $\mathcal X$,
then a metric on $\mathcal Y$ can be obtained by restricting the metric from~$\mathcal X$. 
In other words, 
the distance between two points of $\mathcal Y$ is defined to be the distance between these points in $\mathcal X$.
This way any subset of a metric space can be also considered as a metric space. 

\begin{thm}{Definition}\label{def:line}
A subset $\ell$ of metric space is called a \index{line}\emph{line} if it is isometric to the real line.
\end{thm}

A triple of points that lie on one line is called \index{collinear points}\emph{collinear}.
Note that if $A$, $B$, and  $C$ are  collinear, $AC\ge AB$, and $AC\ge BC$, then $AC\z= AB+BC$.

Some metric spaces have no lines; for example, discrete metrics.
The picture shows examples of lines on the Manhattan plane $(\mathbb{R}^2,d_1)$. 
\begin{figure}[!ht]
\centering
\includegraphics{mppics/pic-2}
\end{figure}

\begin{thm}{Exercise}\label{ex:y=|x|}
Consider the graph $y=|x|$ in $\mathbb{R}^2$.
In which of the following spaces 
(a) $(\mathbb{R}^2,d_1)$, 
(b) $(\mathbb{R}^2,d_2)$, 
(c) $(\mathbb{R}^2,d_\infty)$ 
does it form a line? 
Why?
\end{thm}

\begin{thm}{Exercise}\label{ex:line-motion}
Show that any motion maps a line to a line. 
\end{thm}

\section{Half-lines and segments}

Assume there is a line $\ell$ passing thru
two distinct points $P$ and $Q$.
In this case, we might denote $\ell$ as $(PQ)$.
There might be more than one line thru $P$ and $Q$,
but if we write \index{60@$(PQ)$, $[PQ)$, $[PQ]$}$(PQ)$ we assume that we made a choice of such a line. 

We will denote by $[P Q)$ the \index{half-line}\emph{half-line}
that starts at $P$ and contains~$Q$. 
Formally speaking, $[P Q)$ is a subset of $(P Q)$ that corresponds to $[0,\infty)$ under an isometry $f\:(P Q)\to \mathbb{R}$ such that $f(P)=0$ and $f(Q)>0$.

The subset of line $(P Q)$ between $P$ and $Q$ is called the \index{segment}\emph{segment between} $P$ and $Q$; it is denoted by~$[P Q]$.
Formally, the segment can be defined as the intersection of two half-lines: $[P Q]=[P Q)\cap[Q P)$.

\begin{thm}{Exercise}\label{ex:trig==}
Show that 
\begin{enumerate}[(a)]
\item if $X\in [PQ)$, then 
$QX=|PX-PQ|$;
\item if $X\in [PQ]$, then 
$PX+XQ=PQ$.
\end{enumerate}

\end{thm}


\section{Angles}

Our next goal is to introduce angles and angle measures; 
after that, the statement \textit{``we can measure angles''} will become rigorous;
see (\ref{preaxiomIII}) in Section~\ref{preaxioms}.

An ordered pair of half-lines that start at the same point is called an \index{angle}\emph{angle}.
The angle $AOB$ (also denoted by \index{10@$\angle$, $\measuredangle$}$\angle AOB$) is the pair of half-lines $[OA)$ and $[OB)$.
In this case, the point $O$ is called the \index{vertex!of angle}\emph{vertex} of the angle.

Intuitively, the angle measure tells how much one has to rotate the first half-line counterclockwise, so it gets the position of the second half-line of the angle. 
The full turn is assumed to be $2\cdot\pi$;
it corresponds to the angle measure in radians.%
\footnote{For a while you may think that $\pi$ is a positive real number that measures the size of a half-turn in certain units. Its value $\pi\approx 3.14$ will not be important.}

The angle measure of $\angle AOB$ is denoted by $\measuredangle AOB$;
it is a real number in the interval $(-\pi,\pi]$. 

\begin{wrapfigure}{o}{25mm}
\vskip-0mm
\centering
\includegraphics{mppics/pic-4}
\end{wrapfigure}

The notations $\angle AOB$ and $\measuredangle AOB$ look similar;
they also have close but different meanings which better not be confused.
For example, the equality 
$\angle AOB=\angle A'O'B'$
means that
$[OA)=[O'A')$ and $[OB)\z=[O'B')$;
in particular, $O=O'$.
On the other hand, the equality 
$\measuredangle AOB\z=\measuredangle A'O'B'$ 
means only equality of two real numbers;
in this case, $O$ may be distinct from~$O'$.

Here is the first property of angle measure which will become a part of the axiom.
\textit{Given a half-line $[O A)$ and $\alpha\in(-\pi,\pi]$ there is a unique half-line $[O B)$ such that $\measuredangle A O B= \alpha$.}



\section[\texorpdfstring{Reals modulo $2\cdot\pi$}{Reals modulo 2·π}]{Reals modulo $\bm{2\cdot\pi}$}



Consider three half-lines starting from the same point, $[O A)$, $[O B)$, and $[O C)$.
They make three angles $A O B$, $B O C$, and $A O C$,
so the value $\measuredangle A O C$ should coincide with
the sum $\measuredangle A O B+\measuredangle B O C$ up to full rotation.
This property will be expressed by the formula 
$$\measuredangle A O B+\measuredangle B O C\equiv \measuredangle A O C,$$
where \index{34@$\equiv$}``$\equiv$'' is a new notation which we are about to introduce.
The last identity will become a part of the axioms.

We will write $\alpha\equiv\beta\pmod{2\cdot\pi}$, or briefly
\begin{align*}
\alpha&\equiv\beta
\end{align*}
if $\alpha=\beta+2\cdot\pi\cdot n$
for an integer~$n$.
In this case, we say 
$$\textit{``$\alpha$ is equal to $\beta$ modulo $2\cdot\pi$''}.$$
For example, $-\pi
\equiv
\pi\equiv 3\cdot\pi$ and $\tfrac12\cdot\pi
\equiv
-\tfrac32\cdot\pi$.

The introduced relation ``$\equiv$'' behaves as an equality sign,
but
\[\dots\equiv\alpha-2\cdot\pi\equiv \alpha\equiv \alpha+2\cdot\pi\equiv \alpha+4\cdot\pi\equiv\dots;\] 
that is, if the angle measures differ by full turn,
then they are considered to be the same.

With ``$\equiv$'', we can do addition, subtraction, and multiplication with integer numbers without getting into trouble.
That is, if
$$\alpha\equiv\beta
\quad
\text{and}
\quad
\alpha'\equiv \beta',$$ 
then
$$\alpha+\alpha'\equiv\beta+\beta',
\quad
\alpha-\alpha'\equiv \beta-\beta'
\quad 
\text{and}
\quad
n\cdot\alpha\equiv n\cdot\beta$$
for any integer~$n$.
But ``$\equiv$'' does not in general respect multiplication with non-integer numbers; for example, 
$$\pi
\equiv 
-\pi
\quad
\text{but}
\quad
\tfrac12\cdot\pi
\not\equiv
-\tfrac12\cdot\pi.$$ 

\begin{thm}{Exercise}\label{ex:2a=0}
Show that $2\cdot\alpha\equiv0$ if and only if $\alpha\equiv0$ or $\alpha\equiv\pi$.
\end{thm}

\section{Continuity}

The angle measure is also assumed to be continuous.
Namely, the following property of angle measure will become a part of the axioms:

\textit{The function}
$$\measuredangle\:(A,O,B)\mapsto\measuredangle A O B$$
\textit{is continuous at any triple of points $(A,O,B)$
such that $O\ne A$ and $O\ne B$ and $\measuredangle A O B\ne\pi$.}

To explain this property, we need to extend the notion of {}\emph{continuity} to functions between metric spaces.
The definition is a straightforward generalization of the standard definition for real-to-real functions.

Further, let $\mathcal X$ and $\mathcal Y$ be two metric spaces,
and $d_{\mathcal X}$, $d_{\mathcal Y}$ be their metrics.

A map $f\:\mathcal X\to\mathcal Y$ is called \index{continuous}\emph{continuous} at point $A\in \mathcal X$
if, for any $\epsilon>0$, there is $\delta>0$, such that 
\[d_{\mathcal X}(A,A')
<
\delta
\quad
\Rightarrow
\quad
d_{\mathcal Y}(f(A),f(A'))
<
\epsilon.\]
(Informally it means that sufficiently small changes of $A$ result in arbitrarily small changes of $f(A)$.)

A map $f\:\mathcal X\to\mathcal Y$ is called \index{continuous}\emph{continuous} if it is continuous at every point $A\in \mathcal X$.

One may define a continuous map of several variables the same way.
Assume $f$ is a function that returns a point in $\mathcal Y$ for a triple of points $(A,B,C)$
in~$\mathcal X$.
The map $f$ might be defined only for some triples in~$\mathcal X$.

Assume $f(A,B,C)$ is defined.
Then, we say that $f$ is continuous at the triple $(A,B,C)$ 
if, for any $\epsilon>0$, there is $\delta>0$ such that 
\[d_{\mathcal Y}(f(A,B,C),f(A',B',C'))<\epsilon.\]
if $d_{\mathcal X}(A,A')<\delta$, $d_{\mathcal X}(B,B')<\delta$, and $d_{\mathcal X}(C,C')<\delta$.


\begin{thm}{Exercise}\label{ex:dist-cont}
Let $\mathcal{X}$ be a metric space.
\begin{enumerate}[(a)]
\item\label{ex:dist-cont:a} Let $A\in \mathcal{X}$ be a fixed point.
Show that the function 
$$f\:B\mapsto
d_{\mathcal{X}}(A,B)$$ 
is continuous at any point~$B$.
\item Show that the function $g\:(A,B)\mapsto d_{\mathcal{X}}(A,B)$ is continuous at any pair $A,B\in \mathcal{X}$.
\end{enumerate}

\end{thm}

\begin{thm}{Exercise}\label{ex:comp+cont}
Let $\mathcal{X}$, $\mathcal{Y}$, and $\mathcal{Z}$ be metric spaces.
Assume that the functions $f\:\mathcal{X}\to\mathcal{Y}$
and $g\:\mathcal{Y}\to\mathcal{Z}$ are continuous at any point,
and $h=g\circ f$ is their composition;
that is, $h(A)=g(f(A))$ for any $A\in \mathcal{X}$.
Show that $h\:\mathcal{X}\to\mathcal{Z}$ is continuous at any point.
\end{thm}

\begin{thm}{Exercise}\label{ex:isom-cont}
Show that any distance-preserving map is continuous.
\end{thm}

\section{Congruent triangles}
\label{sec:cong-triangles}

Our next goal is to find a rigorous meaning for statement (\ref{preaxiomIV}) in Section~\ref{preaxioms}.
To do this, we introduce the notion of congruent triangles
so instead of \textit{``if we rotate or shift we will not see the difference''} we say that for triangles, the side-angle-side congruence holds.

An \textit{ordered} triple of distinct points in a metric space $\mathcal{X}$, 
say $A,B,C$,
is called a \index{triangle}\emph{triangle $ABC$}\label{page:def:triangle} (briefly \index{20@$\triangle$}$\triangle A B C$).
Note that the triangles $A B C$ and $A C B$ are different.

Two triangles $A' B' C'$ and $A B C$ are  
\index{triangle!congruent triangles}
\index{congruent!triangles}\emph{congruent}
(briefly  \index{32@$\cong$}$\triangle A' B' C'\z\cong\triangle A B C$) if there is a motion $f\:\mathcal{X}\to\mathcal{X}$ such that 
\[A'\z=f(A),
\quad
B'=f(B)
\quad
\text{and}
\quad
C'=f(C).\]

Let $\mathcal X$ be a metric space,
and $f,g\:\mathcal X\to\mathcal X$ be two motions.
Note that the inverse $f^{-1}:\mathcal X\to\mathcal X$,
as well as the composition $f\circ g:\mathcal X\to\mathcal X$,
are also motions.

It follows that ``$\cong$'' is an \index{equivalence relation}\emph{equivalence relation};
that is, any triangle is congruent to itself, 
and the following two conditions hold:
\begin{itemize} 
\item If $\triangle A' B' C'\z\cong\triangle A B C$, then $\triangle A B C\z\cong\triangle A' B' C'$.
\item If $\triangle A'' B'' C''\z\cong\triangle A' B' C'$ and $\triangle A' B' C'\z\cong\triangle A B C$,
then 
$$\triangle A'' B'' C''\cong\triangle A B C.$$
\end{itemize}


Note that if $\triangle A' B' C'\z\cong\triangle A B C$,
then $AB\z=A'B'$,
$BC=B'C'$ and $CA=C'A'$.

For a discrete metric, as well as some other metrics, 
the converse also holds.
The following example shows that it does not hold in the Manhattan plane:

\parbf{Example.}\label{example:isometric but not congruent} Consider three points 
$A=(0,1)$, $B=(1,0)$, and $C\z=(-1,0)$ on the Manhattan plane $(\mathbb{R}^2,d_1)$.
Note that
$$d_1(A,B)=d_1(A,C)=d_1(B,C)=2.$$

On one hand,
$$\triangle ABC\cong \triangle ACB.$$
Indeed, the map $(x,y)\z\mapsto (-x,y)$ is a motion of $(\mathbb{R}^2,d_1)$
that sends $A\z\mapsto A$, $B\mapsto C$, and $C\z\mapsto B$.

On the other hand,
$$\triangle ABC\ncong \triangle BCA.$$
Indeed, arguing by contradiction, assume that $\triangle ABC\cong \triangle BCA$; that is, there is a motion $f$ of $(\mathbb{R}^2,d_1)$ that sends $A\mapsto B$, $B\mapsto C$, and $C\mapsto A$.

\begin{wrapfigure}[6]{o}{33mm}
\vskip-5mm
\centering
\includegraphics{mppics/pic-6}
\end{wrapfigure}

We say that $M$ is a midpoint of $A$ and $B$ if 
\[d_1(A,M)=d_1(B,M)=\tfrac12\cdot d_1(A,B).\]
Note that a point $M$ is a midpoint of $A$ and $B$ if and only if $f(M)$ is a midpoint of $B$ and~$C$.

The set of midpoints for $A$ and $B$ is infinite, it contains all points $(t,t)$ for $t\in[0,1]$ (it is the gray segment in the picture above).
On the other hand, the midpoint for $B$ and $C$ is unique (it is the black point in the picture).
Thus, the map $f$ cannot be bijective --- a contradiction.

\begin{thm}{Exercise}\label{ex:ncong}
Consider a metric space with four points $\{A,B,C,D\}$ and metric defined by $AB=AC=AD=BC=BD=1$, and 
$CD=2$.
Show that (a) $\triangle ABC\cong \triangle BAC$, and (b) $\triangle ABC\ncong \triangle BCA$.
\end{thm}


\chapter{Inversion}\label{chap:inversion}

Let $\Omega$ be the circle with center $O$ and radius~$r$.
The \index{inversion}\emph{inversion} of a point $P$ across $\Omega$ is the point $P'\in[OP)$ such that
$$OP\cdot OP'=r^2.$$
In this case, the circle $\Omega$ will be called the 
\index{inversion!circle of inversion}\emph{circle of inversion}, 
and its center $O$ is called the \index{inversion!center of inversion}\emph{center of inversion}.

The inverse of $O$ is undefined.

If $P$ is inside $\Omega$, then $P'$ is outside and the other way around. 
Furthermore, $P=P'$ if and only if $P\in \Omega$.

{

\begin{wrapfigure}{r}{37mm}
\vskip-6mm
\centering
\includegraphics{mppics/pic-158}
\end{wrapfigure}

Note that the inversion maps $P'$ back to~$P$.

\begin{thm}{Exercise}\label{ex:constr-inversion}
Let $\Omega$ be a circle centered at~$O$.
Suppose that a line $(PT)$ is tangent to $\Omega$ at~$T$.
Let $P'$ be a footpoint of $T$ on $(OP)$.

Show that $P'$ is the inverse of $P$ across $\Omega$.
\end{thm}

}

\begin{thm}{Lemma}\label{lem:inversion-sim}
Let $\Gamma$ be a circle centered at~$O$.
Assume $A'$ and $B'$ are the inverses of $A$ and $B$ across~$\Gamma$.
Then 
$$\triangle O A B\sim\triangle O B' A'.$$
Moreover,
$$\begin{aligned}
\measuredangle AOB&\equiv -\measuredangle B'OA',
\\
\measuredangle OBA&\equiv -\measuredangle OA'B',
\\
\measuredangle BAO&\equiv -\measuredangle A'B'O.
\end{aligned}\eqlbl{eq:-angle}$$

\end{thm}

\parit{Proof.}
Let $r$ be the radius of the circle of the inversion.

\begin{wrapfigure}[12]{r}{32mm}
\centering
\includegraphics{mppics/pic-160}
\end{wrapfigure}

By the definition of an inversion, 
\begin{align*}
OA\cdot OA'=OB\cdot OB'=r^2.
\end{align*}
Therefore, 
$$\frac{OA}{OB'}=\frac{OB}{OA'}.$$

Clearly,
$$\measuredangle AOB= \measuredangle A'OB'\equiv -\measuredangle B'OA'.\eqlbl{eq:-AOB}$$
From SAS, we get that
$$\triangle O A B\z\sim\triangle O B' A'.$$

Applying Theorem~\ref{thm:signs-of-triug} and \ref{eq:-AOB},
we get \ref{eq:-angle}.
\qeds

\begin{thm}{Exercise}%
\label{ex:appolo-circ}
Let $P'$ be an inverse of $P$ across a circle $\Gamma$.
Assume that $P\ne P'$.
Show that the value $\frac{PX}{P'X}$ is the same for all $X\in\Gamma$.
\end{thm}

The converse to the exercise above also holds.
Namely, given a positive real number $k\ne 1$ 
and two distinct points $P$ and $P'$
the locus of points $X$ such that $\frac{PX}{P'X}=k$ forms a circle which is called the \index{Apollonian circle}\emph{Apollonian circle} (see also Section \ref{sec:Apollonian circle}).
In this case, $P'$ is the inverse of $P$ across the Apollonian circle.

\begin{thm}{Exercise}%
\label{ex:incenter+inversion=orthocenter}
Let $A'$, $B'$, and $C'$ be images of $A$, $B$, and $C$ 
under the inversion across the incircle of $\triangle A B C$.
Show that the incenter of $\triangle A B C$ 
is the orthocenter of $\triangle A' B' C'$.
\end{thm}

\begin{thm}[!]{Exercise}\label{ex:consturuction-of-inversion}
Construct an inverse of a given point across a given circle using ruler and compass.
\end{thm}

\section{The cross-ratio}

The following theorem lists quantities that do not change after inversion.

\begin{thm}{Theorem}\label{lem:inverse-4-angle}
Let $ABCD$ and $A'B'C'D'$  be two quadrangles
such that the points $A'$, $B'$, $C'$, and $D'$ are the inverses of $A$, $B$, $C$, and $D$ respectively.
Then 
\begin{enumerate}[(a)]
\item\label{lem:inverse-4-angle:cross-ratio} $$\frac{AB\cdot CD}{BC\cdot DA}= \frac{A'B'\cdot C'D'}{B'C'\cdot D'A'}.$$
\item\label{lem:inverse-4-angle:angle} 
$$\measuredangle ABC+\measuredangle CDA\equiv -(\measuredangle A'B'C'+\measuredangle C'D'A').$$
\item\label{lem:inverse-4-angle:inscribed}
If the quadrangle $ABCD$ is inscribed, 
then so is $\square A'B'C'D'$.
\end{enumerate}
\end{thm}

The number $\frac{AB \cdot CD}{BC \cdot DA}$ from \textit{(\ref{lem:inverse-4-angle:cross-ratio})} is called the \index{cross-ratio}\emph{cross-ratio} of the four points $A$, $B$, $C$, and $D$.

\parit{Proof; (\ref{lem:inverse-4-angle:cross-ratio}).}
Let $O$ be the center of the inversion.
According to Lemma~\ref{lem:inversion-sim},
$\triangle AOB\z\sim \triangle B'OA'$.
Therefore, 
\begin{align*}
&&\frac{AB}{A'B'} &=\frac{OA}{OB'}.
\intertext{Analogously,}
\frac{BC}{B'C'}&=\frac{OC}{OB'},&
\frac{CD}{C'D'}&=\frac{OC}{OD'},&
\frac{DA}{D'A'}&=\frac{OA}{OD'}.
\end{align*}

Therefore, 
\begin{align*}
\frac{AB}{A'B'}\cdot \frac{CD}{C'D'}&=\frac{OA}{OB'}\cdot\frac{OC}{OD'}=
\frac{OC}{OB'}\cdot\frac{OA}{OD'}
=\frac{BC}{B'C'}\cdot\frac{DA}{D'A'}.
\end{align*}
Hence \textit{(\ref{lem:inverse-4-angle:cross-ratio})} follows.

\parit{(\ref{lem:inverse-4-angle:angle}).}
According to Lemma~\ref{lem:inversion-sim},
\[\begin{aligned}
\measuredangle ABO&\equiv -\measuredangle B'A'O,
&
\measuredangle OBC&\equiv -\measuredangle OC'B',\\
\measuredangle CDO&\equiv -\measuredangle D'C'O,
&
\measuredangle ODA&\equiv -\measuredangle OA'D'.
\end{aligned}\eqlbl{eq:angle=-angle}\]
By Axiom~\ref{def:birkhoff-axioms:2b},
\begin{align*}
\measuredangle ABC&\equiv\measuredangle ABO+\measuredangle OBC,
&
\measuredangle D'C'B'&\equiv\measuredangle D'C'O+\measuredangle OC'B',
\\
\measuredangle CDA&\equiv\measuredangle CDO+\measuredangle ODA,
&
\measuredangle B'A'D'&\equiv\measuredangle B'A'O+\measuredangle OA'D'.
\end{align*}
Therefore, 
summing the four identities in \ref{eq:angle=-angle}, we get that
\begin{align*}
\measuredangle ABC+\measuredangle CDA
&\equiv -(\measuredangle D'C'B'+\measuredangle B'A'D').
\intertext{Applying Axiom~\ref{def:birkhoff-axioms:2b} and Exercise~\ref{ex:quadrangle}, we get that}
\measuredangle A'B'C'+\measuredangle C'D'A'
&\equiv -(\measuredangle B'C'D'+\measuredangle D'A'B')\equiv
\\
&\equiv \measuredangle D'C'B'+\measuredangle B'A'D'.
\end{align*}
Hence \textit{(\ref{lem:inverse-4-angle:angle})} follows.

\parit{(\ref{lem:inverse-4-angle:inscribed}).}
Follows by \textit{(\ref{lem:inverse-4-angle:angle})} and Corollary~\ref{cor:inscribed-quadrangle}.
\qeds

\section{Inversive plane and circlines}

Let $\Omega$ be a circle with the center $O$ and the radius~$r$.
Consider the inversion across~$\Omega$.

Recall that the inverse of $O$ is undefined.
To deal with this problem it is useful to add an extra point to the plane, which will be called the \index{point!at infinity}\emph{point at infinity}, denoted as~\index{50@$\infty$}$\infty$.
We can assume that $\infty$ is the inverse of $O$ and the other way around.

The Euclidean plane with the added point at infinity is called the \index{plane!inversive plane}\index{inversive!plane}\emph{inversive plane}.

We will always assume that every line and half-line contains~$\infty$.

Recall that
\index{circline}\emph{circline}
means \textit{circle or line}.
Therefore we may say 
\textit{``if a circline contains $\infty$, then it is a line''} or \textit{``a circline that does not contain $\infty$  is a circle''}.

According to Theorem~\ref{thm:circumcenter}, 
for every $\triangle ABC$ there is a unique circline that passes thru $A$, $B$, and~$C$
(if $\triangle ABC$ is degenerate, then this is a line, and if not it is a circle).

\begin{thm}{Theorem}\label{thm:inverse-cline}
In the inversive plane, the inverse of a circline is a circline.
\end{thm}

Exercise~\ref{ex:inversions-inversive} gives a partial converse to the theorem.

\parit{Proof.}
Let $O$ be the center of the inversion, and let $r$ be its radius.

Choose a circline $\Gamma$ and three distinct points $A$, $B$, and $C$ on it.
We may assume that $A \ne O$, $B \ne O$, and $C \ne O$.
(If~$\triangle ABC$ is nondegenerate, then $\Gamma$ is the circumcircle of $\triangle ABC$; if $\triangle ABC$ is degenerate, then $\Gamma$ is the line passing thru $A$, $B$, and~$C$.)

Let $A'$, $B'$, and $C'$ be the inverses of $A$, $B$, and $C$, respectively,
and let $\Gamma'$ be the circline passing thru $A'$, $B'$, and~$C'$.

Assume $D$ is a point in the inversive plane, distinct from $A$, $B$, $C$, $O$, and~$\infty$.
Let $D'$ be the inverse of~$D$.
By Theorem~\ref{lem:inverse-4-angle}\textit{\ref{lem:inverse-4-angle:inscribed}},
$D' \in \Gamma'$ if and only if $D \in \Gamma$.

It remains to prove that
$\infty \in \Gamma$ $\Leftrightarrow$ $O \in \Gamma'$, which is equivalent to
$O \in \Gamma$ $\Leftrightarrow$ $\infty \in \Gamma'$.

The condition $\infty \in \Gamma$ means that $\Gamma$ is a line;
equivalently, for any $\epsilon > 0$, the circline $\Gamma$ contains a point $P \ne \infty$ such that $OP > r/\epsilon$.
For the inversion $P' \in \Gamma'$ of $P$, we have $OP' = r^2 / OP < r \cdot \epsilon$.
That is, the circline $\Gamma'$ contains points arbitrarily close to~$O$.
It follows that $O \in \Gamma'$.
In other words,
$\infty \in \Gamma$
$\Rightarrow$
$O \in \Gamma'$.

The implication
$\infty \in \Gamma$
$\Leftarrow$
$O \in \Gamma'$ is proved in the same way.
We can choose $P' \ne O$ on $\Gamma'$ such that $OP' < r \cdot \epsilon$.
Its inverse $P$ lies on $\Gamma$, and $OP = r^2 / OP' > r/\epsilon$.
Since $\epsilon>0$ is arbitrary, we have $\infty \in \Gamma$.
\qeds



\begin{thm}[!]{Exercise}\label{ex:inv-center not=center-inv}
Assume that a circle $\Gamma'$ is an inverse of a circle $\Gamma$ centered at $Q$.
Suppose $Q'$ denotes the inverse of~$Q$.
Show that $Q'$ is not the center of~$\Gamma'$.
\end{thm}

{%%%!!! move pic up if possible

\begin{wrapfigure}{r}{37mm}
\vskip-8mm
\centering
\includegraphics{mppics/pic-162}
\end{wrapfigure}

Assume that a \index{circumtool}\emph{circumtool} is a geometric construction tool 
that produces a circline passing thru three given points.

\begin{thm}{Exercise}\label{ex:circumtool}
Show that with only a circumtool,
it is impossible to construct the center of a given circle.
\end{thm}

}

\begin{thm}{Exercise}\label{ex:tangent-circ->parallels}
Show that for every pair of tangent circles in the inversive plane, there is an inversion that sends them to a pair of parallel lines.
\end{thm}



\begin{thm}{Theorem}\label{thm:inverse}
Consider an inversion of the inversive plane across a circle $\Omega$ centered at~$O$. 
Then 
\begin{enumerate}[(a)]
\item\label{thm:inverse:line-line}
A line passing thru $O$ is inverted into itself.
\item\label{thm:inverse:line} 
A line not passing thru $O$ is inverted into a circle that passes thru $O$, and the other way around.
\item\label{thm:inverse:circle} 
A circle not passing thru $O$ 
is inverted into a circle not passing thru~$O$. 
\end{enumerate}
\end{thm}

\parit{Proof.}
In the proof, we use Theorem~\ref{thm:inverse-cline} without mentioning it.

\parit{(\ref{thm:inverse:line-line}).}
If a line passes thru $O$, then it contains both $\infty$ and~$O$.
Therefore, its inverse also contains $\infty$ and~$O$.
In particular, the image is a line passing thru~$O$.

\parit{(\ref{thm:inverse:line}).}
Since each line $\ell$ passes thru $\infty$, its image $\ell'$ has to contain~$O$.
If the line does not contain $O$, 
then $\ell'\not\ni \infty$;
that is, $\ell'$ is not a line.
Therefore, $\ell'$ is a circle that passes thru~$O$. 

\parit{(\ref{thm:inverse:circle}).}
If the circle $\Gamma$ does not contain $O$, 
then its image $\Gamma'$ does not contain~$\infty$.
Therefore, $\Gamma'$ is a circle.
Since  $\Gamma\not\ni\infty$ we get that $\Gamma' \not\ni O$.
Hence the result.
\qeds

\section{The method of inversion}

Here is an application of inversion,
which we include as an illustration;
we will not use it further in the book.

\begin{thm}{Ptolemy's identity}\label{ptolemy-id}
Let $ABCD$ be an inscribed quadrangle.
Assume that points $A$, $B$, $C$, and $D$ appear on the circline in the same order.
Then 
$$ AB\cdot CD+ BC\cdot DA=AC\cdot BD.$$

\end{thm}

\parit{Proof.}
Assume the points $A$, $B$, $C$, and $D$ lie on one line in this order.

\begin{wrapfigure}{o}{39mm}
\centering
\includegraphics{mppics/pic-164}
\end{wrapfigure}

Set $x\z=AB$, $y=BC$, $z\z=CD$.
Note that
$$x\cdot z+y\cdot (x+y+z)=(x+y)\cdot(y+z).$$
Since $AC\z=x+y$, $BD=y+z$, and $DA\z=x+y+z$,
it proves the identity.

\begin{wrapfigure}{o}{39mm}
\centering
\includegraphics{mppics/pic-166}
\end{wrapfigure}

It remains to consider the case when the quadrangle $ABCD$ is inscribed in a circle, say~$\Gamma$. 

The identity can be rewritten as 
$$\frac{AB\cdot DC}{ BD\cdot CA}+ \frac{BC\cdot AD}{CA\cdot DB}=1.$$
On the left-hand side we have two cross-ratios.
According to Theorem~\ref{lem:inverse-4-angle}\textit{\ref{lem:inverse-4-angle:cross-ratio}}, the left-hand side does not change if we apply an inversion to each point.

Consider an inversion across a circle centered at point $O$ that lies on $\Gamma$ between $A$ and~$D$.
By 
Theorem~\ref{thm:inverse},
this inversion maps $\Gamma$ to a line.
This reduces the problem to the case when $A$, $B$, $C$, and $D$ lie on one line, which was already considered.
\qeds

{

\begin{wrapfigure}{r}{27mm}
\vskip-4mm
\centering
\includegraphics{mppics/pic-168}
\vskip4mm
\includegraphics{mppics/pic-170}
\end{wrapfigure}


In the proof above, we rewrite Ptolemy's identity in a form that is invariant with respect to inversion 
and then apply an inversion which makes the statement evident.
The solution of the following exercise is based on the same idea;
one has to make a right choice of inversion.



\begin{thm}{Exercise}\label{ex:4-circles}
Assume that four circles are mutually tangent to each other.
Show that four (among six) of their points of tangency lie on one circline.
\end{thm}


\begin{thm}{Advanced exercise}\label{ex:inverse}
Assume that three circles are tangent to each other and to two parallel lines as shown in the picture.

Show that $(AB)$ is tangent to two circles at~$A$.
\end{thm}

}

\section{Perpendicular circles}

Assume two circles $\Gamma$ and $\Omega$ intersect at two points $X$ and~$Y$.
Let $\ell$ and $m$ be the tangent lines at $X$ to $\Gamma$ and $\Omega$ respectively.
Analogously, let $\ell'$ and $m'$ be the tangent lines at $Y$ to $\Gamma$ and~$\Omega$.

From Exercise~\ref{ex:two-arcs}, we can deduce that  
 $\ell\perp m$ if and only if $\ell'\perp m'$.

We say that the circle $\Gamma$ is {}\emph{perpendicular} to the circle $\Omega$ 
(briefly \index{38@$\perp$}$\Gamma\perp \Omega$)
if they intersect and the lines tangent to the circles at one point 
(and therefore, both points) 
of intersection are perpendicular.

Similarly, we say that the circle $\Gamma$ is perpendicular to the line $\ell$ (briefly $\Gamma\perp \ell$)
if $\Gamma\cap\ell\ne \emptyset$ and $\ell$ is perpendicular to the tangent lines to $\Gamma$ at one point (and therefore, both points) of intersection.
According to Lemma~\ref{lem:tangent}, 
this happens only if the line $\ell$ passes thru the center of~$\Gamma$.

Now we can talk about \index{perpendicular!circlines}\emph{perpendicular circlines}.

\begin{thm}{Theorem}\label{thm:perp-inverse}
Assume $\Gamma$ and $\Omega$ are distinct circles. 
Then $\Omega\perp\Gamma$ if and only if the circle $\Gamma$ coincides with its inversion across~$\Omega$.
\end{thm}

\begin{wrapfigure}[7]{o}{29mm}
\vskip-4mm
\centering
\includegraphics{mppics/pic-172}
\end{wrapfigure}

\parit{Proof.} 
Suppose that $\Gamma'$ denotes the inverse of~$\Gamma$.

\parit{``Only if'' part.}
Let $O$ be the center of $\Omega$
and $Q$ be the center of~$\Gamma$.
Let $X$ and $Y$ denote the points of intersections of  $\Gamma$ and~$\Omega$.
By Lemma~\ref{lem:tangent}, $\Gamma\perp\Omega$ if and only if $(OX)$ and $(OY)$ are tangent to~$\Gamma$.

Since $O\ne X$, Lemma \ref{lem:perp<oblique} implies that $O$ lies outside of~$\Gamma$.
By Theorem \ref{thm:inverse}\textit{\ref{thm:inverse:circle}}, $\Gamma'$ is a circle.

Note that $\Gamma'$ is also tangent to $(OX)$ and $(OY)$ at $X$ and $Y$ respectively. 
It follows that $X$ and $Y$ are the footpoints of the center of $\Gamma'$ on $(OX)$ and $(OY)$.
Therefore, both $\Gamma'$ and $\Gamma$ have the center~$Q$.
Finally, $\Gamma'=\Gamma$, since both circles pass thru~$X$.

\parit{``If'' part.}
Assume $\Gamma=\Gamma'$.

Since $\Gamma\ne \Omega$, there is a point $P$ that lies on $\Gamma$, but not on~$\Omega$.
Let $P'$ be the inverse of $P$ across~$\Omega$.
Since $\Gamma=\Gamma'$, we have that $P'\in \Gamma$.
In particular, the half-line $[OP)$ intersects $\Gamma$ at two points.
By Exercise~\ref{ex:inside-outside}, 
 $O$ lies outside of~$\Gamma$.

As $\Gamma$ has points inside and outside of $\Omega$,
the circles $\Gamma$ and $\Omega$ intersect.
The latter follows from Exercise~\ref{ex:intersecting-circles-3}.

Let $X$ be a point of their intersection.
We need to show that $(OX)$ is tangent to $\Gamma$, which means $X$ is the only intersection point of $(OX)$ and~$\Gamma$.

Assume $Z$ is another point of intersection of $(OX)$ and~$\Gamma$.
Since $O$ is outside of $\Gamma$, 
the point $Z$ lies on the half-line $[OX)$.

Suppose that $Z'$ denotes the inverse of $Z$ across~$\Omega$.
Clearly, the three points $Z, Z', X$ lie on $\Gamma$ and $(OX)$, which contradicts Lemma~\ref{lem:line-circle}.
\qeds 

It is convenient to define the 
\index{inversion!inversion across the line}\emph{inversion across the line} $\ell$
as the reflection across $\ell$.
This way we can talk about \index{inversion!inversion across the circline}\emph{inversion across an arbitrary circline}.

\begin{thm}{Corollary}\label{cor:perp-inverse-clines}
Let $\Omega$  and $\Gamma$ be distinct circlines in the inversive plane.
Then
the inversion across $\Omega$ sends $\Gamma$ to itself if and only if $\Omega\perp\Gamma$.
\end{thm}

\parit{Proof.}
By Theorem~\ref{thm:perp-inverse}, it is sufficient to consider the case when $\Omega$ or $\Gamma$ is a line.

Assume $\Omega$ is a line, so the inversion across $\Omega$ is a reflection.
In this case, the statement follows from Corollary~\ref{cor:reflection+angle}.

If $\Gamma$ is a line, 
then the statement follows from Theorem~\ref{thm:inverse}.
\qeds

\begin{thm}{Corollary}\label{cor:perp-inverse}
Let $P$ and $P'$ be two distinct points
such that $P'$ is the inverse of $P$ across the circle~$\Omega$.
If a circline $\Gamma$ passes thru $P$ and~$P'$, then $\Gamma\perp\Omega$.
\end{thm}

\parit{Proof.} 
Without loss of generality, we may assume that $P$ is inside and $P'$ is outside~$\Omega$.
By Theorem~\ref{thm:abc}, $\Gamma$ intersects $\Omega$.
Suppose that $A$ denotes a point of intersection.

Suppose that $\Gamma'$ denotes the inverse of~$\Gamma$.
Since $A$ is a self-inverse, the points $A$, $P$, and $P'$ lie on~$\Gamma'$.
By Exercise~\ref{ex:unique-cline},
$\Gamma'=\Gamma$
and by Theorem~\ref{thm:perp-inverse}, $\Gamma\perp\Omega$.
\qeds

\begin{thm}{Corollary}\label{cor:h-line} 
Let $P$ and $Q$ be two distinct points inside a circle~$\Omega$.
Then there is a unique circline $\Gamma$ perpendicular to $\Omega$ that passes thru $P$ and~$Q$.  
\end{thm}

\parit{Proof.}
Let $P'$ be the inverse of the point $P$ across the circle~$\Omega$.
According to Corollary~\ref{cor:perp-inverse},
if a circline that passes thru $P$ and $Q$ is perpendicular to $\Omega$, then it passes thru~$P'$, and the converse holds as well.

Note that $P'$ lies outside of~$\Omega$.
Therefore, the points $P$, $P'$, and $Q$ are distinct.

According to Exercise~\ref{ex:unique-cline},
there is a unique circline passing thru $P$, $Q$, and~$P'$.
Hence the result.
\qeds

\begin{thm}{Exercise}\label{ex:inscribed+inv}
Let $P$, $Q$, $P'$, and $Q'$ be points in the Euclidean plane.
Assume $P'$ and $Q'$ are inverses of $P$ and $Q$ respectively.
Show that the quadrangle $PQP'Q'$ is inscribed.
\end{thm}

\begin{thm}{Exercise}\label{ex:centers-of-perp-circles}
Let $\Omega_1$ and $\Omega_2$ be two perpendicular circles with centers at $O_1$ and $O_2$ respectively.
Show that the inverse of $O_1$ across $\Omega_2$ 
coincides with 
the inverse of $O_2$ across~$\Omega_1$.
\end{thm}

\begin{thm}{Exercise}\label{ex:4-th-perp-circ}
Three distinct circles --- $\Omega_1$, $\Omega_2$, and $\Omega_3$, intersect at two points --- $A$ and~$B$.
Assume that a circle $\Gamma$ is perpendicular to $\Omega_1$ and $\Omega_2$.
Show that $\Gamma\perp\Omega_3$.
\end{thm}

Let us consider two new construction tools:
the \index{circumtool}\emph{circumtool} that constructs a circline thru three given points, 
and the \index{inversor}\emph{inversor} --- a tool that constructs an inverse of a given point across a given circline.

\begin{thm}{Exercise}\label{ex:construction-perp-clines}
Given  two circles $\Omega_1$, $\Omega_2$ and a point $P$ that does not lie on the circles,
use only circumtool and inversor to construct a circline $\Gamma$ thru $P$, 
and perpendicular to both $\Omega_1$ and $\Omega_2$.
\end{thm}

\begin{thm}{Advanced exercise}\label{ex:3-construction-perp-clines}
Given  three disjoint circles $\Omega_1$, $\Omega_2$, and $\Omega_3$,
use only circumtool and inversor to construct a circline $\Gamma$ that is perpendicular to each circle $\Omega_1$, $\Omega_2$, and $\Omega_3$.

Think what to do if two of the circles intersect.
\end{thm}

\section{Angles after inversion}

\begin{thm}{Proposition}
In the inversive plane, an inverse of an arc is an arc.

\end{thm}

\parit{Proof.} 
Consider four distinct points $A$, $B$, $C$, and $D$; 
let $A'$, $B'$, $C'$, and $D'$  be their inverses.
We need to show that $D$ lies on the arc $ABC$ if and only if $D'$ lies on the arc $A'B'C'$.
According to Proposition~\ref{prop:arcs},
the latter is equivalent to the following:
$$\measuredangle ADC= \measuredangle ABC
\quad
\iff
\quad  
\measuredangle A'D'C'= \measuredangle A'B'C'.$$
The latter follows from Theorem~\ref{lem:inverse-4-angle}\textit{\ref{lem:inverse-4-angle:angle}}.
\qeds

The following theorem states that the angle between arcs changes only its sign after the inversion.

{

\begin{wrapfigure}{r}{55mm}
\vskip-6mm
\centering
\includegraphics{mppics/pic-174}
\end{wrapfigure}

\begin{thm}{Theorem}\label{thm:angle-inversion}
Let $AB_1C_1$, $AB_2C_2$ be two arcs in the inversive plane,
and arcs $A'B_1'C_1'$, $A'B_2'C_2'$ be their inverses.
Let $[AX_1)$ and $[AX_2)$ be the half-lines tangent to $AB_1C_1$ and  $AB_2C_2$ at $A$,
and
$[A'Y_1)$ and $[A'Y_2)$ be the half-lines tangent to $A'B_1'C_1'$ and  $A'B_2'C_2'$ at~$A'$.
Then
$$\measuredangle X_1AX_2\equiv-\measuredangle Y_1A'Y_2.$$

\end{thm}

}

{

\begin{wrapfigure}[6]{o}{21mm}
\vskip-6mm
\centering
\includegraphics{mppics/pic-176}
\end{wrapfigure}

The \index{angle!between arcs}\emph{angle between arcs} can be defined as the angle between their tangent half-lines at the common endpoint.
Therefore under inversion, the angles between arcs are preserved up to sign.


From Exercise~\ref{ex:tangent-lim}, it follows that the angle between arcs with the common endpoint $A$ is the limit of $\measuredangle P_1AP_2$ where $P_1$ and $P_2$ are points approaching $A$ along the corresponding arcs. 
This observation can be used to define the angle between a pair of curves emerging from one point.
It turns out that under inversion, angles between curves are also preserved up to sign.

}

\parit{Proof.}
By Proposition~\ref{prop:arc(angle=tan)},
\begin{align*}
\measuredangle X_1AX_2&\equiv\measuredangle X_1AC_1+\measuredangle C_1AC_2+\measuredangle C_2AX_2\equiv
\\
&\equiv(\pi-\measuredangle C_1B_1A)+\measuredangle C_1AC_2+(\pi-\measuredangle AB_2 C_2)\equiv
\\
&
\equiv -(\measuredangle C_1B_1A+\measuredangle AB_2 C_2 +\measuredangle C_2 A C_1)\equiv
\\
&\equiv 
-(\measuredangle C_1B_1A+\measuredangle AB_2 C_1)
-(\measuredangle C_1B_2C_2 +\measuredangle C_2 A C_1).
\intertext{The same way, we get}
\measuredangle Y_1A'Y_2
&\equiv-(\measuredangle C_1'B_1'A'+\measuredangle A'B_2' C_1')
-(\measuredangle C_1'B_2'C_2' +\measuredangle C_2' A' C_1').
\end{align*}

By Theorem~\ref{lem:inverse-4-angle}\textit{\ref{lem:inverse-4-angle:angle}},
\begin{align*}
\measuredangle C_1B_1A+\measuredangle AB_2 C_1&\equiv-(\measuredangle C_1'B_1'A'+\measuredangle A'B_2' C_1'),
\\
\measuredangle C_1B_2C_2 +\measuredangle C_2 A C_1&\equiv-(\measuredangle C_1'B_2'C_2' +\measuredangle C_2' A' C_1'),
\end{align*}
and hence the result.\qeds

\begin{thm}{Corollary}\label{cor:invese-comp}
Let $P$ be an inverse of a point $Q$ across a circle $\Gamma$.
Assume that $P'$, $Q'$, and $\Gamma'$ 
are the inverses of  $P$, $Q$, and $\Gamma$ across another circle $\Omega$.
Then $P'$ is the inverse  of $Q'$ across~$\Gamma'$.
\end{thm}

{

\begin{wrapfigure}{r}{45mm}
\vskip-6mm
\centering
\includegraphics{mppics/pic-178}
\end{wrapfigure}

\parit{Proof.}
If $P=Q$, then $P'=Q'\z\in\Gamma'$. 
Therefore, $P'$ is the inverse of $Q'$ across~$\Gamma'$.

It remains to consider the case $P\z\ne Q$. 
Let $\Delta_1$ and $\Delta_2$ be two distinct circles that intersect at $P$ and~$Q$.
According to Corollary~\ref{cor:perp-inverse}, 
$\Delta_1\perp\Gamma$ and $\Delta_2\perp\Gamma$.

Let $\Delta_1'$ and $\Delta_2'$ denote the inverses of $\Delta_1$ and $\Delta_2$ across~$\Omega$.
Clearly, $\Delta_1'$ meets $\Delta_2'$ at $P'$ and~$Q'$.

By Theorem~\ref{thm:angle-inversion},  $\Delta_1'\perp\Gamma'$ and $\Delta_2'\z\perp\Gamma'$.
By Corollary~\ref{cor:perp-inverse-clines}, $P'$ is the inverse of $Q'$ across~$\Gamma'$.
\qeds

}

\backmatter
%\addtocontents{toc}{\protect\contentsline{part}{\protect\numberline{}References}{}{}}

\chapter{Hints}
%\addcontentsline{toc}{chapter}{Hints}
%\subsection*{Chapter~\ref{chap:metr}}

\raggedcolumns\setlength{\multicolsep}{-7mm}
\spell{\begin{multicols}{2}}{}

\refstepcounter{chapter}
\setcounter{eqtn}{0}

\parbf{\ref{ex:dist-square}.} Check the triangle inequality for $A\z=0$, $B\z=1$, and $C\z=2$.

\parbf{\ref{ex:d_1+d_2+d_infty}.} 
Check all the conditions in \ref{def:metric-space}.
Let us discuss the triangle inequality --- the remaining conditions are self-evident.

Let $A=(x_A,y_A)$, $B=(x_B,y_B)$, and $C=(x_C,y_C)$.
Set 
\begin{align*}
x_1&=x_B-x_A, 
&
y_1&=y_B-y_A,
\\
x_2&=x_C-x_B,
&
y_2&=y_C-y_B.
\end{align*}

\parit{(a).}
The inequality
$$d_1(A,C)\le d_1(A,B)+d_1(B,C)$$
can be written as 
$$|x_1+x_2|+|y_1+y_2|
\le 
|x_1|+|y_1|+|x_2|+|y_2|.$$
The latter follows since $|x_1+x_2|\le |x_1|+|x_2|$ 
and
$|y_1+y_2|\le |y_1|+|y_2|$.

\parit{(b).}
The inequality
$$d_2(A,C)\le d_2(A,B)+d_2(B,C)\eqlbl{eq:trig-inq-d2}$$
can be written as 
\begin{align*}
&\sqrt{\bigl(x_1+x_2\bigr)^2+\bigl(y_1+y_2\bigr)^2}\le
\\
&\qquad\le
\sqrt{x_1^2+y_1^2}+\sqrt{x_2^2+y_2^2}.
\end{align*}
Take the square of the left and the right-hand sides,
simplify,
take the square again, and simplify again.
You should get the following inequality:
$$0
\le 
(x_1\cdot y_2-x_2\cdot y_1)^2,$$
which is equivalent to \ref{eq:trig-inq-d2}
and evidently true.

\parit{(c).}
The inequality
$$d_\infty(A,C)\le d_\infty(A,B)+d_\infty(B,C)$$
can be written as 
$$
\begin{aligned}
&\max\{|x_1+x_2|,|y_1+y_2|\}\le
\\
&\qquad\le 
\max\{|x_1|,|y_1|\}+\max\{|x_2|,|y_2|\}.
\end{aligned}
\eqlbl{eq:max-trig}$$
Without loss of generality, we may assume that 
$$\max\{|x_1+x_2|,|y_1+y_2|\}=|x_1+x_2|.$$
Furthermore,
\begin{align*}
|x_1+x_2|&\le |x_1|+|x_2|\le 
\\
&\le\max\{|x_1|,|y_1|\}+\max\{|x_2|,|y_2|\}.
\end{align*}
Hence \ref{eq:max-trig} follows.

\parbf{\ref{ex:4-triangle}.} Sum up four triangle inequalities.

\parbf{\ref{ex:dist-preserv=>injective}.}
If $A\ne B$, then $d_\mathcal{X}(A,B)>0$.
Since $f$ is distance-preserving,
$$d_\mathcal{Y}(f(A),f(B))=d_\mathcal{X}(A,B).$$
Therefore, $d_\mathcal{Y}(f(A),f(B))>0$; hence $f(A)\z\ne f(B)$.

\parbf{\ref{ex:motion-of-R}.}
Set $f(0)=a$ and $f(1)=b$.
Show that $b=a+1$ or $a-1$.
Moreover, $f(x)=a\pm x$, and at the same time, $f(x)=b\pm(x-1)$ for any~$x$.

Suppose $b=a+1$. 
Show that 
$f(x)=a+x$ for any~$x$.

In the same way, if $b=a-1$, 
show that 
$f(x)=a-x$ for any~$x$.

\parbf{\ref{ex:d_1=d_infty}.} 
Show that the map $(x,y)\mapsto (x+y,x-y)$ is an isometry  $(\mathbb{R}^2,d_1)\z\to (\mathbb{R}^2,d_\infty)$.
You need to check if this map is bijective and distance-preserving.

\parbf{\ref{ad-ex:motions of Manhattan plane}.} 
First prove that \textit{two points $A=(x_A,y_A)$ and $B\z=(x_B,y_B)$ on the Manhattan plane have a unique midpoint if and only if $x_A=x_B$ or $y_A=y_B$}; compare with the example in \ref{sec:cong-triangles}. 

Use the above statement to prove that
any motion of the Manhattan plane 
can be written in one of the following eight ways:
\begin{align*}
(x,y)&\mapsto (\pm x+a,\pm y+b)
\shortintertext{or} 
(x,y)&\mapsto (\pm y+b,\pm x+a),
\end{align*}
for fixed real numbers $a$ and~$b$.
In each case, we have 4 choices of signs, so for a fixed pair $(a,b)$ we have 8 distinct motions.

\parbf{\ref{ex:y=|x|}.}
Assume three points $A$, $B$, and $C$ lie on one line.
Note that in this case one of three triangle inequalities involving the points $A$, $B$, and $C$ becomes an equality.


Set $A=(-1,1)$, $B=(0,0)$, and $C=(1,1)$.
Show that for $d_1$ and $d_2$
all the triangle inequalities with the points $A$, $B$, and $C$ are strict.
It follows that the graph is not a line.

For $d_\infty$ show that $(x,|x|)\mapsto x$ is a isometry from the graph to~$\mathbb{R}$.
Conclude that the graph is a line in $(\mathbb{R}^2,d_\infty)$.

\parbf{\ref{ex:line-motion}.}
Spell the definitions of line and motion.

\parbf{\ref{ex:trig==}.}
Fix an isometry $f\:(P Q)\to \mathbb{R}$ such that $f(P)=0$ and $f(Q)\z=q>0$.

Assume that $f(X)=x$.
By the definition of a half-line $X\in[PQ)$ if and only if $x\ge 0$.
Show that the latter holds if and only if 
$|x-q|=\bigl||x|-|q|\bigr|$.
Hence \textit{(a)} follows.

To prove\textit{(b)}, observe that $X\in [PQ]$ if and only if $0\le x\le q$.
Show that the latter holds if and only if 
$|x-q|+|x|=|q|$.

\parbf{\ref{ex:2a=0}.}
The equation
$2\cdot\alpha\equiv 0$
means that $2\cdot\alpha\z=2\cdot k\cdot\pi$ for an integer~$k$.
Therefore,
$\alpha=k\cdot\pi$.

Equivalently, $\alpha=2\cdot n\cdot \pi$ or $\alpha=(2\cdot n+1)\cdot \pi$ for an integer~$n$.
In these cases, we have $\alpha\equiv 0$ or $\alpha\equiv \pi$ respectively.

\parbf{\ref{ex:a+b==c}.}
Observe that $\gamma'=\alpha+\beta\in [0,2\cdot \pi]$.
Show and use that if $\gamma'\equiv\gamma$, then $\gamma'=\gamma$.

\parbf{\ref{ex:dist-cont}.} \textit{(a).}
By the triangle inequality,
$|f(A')\z-f(A)|\le d(A',A)$.
Therefore, we can take $\delta=\epsilon$.

\parit{(b).}
By the triangle inequality,
\begin{align*}
&|g(A',B')-g(A,B)|
\le 
\\
&\le|g(A',B')-g(A,B')|+
\\
&\quad+
|g(A,B')-g(A,B)|\le
\\
&\le d(A',A)+d(B',B).
\end{align*}
Therefore, we can take $\delta=\tfrac\epsilon2$.

\parbf{\ref{ex:comp+cont}.}
Fix $A\in \mathcal{X}$ and $B\in\mathcal{Y}$
such that $f(A)=B$.

Fix $\epsilon>0$.
Since $g$ is continuous at $B$, there is a positive value $\delta_1$ such that 
$$d_{\mathcal{Z}}(g(B'),g(B))<\epsilon
\quad
\text{if}
\quad
d_{\mathcal{Y}}(B',B)<\delta_1.$$ 

Since $f$ is continuous at $A$, there is $\delta_2>0$ such that 
$$d_{\mathcal{Y}}(f(A'),f(A))\z<\delta_1
\quad
\text{if}
\quad
d_{\mathcal{X}}(A',A)<\delta_2.$$ 

Since $f(A)=B$, we get that
$$d_{\mathcal{Z}}(h(A'),h(A))<\epsilon
\quad
\text{if}
\quad
d_{\mathcal{X}}(A',A)<\delta_2.$$ 
Hence the result.

\parbf{\ref{ex:ncong}.} \textit{(a).}
Show that $A\mapsto B$, $B\mapsto A$, $C\mapsto C$, $D\mapsto D$ defines a motion of the space;
conclude that $\triangle ABC\cong \triangle BAC$.

\parit{(b).} Suppose $\triangle ABC\cong \triangle BCA$; so, there is a motion $m$ that maps $A\mapsto B$, $B\mapsto C$, and $C\mapsto A$.
Show that $m\:D\mapsto D$; arrive at a contradiction.


%\subsection*{Chapter~\ref{chap:axioms}}
\refstepcounter{chapter}
\setcounter{eqtn}{0}

\parbf{\ref{ex:infinite}.} By Axiom~\ref{def:birkhoff-axioms:0}, there are at least two points in the plane.
Therefore, by Axiom~\ref{def:birkhoff-axioms:1}, 
the plane contains a line. 
To prove \textit{(a)}, it remains to note that a line is an infinite set of points.
To prove \textit{(b)} apply Axiom~\ref{def:birkhoff-axioms:2} in addition.

\parbf{\ref{ex:[OA)=[OA')}.}
By Axiom~\ref{def:birkhoff-axioms:1},
$(OA)=(OA')$.
Therefore, the statement boils down to the following:

\textit{Assume $f\:\mathbb{R}\to \mathbb{R}$ is a motion of the line that sends $0\mapsto 0$ and one positive number to a positive number, then $f$ is an identity map.}

The latter follows from \ref{ex:motion-of-R}.

\parbf{\ref{ex:2.4}.}
By \ref{lem:AOA=0},
$\measuredangle AOA=0$.
It remains to apply Axiom~\ref{def:birkhoff-axioms:2a}.

\parbf{\ref{ex:lineAOB}.}
Apply \ref{lem:AOA=0},
\ref{thm:straight-angle},
and \ref{ex:2a=0}.

\parbf{\ref{ex:ABCO-line}.}
By Axiom~\ref{def:birkhoff-axioms:2b},
$2\cdot\measuredangle BOC
\equiv 
2\cdot\measuredangle AOC\z-2\cdot \measuredangle AOB
\equiv 0$.
By \ref{ex:2a=0}, 
this implies that 
$\measuredangle BOC$ is either $0$ or~$\pi$.
It remains to apply \ref{ex:2.4} and \ref{thm:straight-angle} respectively in these two cases.

\parbf{\ref{ex:infinite-number-of-lines}.}
Fix two points $A$ and $B$ as provided by Axiom~\ref{def:birkhoff-axioms:0}.

Choose a real number $0<\alpha<\pi$.
By Axiom~\ref{def:birkhoff-axioms:2a} there is a point $C$ such that $\measuredangle ABC\z=\alpha$.

Use \ref{cor:degenerate=pi} to show that $\triangle ABC$ is nondegenerate.

%\parbf{\ref{ex:O-mid-AB+CD}.}
%Applying \ref{prop:vert}, we get that $\measuredangle AOC\z= \measuredangle BOD$.
%It remains to apply Axiom~\ref{def:birkhoff-axioms:3}.

\parbf{\ref{ex:refelection-of-line}.} Apply \ref{prop:point-reflection} and \ref{ex:line-motion}.

%\subsection*{Chapter~\ref{chap:half-planes}}
\refstepcounter{chapter}
\setcounter{eqtn}{0}

\parbf{\ref{ex:AOB+<=>BOA-}.}
Set $\alpha=\measuredangle AOB$ 
and 
$\beta=\measuredangle BOA$.
Note that $\alpha=\pi$ if and only if $\beta=\pi$.
Otherwise, $\alpha=-\beta$.
Hence the result.

\parbf{\ref{ex:PP(PN)}.}
Set $\alpha=\measuredangle ABC$, $\beta=\measuredangle A'B'C'$.
Since $2\cdot\alpha\equiv 2\cdot \beta$, \ref{ex:2a=0} implies that
 $\alpha\equiv \beta$ or $\alpha\equiv \beta+\pi$.
In the latter case, the angles have opposite signs which is impossible.

Since $\alpha,\beta\in(-\pi,\pi]$, the equality $\alpha\equiv \beta$ implies $\alpha= \beta$.

\parbf{\ref{ex:between}.}
Apply \ref{thm:signs-of-triug} to $\triangle PAB$, $\triangle PBC$, and $\triangle PAC$.

For the second part, argue as in \ref{ex:a+b==c} to show that if three numbers $\alpha,\beta,\gamma\in (-\pi,\pi)$ have the same sign and
$\alpha+\beta\equiv\gamma$, then
$|\alpha|+|\beta|=|\gamma|$.
Apply it to $\alpha=\measuredangle APB$,
$\beta=\measuredangle BPC$,
and
$\gamma=\measuredangle APC$.





\parbf{\ref{ex:vert-intersect}.}
Note that $O$ and $A'$
lie on the same side of~$(AB)$.
Similarly, $O$ and $B'$
lie on the same side of~$(AB)$.
Hence the result.

\parbf{\ref{ex:signs-PXQ-PYQ}.}
Apply \ref{thm:signs-of-triug} for $\triangle PQX$ and $\triangle PQY$ and then 
apply \ref{cor:half-plane}\textit{\ref{cor:half-plane:angle}}.

\parbf{\ref{ex:chevinas}.} 
We can assume that $A'\z\ne B,C$ and $B'\ne A, C$;
otherwise, the statement trivially holds.

Note that $(BB')$ does not intersect $[A'C]$.
Applying Pasch's theorem (\ref{thm:pasch}) for $\triangle AA'C$ and $(BB')$, we get that 
$(BB')$ intersects $[AA']$; denote the point of intersection by $M$.

Тhe same way, we get that $(AA')$ intersects $[BB']$;
that is, $M$ lies on $[AA']$ and $[BB']$.

\parbf{\ref{ex:Z}.}
Assume that $Z$ is the point of intersection.

Note that $Z\ne P$ and $Z\ne Q$.
Therefore, $Z\notin (PQ)$.

Show that $Z$ and $X$ lie on one side of~$(PQ)$.
Repeat the argument to show that $Z$ and $Y$ lie on one side of~$(PQ)$.
It follows that $X$ and $Y$ lie on the same side of $(PQ)$ --- a contradiction.

\parbf{\ref{ex:intersecting-circles-3}.} The ``only-if'' part follows from the triangle inequality.
To prove the ``if'' part,  
observe that \ref{thm:abc} implies the existence of a triangle with sides $r_1$, $r_2$, and $d$.
Use this triangle to show that there is a point $X$ such that $O_1X=r_1$ and $O_2X=r_2$, where $O_1$ and $O_2$ are the centers of the corresponding circles.

%\subsection*{Chapter~\ref{chap:cong}}
\refstepcounter{chapter}
\setcounter{eqtn}{0}

\parbf{\ref{ex:equilateral}.}
Apply \ref{thm:isos} twice.

\parbf{\ref{ex:SMS}.} 
Let $D$ and $D'$ be reflections of
$C$ and $C'$ across $M$ and $M'$ respectively.
Show that $\triangle BCD\z\cong \triangle B'C'D'$ and use it to prove that $\triangle A' B' C'\z\cong\triangle A B C$.

\parbf{\ref{ex:isos-sides}.} \textit{(a)} Apply SAS.

\parit{(b)} Use \textit{(a)} and apply SSS.

\parbf{\ref{ex:ABC-motion}.}
Without loss of generality, we may assume that $X$ is distinct from $A$, $B$, and~$C$.
Set $f(X)=X'$; assume $X'\ne X$.

Note that $AX=AX'$, $BX=BX'$, and $CX=CX'$.
By SSS we get that $\measuredangle ABX\z=\pm\measuredangle ABX'$.
Since $X\ne X'$, we get that
$\measuredangle ABX\equiv - \measuredangle ABX'$.
In the same way, we get that 
$\measuredangle CBX\equiv - \measuredangle CBX'$.
Subtracting these two identities from each other, we get that
$\measuredangle ABC\equiv -\measuredangle ABC$.
Conclude that $\measuredangle ABC=0$ or $\pi$.
That is, $\triangle ABC$ is degenerate --- a contradiction. 

\parbf{\ref{ex:motion}.} Construct three circles with centers at $A'$, $B'$, and $C'$ and radii $AX$, $BX$, and $CX$, respectively.
The point $X'$ is their common point.

%\parbf{\ref{ex:3-isos}.} Use SAS to show $\triangle ABA'\z\cong \triangle C'BC$, $\triangle BCB'\z\cong \triangle A'CA$, and $\triangle CAC'\z\cong \triangle B'CB$.

%\subsection*{Chapter~\ref{chap:perp}}
\refstepcounter{chapter}
\setcounter{eqtn}{0}


\parbf{\ref{ex:acute-obtuce}.} 
By Axiom~\ref{def:birkhoff-axioms:2b} and Theorem \ref{thm:straight-angle}, we have
$\measuredangle AOX\z+\measuredangle XOB\z\equiv\pi$; in particular, $|\measuredangle AOX\z+\measuredangle XOB|\ge\pi$.
Conclude that both values $\measuredangle AOX$ and $\measuredangle XOB$ are nonnegative or nonpositive.

In each case, argue as in \ref{ex:a+b==c} to show that $|\measuredangle AOX|\z+|\measuredangle XOB|=\pi$ and draw a conclusion.

\parbf{\ref{ex:pbisec-side}.}
Assume $X$ and $A$ lie on the same side of~$\ell$.

\begin{wrapfigure}{r}{27mm}
\vskip-4mm
\centering
\includegraphics{mppics/pic-322}
\end{wrapfigure}

Note that $A$ and $B$ lie on opposite sides of~$\ell$.
Therefore, by \ref{cor:half-plane}, 
$[AX]$ does not intersect $\ell$ 
and $[BX]$ intersects $\ell$;
let $Y$ be the intersection point.

By the triangle inequality, $BX=AY\z+YX\z\ge AX$.
Since $X\notin\ell$, by \ref{thm:perp-bisect} we have $AX\ne BX$.
Therefore $BX> AX$.

This way we have proved the ``if'' part.
To prove the ``only if'' part, you need to switch $A$ and $B$ and
repeat the above argument.

\parbf{\ref{ex:side-angle}.}
Apply \ref{ex:pbisec-side}, \ref{ex:vert-intersect}, and \ref{thm:isos}.

\parbf{\ref{ex:pbisec-motion}.} Show that $FX=FY$ and apply \ref{thm:perp-bisect}.

\begin{wrapfigure}{r}{27mm}
\vskip-4mm
\centering
\includegraphics{mppics/pic-326}
\end{wrapfigure}

\parbf{\ref{ex:construction-perpendicular}.}
See the picture; the given data is marked in bold.

\parbf{\ref{ex:2-reflections}.}
Note that
\begin{align*}
\measuredangle XBA&=\measuredangle ABP,
\\
\measuredangle PBC&=\measuredangle CBY.
\end{align*}
Therefore,
\begin{align*}
\measuredangle XBY
&\equiv
\measuredangle XBP+\measuredangle PBY\equiv
\\
&\equiv
 2\cdot(\measuredangle ABP+\measuredangle PBC)\equiv
\\
&
\equiv
 2\cdot \measuredangle ABC.
\end{align*}


\parbf{\ref{ex:3-reflections}.}
Choose an arbitrary nondegenerate triangle $ABC$.
Suppose that $\triangle \hat A \hat B\hat C$ denotes its image after the motion.

If $A\ne \hat A$, apply the reflection across the perpendicular bisector of~$[A\hat A]$.
This reflection sends $A$ to~$\hat A$.
Let $B'$ and $C'$ denote the reflections of $B$ and $C$ respectively.

If $B'\ne \hat B$, apply the reflection across the perpendicular bisector of~$[B'\hat B]$.
This reflection sends $B'$ to~$\hat B$.
Note that $\hat A\hat B=\hat AB'$;
that is, $\hat A$ lies on the perpendicular bisector. 
Therefore, $\hat A$ reflects to itself.
Suppose that $C''$ denotes the reflection of~$C'$.

Finally, if $C''\ne \hat C$, apply the reflection across $(\hat A\hat B)$.
Note that $\hat A\hat C\z=\hat AC''$ and $\hat B\hat C\z=\hat BC''$;
that is, $(\hat A\hat B)$ is the perpendicular bisector of $[C''\hat C]$.
Therefore, this reflection sends $C''$ to~$\hat C$.

Apply \ref{ex:ABC-motion} to show that the composition of the constructed reflections coincides with the given motion.

\parbf{\ref{ex:right-acute}.}
\textit{(a).} Apply \ref{lem:perp<oblique} and \ref{ex:side-angle}.

\parit{(b).} If the angle at $C$ is straight, then apply \ref{cor:degenerate=pi}.
Otherwise, show that there is a point $D\in [AB]$ such that $\angle ACD$ is right.
Applying \textit{(a)}, show that the angle at $A$ is acute. Reapeed it for~$B$.

\parbf{\ref{ex:obtuce}.}
We may assume that $P \ne X$; otherwise, the statement is trivial.

If $\angle PXV$ is right, then the statement follows from \ref{lem:perp<oblique}.
Otherwise, draw the line $(XY)$ perpendicular to~$(PX)$.
Show that $V$ and $W$ lie on opposite sides of $(XY)$.

Without loss of generality, assume that $P$ and $W$ lie on opposite sides of $(XY)$.
Let $W'$ be the point of intersection of $[PW]$ and $(XY)$.
Show and use that $PW>PW'>PX$.

\parbf{\ref{ex:PMQ}.}
Observe that $PM+MQ=P'M+MQ\z\ge P'Q$ and argue as in \ref{lem:perp<oblique}.



\parbf{\ref{ex:inside-outside}.}
Apply \ref{ex:obtuce} twice to $X$, $Y$, $P$, and the center.

{

\begin{wrapfigure}[6]{r}{27mm}
\vskip-6mm
\centering
\includegraphics{mppics/pic-328}
\end{wrapfigure}

\parbf{\ref{ex:chord-perp}.} Apply \ref{thm:perp-bisect}.

\parbf{\ref{ex:center}.}
See the picture; the given circle is marked in bold.

\parbf{\ref{ex:two-circ}.} Use \ref{ex:chord-perp} and \ref{perp:ex+un}.

}

\parbf{\ref{ex:tangent-circles}}; \textit{(\ref{ex:tangent-circles:a}).}
Assume that $Q$ is another point on both circles.
Show and use that $Q$ is the reflection of $P$ across~$(OO')$.

\parit{(\ref{ex:tangent-circles:b}).} Apply \textit{(\ref{ex:tangent-circles:a})} and \ref{lem:tangent}.

\parit{(\ref{ex:tangent-circles-2}).} Apply \textit{(\ref{ex:tangent-circles:a})} and \ref{cor:degenerate-trig}.

\parbf{\ref{ex:tangent-circles-3}.}
Let $A$ and $B$ be the points of intersection.
Note that the centers lie on the perpendicular bisector of the segment~$[AB]$.

\parbf{\ref{ex:tangent}+\ref{ex:tangent-circle}.}
The given data is marked in bold.

\begin{Figure}
\begin{minipage}{.49\textwidth}
\centering
\includegraphics{mppics/pic-330}
\end{minipage}
\hfill
\begin{minipage}{.49\textwidth}
\centering
\includegraphics{mppics/pic-332}
\end{minipage}
\end{Figure}

%\subsection*{Chapter~\ref{chap:parallel}}
\refstepcounter{chapter}
\setcounter{eqtn}{0}

\parbf{\ref{ex:mid-triangle}.} Apply the SAS similarity condition to show that
$\triangle AC'B'\sim\triangle ABC$, $\triangle C'BA'\sim\triangle ABC$, and $\triangle B'A'C\sim\triangle ABC$.
Show that in each case, the similarity coefficient is $\tfrac12$.
Conclude that $A'B'=\tfrac12\cdot AB$, $B'C'=\tfrac12\cdot BC$, and $C'A'\z=\tfrac12\cdot CA$.
Apply the SSS similarity condition.

\parbf{\ref{ex:k*triangle}.} Apply the SAS similarity condition to show that
$\triangle OA'B'\sim\triangle OAB$,
$\triangle OB'C'\sim\triangle OBC$,
and $\triangle OC'A'\sim\triangle OCA$.
Show that in each case, the similarity coefficient is $k$.
Conclude that $A'B'=k\cdot AB$, $B'C'=k\cdot BC$, and $C'A'\z=k\cdot CA$.
Apply the SSS similarity condition.

\parbf{\ref{ex:angle-preserving-euclid}.}
By the AA similarity condition, the transformation multiplies the sides of any nondegenerate triangle by the same number; we need to show that this number does not depend on the triangle. 

Note that for any two nondegenerate triangles that share one side, this number is the same.
Apply this observation to a chain of triangles.

\parbf{\ref{ex:pyth}.}
Apply that $\triangle ADC\sim \triangle CDB$.

\parbf{\ref{ex:pyth-conv}.}
Apply the Pythagorean theorem (\ref{thm:pyth}) and the SSS congruence condition.

\parbf{\ref{ex:two-pairs-sim}.}
By the AA similarity condition (\ref{prop:sim}), $\triangle AYC\z\sim \triangle BXC$.
Conclude that 
$\frac{YC}{AC}=\frac{XC}{BC}$.
Apply the SAS similarity condition to show that $\triangle ABC\z\sim \triangle YXC$.

Similarly, apply AA and the equality of vertical angles to prove that $\triangle AZX\sim \triangle BZY$ and use SAS to show that $\triangle ABZ\z\sim \triangle YXZ$.

\parbf{\ref{ex:ABC+D}.}
Show and use that $\triangle ABC\sim \triangle CBD$.

\parbf{\ref{ex:right-perp-bi}.}
Show and use that $\triangle AQM\sim \triangle ABC\z\sim\triangle PBM$.

%\parbf{\ref{ex:footpoints}.} Show and use that $\triangle XAA'\sim \triangle XBB'\sim \triange XCC'$

%\subsection*{Chapter~\ref{chap:angle-sum}}
\refstepcounter{chapter}
\setcounter{eqtn}{0}


\parbf{\ref{ex:perp-perp}.}
Apply \ref{prop:perp-perp} to show that $k\parallel m$.
By \ref{cor:parallel-1}, $k\parallel n\z\Rightarrow m\parallel n$.
The latter contradicts that $m\perp n$.

\parbf{\ref{ex:construction-parallel}.}
Repeat the construction in \ref{ex:construction-perpendicular} twice.

\parbf{\ref{ex:parallel-angles}.}
Apply Axiom~\ref{def:birkhoff-axioms:2b}, \ref{thm:parallel-2}, and \ref{ex:2a=0}.

\begin{wrapfigure}{r}{23mm}
\vskip-0mm
\centering
\includegraphics{mppics/pic-333}
\end{wrapfigure}

\parbf{\ref{ex:smililar+parallel}.}
Since  $\ell\parallel (AC)$, it cannot cross $[AC]$.
By Pasch's theorem (\ref{thm:pasch}), $\ell$ has to cross another side of $\triangle ABC$.
Therefore $\ell$ crosses $[BC]$; denote the point of intersection by $Q$.



Use the transversal property (\ref{thm:parallel-2}) to show that $\measuredangle BAC= \measuredangle BPQ$.
The same argument shows that $\measuredangle ACB\z= \measuredangle PQB$; it remains to apply the AA similarity condition.



\parbf{\ref{ex:trisection}.}
Assume we need to trisect segment $[AB]$.
Construct a line $\ell\ne (AB)$ with four points $A,C_1,C_2, C_3$
such that $C_1$ and $C_2$ trisect $[AC_3]$.
Draw the line $(BC_3)$
and draw parallel lines thru $C_1$ and~$C_2$.
The points of intersections of these two lines with $(AB)$ trisect the segment $[AB]$.

\parbf{\ref{ex:|3sum|}.}
If $\triangle ABC$ is degenerate, then one of the angle measures is $\pi$, and the other two are~$0$.
Hence the result.

Assume $\triangle ABC$ is nondegenerate.
Set $\alpha\z=\measuredangle CAB$, $\beta=\measuredangle ABC$, and $\gamma=\measuredangle BCA$.

By \ref{thm:signs-of-triug},
we may assume that $0<\alpha,\beta,\gamma<\pi$.
Therefore, 
\[0<\alpha+\beta+\gamma<3\cdot\pi.
\eqlbl{eq:|3|<3pi}\]

By \ref{thm:3sum},
$$\alpha+\beta+\gamma\equiv\pi.\eqlbl{eq:|3|==pi}$$

From \ref{eq:|3|<3pi} and \ref{eq:|3|==pi} the result follows.

\parbf{\ref{ex:pent}.}
Apply \ref{thm:3sum} or \ref{ex:|3sum|} twice and \ref{thm:isos} trice.

\parbf{\ref{ex:right-isos}.}
Apply \ref{thm:isos} twice and \ref{thm:3sum} once. 

%\parbf{\ref{ex:pi/4-isos}.} 
%Suppose that $O$ denotes the center of the circle.

%\begin{wrapfigure}{r}{28mm}
%\vskip-5mm
%\centering
%\includegraphics{mppics/pic-334}
%\end{wrapfigure} 

%Note that $\triangle AOX$ is isosceles and $\angle OXC$ is right.
%Applying \ref{thm:3sum} and \ref{thm:isos} and simplifying, you should get $4\cdot \measuredangle CAX \equiv \pi$.

%Show that $\angle CAX$ is acute.
%Conclude that $\measuredangle CAX\z=\pm\tfrac\pi4$.

\parbf{\ref{ex:quadrangle}.}
Apply \ref{thm:3sum} to $\triangle ABC$ and $\triangle BDA$.

\parbf{\ref{ex:4parallels}.}
Denote by $M$ the center of symmetry of $\square ABCD$;
it exists by \ref{lem:parallelogram}.
Let $P'$ be the reflection of $P$ across $M$.
Show and use that $a=(AP')$, $b=(BP')$, $c=(CP')$, and $d=(DP')$.

\parbf{\ref{ex:romb}.}
Since $\triangle ABC$ is isosceles, $\measuredangle CAB\z=\measuredangle BCA$.
 
By SSS, $\triangle ABC\cong \triangle CDA$.
Therefore, 
$\pm\measuredangle DCA= \measuredangle BCA=\measuredangle CAB$.

Since $D\ne C$, we get ``$-$'' in the last formula.
Use the transversal property (\ref{thm:parallel-2}) to show that $(AB)\parallel (CD)$. Repeat the argument to show that $(AD)\z\parallel(BC)$ 

\parbf{\ref{ex:rectangle}.} 
By \ref{lem:parallelogram} and SSS, 
$AC=BD$
if and only if
$\measuredangle ABC=\pm \measuredangle BCD$.
By the transversal property~(\ref{thm:parallel-2}), 
$\measuredangle ABC+\measuredangle BCD\equiv \pi$.

Therefore, 
$AC=BD$
if and only if
$\measuredangle ABC
=\measuredangle BCD
=\pm\tfrac\pi2$.

\parbf{\ref{ex:romb2}.} 
Fix a parallelogram $ABCD$.
By \ref{lem:parallelogram},
its diagonals $[AC]$ and $[BD]$ have a common midpoint; denote it by~$M$.

Use SSS and \ref{lem:parallelogram} to show the following:
$AB=CD\ \iff\ \triangle AMB
\cong
\triangle AMD\ \z\iff\ \measuredangle AMB
=
\pm\tfrac\pi2$.

\parbf{\ref{ex:inscribed-rhombus}.}
Show and use that $\triangle CZY\sim\triangle YXB$.

\parbf{\ref{ex:coordinates}.} \textit{(a).} Use the uniqueness of the parallel line (\ref{thm:parallel}).

\parit{(b)} Use \ref{lem:parallelogram} and the Pythagorean theorem (\ref{thm:pyth}).

\parbf{\ref{ex:abc}.}
Set $A=(0,0)$, $B=(c,0)$, and $C=(x,y)$.
Clearly, $AB=c$,
$AC^2=x^2+y^2$ and $BC^2\z=(c-x)^2+y^2$.

It remains to show that there is a pair of real numbers $(x,y)$ 
that satisfy the following system of equations:
$$
\left\{
\begin{aligned}
b^2&=x^2+y^2
\\
a^2&=(c-x)^2+y^2
\end{aligned}
\right.
$$
if $0<a\le b\le c\le a+c$.

\parbf{\ref{ex:line-coord}.} Note that $MA=MB$ if and only if
\[(x-x_A)^2+(y-y_A)^2=(x-x_B)^2+(y-y_B)^2,\]
where $M=(x,y)$. 
To prove the first part, simplify this equation.
For the remaining parts use that any line is a perpendicular bisector of some line segment.

\parbf{\ref{ex:circle-coord}.} Rewrite it the following way and think 
\[(x+\tfrac a2)^2+(y+\tfrac b2)^2=(\tfrac a2)^2+(\tfrac b2)^2-c.\]


\parbf{\ref{ex:apolonnius}.}
We can choose the coordinates so that $B=(0,0)$ and $A=(a,0)$ for some $a>0$.
If $M=(x,y)$, then the equation $AM=k\cdot BM$ can be written in coordinates as 
\[k^2\cdot(x^2+y^2)=(x-a)^2+y^2.\]
It remains to rewrite this equation as in \ref{ex:circle-coord}.

\parbf{\ref{ex:apolonnius-construction}.}
Assume $M\notin(AB)$.
Show and use that the points $P$ and $Q$ constructed in the following picture lie on the Apollonian circle.

\begin{Figure}
\centering
\includegraphics{mppics/pic-335}
\end{Figure}


%\subsection*{Chapter~\ref{chap:triangle}}
\refstepcounter{chapter}
\setcounter{eqtn}{0}

\parbf{\ref{ex:unique-cline}.}
Apply \ref{thm:circumcenter} and \ref{thm:perp-bisect}.

\parbf{\ref{ex:orthic-4}.}
Note that $(AC)\perp (BH)$ and $(BC)\z\perp (AH)$ and apply \ref{thm:orthocenter}.

(Note that each of $A,B,C,H$ is the orthocenter of the remaining three; such a quadruple of points $A,B,C,H$ is called an \index{orthocentric system}\emph{orthocentric system}.)

\parbf{\ref{ex:orthic-sim}.}
Show that $\triangle AA'C \sim \triangle BB'C$.
Use this to show that $\triangle ABC \sim \triangle A'B'C$.
Repeat the argument for $\triangle AB'C'$ and $\triangle A'BC'$.

\parbf{\ref{ex:midle}.}
Use the idea from the proof of \ref{thm:centroid}
to show that $(XY)\z\parallel (AC)\z\parallel (VW)$ and
$(XW)\z\parallel (BD)\z\parallel (YV)$.

\parbf{\ref{ex:euler-line}.}
Read about homothety.

Applying \ref{thm:centroid}, show that the homothety centered at $M$ with a coefficient of $-\tfrac{1}{2}$ maps $\triangle ABC$ onto its medial triangle $\triangle A'B'C'$.
Show that the homothety also maps the orthocenter of $\triangle ABC$ to the orthocenter of $\triangle A'B'C'$.

Arguing as in \ref{thm:orthocenter}, show that the orthocenter of $\triangle A'B'C'$ coincides with the circumcenter of $\triangle ABC$.
Make a conclusion.

\parbf{\ref{ex:perp-bisectors}.}
Let $(BX)$ and $(BY)$ be the internal and external bisectors of $\angle ABC$.
Then 
\begin{align*}
2\cdot \measuredangle XBY&\equiv2\cdot \measuredangle XBA+2\cdot \measuredangle ABY\equiv
\\
&\equiv
\measuredangle CBA+\pi+2\cdot \measuredangle ABC\equiv
\\
&\equiv\pi+\measuredangle CBC=\pi
\end{align*}
and hence the result.

\parbf{\ref{ex:bisect=altitude}.}
Apply ASA to the two triangles that the bisector cuts from the original triangle. 

\parbf{\ref{ex:ext-disect}.} 
If $E$ is the point of intersection of $(BC)$ 
with the external bisector of $\angle BAC$, then 
$\frac{AB}{AC}\z=\frac{EB}{EC}$.
It can be proved along the same lines as \ref{lem:bisect-ratio}.

\parbf{\ref{ex:bisect=median}.}
Apply \ref{lem:bisect-ratio}.
See also the solution of \ref{ex:abs-bisect=median}.

\parbf{\ref{ex:bisector-parallel}.}
Apply \ref{thm:isos}, \ref{thm:parallel-2}, and \ref{lem:parallelogram}.

\parbf{\ref{ex:2x=b+c-a}.}
Let $I$ be the incenter.
By SAS, we get that $\triangle AIZ\z\cong\triangle AIY$.
Therefore, $AY=AZ$.
In the same way, we get that $BX=BZ$ and $CX=CY$.
Hence the result.

\parbf{\ref{ex:orthic-triangle}.}
Argue as in \ref{ex:orthic-sim} to show that $\triangle A'B'C\z\sim \triangle ABC\z\sim\triangle AB'C'$.
Conclude that $(BB')$ bisects $\angle A'B'C'$.

If $\triangle ABC$ is obtuse, then its orthocenter coincides with one of the \index{excenter}\emph{excenters} of $\triangle A'B'C'$;
that is, 
the point of intersection of two external and one internal bisectors of $\triangle A'B'C'$.

\parbf{\ref{ex:bisector-incenter}.}
Apply \ref{lem:bisect-ratio} twice.
Use the obtained identity to show that the angle bisector at $C$ passes thru $I$.

%\subsection*{Chapter~\ref{chap:inscribed-angle}}
\refstepcounter{chapter}
\setcounter{eqtn}{0}

\parbf{\ref{ex:inscribed-angle}.} \textit{(a).}
Apply \ref{thm:inscribed-angle} for $\angle XX'Y$ and $\angle X'YY'$
and \ref{thm:3sum} for $\triangle PYX'$.

\parit{(b)} If $P$ is inside of $\Gamma$, then $P$ lies between $X$ and $X'$ and between $Y$ and $Y'$.
In this case, $\angle XPY$ is vertical to $\angle X'PY'$.

If $P$ is outside of $\Gamma$ then $[PX)\z=[PX')$ and $[PY)=[PY')$.
In both cases we have that $\measuredangle XPY=\measuredangle X'PY'$.

Applying \ref{thm:inscribed-angle} and \ref{ex:ABCO-line}, we get
\pagebreak[0]
\begin{align*}
2\cdot \measuredangle Y'X'P
&\equiv
2\cdot \measuredangle Y'X'X\equiv 
\\
\equiv
2\cdot\measuredangle Y'YX
&\equiv
2\cdot\measuredangle PYX.
\end{align*}
According to \ref{thm:signs-of-triug}, $\angle Y'X'P$ and $\angle PYX$ have the same sign;
therefore
$\measuredangle Y'X'P\z= \measuredangle PYX$.
It remains to apply the AA similarity condition.

\parit{(c)} Apply \textit{(b)} assuming $[YY']$ is the diameter of~$\Gamma$. 

\parbf{\ref{ex:inscribed-hex}.} Apply \ref{ex:inscribed-angle}\textit{\ref{ex:inscribed-angle:b}}
thrice.

\parbf{\ref{ex:altitudes-circumcircle}.}
Let $X$ and $Y$ be the footpoints of the altitudes from $A$ and~$B$.
Suppose that $O$ denotes the circumcenter.
 
By AA, $\triangle A X C\sim \triangle B Y C$.
Thus 
\begin{align*}
\measuredangle A'OC
&\equiv 
2\cdot \measuredangle A' A C
\equiv
\\
&\equiv-2\cdot\measuredangle B' B C
\equiv
\\
&\equiv-\measuredangle B'OC.
\end{align*}

By SAS, $\triangle A'OC\cong\triangle B'OC$.
Therefore, $A'C=B'C$.

\parbf{\ref{ex:two-chords}.}
Apply the transversal property (\ref{thm:parallel-2}) and the theorem on inscribed angles (\ref{thm:inscribed-angle}).

Alternatively, apply \ref{cor:reflection+angle} to the reflection across the common perpendicular to the lines passing thru the center of the circle.

\begin{wrapfigure}[9]{r}{20mm}
\vskip-2mm
\centering
\includegraphics{mppics/pic-336}
\end{wrapfigure}

\parbf{\ref{ex:tangent-construction-inscribed}.} Apply \ref{cor:right-angle-diameter} and \ref{lem:tangent}.

\parbf{\ref{ex:perp-construction-inscribed}.} Apply \ref{cor:right-angle-diameter}.

\parbf{\ref{ex:altitude+circles}.}
Show and use that $\angle ABX$ and $\angle ABY$ are right.
(We may assume that the points $A$, $B$, $X$, and $Y$ are distinct; otherwise, there is nothing to prove.)


\parbf{\ref{ex:perpendicular-ruler}.}
Guess the construction from the picture.
To prove it,
apply \ref{thm:orthocenter} and \ref{cor:right-angle-diameter}.

\parbf{\ref{ex:tnagents+midpoint}.}
Let $O$ be the center of $\Gamma$.
Use \ref{cor:right-angle-diameter} to show that the points lie on the circle with diameter $[PO]$.

\parbf{\ref{ex:VVAA}.} 
Note that $\measuredangle AA'B\z=\pm\tfrac\pi2$ and $\measuredangle AB'B\z=\pm\tfrac\pi2$.
Then apply \ref{cor:inscribed-quadrangle}
to $\square AA'BB'$.

\parbf{\ref{ex:secant-circles}.}
Apply \ref{cor:inscribed-quadrangle} twice for $\square ABYX$ and $\square ABY'X'$ and use the transversal property (\ref{thm:parallel-2}).

\parbf{\ref{ex:perim+angle+side}.}
Construct $\triangle AXC$ such that $AC=b$, $AX=p-b$, and $\measuredangle AXC=\tfrac12\cdot \beta$.
Note that point $B$ at the intersection of $AX$ and the perpendicular bisector of $[CX]$ solves the problem. 


\parbf{\ref{ex:inaccuracy}.}
One needs to show that $(A'B') \nparallel (XP)$; otherwise, the first line in the proof does not make sense.
By the transversal property, this means that
\[
2 \cdot \measuredangle XPA' + 2 \cdot \measuredangle PA'B' \not\equiv 0.
\eqlbl{eq:2<B'A'P+2<A'PX}
\]
This should be done later in the proof, right before we use point $Y$.

Instead of $2 \cdot \measuredangle AXY \equiv 2 \cdot \measuredangle AA'Y$, we should write
\[
2 \cdot \measuredangle AXP \equiv 2 \cdot \measuredangle PA'B'.
\eqlbl{eq:2<AXP=2<AA'B'}
\]
Since $\triangle XAP$ has a right angle at $A$, we have
\[
\measuredangle AXP + \measuredangle XPA \equiv \pm \tfrac{\pi}{2}.
\]
Since $2 \cdot \measuredangle XPA' \equiv 2 \cdot \measuredangle XPA$, \ref{eq:2<AXP=2<AA'B'} implies that
\[
2 \cdot \measuredangle XPA' + 2 \cdot \measuredangle PA'B' \equiv \pi,
\]
and \ref{eq:2<B'A'P+2<A'PX} follows.

Additionally, we implicitly use the following identities:
\begin{align*}
2 \cdot \measuredangle AXP &\equiv 2 \cdot \measuredangle AXY, \\
2 \cdot \measuredangle ABP &\equiv 2 \cdot \measuredangle ABB', \\
2 \cdot \measuredangle AA'B' &\equiv 2 \cdot \measuredangle AA'Y.
\end{align*}

\parbf{\ref{ex:equilateral-2}.}
By \ref{cor:right-angle-diameter},
the points $L$, $M$, and $N$ lie on the circle $\Gamma$ with diameter~$[OX]$.
It remains to apply \ref{thm:inscribed-angle} for the circle $\Gamma$ 
and two inscribed angles with a common vertex at~$O$.

\parbf{\ref{ex:median-angle}.}
Observe that the points $A$, $A'$, $B$, and $B'$ line on one circle.
Show that $(AB)\parallel (B'A')$.
Applying \ref{ex:two-chords}, show that $AB'=BA'$ and therefore $AC=BC$.


\parbf{\ref{ex:simson}.}
Let $X$, $Y$, and $Z$ denote the footpoints of $P$ on $(BC)$, $(CA)$, and $(AB)$ respectively.

\begin{Figure}
\centering
\includegraphics{mppics/pic-338}
\end{Figure}

Show that $\square AZPY$, $\square BXPZ$, $\square CYPX$, and $\square ABCP$ are inscribed.
Use this to show that
\begin{align*}
2\cdot \measuredangle CXY&\equiv 2\cdot \measuredangle CPY,
&
2\cdot \measuredangle BXZ&\equiv 2\cdot \measuredangle BPZ,
\\
2\cdot \measuredangle YAZ&\equiv 2\cdot \measuredangle YPZ,
&
2\cdot \measuredangle CAB&\equiv 2\cdot \measuredangle CPB.
\end{align*}

Conclude that 
$2\cdot \measuredangle CXY\equiv 2\cdot \measuredangle BXZ$
and hence $X$, $Y$, and $Z$ lie on one line.


%\parbf{\ref{ex:arc-tan-straight}.}
%By \ref{thm:3sum},
%$$\measuredangle ABC+\measuredangle BCA+\measuredangle CAB\equiv \pi.$$
%It remains to apply
%\ref{prop:arc(angle=tan)} twice.

\parbf{\ref{ex:a+b=c}.}
Show that $P$ lies on the arc opposite from $ACB$;
conclude that
$\measuredangle APC\z=\measuredangle CPB\z=\pm\tfrac\pi3$.

{

\begin{wrapfigure}[8]{r}{25mm}
\vskip-3mm
\centering
\includegraphics{mppics/pic-337}
\end{wrapfigure}

Choose a point $A'\z\in [PC]$ such that $PA'\z=PA$.
Note that $\triangle APA'$ is equilateral.
Prove and use that $\triangle AA'C\z\cong \triangle APB$.

\parbf{\ref{ex:tangent-arc}.}
If $C\in (AX)$, then the arc is the line segment $[AC]$ or the union of two half-lines in $(AX)$ with vertices at $A$ and~$C$.

}

Assume $C\notin (AX)$.
Let $\ell$ be the perpendicular line dropped from $A$ to $(AX)$ and $m$ be the perpendicular bisector of~$[AC]$.

Note that $\ell\nparallel m$;
suppose they meet at $O$.
Note that the circle with center $O$ passing thru $A$ is also passing thru $C$ and tangent to~$(AX)$.


Note that one of the two arcs with endpoints $A$ and $C$ is tangent to~$[AX)$.

The uniqueness follows from \ref{prop:arc(angle=tan)}.

\parbf{\ref{ex:tangent-lim}.} Use \ref{prop:arc(angle=tan)} and \ref{thm:3sum} to show that 
$\measuredangle XAY\z=\measuredangle ACY$.
By Axiom~\ref{def:birkhoff-axioms:2c}, $\measuredangle ACY\to 0$ as $AY\to 0$;
hence the result.

\begin{wrapfigure}[7]{r}{25mm}
\vskip-6mm
\centering
\includegraphics{mppics/pic-340}
\end{wrapfigure}

\parbf{\ref{ex:two-arcs}.} 
Apply \ref{prop:arc(angle=tan)} twice.

(Alternatively, apply \ref{cor:reflection+angle} for the reflection across the perpendicular bisector of $[AC]$.)

\parbf{\ref{ex:3x120}.} Guess a construction from the picture.
To show that it produces the needed point, apply~\ref{thm:inscribed-angle}.

%\subsection*{Chapter~\ref{chap:inversion}}
\refstepcounter{chapter}
\setcounter{eqtn}{0}

\parbf{\ref{ex:constr-inversion}.}
By \ref{lem:tangent}, $\angle OTP'$ is right. 
Therefore, $\triangle OPT\z\sim \triangle OTP'$
and in particular
$OP\cdot OP'\z=OT^2$,
and hence the result.

\parbf{\ref{ex:appolo-circ}.}
Suppose that $O$ denotes the center of $\Gamma$.
Assume that $X,Y\z\in \Gamma$;
in particular, $OX\z=OY$.



Note that the inversion sends $X$ and $Y$ to themselves.
By \ref{lem:inversion-sim},
$$\triangle OPX\z\sim \triangle OXP'
\quad
\text{and}
\quad
\triangle OPY\sim \triangle OYP'.$$
Therefore, 
$\frac{PX}{P'X}=\frac{OP}{OX}=\frac{OP}{OY}=\frac{PY}{P'Y}$
and hence the result.

\parbf{\ref{ex:incenter+inversion=orthocenter}.}
By \ref{lem:inversion-sim},
\begin{align*}
\measuredangle IA'B'&\equiv -\measuredangle IBA,
&
\measuredangle IB'A'&\equiv -\measuredangle IAB,
\\
\measuredangle IB'C'&\equiv -\measuredangle ICB,
&
\measuredangle IC'B'&\equiv -\measuredangle IBC,
\\
\measuredangle IC'A'&\equiv -\measuredangle IAC,
&
\measuredangle IA'C'&\equiv -\measuredangle ICA.
\end{align*}

It remains to apply the theorem on the sum of angles of triangle (\ref{thm:3sum})
to show that $(A'I)\z\perp (B'C')$, 
$(B'I)\z\perp (C'A')$
and
$(C'I)\z\perp (B'A')$.

\parbf{\ref{ex:consturuction-of-inversion}.}
Guess the construction from the picture.

\begin{Figure}
\vskip-0mm
\centering
\includegraphics{mppics/pic-342}
\end{Figure}

\parbf{\ref{ex:inv-center not=center-inv}.}
Assume $r\z>0$, $x\ne 0$, and $y\ne 0$.
Show that
$$\frac{r^2}{(x+y)/2}
=
\left(\frac {r^2}x+\frac {r^2}y\right)/2\ \iff\ x=y.$$

Suppose $\ell$ denotes the line passing thru $Q$, $Q'$, and the center of the inversion $O$.
Choose an isometry $\ell\to\mathbb{R}$ that sends $O$ to $0$;
assume $x,y\in \mathbb{R}$ are the values of $\ell$ for the two points in $\ell\cap\Gamma$;
note that $x\ne y$.
Assume $r$ is the radius of the circle of inversion.
Then the left-hand side above is the coordinate of $Q'$ 
and the right-hand side is the coordinate of the center of $\Gamma'$.

\parbf{\ref{ex:circumtool}.}
A solution is given in \ref{sec:verification}.

\parbf{\ref{ex:tangent-circ->parallels}.}
Apply an inversion across a circle with the center at the only point of intersection of the circles;
then use \ref{thm:inverse}.

\parbf{\ref{ex:4-circles}.}
Label the points of tangency as in the picture.
Apply an inversion with the center at $P$. 
Observe that the two circles that tangent are at $P$ become parallel lines and 
the remaining two circles are tangent to each other and these two parallel lines.

\begin{Figure}
\centering
\includegraphics{mppics/pic-344}
\end{Figure}

Note that the points of tangency $A'$, $B'$, $X'$, and $Y'$ with the parallel lines are vertices of a square;
in particular, they lie on one circle.
These points are images of $A$, $B$, $X$, and $Y$ under the inversion.
By \ref{thm:inverse-cline}, the points $A$, $B$, $X$, and $Y$ also lie on one circline.

\parbf{\ref{ex:inverse}.} 
Apply the inversion across a circle with center~$A$. 
Point $A$ will go to infinity; the two circles tangent at $A$ will become parallel lines
while the two parallel lines will become circles tangent at~$A$.

\begin{Figure}
\centering
\includegraphics{mppics/pic-346}
\end{Figure}

It remains to show that the dashed line $(AB')$ in the picture is parallel to the other two lines.

\parbf{\ref{ex:inscribed+inv}.}
Apply \ref{lem:inverse-4-angle}\textit{\ref{lem:inverse-4-angle:angle}}, 
\ref{ex:quadrangle},
and \ref{thm:inscribed-angle}.

\parbf{\ref{ex:centers-of-perp-circles}.}
Suppose that $T\in \Omega_1\cap\Omega_2$.
Let $P$ be the footpoint of $T$ on~$(O_1O_2)$.
Show that
$\triangle O_1PT
\z\sim \triangle O_1TO_2
\z\sim \triangle TPO_2$.
Conclude that $P$ coincides with the inverses of $O_1$ across $\Omega_2$ and of $O_2$ across~$\Omega_1$.

\parbf{\ref{ex:4-th-perp-circ}.}
Since $\Gamma\perp\Omega_1$ and $\Gamma\perp\Omega_2$,
Corollary~\ref{cor:perp-inverse-clines} 
implies that
the circles $\Omega_1$ and $\Omega_2$ are inverted in $\Gamma$ 
to themselves.
Conclude that $A$ and $B$ are inverses of each other.

Since $\Omega_3\ni A,B$,
Corollary~\ref{cor:perp-inverse} implies that
$\Omega_3\perp \Gamma$.

\parbf{\ref{ex:construction-perp-clines}.}
Let $P_1$ and $P_2$ be the inverses of $P$ 
across $\Omega_1$ and~$\Omega_2$.
Apply \ref{cor:perp-inverse} and \ref{thm:perp-inverse}
to show that a circline $\Gamma$ passing thru $P$, $P_1$, and $P_3$ is a solution.

\parbf{\ref{ex:3-construction-perp-clines}.}
All circles perpendicular to $\Omega_1$ and $\Omega_2$ pass thru a fixed point~$P$.
Try to construct~$P$.

If two of the circles intersect, try to apply \ref{cor:invese-comp}.

%\subsection*{Chapter~\ref{chap:non-euclid}}
\refstepcounter{chapter}
\setcounter{eqtn}{0}

\parbf{\ref{ex:tangent-angle-neutral}.}
One cannot apply \ref{thm:3sum} (its proof relies on \ref{thm:parallel-2}, which in turn uses \ref{thm:parallel}, and that depends on the SAS similarity condition, which is essentially Axiom~\ref{def:birkhoff-axioms:4}).

\parbf{\ref{ex:abs-bisect=median}.}
Suppose $D$ is the midpoint of~$[BC]$.
Assume that $(AD)$ is the angle bisector at~$A$.

Let $A'\in [AD)$ be the reflection of $A$ across~$D$.
Note that $\triangle CAD\z\cong\triangle BA'D$.
In particular, $\measuredangle BAA'=\measuredangle AA'B$.
Apply \ref{thm:isos} to  $\triangle ABA'$.

\parbf{\ref{ex:abs-inscibed}.}
The statement is evident if $A$, $B$, $C$, and $D$ lie on one line.

In the remaining case, suppose that $O$ denotes the circumcenter.
Apply \ref{thm:isos} to
$\triangle AOB$,
$\triangle BOC$, 
$\triangle COD$, 
and
$\triangle DOA$. 

\textit{(In the Euclidean plane the statement follows from \ref{cor:inscribed-quadrangle} and \ref{ex:quadrangle},
but these statements cannot be used in the neutral plane.)}

\parbf{\ref{ex:parallel-abs}.}
Arguing by contradiction, 
assume 
$2\cdot(\measuredangle ABC+\measuredangle BCD)\equiv0$, 
but $(AB)\z\nparallel(C D)$.
Let $Z\in (AB)\cap(CD)$.

Note that 
$
2\cdot \measuredangle ABC\equiv 2\cdot \measuredangle ZBC,
$ and
$2\cdot \measuredangle BCD\equiv 2\cdot \measuredangle BCZ$.

Apply \ref{prop:2sum} to $\triangle ZBC$ and try to reach a contradiction.

\parbf{\ref{ex:SAA}.}
Choose $C''\in [B'C')$ such that $B'C''\z=BC$.

Use SAS to show that $\triangle ABC\cong \triangle A'B'C''$.
Conclude that $\measuredangle B'C'A'\z= \measuredangle B'C''A'$.

\begin{Figure}
\vskip-0mm
\centering
\includegraphics{mppics/pic-348}
\end{Figure}

Therefore, it is sufficient to show that $C''\z=C'$.
If $C'\z\ne C''$ apply \ref{prop:2sum} to $\triangle A'C'C''$ and try to reach at a contradiction.

%(This proof was given in Euclid's Elements \cite[Book I, Proposition 26]{euclid}.)


\parbf{\ref{ex:chev<side}.} 
Use \ref{ex:side-angle} and \ref{prop:2sum}.

(Alternatively, follow the solution of \ref{ex:inside-outside}.)

\parbf{\ref{ex:neutral-quadrangle}.}
Set $a=AB$, $b=BC$, $c=CD$, and $d=DA$; we need to show that $c\ge a$.

Mimic the proof of \ref{thm:3sum-a} for the shown fence made from copies of quadrangle $ABCD$.
You should get that $a\le c+\tfrac dn$ for any integer $n\ge 1$, and this implies the needed inequality.

\begin{Figure}
\vskip-0mm
\centering
\includegraphics{mppics/pic-349}
\end{Figure}

\parit{Alternative way.}
By \ref{lem:perp<oblique}, we can assume that $\angle BCD$ is right.
Use \ref{thm:3sum-a} to show that $|\measuredangle CBD|\ge|\measuredangle ADB|$.
Use \ref{prop:angle-side} and the construction in the neutral proof of \ref{thm:hypotenuse-leg} to show that $2\cdot c\ge 2\cdot a$.

\parbf{\ref{ex:defect}.}
Note that 
$|\measuredangle ADC|+|\measuredangle CDB|=\pi$.
Then apply the definition of the defect.

\parbf{\ref{ex:defect=}.}
Show that $\triangle AMX\cong \triangle BMC$. 
Apply \ref{ex:defect} to $\triangle ABC$ and $\triangle AXC$.


\parbf{\ref{ex:neutral-rectangle}.}
Observe that the total sum of absolute values of angle measures in $\triangle ABC$ and $\triangle CDA$ is at least $2\cdot\pi$.
Apply \ref{thm:3sum-a} to show that
\[\defect(\triangle ABC)=\defect(\triangle CDA)=0.\]
Use it to show that $\measuredangle CAB=\measuredangle ACD$ and $\measuredangle ACB=\measuredangle CAD$.
By ASA, $\triangle ABC\cong\triangle CDA$, and, in particular, $AB=CD$.

(Note that it also follows from \ref{ex:neutral-quadrangle}.)

\parbf{\ref{ex:neutral-rectangle+}.}
Choose a triangle $\triangle_0$.
Show that $\triangle_0$ can be covered by a large rectangle $\square$ that is a union of several copies of the given rectangle.
Further, subdivide $\square$ into triangles $\triangle_0,\dots, \triangle_n$.
Show that the sum of the defects of $\triangle_i$ is zero and use that defects are nonnegative.

\begin{Figure}
\vskip-0mm
\centering
\includegraphics{mppics/pic-1114}
\end{Figure}

\parit{Remark.}
This is a key step in the proofs of parts \textit{(\ref{thm:=IV:defect})} and \textit{(\ref{thm:=IV:rectangle})} in Theorem~\ref{thm:=IV}.


%\subsection*{Chapter~\ref{chap:poincare}}
\refstepcounter{chapter}
\setcounter{eqtn}{0}

\parbf{\ref{ex:ideal-line-unique}.} 
Consider the ideal points $A$ and $B$ be the ideal points of the h-line~$\ell$. 
Note that the center of the Euclidean circle containing $\ell$ lies 
at the intersection of the lines tangent to the absolute at the ideal points of~$\ell$.

\parbf{\ref{ex:1ideal-line-unique}.}
Assume $A$ is an ideal point of the h-line $\ell$
and $P\in \ell$.
Suppose that $P'$ denotes the inverse of $P$ across the absolute.
By \ref{cor:perp-inverse-clines},
$\ell$ lies in the intersection of the h-plane and the (necessarily unique) circline 
passing thru $P$, $A$, and~$P'$.

\parbf{\ref{ex:line/h-line}.} 
Let $\Omega$ and $O$ denote the absolute and its center. 

Let $\Gamma$ be the circline containing~$[PQ]_h$.
Note that $[PQ]_h=[PQ]$ if and only if $\Gamma$ is a line.

Suppose that $P'$ denotes the inverse of $P$ across~$\Omega$.
Note that $O$, $P$, and $P'$ lie on one line.

By the definition of an h-line, $\Omega\perp \Gamma$.
By \ref{cor:perp-inverse-clines}, $\Gamma$ passes thru $P$ and~$P'$. 
Therefore, $\Gamma$
is a line if and only if it passes thru~$O$.

\parbf{\ref{ex:h-dist-eq}.}
Assume that the absolute is a unit circle.

Set $a\z=OX\z=OY$.
Note that $0<a<\tfrac12$,
$
OX_h=\ln \tfrac{1+a}{1-a}$,
and
$XY_h=\ln \tfrac{(1+2\cdot a)\cdot(1-a)}{(1-2\cdot a)\cdot(1+a)}$.
Verify that the inequalities 
\[1<
\tfrac{1+a}{1-a}
<
\tfrac{(1+2\cdot a)\cdot(1-a)}{(1-2\cdot a)\cdot(1+a)}\]
hold when $0<a<\tfrac12$.

\parbf{\ref{ex:h-perp-unique}.} 
Spell the meaning of the terms ``perpendicular'' and ``h-line'' and then apply \ref{ex:construction-perp-clines}.

\parbf{\ref{ex:h-circle=circle}.}
Choose the vertices $P$, $Q$, and $R$ on a Euclidean circle that intersects the absolute and is not orthogonal to it.
Apply~\ref{lem:h-circle=circle}.

\begin{wrapfigure}[9]{r}{34mm}
\vskip-5mm
\centering
\includegraphics{mppics/pic-350}
\end{wrapfigure}

\parbf{\ref{ex:3-h-lines}.}
Choose the required h-lines from the picture.

\parbf{\ref{ex:O-h-dist}.} Use \ref{lem:O-h-dist}.

\parbf{\ref{ex:cosh}.}
By \ref{cor:invese-comp} and \ref{lem:inverse-4-angle},
the right-hand sides of the identities 
remain unchanged under an inversion across a circle perpendicular to the absolute.

As usual, we assume that the absolute is a unit circle.
Let $O$ be the h-midpoint of $[PQ]_h$.
By the main observation (\ref{thm:main-observ})
we can assume that $O$ is the center of the absolute.
In this case, $O$ is also the Euclidean midpoint of $[PQ]$.%

Set $a=OP=OQ$; in this case, we have
\begin{align*}
PQ&=2\cdot a,
&
PP'=QQ'&=\tfrac1a-a,
\\
P'Q'&=2\cdot \tfrac1a,
&
PQ'=QP'&=\tfrac1a+a.
\end{align*}
and 
\[PQ_h=\ln \tfrac{(1+a)^2}{(1-a)^2}=2\cdot \ln \tfrac{1+a}{1-a}.\]
Therefore
\begin{align*}
\cosh[\tfrac12\cdot PQ_h]
&=\tfrac12\cdot(\tfrac{1+a}{1-a}+\tfrac{1-a}{1+a})=
\\
&=\tfrac{1+a^2}{1-a^2};
\end{align*} 
\begin{align*}
\sqrt{\frac{PQ'\cdot P'Q}{PP'\cdot QQ'}}
&=\frac{\frac1a+a}{\frac1a-a}=
\\
&=\tfrac{1+a^2}{1-a^2}.
\end{align*} 
Hence part \textit{(\ref{ex:cosh/2})} follows.
Similarly,
\begin{align*}
\sinh[\tfrac12\cdot PQ_h]
&=\tfrac12\cdot\left(\tfrac{1+a}{1-a}-\tfrac{1-a}{1+a}\right)=
\\
&=\tfrac{2\cdot a}{1-a^2};
\\
\sqrt{\frac{PQ\cdot P'Q'}{PP'\cdot QQ'}}
&=\frac{2}{\frac1a-a}=
\\
&=\tfrac{2\cdot a}{1-a^2}.
\end{align*} 
Hence part \textit{(\ref{ex:coshsinh})} follows.

Parts \textit{(\ref{ex:coshtanh})} and \textit{(\ref{ex:coshcosh})} follow from \textit{(\ref{ex:cosh/2})}, \textit{(\ref{ex:coshsinh})}, the definition of a hyperbolic tangent, and the double-argument identity for hyperbolic cosine, see \ref{double-argument}.

(We could also move $Q$ to the center of absolute.
In this case, the derivations are simpler.
But since $Q'Q=Q'P=Q'P'=\infty$, one has to justify that $\tfrac\infty\infty=1$ every time.)

%\subsection*{Chapter~\ref{chap:h-plane}}
\refstepcounter{chapter}
\setcounter{eqtn}{0}

\parbf{\ref{ex:lambert-parallelism}.}
Use \ref{prop:perp-perp} to show that $(AB)_h\parallel (CD)_h$.
Apply the definition of the angle of parallelism.

\parbf{\ref{ex:ultra-parallel}}; \textit{``only-if'' part.}
Suppose $\ell$ and $m$ are ultraparallel; that is, they do not intersect and have distinct ideal points.

Denote the ideal points by $A$, $B$, $C$, and $D$;
we may assume that they appear on the absolute in the same order.
In this case, the h-line with ideal points $A$ and $C$ intersects the h-line with ideal points $B$ and $D$.
Let $O$ be their point of intersection.
By \ref{thm:main-observ}, we can assume that $O$ is the center of absolute.
Note that $\ell$ is the reflection of $m$ across $O$ in the Euclidean sense.

\begin{Figure}
\vskip-0mm
\centering
\includegraphics{mppics/pic-359}
\end{Figure}

Drop an h-perpendicular $n$ from $O$ to $\ell$ and
show that $n\perp m$.

\parit{``If'' part.} 
Suppose $n$ is a common perpendicular.
Let $L$ and $M$ be its points of intersection with $\ell$ and $m$ respectively.

Let $O$ be the center of the absolute.
By \ref{thm:main-observ}, we can assume that $O$ is the h-midpoint of $L$ and $M$.
Notice that in this case $\ell$ is the reflection of $m$ across $O$ in the Euclidean sense.
It follows that the ideal points of the h-lines $\ell$ and $m$ are symmetric to each other.
Therefore, if one pair of them coincides, then the other pair does too. 
By \ref{ex:ideal-line-unique}, $\ell=m$, which contradicts the assumption $\ell\ne m$.

\parbf{\ref{ex:right-angle-parallelism}.} Show that the angle of parallelism at $C$ with respect to $(AB)_h$ is less than $\tfrac\pi4$, and apply \ref{prop:angle-parallelism}.
You can use approximations such as $\sqrt2\approx 1.414$ and $e\approx2.718$.


\parbf{\ref{ex:small-angle}.}
By the triangle inequality, the h-distance from $B$ to $(AC)_h$ is at least 50.
By \ref{prop:angle-parallelism}, $|\measuredangle_h ABC|<2\cdot\phi$, where $\phi$ is the angle of parallelism at the distance $50$.
Show that $\cos\phi\z=\tfrac{e^{100}-1}{e^{100}+1}$ and
use it to estimate~$\phi$.
You may use that $\cos\phi\z\le 1-\tfrac1{10}\cdot\phi^2$ for $|\phi|\z<\tfrac\pi2$ and $e^3>10$.

\parbf{\ref{ex:side-sup}.}
Note that the angle of parallelism at $B$ with respect to $(CD)_h$ is greater than $\tfrac\pi4$,
and it approaches to $\tfrac\pi4$ as $CD_h\to\infty$.

Applying \ref{prop:angle-parallelism},
we get that
$$BC_h<\tfrac12\cdot\ln\frac{1+\frac1{\sqrt{2}}}{1-\frac1{\sqrt{2}}}=\ln\left(1+\sqrt{2}\right).$$

The right-hand side is the limit of $BC_h$ as $CD_h\to\infty$.
Therefore, $\ln\left(1+\sqrt{2}\right)$ is the optimal upper bound.


\parbf{\ref{ex:equidistant-reflection}.}
Let $P'$ be another h-point.
Denote by $R$ and $R'$ the h-footpoint of $P$ and $P'$ on $m$.

Show and use that $P'$ is an h-reflection of $P$ across $Q\in m$ if and only if $P$ and $P'$ lie on the opposite sides of $m$, 
$PR_h=P'R'_h$, and $Q$ is the h-midpoint of $[RR']_h$. 
You may need \ref{prop:vert}, SAS, and SAA (see \ref{ex:SAA}).

\parbf{\ref{ex:right-trig-horocycle}.}
As usual, we assume that the absolute is a unit circle. 

Consider a hyperbolic triangle $PQR$
with a right angle at $Q$, where  $PQ_h\z=QR_h$
and the vertices $P$, $Q$, and $R$ 
lie on a horocycle.

\begin{wrapfigure}{r}{32mm}
\vskip-2mm
\centering
\includegraphics{mppics/pic-352}
\end{wrapfigure}

We may assume that $Q$ is the center of the absolute.
In this case, $\measuredangle_hPQR\z=\measuredangle PQR\z=\pm\tfrac\pi2$ and $PQ=QR$.

Note that the Euclidean circle passing thru $P$, $Q$, and $R$ is tangent to the absolute.
Conclude that $PQ=\tfrac1{\sqrt2}$. 
Apply \ref{lem:O-h-dist} to find $PQ_h$.


\parbf{\ref{ex:angle-preserving-hyp}.}
Apply the AAA-congruence condition (\ref{thm:AAA}).

\parbf{\ref{ex:circum}.}
Apply \ref{prop:circum}.
Use that the function $r\mapsto e^{-r}$ is decreasing and $e\z>2$.

\parbf{\ref{ex:c+1>a+b}.}
Apply the hyperbolic Pythagorean theorem and the definition of a hyperbolic cosine.
The following observations should help:
\begin{itemize}
 \item The function $x\mapsto e^x$ is an increasing positive function.
 \item By the triangle inequality,  we have
 \[-c\le a-b\quad \text{and}\quad  -c\le b-a.\]
\end{itemize}

%\subsection*{Chapter~\ref{chap:trans}}
\refstepcounter{chapter}
\setcounter{eqtn}{0}

\parbf{\ref{ex:affine-par}.}
Assume the distinct lines $\ell$ and $m$ 
are mapped to intersecting lines $\ell'$ and~$m'$.
Suppose $P'$ denotes their point of intersection.

Let $P$ be the inverse image of~$P'$.
By the definition of an affine map, $P$ must lie on both $\ell$ and $m$.
Make a concussion.

\parbf{\ref{ex:afine-linear}.}
In each case, check that the map is a bijection and apply \ref{ex:line-coord}.

\parbf{\ref{ex:collinear=affine}.}
Choose a line $(AB)$.

Assume $X'\z\in (A'B')$ for some $X\z\notin(AB)$.
Since $P\mapsto P'$ maps collinear points to collinear, 
the three lines $(AB)$, $(AX)$, and $(BX)$ are mapped to~$(A'B')$.
Furthermore, any line connecting a pair of points on these three lines is also mapped to~$(A'B')$.
Use it to show that the entire plane is mapped to $(A'B')$.
The latter contradicts that the map is a bijection.

{

\begin{wrapfigure}{r}{22mm}
\vskip-3mm
\centering
\includegraphics{mppics/pic-353}
\end{wrapfigure}

By the assumption, if $X\z\in (AB)$, then $X'\z\in (A'B')$.
From above, if $X\notin (AB)$, then $X'\z\notin (A'B')$.
Use it to prove the second statement.

}

\parbf{\ref{ex:circle=affine}.}
Observe that $\alpha$ maps noncollinear triples of points to noncollinear ones.
Therefore, $\alpha^{-1}$ maps collinear triples to collinear ones.
It remains to apply \ref{ex:collinear=affine}.

{

\begin{wrapfigure}{r}{28mm}
\vskip-6mm
\centering
\includegraphics{mppics/pic-354}
\end{wrapfigure}

\parbf{\ref{ex:midpoint-affine}.}
It is sufficient to construct the midpoint of $[AB]$
with a ruler and a parallel tool.
Guess the construction from the picture.

}

\parbf{\ref{ex:R-hom}.}
Let $O$, $E$, $A$, and $B$ be the points with coordinates $(0,0)$, $(1,0)$, $(a,0)$, and $(b,0)$ respectively.

To construct a point $W$ with coordinates $(a+b,0)$ (or $(a-b,0)$) try to construct two parallelograms $OAPQ$ and $BWPQ$ (or $WBPQ$).

To construct $Z$ with coordinates $(a\cdot b,0)$
choose a line $(OE')\ne (OE)$
and try to construct points $A'\in (OE')$
and $Z \in(OE)$
so that 
$\triangle OEE'\z\sim \triangle OAA'$ and $\triangle OE'B\z\sim \triangle OA'Z$.
Likewise, construct the point $(\tfrac ab,0)$.

\parbf{\ref{ex:center-circ-affine}.}
Draw two parallel chords $[XX']$ and~$[YY']$.
Let $Z\z\in(XY)\z\cap (X'Y')$ and $Z'\z\in (XY')\cap (X'Y)$.
Note that $(ZZ')$ passes thru the center.

Repeat the same construction for another pair of parallel chords.
The center lies at the intersection of the obtained lines.

\parbf{\ref{ex:affine-perp}.}
Assume a construction produces two perpendicular lines.
Apply a shear map that changes the angle between the lines (see \ref{ex:afine-linear}\textit{\ref{ex:afine-linear:shear}}).

Note that it transforms the construction into the same construction for other free choices of points.
Therefore, our construction does not generally produce perpendicular lines.
(It might produce a perpendicular line only by coincidence.)
 
\parbf{\ref{ex:parallelogram-rule}.}
The first part follows from \ref{ex:affine-par}.

Suppose $A$, $B$, $X$, and $Y$ are not collinear;
in this case, $\square ABYX$ is a parallelogram.
By the parallelogram rule, the only-if part follows.

Now suppose $A$, $B$, $X$, and $Y$ lie on line $\ell$.
Choose two additional points $P,Q\notin\ell$ such that 
\[\overrightarrow{XY}=\overrightarrow{PQ}
\quad\text{and therefore}\quad 
\overrightarrow{PQ}=\overrightarrow{AB}.\]
From above we get 
\[\overrightarrow{X'Y'}=\overrightarrow{P'Q'}
\quad\text{and}\quad 
\overrightarrow{P'Q'}=\overrightarrow{A'B'}.\]
Hence the only-if part follows in the general case.

The if part follows since the inverse of an affine transformation is also affine.
 
\parbf{\ref{ex:affine-continuous} and \ref{ex:affine-coordinates}.}
Choose a coordinate system and apply the fundamental theorem of affine geometry (\ref{thm:fundamental-theorem-of-affine-geometry}) for the points $O=(0,0)$, $X=(1,0)$, and $Y=(0,1)$.

\parbf{\ref{ex:preserved-circle}.} 
Use \ref{ex:center-circ-affine}, to show that $O'=O$.
Choose a coordinate system with origin at $O$.
By \ref{ex:affine-coordinates}, 
\[\alpha\:(x,y)\mapsto(a\cdot x+b\cdot y,\ c\cdot x+d\cdot y).\]
Use that $\alpha(\Gamma)=\Gamma$ to show at $(\begin{smallmatrix}a&b\\c&d\end{smallmatrix})\cdot (\begin{smallmatrix}a&c\\b&d\end{smallmatrix})\z=(\begin{smallmatrix}1&0\\0&1\end{smallmatrix})$.
Conclude that $\alpha$ is a motion.

\parbf{\ref{ex:inversions-inversive}.}
Observe that any reflection meets the condition.
By Theorem~\ref{thm:inverse-cline}, any inversion meets the condition as well.
Therefore, the same holds for any composition of inversions and reflections.

To prove the converse, choose a bijection $\alpha$ that maps circlines to circlines.
Show that we can compose $\alpha$ with several inverses and reflections to obtain a bijection $\alpha'$ such that $\alpha'(\infty)=\infty$ and $\alpha(\Gamma)=\Gamma$ for some circle $\Gamma$.

By \ref{ex:preserved-circle}, $\alpha'$ is a motion.
By \ref{ex:3-reflections}, $\alpha'$ is a composition of reflections.
Therefore so is $\alpha$.

\parbf{\ref{ex:f(1)=1}.}
Set $a=f(1)$ and $b=f(0)$.
Show and use that $a=a^2$ and $b=b^2$.

\parbf{\ref{ex:ceva-affine}.} Apply Menelaus's theorem
for $\triangle AA'B$ with $(CC')$
and
for $\triangle AA'C$ with $(BB')$.

%\subsection*{Chapter~\ref{chap:proj}}
\refstepcounter{chapter}
\setcounter{eqtn}{0}

\parbf{\ref{ex:proj-cross-ratio}.}
To prove \textit{(a)}, apply \ref{prop:affine-linear}.

To prove \textit{(b)}, suppose $P_i=(x_i,y_i)$;
show and use that 
\[\frac{P_1P_2\cdot P_3P_4}{P_2P_3\cdot P_4P_1}
=\left|\frac{(x_1-x_2)\cdot(x_3-x_4)}{(x_2-x_3)\cdot (x_4-x_1)}\right|\]
if all $P_i$ lie on a horizontal line $y=b$, and
\[\frac{P_1P_2\cdot P_3P_4}{P_2P_3\cdot P_4P_1}
=\left|\frac{(y_1-y_2)\cdot(y_3-y_4)}{(y_2-y_3)\cdot (y_4-y_1)}\right|\]
otherwise. (See \ref{ex:cross-ratio-area} for another proof.)

To prove \textit{(c)}, apply \textit{(a)}, \textit{(b)}, and \ref{thm:moving}.

\parbf{\ref{ex:proj-cross-ratio=1}.}
Observe that for the perspective projection from $(AB)$ to $(AC)$ with center at $W$ we have
$A\mapsto A$, $B\mapsto C$, $X\mapsto P$, and $Y\mapsto Q$.
Therefore \ref{ex:proj-cross-ratio} implies \textit{(\ref{ex:proj-cross-ratio=1:=})}.

Applying the same argument with center $V$, we get  
$\frac{AY\cdot BX}{AX\cdot BY}=\frac{AP\cdot CQ}{AQ\cdot CP}$.
Therefore 
$\frac{AX\cdot BY}{AY\cdot BX}=\frac{AY\cdot BX}{AX\cdot BY}$,
hence \textit{(\ref{ex:proj-cross-ratio=1:1})}.


\parbf{\ref{ex:pappus}.}
Assume that $(AB)$ meets $(A'B')$ at~$O$.
Since $(AB')\parallel (BA')$, we get that $\triangle OAB'\z\sim\triangle OBA'$
and
$\frac{OA}{OB}=\frac{OB'}{OA'}$.

Similarly, $(AC')\parallel (CA')\ \Longrightarrow\ \frac{OA}{OC}=\frac{OC'}{OA'}$.

Therefore
$\frac{OB}{OC}=\frac{OC'}{OB'}$.
Applying the SAS similarity condition, we get that
$\triangle OBC'\z\sim\triangle OCB'$.
Consequently, $(BC')\parallel (CB')$.

The case $(AB)\parallel(A'B')$ is similar.

\parbf{\ref{ex:pappus-converse}.} Observe that the statement is equivalent to Pappus' theorem.

\parbf{\ref{ex:desargues-construction};} \textit{(\ref{ex:desargues-construction:desargues}).}
Assume that the parallelogram is formed by the two pairs of parallel lines $(AB)\z\parallel (A'B')$ and $(BC)\parallel(B'C')$ and $\ell=(AC)$ in the notation of Desargues' theorem (\ref{thm:desargues}).

\parit{(\ref{ex:desargues-construction:pappus}).} Suppose that the parallelogram is formed by the two pairs of parallel lines $(AB')\z\parallel (A'B)$ and $(BC')\parallel(B'C)$ and $\ell=(AC')$ in the notation of Pappus' theorem (\ref{thm:pappus}).

\parbf{\ref{ex:dual-configurations};}
\textit{(\ref{ex:dual-configurations:infty}).} Observe and use that $\square ABCD$ and $\square ABDX$ are parallelograms.

\parit{(\ref{ex:dual-configurations:dual}).}
Draw $a=(KN)$, $b=(KL)$, $c=(LM)$, $d=(MN)$, mark $P\in b\cap d$, and continue.

\parbf{\ref{ex:dual-euclid}.}
Assume there is a duality.
Choose two distinct parallel lines $\ell$ and~$m$.
Let $L$ and $M$ be their dual points.
Set $s=(ML)$, then its dual point $S$ has to lie on both $\ell$ and $m$ --- a contradiction.

\parbf{\ref{ex:dula-coordinates}.}
Assume $M=(a,b)$ 
and the line $s$ is given by the equation $p\cdot x+q\cdot y=1$.
Then $M\in s$ is equivalent to $p\cdot a+q\cdot b=1$.

This, in turn, is equivalent to $m\ni S$
where $m$ is the line given by the equation 
$a\cdot x+b\cdot y=1$ and $S=(p,q)$.

To extend this bijection to the whole projective plane, assume that 
(1) the ideal line corresponds to the origin 
and (2) the ideal point is given by the pencil of lines $b\cdot x-a\cdot y=c$ for different values of $c$ corresponds to the line defined by $a\cdot x+b\cdot y=0$.

\parbf{\ref{ex:dual-pappus}.}
Assume one set of concurrent lines $a$, $b$, $c$, 
and another set of concurrent lines $a'$, $b'$, $c'$ are given.
Let
\begin{align*}
P&\in b\cap c',
&
Q&\in c\cap a',
&
R&\in a\cap b',\\
P'&\in b'\cap c,
&
Q'&\in c'\cap a,
&
R'&\in a'\cap b.
\end{align*}
Then the lines $(PP')$, $(QQ')$, and $(RR')$ are concurrent.

\begin{Figure}
\vskip-0mm
\centering
\includegraphics{mppics/pic-356}
\vskip-0mm
\end{Figure}

(The obtained configuration of nine points and nine lines is the same as in the original theorem and the obtained result is its reformulation.)

\parbf{\ref{ex:dual-desargues-construction},} \textit{(\ref{ex:dual-desargues-construction:desargues})}.
Assume $(AA')$ and $(BB')$ are the given lines and $C$ is the given point.
Apply the dual Desargues' theorem (\ref{thm:dual-desargues}) to construct $C'$ so that $(AA')$, $(BB')$, and $(CC')$ are concurrent. 
Since $(AA')\z\parallel (BB')$, 
we get that 
$(AA')\z\parallel (BB')\z\parallel (CC')$.

\parit{\textit{(\ref{ex:dual-desargues-construction:pappus})}.} Assume that $P$ is the given point and $(R'Q)$, $(P'R)$ are the given parallel lines.
Try to construct point $Q'$ as in the dual Pappus' theorem (see the solution of \ref{ex:dual-pappus}).

\parbf{\ref{ex:revert}.} Suppose $p=(QR)$; denote by $q$ and $r$ the dual lines produced by the construction.
Then, by \ref{clm:polar}, $P$ is the point of intersection of $q$ and $r$.

\parbf{\ref{ex:tangent ruler}.}
The line $v$ polar to $V$ is tangent to~$\Gamma$.
Since $V\in p$, by \ref{clm:polar}, we get that $P\in v$;
that is, $(PV)=v$.
Hence the statement follows.

\parbf{\ref{ex:concentric-circ}.}
Choose a point $P$ outside of the larger circle.
Construct the lines dual to $P$ for both circles.
Note that these two lines are parallel. 

Assume that the lines intersect the bigger circle at two pairs of points $X$, $X'$ and $Y$, $Y'$.
Let $Z\z\in (XY)\cap (X'Y')$.
Note that the line $(PZ)$ passes thru the common center.

\begin{Figure}
\vskip-0mm
\centering
\includegraphics{mppics/pic-370}
\vskip-6mm
\end{Figure}

The center is the intersection of $(PZ)$ and another line constructed in the same way.

\parbf{\ref{ex:proj-perp}.} 
Construct polar lines for two points on~$\ell$.
Denote their intersection by $L$.
Note that $\ell$ is polar to $L$ and therefore $(OL)\perp \ell$.

%\subsection*{Chapter~\ref{chap:sphere}}
\refstepcounter{chapter}
\setcounter{eqtn}{0}

\parbf{\ref{ex:defect-sphere}.}
Apply \ref{lem:area-spher-triangle}.

\parbf{\ref{ex:s-medians}.}
\textit{(\ref{ex:s-medians:a})}.
Observe and use that 
$OA'\z=OB'\z=OC'$.

\parit{(\ref{ex:s-medians:b}).} Notice that the medians of spherical triangle $ABC$ 
map to the medians of Euclidean a triangle $A'B'C'$.
It remains to apply \ref{thm:centroid} for $\triangle A'B'C'$.

\parbf{\ref{ex:s-altitudes}.}
Apply the reflection in the plane thru $O$, $N$, and $P$.

\parbf{\ref{ex:stereographic-inversion}.}
Apply \ref{thm:inverse}\textit{\ref{thm:inverse:line}}
and \ref{thm:inversion-3d}\textit{\ref{thm:inversion-3d:angle}}.

\parbf{\ref{ex:great-circ}.}
Apply \ref{thm:inversion-3d}\textit{\ref{thm:inversion-3d:b}}.

\parbf{\ref{ex:conform-sphere}.}
Set $z=P'Q'$.
Note that $\tfrac yz\to 1$ as $x\to 0$.

It remains to show that 
$$\lim_{x\to 0} \frac{z}{x}=\frac{2}{1+OP^2}.$$

Recall that the stereographic projection is the inversion across the sphere $\Upsilon$ with the center at the south pole $S$ restricted to the plane $\Pi$.
Show that there is a plane $\Lambda$ passing thru $S$, $P$, $Q$, $P'$, and~$Q'$.
In the plane $\Lambda$, the map $Q\mapsto Q'$ is an inversion across the circle $\Upsilon\cap \Lambda$.

This reduces the problem to Euclidean plane geometry.
The remaining calculations in $\Lambda$ are similar to those in the proof of \ref{lem:conformal}.

\parbf{\ref{ex:cone}.}
Consider the inversion of the base circle across a sphere with its center at the tip of the cone and apply \ref{thm:inversion-3d}.



\parbf{\ref{ex:2(pi/4)=pi/3}.} 
Apply the spherical Pythagorean theorem to show that
$
\cos AB_s\z=\tfrac12
$,
and conclude that $AB_s\z=\tfrac\pi3$.

{

\begin{wrapfigure}{r}{28mm}
\vskip-0mm
\centering
\includegraphics{mppics/pic-358}
\end{wrapfigure}

Alternatively, 
look at the tessellation of the half-sphere in the picture;
it is made from 12 copies of $\triangle_s A B C$ and 4 equilateral spherical triangles.
Due to the symmetry of this tessellation, it follows that $[AB]_s$ occupies $\tfrac16$ of the equator, meaning $AB_s=\tfrac\pi3$.

}

\parbf{\ref{ex:taurinus}.}
Use Euler's formula (\ref{sec:Euler's formula}) to show that $\sin(i\cdot x)=i\cdot\sinh x$ and apply it.
You should get the following answers:
\[\cosh c=\cosh a \cdot \cosh b-\sinh a\cdot \sinh b\cdot \cos\gamma,\]
\[\cos \gamma=-\cos \alpha \cdot \cos \beta+\sin \alpha\cdot \sin \beta \cdot \cosh c,\]
\[\frac{\sin \alpha}{\sinh a}=\frac{\sin \beta}{\sinh b}=\frac{\sin \gamma}{\sinh c}.\]




%\subsection*{Chapter~\ref{chap:klein}}
\refstepcounter{chapter}
\setcounter{eqtn}{0}

%\parbf{\ref{ex:P-->hat-P}.}
%Let $N$, $O$, $S$, $P$, $P'$, and $\hat P$ 
%be as in the picture in \ref{sec:special-bijection}.

%Note that $OQ=\tfrac1x$ and therefore we need to show that $O\hat P=2/(x+\tfrac1x)$.
%To do this, show and use that $\triangle SOP\sim \triangle SP'N\sim \triangle P'\hat PP$ and $2\cdot SO\z=NS$.

\parbf{\ref{ex:hex}.}
Consider the bijection $P\z\leftrightarrow \hat P$ of the h-plane with the absolute~$\Omega$.
Note that $\hat P\z\in [A_iB_i]$ if and only if $P\in\Gamma_i$.

\parbf{\ref{ex:P<->hatP}.} Apply \ref{lem:P-hat-chord} and the definition of h-distance in \ref{sec:conformal-model}.

\parbf{\ref{ex:h-median}.} 
The observation follows since the reflection across the perpendicular bisector of $[PQ]$ is a motion of the Euclidean plane and a motion of the h-plane as well.

We can assume that the center of the circumcircle coincides with the center of the absolute.
In this case, the h-medians of the triangle coincide with the Euclidean medians.
It remains to apply \ref{thm:centroid}.

\begin{Figure}
\centering
\vskip-0mm
\includegraphics{mppics/pic-360}
\end{Figure}

\parbf{\ref{ex:h-altitudes}.} 
Use the projective model.
Assume that two h-altitudes intersect at a point~$H$.
Move $H$ to the center of the absolute.
By \ref{obs:h-p-perp} these h-altitudes become Euclidean altitudes of the triangle.
By \ref{thm:orthocenter}, the remaining Euclidean altitude passes thru $H$.
By \ref{obs:h-p-perp} this Euclidean altitude is also an h-altitude.

\parbf{\ref{ex:klein-perp}.} 
Let $\hat\ell$ and $\hat m$ denote the h-lines in the conformal model that correspond to $\ell$ and $m$.
We need to show that $\hat\ell\perp\hat m$ as arcs in the Euclidean plane.

The point $Z$, where $s$ meets $t$, is the center of the circle $\Gamma$ containing~$\hat\ell$.

If $\bar m$ is passing thru $Z$, then the inversion across $\Gamma$ exchanges the ideal points of~$\hat\ell$.
Consequently, $\hat\ell$ maps to itself. 
Hence the result.

\parbf{\ref{ex:klein-for-angle-parallelism}.}
Let $Q$ be the footpoint of $P$ on the line and $\phi$ be the angle of parallelism. 

{

\begin{wrapfigure}{r}{28mm}
\vskip-2mm
\centering
\includegraphics{mppics/pic-362}
\end{wrapfigure}

We can assume that $P$ is the center of the absolute.
Therefore $PQ=\cos\phi$ and
\[PQ_h=\tfrac12\cdot\ln\frac{1+\cos\phi}{1-\cos\phi}.\]

\parbf{\ref{ex:klein-inradius}.} 
Apply \ref{ex:klein-for-angle-parallelism} for $\phi=\tfrac\pi3$.

}

\parbf{\ref{ex:pyth-h-proj}.}
Note that
$
b=\tfrac12\cdot\ln\frac{1+t}{1-t}$;
therefore
\[
\cosh b
=
\tfrac12\cdot\left(\sqrt{\tfrac{1+t}{1-t}}+\sqrt{\tfrac{1-t}{1+t}}\right)
=
\frac1{\sqrt{1-t^2}}.
\eqlbl{cosh-b}
\]
The same way, we get
\[\begin{aligned}\cosh c&=\frac1{\sqrt{1-u^2}}.
\end{aligned}
\eqlbl{cosh-c}
\]

Let $X$ and $Y$ be the ideal points of~$(BC)_h$.
Applying the Pythagorean theorem (\ref{thm:pyth}) again,
we get that
$CX=CY=\sqrt{1-t^2}$.
Therefore, 
\[
a
=
\tfrac12\cdot\ln\frac{\sqrt{1-t^2}+s}{\sqrt{1-t^2}-s},\]
and
\[
\begin{aligned}
\cosh a&=\tfrac12\cdot
\sqrt{\frac{\sqrt{1-t^2}+s}{\sqrt{1-t^2}-s}}+
\\
&+
\tfrac12\cdot\sqrt{\frac{\sqrt{1-t^2}-s}{\sqrt{1-t^2}+s}}=
\\
&=\frac{\sqrt{1-t^2}}{\sqrt{1-t^2-s^2}}=
\\
&=\frac{\sqrt{1-t^2}}{\sqrt{1-u^2}}.
\end{aligned}
\eqlbl{cosh-a}
\]

Finally, note that \ref{cosh-b}, \ref{cosh-c}, and \ref{cosh-a} imply the theorem.

\parbf{\ref{ex:Boyai-in-Euclid}.}
In the Euclidean plane, the circle $\Gamma_2$ is tangent to $k$; 
that is, $T$ is the only point at intersection of $\Gamma_2$ and $k$.
It defines a unique line $(PT)$ parallel to~$\ell$.

\parbf{\ref{ex:common-perp}.}
Choose two points $P$ and $Q$ on $\ell$.
Drop perpendiculars $p$ and $q$ from $P$ and $Q$ to~$m$
(follow \ref{ex:construction-perpendicular}).
Drop perpendiculars from $P$ to $q$ and from $Q$ to~$p$;
label their point of intersection by~$H$.
(Assume $H$ exists, if not choose other $P$ and $Q$.) 
Drop perpendicular $n$ from $H$ to $m$.

To justify the construction, we need to show that $n\perp \ell$.
To do this, move $H$ to the center of the absolute and apply \ref{obs:h-p-perp}, \ref{ex:klein-perp}, and \ref{thm:orthocenter}.

%\subsection*{Chapter~\ref{chap:complex}}
\refstepcounter{chapter}
\setcounter{eqtn}{0}

\parbf{\ref{ex:|zw|}.} Use that $|z|^2=z\cdot \bar z$ and $\bar z\cdot \bar u=\overline{z\cdot u}$.

\parbf{\ref{ex:ptolemy}.}
Choose a quadrangle $ABCD$.
Assume that $0$, $u$, $v$, and $w$ are complex coordinates of $A$, $B$, $C$, and $D$ respectively.
Rewrite Ptolemy's inequality using $u$, $v$, and $w$.
Deduce this inequality from the provided identity and the triangle inequality.


\parbf{\ref{ex:3-sum-C}.} 
Let $z$, $v$, and $w$ denote the complex coordinates of $Z$, $V$, and $W$ respectively.
Then 
\begin{align*}
&\qquad \measuredangle ZVW+\measuredangle VWZ+\measuredangle WZV\equiv
\\
&\equiv
\arg \tfrac{w-v}{z-v}+\arg \tfrac{z-w}{v-w}+\arg \tfrac{v-z}{w-z}\equiv
\\
&\equiv
\arg \tfrac{(w-v)\cdot(z-w)\cdot(v-z)}{(z-v)\cdot(v-w)\cdot(w-z)}\equiv
\\
&\equiv\arg (-1)\equiv
\pi.
\end{align*}

\parbf{\ref{ex:C-sim}.}
Note and use that 
$
\measuredangle EOV=\measuredangle WOZ\z=\arg v$
and
$\frac{OW}{OZ}=\frac{OZ}{OW}=|v|$.


\parbf{\ref{ex:real-cross-ratio}.}
Note that 
\begin{align*}
&\qquad\arg\frac{(v-u)\cdot(z-w)}{(v-w)\cdot(z-u)}\equiv
\\
&\equiv
\arg\frac{v-u}{z-u}
+
\arg\frac{z-w}{v-w}=
\\
&= \measuredangle ZUV+\measuredangle VWZ.
\end{align*}

The statement follows since the value $\tfrac{(v-u)\cdot(z-w)}{(v-w)\cdot(z-u)}$ is real if and only if 
\[2\cdot\arg\frac{(v-u)\cdot(z-w)}{(v-w)\cdot(z-u)}\equiv0.\]

\parbf{\ref{ex:3-squares}.}
We can choose the complex coordinates so that the points $O$, $E$, $A$, $B$, and $C$ have coordinates
$0$, $1$, $1+i$, $2+i$, and $3+i$ respectively.
Set $\measuredangle EOA=\alpha$, $\measuredangle EOB=\beta$, and $\measuredangle EOC\z=\gamma$.
Note that
\begin{align*}
&\ \ \ \ \alpha+\beta+\gamma\equiv
\\
&\equiv\arg(1+i)+\arg(2+i)+\arg(3+i)\equiv
\\
&\equiv\arg[(1+i)\cdot(2+i)\cdot(3+i)]\equiv
\\
&\equiv\arg [10\cdot i]=
\\
&=\tfrac\pi2.
\end{align*}
Note that these three angles are acute and conclude that $\alpha+\beta+\gamma=\tfrac\pi2$.

\parbf{\ref{ex:6-circles}.}
The identity can be verified by straightforward computations.

By \ref{thm:inscribed-quadrangle-C}, five out of the six cross-ratios in this identity are real.
Consequently, the sixth cross-ratio must also be real.
It remains to apply the theorem again.

\parbf{\ref{ex:4-sim}.}
Use \ref{thm:signs-of-triug} and \ref{cor:half-plane} to show that $\angle UAB$, $\angle BVA$, and $\angle ABW$ have the same sign.
By SAS, we have
\[\frac{AU}{AB}=\frac{VB}{VA}=\frac{BA}{BW},\]
and
\[\measuredangle UAB=\measuredangle BVA=\measuredangle ABW.\]
The latter means that 
\[|\frac{u-a}{b-a}|=|\frac{b-v}{a-v}|=|\frac{a-b}{w-b}|,\]
and
\[\arg\frac{b-a}{u-a}=\arg\frac{a-v}{b-v}=\arg\frac{a-b}{w-b}.\]
It implies the first two equalities in 
\[\frac{b-a}{u-a}=\frac{a-v}{b-v}=\frac{w-b}{a-b}=\frac{w-v}{u-v};\eqlbl{eq:4fractions}\]
the last equality holds since 
\[\frac{(b-a)+(a-v)+(w-b)}{(u-a)+(b-v)+(a-b)}=\frac{w-v}{u-v}.\]

To prove \textit{(b)}, rewrite \ref{eq:4fractions} using angles and distances between the points and apply SAS.

\parbf{\ref{ex:inverse-Mob}.}
Show that the inverse of each elementary transformation is elementary
and use \ref{prop:mob-comp}.

\parbf{\ref{ex:3-point-Mob}.}
The fractional linear transformation
\[f(z)=\frac{(z_1-z_\infty)\cdot(z-z_0)}{(z_1-z_0)\cdot(z-z_\infty)}\]
meets the conditions.

To show the uniqueness, assume there is another fractional linear transformation
$g(z)$ that meets the conditions.
Then the composition
$h=g\circ f^{-1}$ 
is a fractional linear transformation; set
$h(z)=\tfrac{a\cdot z+b}{c\cdot z+d}$.

Note that $h(\infty)=\infty$;
therefore, $c=0$.
Furthermore, $h(0)=0$ implies $b=0$.
Finally, since $h(1)=1$, we get that $\tfrac ad=1$.
Therefore, $h$ is the \index{identity map}\emph{identity};
that is, $h(z)=z$ for any~$z$.
It follows that $g=f$.

\parbf{\ref{ex:inversion-Mob}.}
Let $Z'$ be the inverse of the point $Z$.
Assume that the circle of the inversion has center $W$ and radius~$r$.
Let $z$, $z'$, and $w$ denote the complex coordinates of the points $Z$, $Z'$, and $W$ respectively.

By the definition of an inversion, $\arg (z\z-w)\z=\arg (z'-w)$ and
$|z-w|\cdot|z'-w|=r^2$.
It follows that $(\bar z'-\bar w)\cdot ( z- w)= r^2$.
Equivalently,
\[z'=\overline{\left(\frac{\bar w\cdot z+[r^2-|w|^2]}{z- w}\right)}.\]

 
\parbf{\ref{ex:C-cross-ratio}.}
Check the statement for each elementary transformation.
Then apply \ref{prop:mob-comp}.

\parbf{\ref{ex:schwarz-moebius}.}
Note that $f\:z\mapsto\tfrac{a\cdot z+b}{c\cdot z+d}$ preserves the unit circle $|z|=1$.
Use \ref{cor:invese-comp} and \ref{prop:mob-comp} to show that $f$ commutes with the inversion $z\mapsto 1/\bar z$.
In other words, $1/\overline{f(z)}=f(1/\bar z)$ or
\[\frac{\bar c\cdot \bar z+\bar d}{\bar a\cdot \bar z+\bar b}
=\frac{a/\bar z+b}{c/\bar z+d}\]
for any $z\in\hat{\mathbb{C}}$.
The latter identity leads to the required statement. 
The condition $|w|<|v|$ follows since $f(0)\in\mathbb{D}$.

\parbf{\ref{ex:schwarz-tanh}.} 
Note that the inverses of the points $z$ and $w$ have complex coordinates $1/\bar z$ and $1/\bar w$.
Apply \ref{ex:cosh} and simplify.

The second part follows since the function $x\mapsto \tanh(\tfrac12\cdot x)$ is increasing.

\parbf{\ref{ex:schwarz}.}
Apply Schwarz--Pick theorem for a function $f$ such that $f(0)\z=0$ and then apply \ref{lem:O-h-dist}.

%\subsection*{Chapter~\ref{chap:car}}
\refstepcounter{chapter}
\setcounter{eqtn}{0}

%{
%
%\begin{wrapfigure}{r}{37mm}
%\vskip-4mm
%\centering
%\includegraphics{mppics/pic-364}
%\end{wrapfigure}
%
%\parbf{\ref{ex:simple-apollonius}.}
%Let $O$ be the point of intersection of the lines.
%Construct a circle $\Gamma$, tangent to both lines, that crosses~$[OP)$;
%denote its center by $I$.
%Suppose that $X$ denotes one of the points of intersections of $\Gamma$ and $[OP)$.
%
%Construct a point $I'\in[OI)$ such that $\tfrac{OI'}{OI}=\tfrac{OP}{OX}$.
%Observe that the circle passing thru $P$ and centered at $I'$ is a solution.
%
%}

\parbf{\ref{ex:a2/b}.}
To construct  $\sqrt{a\cdot b}$:
(1) construct points $A$, $B$, and $D\z\in [AB]$
such that $AD=a$ and $BD=b$;
(2) construct the circle $\Gamma$ on the diameter $[AB]$;
(3) draw the line $\ell$ thru $D$ perpendicular to $(AB)$; 
(4) let $C$ be an intersection of $\Gamma$ and~$\ell$.
Then $DC= \sqrt{a\cdot b}$.


The construction of $\tfrac{a^2}b$ is analogous.

\parbf{\ref{ex:5-gon},} \textit{(\ref{ex:5-gon:a})}.
Look at the picture;
show that the angles marked the same way have the same angle measure.

Conclude that $XC=BC$ and $\triangle ABC\z\sim \triangle AXB$.
Therefore 
\[\frac{AB}{AC}=\frac{AX}{AB}=\frac{AC-AB}{AB}=\frac{AC}{AB}-1.\]
It remains to solve for $\frac{AC}{AB}$.

{

\begin{wrapfigure}[7]{r}{27mm}
\vskip-8mm
\centering
\includegraphics{mppics/pic-366}
\end{wrapfigure}

\parit{(\ref{ex:5-gon:b}).}
Choose two points $P$ and $Q$ and use the ruler-and-compass calculator to construct two points $V$ and $W$ such that $VW\z=\tfrac{1+\sqrt5}2\cdot PQ$.
Then construct a pentagon with the sides $PQ$ and diagonals $VW$.

}

\parbf{\ref{ex:trisect-set-square}.} 
Note that with a set-square we can construct a line parallel to a given line thru the given point.
It remains to modify the construction in \ref{ex:midpoint-affine}.

\parbf{\ref{ex:equilateral triangle}.}
Choose a coordinate system so that the given vertices are $(0,0)$ and $(1,0)$.
Show that the remaining vertex is $(\tfrac12,\pm\tfrac{\sqrt{3}}2)$.
Observe that it is an irrational point; apply \ref{thm:set-square-constructible-numbers}.%
\footnote{It is okay to use that $\sqrt{3}$ is irrational without proving it.
But let us explain why it is true. 
Assume the contrary; that is, $\tfrac mn=\sqrt{3}$ for integers $m$ and $n$.
We can assume that $m$ and $n$ do not share a prime factor; in particular, if $m$ is divisible by $3$, then $n$ is not.
Observe that $m^2=3\cdot n^2$.
It follows that $m$ is divisible by 3; that is, $m=3\cdot k$ for an integer $k$.
It follows that $3\cdot k^2=n^2$.
Therefore, $n$ is divisible by 3 --- a contradiction.} 

\parbf{\ref{ex:set-square-bisect}.}
Assume that one can construct a bisector of $\angle AOB$, where $A=(1,0)$, $O=(0,0)$, and $C=(1,1)$.
Let $D$ be the point of intersection of the bisector with the line $(AB)$.
Show that $D$ is an irrational point.
Apply \ref{thm:set-square-constructible-numbers} and arrive at a contradiction.

\parbf{\ref{ex:90-60-30}.}
Suppose that every initial point has coordinates $(a,b\cdot\sqrt{3})$ for rational values $a$ and $b$.
Show and use that any point that can be constructed with the 30°-set-square has coordinates of the same type.

\begin{wrapfigure}[8]{r}{21mm}
\vskip-0mm
\centering
\includegraphics{mppics/pic-368}
\end{wrapfigure}

\parbf{\ref{ex:equilateral triangle-verify},} \textit{(\ref{ex:verify:triangle})}.
Observe that three perpendiculars in the picture meet at one point if and only if the triangle is isosceles.

Use this observation a couple of times to verify that the given triangle is equilateral.

\parit{(\ref{ex:verify:bisector}).}
Suppose that a line $\ell$ passes thru the vertex of the given angle.
Choose a point $P\in \ell$.
Suppose $X$ and $Y$ are the footpoints of $P$ on the sides of the angle.
Show and use that $(XY)\perp \ell$ if and only if $\ell$ bisects the angle.

\parbf{\ref{ex:midpoint-proj}.}
Consider the perspective projection 
$(x,y)\mapsto (\tfrac 1x,\tfrac yx)$ (see \ref{sec:perspective-projection}).
Let $A\z=(1,1)$, $B\z=(3,1)$, and $M\z=(2,1)$.
Note that $M$ is the midpoint of $[AB]$.

Their images are $A'\z=(1,1)$, $B'\z=(\tfrac13,\tfrac13)$, and $M'\z=(\tfrac12,\tfrac12)$.
Clearly, $M'$ is not the midpoint of~$[A'B']$.


\parbf{\ref{ex:comparison}.}
$(a)$ is strictly stronger than $(b)$,
$(b)$ is strictly stronger than $(c)$,
$(a)$ is strictly stronger than $(d)$,
and $(d)$ is not comparable with $(b)$ and $(c)$.
Most of these statements follow from \ref{ex:construction-perpendicular},
\ref{prop:perp-perp},
\ref{ex:consturuction-of-inversion},
\ref{ex:affine-perp},
\ref{ex:equilateral triangle}, 
\ref{prob:center-inversor+circumtool}.

To show that $(d)$ is not stronger than $(c)$, show that one cannot construct a midpoint using the set $(d)$ and use the solution of \ref{ex:midpoint-affine}.
To show that $(b)$ is not stronger than $(d)$, show that given the initial configuration of 6 points 
$(0,0)$, 
$(1,0)$,
$(2,0)$,
$(0,1)$, 
$(1,1)$,
$(2,1)$,
one can construct an equilateral triangle using the set $(d)$ and apply \ref{ex:equilateral triangle}.

%\subsection*{Chapter~\ref{chap:area}}
\refstepcounter{chapter}
\setcounter{eqtn}{0} 

\parbf{\ref{ex:triangle-convex}.}
Assume the contrary; 
that is, there is a point $W\in [XY]$ such that $W\notin\solidtriangle ABC$.

Without  loss of generality, we may assume that $W$ and $A$ lie on opposite sides of the line~$(BC)$.

It implies that both segments $[WX]$ and $[WY]$ intersect $(BC)$.
By Axiom~\ref{def:birkhoff-axioms:1}, $W\in (BC)$ --- a contradiction.

%\parbf{\ref{ex:solid-triangle-sum}.} Without loss of generality, we may assume that the angles $ACB$, $BAC$, and $CBA$ are positive.
%By \ref{ex:signs-PXQ-PYQ}, $X\in \solidtriangle ABC$ if and only if $\measuredangle AXB, \measuredangle BXC,\measuredangle CXA\ge 0$.
%Note that 
%\[\measuredangle AXB+\measuredangle BXC+\measuredangle CXA\equiv0.\]

\parbf{\ref{ex:vertex}.} 
To prove the ``only if'' part, consider the line passing thru the vertex that is parallel to the opposite side.

To prove the ``if'' part, use Pasch's theorem (\ref{thm:pasch}).

\parbf{\ref{ex:solid-square}.}
Assume the contrary; that is, suppose a solid square \( \mathcal{K} \) can be represented as the union of a finite collection of segments \( [A_1B_1],\dots,[A_nB_n] \)
and single-point sets \( \{C_1\},\dots,\{C_k\} \).

Note that \( \mathcal{K} \) contains an infinite number of mutually nonparallel segments.
Therefore, we can choose a segment \( [XY] \) in \( \mathcal{K} \) that is not parallel to any of the segments \( [A_1B_1],\dots,[A_nB_n] \).

It follows that \( [XY] \) has at most one common point with each of the sets \( [A_iB_i] \) and~\( \{C_i\} \).
Since \( [XY] \) contains infinitely many points, we reach a contradiction.

\smallskip

Alternatively, choose a circle \( \Gamma \) inside \( \mathcal{K} \).
Note that \( \Gamma \) contains infinitely many points, but by \ref{lem:line-circle}, it has only finitely many intersection points with any degenerate polygonal set.

Let us also note that the statement will follow from \ref{cor:degenerate} and \ref{thm:area-rect}.

\parbf{\ref{ex:poly-circ}.} 
Show that among elementary sets
only one-point sets can be subsets of a circle.
It remains to note that any circle contains an infinite number of points.

\parbf{\ref{ex:two-parallelograms}.}
Suppose that $E$ denotes the point of intersection of the lines $(BC)$ and~$(C'D')$.

\begin{wrapfigure}[7]{r}{28mm}
\vskip-2mm
\centering
\includegraphics{mppics/pic-372}
\end{wrapfigure}

Use \ref{prop:area-parallelogram} to prove that the following solid parallelograms have the same area:
$\solidsquare ABCD$, $\solidsquare AB'ED$, and $\solidsquare AB'C'D'$.

\parbf{\ref{ex:three-trig}.}
Without loss of generality, we may assume that the angles $ABC$ and $BCA$ are acute.

Let $A'$ and $B'$ denote the footpoints of $A$ and $B$ on $(BC)$ and $(AC)$ respectively.
Note that $h_A=AA'$ and $h_B=BB'$.

Note that $\triangle AA'C\sim \triangle BB'C$;
indeed the angle at $C$ is shared and the angles at $A'$ and $B'$ are right.
In particular
$\frac{AA'}{BB'}=\frac{AC}{BC}$,
or equivalently, $h_A\cdot BC=h_B\cdot AC$.

\begin{Figure}
\vskip-0mm
\begin{minipage}{.49\textwidth}
\centering
\includegraphics{mppics/pic-374}
\end{minipage}
\hfill
\begin{minipage}{.49\textwidth}
\centering
\includegraphics{mppics/pic-376}
\end{minipage}
\end{Figure}

\parbf{\ref{ex:half-parallelogram}.}
Draw the line $\ell$ 
thru $M$ parallel to $[AB]$ and $[CD]$;
it subdivides $\solidsquare ABCD$ into two solid parallelograms
which we will denote as
$\solidsquare ABEF$ and
$\solidsquare CDFE$.
In particular,
\begin{align*}
&\area(\solidsquare ABCD)=
\\
&\qquad=
\area(\solidsquare ABEF)+\area(\solidsquare CDFE).
\end{align*}

By \ref{prop:area-parallelogram} and \ref{thm:area-of-triangle} we get that 
\begin{align*}
\area(\solidtriangle ABM)&=\tfrac12\cdot\area(\solidsquare ABEF),
\\
\area(\solidtriangle CDM)&=\tfrac12\cdot\area(\solidsquare CDFE)
\end{align*}
and hence the result.

\parbf{\ref{ex:area-diag}.}
Let $h_A$ and $h_C$ denote the distances from $A$ and $C$ to the line~$(BD)$ respectively.
According to \ref{thm:area-of-triangle},
\begin{align*}
\area(\solidtriangle ABM)&=\tfrac12\cdot h_A\cdot BM;
\\
\area(\solidtriangle BCM)&=\tfrac12\cdot h_C\cdot BM;
\\
\area(\solidtriangle CDM)&=\tfrac12\cdot h_C\cdot DM;
\\
\area(\solidtriangle ABM)&=\tfrac12\cdot h_A\cdot DM.
\end{align*}
Therefore
\begin{align*}
&\area(\solidtriangle ABM)\cdot \area(\solidtriangle CDM)=
\\
&\qquad=\tfrac14 \cdot h_A\cdot h_C\cdot DM\cdot BM=
\\
&\qquad=\area(\solidtriangle BCM)\cdot \area(\solidtriangle DAM).
\end{align*}

\parbf{\ref{ex:area-inradius}.}
Let $I$ be the incenter of $\triangle ABC$.
Note that $\solidtriangle ABC$
can be subdivided into 
$\solidtriangle IAB$, 
$\solidtriangle IBC$,
and $\solidtriangle ICA$.

It remains to apply \ref{thm:area-of-triangle} 
to each of these triangles and sum up the results.

\parbf{\ref{ex:subdivision}.} Fix a polygonal set $\mathcal{P}$.
Without loss of generality, we may assume that $\mathcal{P}$ is a union of a finite collection of solid triangles.
Cut $\mathcal{P}$ along the extensions of the sides of all the triangles,
it subdivides $\mathcal{P}$ into convex polygons.
Cutting each polygon by diagonals from one vertex produces a subdivision into solid triangles.

\parbf{\ref{ex:pyth-2}.}
Assuming $a>b$,
we have subdivided $\mathcal{K}_c$ into $\mathcal{K}_{a-b}$ and four triangles congruent to~$\mathcal{T}$.
Therefore
\[\area\mathcal{K}_c=\area\mathcal{K}_{a-b}+4\cdot\area\mathcal{T}.
\eqlbl{eq:pyth-2}\]

According to \ref{thm:area-of-triangle},
$\area\mathcal{T}=\tfrac12\cdot a\cdot b$. 
Therefore, the identity \ref{eq:pyth-2} can be written as 
\[c^2=(a-b)^2+2\cdot a\cdot b.\]
Simplifying, we get the Pythagorean theorem.

Case $a=b$ is simpler.
Case $b>a$ can be done in the same way.


\parbf{\ref{ex:sum-3-dist}.} 
If $X$ is a point inside of $\triangle ABC$, then $\solidtriangle ABC$ is subdivided into $\solidtriangle ABX$, $\solidtriangle BCX$, and $\solidtriangle CAX$.
Therefore
\begin{align*}
&\area(\solidtriangle ABX)
+\area(\solidtriangle BCX)+
\\
&\qquad+\area(\solidtriangle CAX)
=\area(\solidtriangle ABC).
\end{align*}

Set $a=AB=BC=CA$.
Let $h_1$, $h_2$, and $h_3$ denote the distances from $X$ to the sides $[AB]$, $[BC]$, and~$[CA]$. 
Then by \ref{thm:area-of-triangle},
\begin{align*}
\area(\solidtriangle ABX)&=\tfrac12\cdot h_1\cdot a,
\\
\area(\solidtriangle BCX)&=\tfrac12\cdot h_2\cdot a,
\\
\area(\solidtriangle CAX)&=\tfrac12\cdot h_3\cdot a.
\end{align*}
Therefore, 
\[h_1+h_2+h_3=\tfrac2a\cdot\area(\solidtriangle ABC).\]

\parbf{\ref{ex:area-medians}.}
Apply \ref{thm:centroid} and \ref{clm:area-ratio} several times.

\parbf{\ref{ex:ceva}.}
Apply \ref{clm:area-ratio} to show that
\[\frac{\area(\solidtriangle ABX)}{\area(\solidtriangle ABB')}=\frac{BX}{BB'}=\frac{\area(\solidtriangle BCX)}{\area(\solidtriangle CBB')}\]
and, therefore,
\[\frac{\area(\solidtriangle ABX)}{\area(\solidtriangle BCX)}=\frac{\area(\solidtriangle ABB')}{\area(\solidtriangle CBB')}=\frac{AB'}{B'C}.\]

It implies the first identity; the rest is analogous.
Multipying the identities, we get the last statement.

\parbf{\ref{ex:cross-ratio-area}.}
To prove \textit{(\ref{ex:cross-ratio-area:a})}, apply \ref{clm:area-ratio} twice to $\solidtriangle OL_iL_j$, $\solidtriangle OL_jM_i$, and $\solidtriangle OM_iM_j$.

To prove part~\textit{(\ref{ex:cross-ratio-area:b})}, use \ref{clm:area-ratio} to rewrite the left-hand side using the areas of $\solidtriangle OL_1L_2$, $\solidtriangle OL_2L_3$, $\solidtriangle OL_3L_4$, and $\solidtriangle OL_4L_1$.
Furthermore, use part \textit{(\ref{ex:cross-ratio-area:a})} to rewrite it using areas of $\solidtriangle OM_1M_2$, $\solidtriangle OM_2M_3$, $\solidtriangle OM_3M_4$, and $\solidtriangle OM_4M_1$ and apply \ref{clm:area-ratio} again to get the right-hand side.


\parbf{\ref{ex:circle-is-quadrable}.}
Let $\mathcal{P}_n$ and $\mathcal{Q}_n$ be the solid regular $n$-gons
so that $\Gamma$ is inscribed in $\mathcal{Q}_n$ and circumscribed around~$\mathcal{P}_n$.
Clearly,
$\mathcal{P}_n\subset\mathcal{D}\subset\mathcal{Q}_n$.

Show that 
$\tfrac{\area\mathcal{P}_n}{\area\mathcal{Q}_n}=(\cos\tfrac\pi n)^2$;
in particular, 
$$\frac{\area\mathcal{P}_n}{\area\mathcal{Q}_n}\to 1
\quad
\text{as}
\quad
n\to\infty.$$
Next show that $\area\mathcal{Q}_n<100$ foe any large $n$.

These two statements imply that
\[(\area\mathcal{Q}_n\z-\area\mathcal{P}_n)\to 0.\]
Hence the result.

\spell{\end{multicols}}{}

\newpage

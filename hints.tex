\backmatter
\addtocontents{toc}{\protect\begin{center}}
\addtocontents{toc}{\large{\bf References}}
\addtocontents{toc}{\protect\end{center}}
\chapter{Hints}
%\addcontentsline{toc}{chapter}{Hints}
%\subsection*{Chapter~\ref{chap:metr}}
\refstepcounter{chapter}
\setcounter{eqtn}{0}

\parbf{Exercise~\ref{ex:d_1+d_2+d_infty}.} 
Only the triangle inequality requires a proof --- 
the rest of conditions in Definition~\ref{def:metric-space} are evident.
Let $A=(x_A,y_A)$, $B=(x_B,y_B)$ and $C=(x_C,y_C)$.
Set 
\begin{align*}
x_1&=x_B-x_A, 
&
y_1&=y_B-y_A,
\\
x_2&=x_C-x_B,
&
y_2&=y_C-y_B.
\end{align*}

\parit{(a).}
The inequality
$$d_1(A,C)\le d_1(A,B)+d_1(B,C)$$
can be written as 
$$|x_1+x_2|+|y_1+y_2|
\le 
|x_1|+|y_1|+|x_2|+|y_2|.$$
The latter follows since $|x_1+x_2|\le |x_1|+|x_2|$ 
and
$|y_1+y_2|\le |y_1|+|y_2|$.

\parit{(b).}
The inequality
$$d_2(A,C)\le d_2(A,B)+d_2(B,C)\eqlbl{eq:trig-inq-d2}$$
can be written as 
$$\sqrt{\bigl(x_1+x_2\bigr)^2+\bigl(y_1+y_2\bigr)^2}
\le 
\sqrt{x_1^2+y_1^2}+\sqrt{x_2^2+y_2^2}.$$
Take the square of the left and the right hand sides,
simplify,
take the square again and simplify again.
You should get the following inequality
$$0
\le 
(x_1\cdot y_2-x_2\cdot y_1)^2,$$
which is equivalent to \ref{eq:trig-inq-d2}
and evidently true.

\parit{(c).}
The inequality
$$d_\infty(A,C)\le d_\infty(A,B)+d_\infty(B,C)$$
can be written as 
$$\max\{|x_1+x_2|,|y_1+y_2|\}
\le 
\max\{|x_1|,|y_1|\}+\max\{|x_2|,|y_2|\}.\eqlbl{eq:max-trig}$$
Without loss of generality, we may assume that 
$$\max\{|x_1+x_2|,|y_1+y_2|\}=|x_1+x_2|.$$
Further,
\begin{align*}
|x_1+x_2|&\le |x_1|+|x_2|\le 
\max\{|x_1|,|y_1|\}+\max\{|x_2|,|y_2|\}.
\end{align*}
Hence \ref{eq:max-trig} follows.

\parbf{Exercise~\ref{ex:dist-preserv=>injective}.}
If $A\ne B$, then $d_\mathcal{X}(A,B)>0$.
Since $f$ is distance-preserving,
$$d_\mathcal{Y}(f(A),f(B))=d_\mathcal{X}(A,B).$$
Therefore, $d_\mathcal{Y}(f(A),f(B))>0$; hence $f(A)\ne f(B)$.

\parbf{Exercise~\ref{ex:motion-of-R}.}
Set $f(0)=a$ and $f(1)=b$.
Note that $b=a+1$ or $a-1$.
Moreover, $f(x)=a\pm x$ and at the same time, $f(x)=b\pm(x-1)$ for any~$x$.

If $b=a+1$, 
it follows that 
$f(x)=a+x$ for any~$x$.

The same way, if $b=a-1$, 
it follows that 
$f(x)=a-x$ for any~$x$.

\parbf{Exercise~\ref{ex:d_1=d_infty}.} 
Show that the map $(x,y)\mapsto (x+y,x-y)$ is an isometry  $(\mathbb{R}^2,d_1)\z\to (\mathbb{R}^2,d_\infty)$.
That is, you need to check if this map is bijective and distance-preserving.

\parbf{Exercise~\ref{ad-ex:motions of Manhattan plane}.} 
First prove that \textit{two points $A=(x_A,y_A)$ and $B\z=(x_B,y_B)$ on the Manhattan plane have a unique midpoint if and only if $x_A=x_B$ or $y_A=y_B$}; compare with the example on page~\pageref{example:isometric but not congruent}. 

Then use above statement to prove that
any motion of the Manhattan plane 
can be written in one of the following two ways
\begin{align*}
(x,y)&\mapsto (\pm x+a,\pm y+b)
&&\text{or} 
&(x,y)&\mapsto (\pm y+b,\pm x+a),
\end{align*}
for some fixed real numbers $a$ and~$b$.
(In each case we have 4 choices of signs, so for fixed pair $(a,b)$ we have 8 distinct motions.)

\parbf{Exercise~\ref{ex:y=|x|}.}
Assume three points $A$, $B$ and $C$ lie on one line.
Note that in this case one of the triangle inequalities with the points $A$, $B$ and $C$ becomes an equality.

Set $A=(-1,1)$, $B=(0,0)$ and $C=(1,1)$.
Show that for $d_1$ and $d_2$
all the triangle inequalities with the points $A$, $B$ and $C$ are strict.
It follows that the graph is not a line.

For $d_\infty$ show that $(x,|x|)\mapsto x$ gives the isometry of the graph to~$\mathbb{R}$.
Conclude that the graph is a line in $(\mathbb{R}^2,d_\infty)$.

\parbf{Exercise~\ref{ex:2mid}.}
Applying the definition of lines,
the problems are reduced to the following.

Assume that $a\ne b$,
find the number of solutions for each of the following two equations
\begin{align*}
|x-a|&=|x-b|
&&\text{and}
&|x-a|&=2\cdot |x-b|.
\end{align*}

Each can be solved by taking the square of the left and the right hand sides.
The numbers of solutions are 1 and 2 correspondingly.

\parbf{Exercise~\ref{ex:trig==}.}
Fix an isometry $f\:(P Q)\to \mathbb{R}$ such that $f(P)=0$ and $f(Q)=q>0$.

Assume that $f(X)=x$.
By the definition of the half-line $X\in[PQ)$ if and only if $x\ge 0$.
Show that the latter holds if and only if 
\[|x-q|=\bigl||x|-|q|\bigr|.\]
Hence the result will follow.

\parbf{Exercise~\ref{ex:2a=0}.}
The equation
$2\cdot\alpha\equiv 0$
means that $2\cdot\alpha=2\cdot k\cdot\pi$ for some integer~$k$.
Therefore,
$\alpha=k\cdot\pi$ for some integer~$k$.

Equivalently, $\alpha=2\cdot n\cdot \pi$ or $\alpha=(2\cdot n+1)\cdot \pi$ for some integer~$n$.
The first identity means that $\alpha\equiv 0$ and the second means that $\alpha\equiv \pi$.

\parbf{Exercise~\ref{ex:dist-cont}.} \textit{(a).}
By the triangle inequality,
$$|f(A')-f(A)|\le d(A',A).$$
Therefore, we can take $\delta=\epsilon$.

\parit{(b).}
By the triangle inequality,
\begin{align*}
|f(A',B')-f(A,B)|
&\le |f(A',B')-f(A,B')|
+|f(A,B')-f(A,B)|
\le
\\
&\le d(A',A)+d(B',B).
\end{align*}
Therefore, we can take $\delta=\tfrac\epsilon2$.

\parbf{Exercise~\ref{ex:comp+cont}.}
Fix $A\in \mathcal{X}$ and $B\in\mathcal{Y}$
such that $f(A)=B$.

Fix $\epsilon>0$.
Since $g$ is continuous at $B$, there is a positive value $\delta_1$ such that 
$$d_{\mathcal{Z}}(g(B'),g(B))<\epsilon
\quad
\text{if}
\quad
d_{\mathcal{Y}}(B',B)<\delta_1.$$ 

Since $f$ is continuous at $A$, there is $\delta_2>0$ such that 
$$d_{\mathcal{Y}}(f(A'),f(A))\z<\delta_1
\quad
\text{if}
\quad
d_{\mathcal{X}}(A',A)<\delta_2.$$ 

Since $f(A)=B$, we get that
$$d_{\mathcal{Z}}(h(A'),h(A))<\epsilon
\quad
\text{if}
\quad
d_{\mathcal{X}}(A',A)<\delta_2.$$ 
Hence the result.

%\subsection*{Chapter~\ref{chap:axioms}}
\refstepcounter{chapter}
\setcounter{eqtn}{0}

\parbf{Exercise~\ref{ex:infinite}.} By Axiom~\ref{def:birkhoff-axioms:0}, there are at least two points in the plane.
Therefore, by Axioms~\ref{def:birkhoff-axioms:1}, 
the plane contains a line. 
It remains to note that line is an infinite set of points.

\parbf{Exercise~\ref{ex:[OA)=[OA')}.}
By Axiom~\ref{def:birkhoff-axioms:1},
$(OA)=(OA')$.
Therefore, the statement boils down two the following:

\textit{Assume $f\:\mathbb{R}\to \mathbb{R}$ is a motion of the line which sends $0\to 0$ and one positive number to a positive number, then $f$ is an identity map.}

The latter follows from Exercise~\ref{ex:motion-of-R}.

\parbf{Exercise~\ref{ex:2.4}.}
By Proposition~\ref{lem:AOA=0},
$\measuredangle AOA=0$.
It remains to apply Axiom~\ref{def:birkhoff-axioms:2a}.

\parbf{Exercise~\ref{ex:lineAOB}.}
Apply Proposition~\ref{lem:AOA=0},
Theorem~\ref{thm:straight-angle} 
and Exercise~\ref{ex:2a=0}.

\parbf{Exercise~\ref{ex:ABCO-line}.}
By Axiom~\ref{def:birkhoff-axioms:2b},
$$2\cdot\measuredangle BOC
\equiv 
2\cdot\measuredangle AOC-2\cdot \measuredangle AOB
\equiv 0.$$
By Exercise~\ref{ex:2a=0}, 
it implies that 
$\measuredangle BOC$ is either $0$ or~$\pi$.
It remains to apply Exercise~\ref{ex:2.4} and Theorem~\ref{thm:straight-angle} correspondingly in these two cases.

\parbf{Exercise~\ref{ex:infinite-number-of-lines}.}
Fix two points $A$ and $B$ provided by Axiom~\ref{def:birkhoff-axioms:0}.

Fix a real number $0<\alpha<\pi$.
By Axiom~\ref{def:birkhoff-axioms:2a} there is a point $C$ such that $\measuredangle ABC=\alpha$.

Use Proposition~\ref{lem:line-line} to show that $\triangle ABC$ is nondegenerate.

\parbf{Exercise~\ref{ex:O-mid-AB+CD}.} 
Applying Proposition~\ref{prop:vert}, we get that
$\measuredangle AOC= \measuredangle BOD$.
It remains to apply Axiom~\ref{def:birkhoff-axioms:3}.

%\subsection*{Chapter~\ref{chap:half-planes}}
\refstepcounter{chapter}
\setcounter{eqtn}{0}

\parbf{Exercise~\ref{ex:AOB+<=>BOA-}.}
Set $\alpha=\measuredangle AOB$ 
and 
$\beta=\measuredangle BOA$.
Note that $\alpha=\pi$ if and only if $\beta=\pi$.
Otherwise $\alpha=-\beta$.
Hence the result.

\parbf{Exercise~\ref{ex:PP(PN)}.}
Set $\alpha=\measuredangle AOB$, $\beta=\measuredangle BOC$ and $\gamma=\measuredangle COA$.
By Axiom~\ref{def:birkhoff-axioms:2b} and Proposition~\ref{lem:AOA=0}, we have
$$\alpha+\beta+\gamma\equiv 0.
\eqlbl{alph+bet+gam}$$

Note that $0<\alpha+\beta<2\cdot\pi$ and $|\gamma|\le \pi$.
If $\gamma> 0$, then \ref{alph+bet+gam} implies
$$\alpha+\beta+\gamma=2\cdot\pi$$
and 
if $\gamma<0$, then \ref{alph+bet+gam} implies
$$\alpha+\beta+\gamma=0.$$

\parbf{Exercise~\ref{ex:vert-intersect}.}
Note that $O$ and $A'$
lie on the same side from~$(AB)$.
Analogously $O$ and $B'$
lie on the same side from~$(AB)$.
Hence the result.

\parbf{Exercise~\ref{ex:signs-PXQ-PYQ}.}
Apply Theorem~\ref{thm:signs-of-triug} for $\triangle PQX$ and $\triangle PQY$ and then 
apply Corollary~\ref{cor:half-plane}\textit{\ref{cor:half-plane:angle}}.

\parbf{Exercise~\ref{ex:chevinas}.} 
Note that it is sufficient to consider the cases when $A'\z\ne B,C$ and $B'\ne A, C$.

Apply Pasch's theorem (\ref{thm:pasch}) twice:
(1) for $\triangle AA'C$ and $(BB')$, and 
(2) for $\triangle BB'C$ and~$(AA')$.

\parbf{Exercise~\ref{ex:Z}.}
Assume that $Z$ is the point of intersection.

Note that $Z\ne P$ and $Z\ne Q$.
Therefore, $Z\notin (PQ)$.

Show that $Z$ and $X$ lie on one side from~$(PQ)$.
Repeat the argument to show that $Z$ and $Y$ lie on one side from~$(PQ)$.
It follows that $X$ and $Y$ lie on the same side from $(PQ)$ --- a contradiction.

\parbf{Exercise~\ref{ex:intersecting-circles-3}.} The ``only-if'' part follows from the triangle inequality.
To prove ``if'' part apply Theorem \ref{thm:abc} for the values $r_1$, $r_2$ and $d$.


%\subsection*{Chapter~\ref{chap:cong}}
\refstepcounter{chapter}
\setcounter{eqtn}{0}

\parbf{Exercise~\ref{ex:equilateral}.}
Apply Theorem~\ref{thm:isos} twice.

\parbf{Exercise~\ref{ex:SMS}.} 
Consider the points $D$ and $D'$, such that 
$M$ is the midpoint of $[AD]$
and 
$M'$ is the midpoint of~$[A'D']$.
Show that $\triangle ABD\z\cong \triangle A'B'D'$ and use it to prove that $\triangle A' B' C'\z\cong\triangle A B C$.

\parbf{Exercise~\ref{ex:isos-sides}.} \textit{(a)} Apply SAS.

\parit{(b)} Use \textit{(a)} and apply SSS.

\parbf{Exercise~\ref{ex:degenerate-trig}.}
Choose $B'\in [AC]$ such that $AB=AB'$.
Note that $BC\z=B'C$.
By SSS, 
 $\triangle ABC\cong \triangle AB'C$.

\parbf{Exercise~\ref{ex:ABC-motion}.}
Without loss of generality, we may assume that $X$ is distinct from $A$, $B$ and~$C$.
Set $f(X)=X'$; assume $X'\ne X$.

Note that $AX=AX'$, $BX=BX'$ and $CX=CX'$.
By SSS we get that $\measuredangle ABX\z=\pm\measuredangle ABX'$.
Since $X\ne X'$, we get that
$$\measuredangle ABX\equiv - \measuredangle ABX'.$$
The same way we get that 
$$\measuredangle CBX\equiv - \measuredangle CBX'.$$
Subtracting these two identities from each other, we get that
$$\measuredangle ABC\equiv -\measuredangle ABC.$$
Conclude that $\measuredangle ABC=0$ or $\pi$.
That is, $\triangle ABC$ is degenerate --- a contradiction. 

%\subsection*{Chapter~\ref{chap:perp}}
\refstepcounter{chapter}
\setcounter{eqtn}{0}

\parbf{Exercise~\ref{ex:acute-obtuce}.} 
By Axiom~\ref{def:birkhoff-axioms:2b} and Theorem~\ref{thm:straight-angle}, we have
\[\measuredangle XOA-\measuredangle XOB\equiv\pi.\]
Since $|\measuredangle XOA|,|\measuredangle XOB|\le \pi$, we get that
\[|\measuredangle XOA|+|\measuredangle XOB|=\pi.\]
Hence the statement follows. 

\parbf{Exercise~\ref{ex:pbisec-side}.}
Assume $X$ and $A$ lie on the same side from~$\ell$.

Note that $A$ and $B$ lie on opposite sides of~$\ell$.
Therefore, by Corollary~\ref{cor:half-plane}, 
$[AX]$ does not intersect $\ell$ 
and $[BX]$ intersects $\ell$;
let $Y$ denotes the intersection point.

\begin{wrapfigure}[7]{r}{25mm}
\begin{lpic}[t(-0mm),b(0mm),r(0mm),l(0mm)]{pics/pbisec-side(1)}
\lbl[tr]{1,1;$A$}
\lbl[tl]{23,1;$B$}
\lbl[br]{7,21;$X$}
\lbl[l]{13.5,15;$Y$}
\lbl[l]{13,21;$\ell$}
\end{lpic}
\end{wrapfigure}

Note that $Y\notin [AX]$.
By Exercise~\ref{ex:degenerate-trig},
$$BX=AY+YX>AX.$$

This way we proved the ``if'' part.
To prove the ``only if'' part, it remains to switch $A$ and $B$,
repeat the above argument and apply Theorem~\ref{thm:perp-bisect}.

\parbf{Exercise~\ref{ex:side-angle}.}
Apply Exercise~\ref{ex:pbisec-side}, Theorem~\ref{thm:ASA} and Exercise~\ref{ex:PP(PN)}.

\parbf{Exercise~\ref{ex:3-reflections}.}
Choose an arbitrary nondegenerate triangle $ABC$.
Let $\triangle \hat A \hat B\hat C$ denotes its image after the motion.

If $A\ne \hat A$, apply the reflection in the perpendicular bisector of~$[A\hat A]$.
This reflection sends $A$ to~$\hat A$.
Let $B'$ and $C'$ denote the reflections of $B$ and $C$ correspondingly.

If $B'\ne \hat B$, apply the reflection in the perpendicular bisector of~$[B'\hat B]$.
This reflection sends $B'$ to~$\hat B$.
Note that $\hat A\hat B=\hat AB'$;
that is, $\hat A$ lies on the perpendicular bisector. 
Therefore, $\hat A$ reflects to itself.
Let $C''$ denotes the reflection of~$C'$.

Finally, if $C''\ne \hat C$, apply the reflection in $(\hat A\hat B)$.
Note that $\hat A\hat C\z=\hat AC''$ and $\hat B\hat C=\hat BC''$;
that is, $(AB)$ is the perpendicular bisector of $[C''\hat C]$.
Therefore, this reflection sends $C''$ to~$\hat C$.

Apply Exercise~\ref{ex:ABC-motion} to show that the composition of the constructed reflections coincides with the given motion.

\parbf{Exercise~\ref{ex:2-reflections}.}
Note that $\measuredangle XBA=\measuredangle ABP$, $\measuredangle PBC=\measuredangle CBY$.
Therefore,
\begin{align*}
\measuredangle XBY
&\equiv
\measuredangle XBP+\measuredangle PBY
\equiv
 2\cdot(\measuredangle ABP+\measuredangle PBC)
\equiv
 2\cdot \measuredangle ABC.
\end{align*}

{

\begin{wrapfigure}{r}{21mm}
\begin{lpic}[t(-0mm),b(0mm),r(0mm),l(0mm)]{pics/ex-obtuce(1)}
\lbl[t]{20,8;$A$}
\lbl[tr]{5.5,8;$B$}
\lbl[br]{2,21;$C$}
\lbl[tl]{8,4;$D$}
\lbl[bl]{8,18;$X$}
\end{lpic}
\end{wrapfigure}

\parbf{Exercise~\ref{ex:obtuce}.}
If $\angle ABC$ is right, the statement follows from Lemma~\ref{lem:perp<oblique}.
Therefore, we can assume that $\angle ABC$ is obtuse.

Draw a line $(BD)$ perpendicular to~$(BA)$.
Since $\angle ABC$ is obtuse, 
the angles $DBA$ and $DBC$ have opposite signs.

By Corollary~\ref{cor:half-plane},
$A$ and $C$ lie on opposite sides from~$(BD)$.
In particular, $[AC]$ intersects $(BD)$ at a point; denote it by~$X$.

Note that $AX<AC$ and by Lemma~\ref{lem:perp<oblique}, $AB\le AX$.

}

\parbf{Exercise~\ref{ex:inside-outside}.}
Let $O$ be the center of the circle.
Note that we can assume that $O\ne P$.

Assume $P$ lies between $X$ and~$Y$.
By Exercise~\ref{ex:acute-obtuce}, we can assume that $\angle OPX$ is right or obtuse.
By Exercise~\ref{ex:obtuce}, $OP<OX$; 
that is, $P$ lies inside~$\Gamma$.

If $P$ does not lie between $X$ and $Y$, we can assume that $X$ lies between $P$ and~$Y$.
Since $OX=OY$, Exercise~\ref{ex:obtuce} implies that $\angle OXY$ is acute.
Therefore, $\angle OXP$ is obtuse.
Applying Exercise~\ref{ex:obtuce} again we get that $OP\z>OX$;
that is, $P$ lies outside~$\Gamma$.

\parbf{Exercise~\ref{ex:chord-perp}.} Apply Theorem~\ref{thm:perp-bisect}.

\parbf{Exercise~\ref{ex:two-circ}.} Use Exercise~\ref{ex:chord-perp} and Theorem~\ref{perp:ex+un}.

\parbf{Exercise~\ref{ex:tangent-circles}.} 
Let $P'$ be the reflection of $P$ in~$(OO')$.
Note that $P'$ lies on both circles and $P'\ne P$ if and only if $P\notin(OO')$.

\parbf{Exercise~\ref{ex:tangent-circles-2}.}  Apply Exercise~\ref{ex:tangent-circles}.

\parbf{Exercise~\ref{ex:tangent-circles-3}.}
Let $A$ and $B$ be the points of intersection.
Note that the centers lie on the perpendicular bisector of the segment~$[AB]$.

\begin{minipage}{.48\textwidth}
\centering
\begin{lpic}[t(7mm),b(0mm),r(0mm),l(0mm)]{pics/perp-constr(1)}
\lbl[b]{20.5,50;{\bf Exercise~\ref{ex:construction-perpendicular}.}}
\end{lpic}
\end{minipage}
\hfill
\begin{minipage}{.48\textwidth}
 \begin{lpic}[t(7mm),b(0mm),r(0mm),l(5mm)]{pics/center(1)}
\lbl[b]{20.5,50;{\bf Exercise~\ref{ex:center}.}}
\end{lpic}
\end{minipage}
\medskip
\begin{minipage}{.48\textwidth}
\centering
\begin{lpic}[t(7mm),b(0mm),r(0mm),l(0mm)]{pics/tangent-constr(1)}
\lbl[b]{20.5,50;{\bf Exercise~\ref{ex:tangent}.}}
\end{lpic}
\end{minipage}
\hfill
\begin{minipage}{.48\textwidth}
\centering
\begin{lpic}[t(7mm),b(0mm),r(0mm),l(5mm)]{pics/tangent-circ-constr(1)}
\lbl[b]{20.5,50;{\bf Exercise~\ref{ex:tangent-circle}.}}
\end{lpic}
\end{minipage}
\medskip
%\subsection*{Chapter~\ref{chap:parallel}}
\refstepcounter{chapter}
\setcounter{eqtn}{0}

\parbf{Exercise~\ref{ex:angle-preserving-euclid}.}
By the AA similarity condition, the transformation multiplies the sides of any nondegenerate triangle by some number which may depend on the triangle. 

Note that for any two nondegenerate triangles which share one side this number is the same.
Applying this observation to a chain of triangles leads to a solution.

\parbf{Exercise~\ref{ex:pyth}.}
Apply that $\triangle ADC\sim \triangle CDB$.

\parbf{Exercise~\ref{ex:pyth-conv}.}
Apply the Pythagorean theorem (\ref{thm:pyth}) and the SSS congruence condition.

%(???

\parbf{Exercise~\ref{ex:two-pairs-sim}.}
By the AA similarity condition (\ref{prop:sim}), $\triangle AYC\z\sim \triangle BXC$.

Conclude that 
\[\frac{YC}{AC}=\frac{XC}{BC}.\]
Apply the SAS similarity condition to show that $\triangle ABC\z\sim \triangle YXC$.

Use AA and equality of vertical angles to prove that $\triangle AZX\sim \triangle BZY$.

%???)

%\subsection*{Chapter~\ref{chap:angle-sum}}
\refstepcounter{chapter}
\setcounter{eqtn}{0}


\parbf{Exercise~\ref{ex:perp-perp}.}
Apply Proposition~\ref{prop:perp-perp} to show that $k\parallel m$.
By Corollary~\ref{cor:parallel-1}, $k\parallel n\z\Rightarrow m\parallel n$.
The latter contradicts that $m\perp n$.

\parbf{Exercise~\ref{ex:construction-parallel}.}
Repeat the construction in Exercise~\ref{ex:construction-perpendicular} twice.

\parbf{Exercise~\ref{ex:smililar+parallel}.}
By the transversal property~\ref{thm:parallel-2},
\[\measuredangle B'BC \equiv \pi -\measuredangle C'B'B.\]

Since $B'$ lies between $A$ and $B$, we get that 
$\measuredangle ABC=\measuredangle B'BC$ and $\measuredangle AB'C'+\measuredangle C'B'B\z\equiv \pi$.
Hence $\measuredangle ABC= \measuredangle AB'C'$.

The same way we can prove that 
$\measuredangle BCA= \measuredangle B'C'A$.
It remains to apply the AA similarity condition.

\parbf{Exercise~\ref{ex:trisection}.}
Assume we need to trisect segment $[AB]$.
Construct a line $\ell\ne (AB)$ with four points $A,C_1,C_2, C_3$
such that $C_1$ and $C_2$ trisect $[AC_3]$.
Draw the line $(BC_3)$
and draw parallel lines thru $C_1$ and~$C_2$.
The points of intersections of these two lines with $(AB)$ trisect the segment $[AB]$.

\parbf{Exercise~\ref{ex:pent}.}
Apply twice Theorem~\ref{thm:isos} and twice Theorem~\ref{thm:3sum}.


\parbf{Exercise~\ref{ex:|3sum|}.}
If $\triangle ABC$ is degenerate, then one of the angle measures is $\pi$ and the other two are~$0$.
Hence the result.

Assume $\triangle ABC$ is nondegenerate.
Set $\alpha=\measuredangle CAB$, $\beta=\measuredangle ABC$ and $\gamma=\measuredangle BCA$.

By Theorem~\ref{thm:signs-of-triug},
we may assume that $0<\alpha,\beta,\gamma<\pi$.
Therefore, 
$$0<\alpha+\beta+\gamma<3\cdot\pi.\eqlbl{eq:|3|<3pi}$$

By Theorem~\ref{thm:3sum},
$$\alpha+\beta+\gamma\equiv\pi.\eqlbl{eq:|3|==pi}$$

From \ref{eq:|3|<3pi} and \ref{eq:|3|==pi} the result follows.

\parbf{Exercise~\ref{ex:right-isos}.}
Apply twice Theorem~\ref{thm:isos} and once Theorem~\ref{thm:3sum}. 

\begin{wrapfigure}{o}{25mm}
\begin{lpic}[t(-0mm),b(0mm),r(-0mm),l(1mm)]{pics/pi4-isos(1)}
\lbl[tr]{1,0;$A$}
\lbl[b]{12,29.5;$B$}
\lbl[tl]{21,0;$C$}
\lbl[r]{4,12;$O$}
\lbl[lb]{16,17;$X$}
\end{lpic}
\end{wrapfigure}

\parbf{Exercise~\ref{ex:pi/4-isos}.} 
Let $O$ denotes the center of the circle.

Note that $\triangle AOX$ is isosceles
and $\angle OXC$ is right.
Applying \ref{thm:3sum} and \ref{thm:isos} and simplifying, you should get
\[
4\cdot \measuredangle CAX
\equiv
\pi.
\]

Show that $\angle CAX$ has to be acute.
It follows then that 
$\measuredangle CAX\z=\pm\tfrac\pi4$.

\parbf{Exercise~\ref{ex:quadrilateral}.}
Apply Theorem~\ref{thm:3sum} to $\triangle ABC$ and $\triangle BDA$.


\parbf{Exercise~\ref{ex:romb}.}
Since $\triangle ABC$ is isosceles, $\measuredangle CAB=\measuredangle BCA$.
 
By SSS, $\triangle ABC\cong \triangle CDA$.
Therefore, 
$$\pm\measuredangle DCA= \measuredangle BCA=\measuredangle CAB.$$

Since $D\ne C$, we get ``$-$'' in the last formula.
Use the transversal property (\ref{thm:parallel-2}) to show that $(AB)\parallel (CD)$. Repeat the argument to show that $(AD)\z\parallel(BC)$ 

\parbf{Exercise~\ref{ex:diad-par}.}
Apply Lemma~\ref{lem:parallelogram} together with
the transversal property (\ref{thm:parallel-2}) to the diagonals and a pair of opposite sides.
After that use the ASA-congruence condition (\ref{thm:ASA}).

\parbf{Exercise~\ref{ex:rectangle}.} 
By Lemma~\ref{lem:parallelogram} and SSS, 
\[AC=BD
\quad
\iff
\quad
\measuredangle ABC=\pm \measuredangle BCD.\]
By the transversal property~(\ref{thm:parallel-2}), 
\[\measuredangle ABC+\measuredangle BCD\equiv \pi.\]

Therefore, 
\[AC=BD
\quad
\iff
\quad 
\measuredangle ABC
=\measuredangle BCD
=\pm\tfrac\pi2.\]

\parbf{Exercise~\ref{ex:romb2}.} 
Fix a parallelogram $ABCD$.
By Exercise~\ref{ex:diad-par},
its diagonals $[AC]$ and $[BD]$ have a common midpoint; denote it by~$M$.

Use SSS and Lemma \ref{lem:parallelogram} to show that
\[AB=CD
\quad
\iff
\quad
\triangle AMB
\cong
\triangle AMD
\quad
\iff
\quad
\measuredangle AMB
=
\pm\tfrac\pi2.\]

\parbf{Exercise~\ref{ex:coordinates}.} \textit{(a).} Use the uniqueness of the parallel line (Theorem~\ref{thm:parallel}).

\parit{(b)} Use Lemma~\ref{lem:parallelogram} and the Pythagorean theorem (\ref{thm:pyth}).

\parbf{Exercise~\ref{ex:abc}.}
Set $A=(0,0)$, $B=(c,0)$ and $C=(x,y)$.
Clearly, $AB=c$,
$AC^2=x^2+y^2$ and $BC^2=(c-x)^2+y^2$.

It remains to show that there is a pair of real numbers $(x,y)$ 
which satisfy the following system of equations 
$$
\left\{
\begin{aligned}
b^2&=x^2+y^2
\\
a^2&=(c-x)^2+y^2
\end{aligned}
\right.
$$
if $0<a\le b\le c\le a+c$.



%???(

\parbf{Exercise~\ref{ex:circle-coord}.} Rewrite the equation as 
\[(x+\tfrac a2)^2+(y+\tfrac b2)^2=(\tfrac a2)^2+(\tfrac b2)^2-c\]
and think.

\parbf{Exercise~\ref{ex:apolonnius}.}
We can choose the coordinates so that $B=(0,0)$ and $A=(a,0)$ for some $a>0$.
If $X=(x,y)$, then the equation $AX=2\cdot BX$ can be written in coordinates as 
\[4\cdot(x^2+y^2)=(x-a)^2+y^2.\]
It remains to rewrite this equation as in Exercise~\ref{ex:circle-coord}.

%???)

%\subsection*{Chapter~\ref{chap:triangle}}
\refstepcounter{chapter}
\setcounter{eqtn}{0}

\parbf{Exercise~\ref{ex:unique-cline}.}
Apply Theorem~\ref{thm:circumcenter} and Theorem~\ref{thm:perp-bisect}.

\parbf{Exercise~\ref{ex:orthic-4}.}
Note that $(AC)\perp (BH)$ and $(BC)\perp (AH)$ and apply Theorem~\ref{thm:orthocenter}.

\parbf{Exercise~\ref{ex:midle}.}
Use the idea from the proof of Theorem~\ref{thm:centroid}
to show that $(XY)\parallel (AC)\z\parallel (VW)$ and
$(XV)\parallel (BD)\z\parallel (YW)$.

\parbf{Exercise~\ref{ex:perp-bisectors}.}
Let $(BX)$ and $(BY)$ be the internal and external bisectors of $\angle ABC$.
Then 
\begin{align*}
2\cdot \measuredangle XBY&\equiv2\cdot \measuredangle XBA+2\cdot \measuredangle ABY\equiv
\\
&\equiv\measuredangle CBA+\pi+2\cdot \measuredangle ABC\equiv
\\
&\equiv \pi+\measuredangle CBC=\pi
\end{align*}
and hence the result.

\parbf{Exercise~\ref{ex:ext-disect}.} 
If $E$ is the point of intersection of $(BC)$ 
with the external bisector of $\angle BAC$, then 
$$\frac{AB}{AC}=\frac{EB}{EC}.$$
It can be proved along the same lines as Lemma~\ref{lem:bisect-ratio}.

\parbf{Exercise~\ref{ex:bisect=median}.}
Apply Lemma~\ref{lem:bisect-ratio}.
Also see the solution of Exercise~\ref{ex:abs-bisect=median}.

\parbf{Exercise~\ref{ex:bisect=altitude}.}
Apply ASA for the two triangles which the bisector cuts from the original triangle. 

\parbf{Exercise~\ref{ex:2x=b+c-a}.}
Let $I$ be the incenter.
By SAS, we get that $\triangle AIZ\z\cong\triangle AIY$.
Therefore, $AY=AZ$.
The same way we get that $BX=BZ$ and $CX=CY$.
Hence the result.

\parbf{Exercise~\ref{ex:orthic-triangle}.}
Let $\triangle ABC$ be the given acute triangle and $\triangle A'B'C'$ 
be its orthic triangle.
Note that $\triangle AA'C\sim\triangle BB'C$.
Use it to show that $\triangle A'B'C\sim \triangle ABC$.

The same way we get that $\triangle AB'C'\sim \triangle ABC$.
It follows that $\measuredangle A'B'C\z=\measuredangle AB'C'$.
Conclude that $(BB')$ bisects $\angle A'B'C'$.

If $\triangle ABC$ is obtuse, then its orthocenter coincides with one of the \index{excenter}\emph{excenters} of $\triangle ABC$;
that is, 
the point of intersection of two external and one internal bisectors of $\triangle ABC$.
 
\parbf{Exercise~\ref{ex:bisector-parallel}.} Apply \ref{thm:isos}, \ref{thm:parallel-2} 
and \ref{lem:parallelogram}.

%\subsection*{Chapter~\ref{chap:inscribed-angle}}
\refstepcounter{chapter}
\setcounter{eqtn}{0}

\parbf{Exercise~\ref{ex:inscribed-angle}.} \textit{(a).}
Apply Theorem~\ref{thm:inscribed-angle} for $\angle XX'Y$ and $\angle X'YY'$
and Theorem~\ref{thm:3sum} for $\triangle PYX'$.

\parit{(b)} If $P$ is inside of $\Gamma$ then $P$ lies between $X$ and $X'$ and between $Y$ and $Y'$ in this case $\angle XPY$ is vertical to $\angle X'PY'$.
If $P$ is outside of $\Gamma$ then $[PX)\z=[PX')$ and $[PY)=[PY')$.
In both cases we have $\measuredangle XPY=\measuredangle X'PY'$.

Applying Theorem~\ref{thm:inscribed-angle} and Exercise~\ref{ex:ABCO-line}, we get that
\[2\cdot \measuredangle Y'X'P
\equiv
2\cdot \measuredangle Y'X'X 
\equiv
2\cdot\measuredangle Y'YX
\equiv
2\cdot\measuredangle PYX.\]
According to Theorem~\ref{thm:signs-of-triug}, $\angle Y'X'P$ and $\angle PYX$ have the same sign;
therefore
$$\measuredangle Y'X'P= \measuredangle PYX.$$
It remains to apply the AA similarity condition.

\parit{(c)} Apply \textit{(b)} assuming $[YY']$ is the diameter of~$\Gamma$. 

\parbf{Exercise~\ref{ex:inscribed-hex}.} Apply Exercise~\ref{ex:inscribed-angle}\textit{\ref{ex:inscribed-angle:b}}
three times.

\parbf{Exercise~\ref{ex:altitudes-circumcircle}.}
Let $X$ any $Y$ be the foot points of the altitudes from $A$ and~$B$.
Let $O$ denotes the circumcenter.
 
By AA condition, $\triangle A X C\sim \triangle B Y C$.
Thus 
\begin{align*}
\measuredangle A'OC
&\equiv 
2\cdot \measuredangle A' A C
\equiv-2\cdot\measuredangle B' B C
\equiv-\measuredangle B'OC.
\end{align*}

By SAS, $\triangle A'OC\cong\triangle B'OC$.
Therefore, $A'C=B'C$.

\parbf{Exercise~\ref{ex:two-right}.}
Construct the circles $\Gamma$ and $\Gamma'$
on the diameters $[AB]$ and $[A'B']$ correspondingly.
By Corollary~\ref{cor:right-angle-diameter},
any point $Z$ in the intersection $\Gamma\cap \Gamma'$ will do.

{

\begin{wrapfigure}[11]{r}{34mm}
\begin{lpic}[t(-5mm),b(-0mm),r(-1mm),l(0mm)]{pics/perp=constr(1)}
\end{lpic}
\end{wrapfigure}

\parbf{Exercise~\ref{ex:VVAA}.} 
Note that $\measuredangle AA'B=\pm\tfrac\pi2$ and $\measuredangle AB'B=\pm\tfrac\pi2$.
Then apply Theorem~\ref{thm:inscribed-quadrilateral}
to $\square AA'BB'$.

If $O$ is the center of the circle, then 
$$\measuredangle AOB\equiv 2\cdot \measuredangle AA'B\equiv\pi.$$
That is, $O$ is the midpoint of~$[AB]$.

\parbf{Exercise~\ref{ex:perpendicular-ruler}.}
Guess the construction from the diagram.
To prove it,
apply Theorem~\ref{thm:orthocenter} and Corollary~\ref{cor:right-angle-diameter}.

\parbf{Exercise~\ref{ex:two-chords}.} Apply the transversal property (\ref{thm:parallel-2}) and the theorem on inscribed angles (\ref{thm:inscribed-angle}).

\parbf{Exercise~\ref{ex:secant-circles}.}
Apply Theorem~\ref{thm:inscribed-quadrilateral} twice for $\square ABYX$ and $\square ABY'X'$ and use the transversal property (\ref{thm:parallel-2}).

}

\parbf{Exercise~\ref{ex:inaccuracy}.}
One needs to show that the lines $(A'B')$ and $(XP)$ are not parallel, otherwise the first line in the proof does not make sense.

In addition, the following identities:
\begin{align*}
2\cdot \measuredangle AXP&\equiv2\cdot \measuredangle AXY,
&
2\cdot \measuredangle ABP&\equiv2\cdot \measuredangle ABB',
&
2\cdot \measuredangle AA'B'&\equiv2\cdot \measuredangle AA'Y.
\end{align*}

\parbf{Exercise~\ref{ex:equilateral-2}.}
By Corollary~\ref{cor:right-angle-diameter},
the points $L$, $M$ and $N$ lie on the circle $\Gamma$ with diameter~$[OX]$.
It remains to apply Theorem~\ref{thm:inscribed-angle} for the circle $\Gamma$ 
and two inscribed angles with vertex at~$O$.

\begin{wrapfigure}[7]{o}{30mm}
\begin{lpic}[t(-6mm),b(-0mm),r(-1mm),l(0mm)]{pics/simson(1)}
\lbl[br]{5,22;$A$}
\lbl[b]{24,24.5;$B$}
\lbl[br]{3,12;$C$}
\lbl[br]{8.5,15;$X$}
\lbl[r]{1,4;$Y$}
\lbl[br]{14.5,23;$Z$}
\lbl[t]{17,0;$P$}
\end{lpic}
\end{wrapfigure}

\parbf{Advanced exercise~\ref{ex:simson}.}
Let $X$, $Y$ and $Z$ denote the foot points of $P$ on $(BC)$, $(CA)$ and $(AB)$ correspondingly.

Notice that $\square AZPY$, $\square BXPZ$, $\square CYPX$ and $\square ABCP$ are inscribed.
Therefore
\begin{align*}
2\cdot \measuredangle CXY&\equiv 2\cdot \measuredangle CPY,
&
2\cdot \measuredangle BXZ&\equiv 2\cdot \measuredangle BPZ,
\\
2\cdot \measuredangle YAZ&\equiv 2\cdot \measuredangle YPZ,
&
2\cdot \measuredangle CAB&\equiv 2\cdot \measuredangle CPB.
\end{align*}
Conclude that 
$2\cdot \measuredangle CXY\equiv 2\cdot \measuredangle BXZ$
and hence $X$, $Y$ and $Z$ lie on one line.

\parbf{Exercise~\ref{ex:arc-tan-straight}.}
By Theorem~\ref{thm:3sum},
$$\measuredangle ABC+\measuredangle BCA+\measuredangle CAB\equiv \pi.$$
It remains to apply
Proposition~\ref{prop:arc(angle=tan)} twice.

\parbf{Exercise~\ref{ex:tangent-arc}.}
If $C\in (AX)$, then the arc is the line segment $[AC]$ or the union of two half-lines in $(AX)$ with vertices at $A$ and~$C$.

Assume $C\notin (AX)$.
Let $\ell$ be the perpendicular line dropped from $A$ to $(AX)$ and $m$ be the perpendicular bisector of~$[AC]$.

Note that $\ell\nparallel m$;
set $O=\ell\cap m$.
Note that the circle with center $O$ passing thru $A$ is also passing thru $C$ and tangent to~$(AX)$.

{

\begin{wrapfigure}{r}{35mm}
\begin{lpic}[t(-4mm),b(0mm),r(0mm),l(0mm)]{pics/3x120(1)}
\end{lpic}
\end{wrapfigure}

Note that one the two arcs with endpoints $A$ and $C$ is tangent to~$[AX)$.

The uniqueness follow from the propositions \ref{prop:arc(angle=angle)}
and \ref{prop:arc(angle=tan)}.

\parbf{Exercise~\ref{ex:tangent-lim}.} Use \ref{prop:arc(angle=tan)} and \ref{thm:3sum} to show that 
\[\measuredangle XAY=\measuredangle ACY.\]
By Axiom \ref{def:birkhoff-axioms:2c}, $\measuredangle ACY\to 0$ as $AY\to 0$;
hence the result.

\parbf{Exercise~\ref{ex:two-arcs}.} 
Apply Proposition~\ref{prop:arc(angle=tan)} twice.

\parbf{Exercise~\ref{ex:3x120}.} Guess the construction from the diagram.
To show that it produces the needed point, apply Theorem \ref{thm:inscribed-angle}.

}

%\subsection*{Chapter~\ref{chap:inversion}}
\refstepcounter{chapter}
\setcounter{eqtn}{0}

\parbf{Exercise~\ref{ex:constr-inversion}.}
By Lemma~\ref{lem:tangent}, $\angle OTP'$ is right. 
Therefore, $\triangle OPT\z\sim \triangle OTP'$
and in particular
$$OP\cdot OP'=OT^2$$
and hence the result.

\parbf{Exercise~\ref{ex:appolo-circ}.}
Let $O$ denotes the center of $\Gamma$.
Suppose $X,Y\in \Gamma$;
in particular, $OX=OY$.

Note that the inversion sends $X$ and $Y$ to themselves.
By Lemma~\ref{lem:inversion-sim},
$$\triangle OPX\z\sim \triangle OXP'
\quad
\text{and}
\quad
\triangle OPY\sim \triangle OYP'.$$
Therefore, 
\[\frac{PX}{P'X}=\frac{OP}{OX}=\frac{OP}{OY}=\frac{PY}{P'Y}\]
and hence the result.

\parbf{Exercise~\ref{ex:incenter+inversion=orthocenter}.}
By Lemma~\ref{lem:inversion-sim},
\begin{align*}
\measuredangle IA'B'&\equiv -\measuredangle IBA,
&
\measuredangle IB'C'&\equiv -\measuredangle ICB,
&
\measuredangle IC'A'&\equiv -\measuredangle IAC,
\\
\measuredangle IB'A'&\equiv -\measuredangle IAB,
&
\measuredangle IC'B'&\equiv -\measuredangle IBC,
&
\measuredangle IA'C'&\equiv -\measuredangle ICA.
\end{align*}

It remains to apply the theorem on the sum of angles of triangle (\ref{thm:3sum})
to show that $(A'I)\z\perp (B'C')$, 
$(B'I)\z\perp (C'A')$
and
$(C'I)\z\perp (B'A')$.

\begin{wrapfigure}[10]{r}{54mm}
\begin{lpic}[t(-0mm),b(0mm),r(0mm),l(0mm)]{pics/inv-constr(1)}
\end{lpic}
\end{wrapfigure}

\parbf{Exercise~\ref{ex:consturuction-of-inversion}.}
Guess the construction from the diagram (the two nonintersecting lines on the diagram are parallel).

\parbf{Exercise~\ref{ex:inv-center not=center-inv}.}
First show that for any $r>0$ and for any real numbers $x,y$ distinct from $0$,
$$\frac{r^2}{(x+y)/2}
=
\left(\frac {r^2}x+\frac {r^2}y\right)/2$$
if and only if $x=y$.

Let $\ell$ denotes the line passing thru $Q$, $Q'$ and the center of the inversion $O$.
Choose an isometry $\ell\to\mathbb{R}$ which sends $O$ to $0$;
assume $x,y\in \mathbb{R}$ are the values of $\ell$ for the two points of intersection $\ell\cap\Gamma$;
note that $x\ne y$.
Assume $r$ is the radius of the circle of inversion.
Then the left hand side above is the coordinate of $Q'$ 
and the right hand side is the coordinate of the center of $\Gamma'$.


\begin{wrapfigure}[11]{r}{65mm}
\begin{lpic}[t(-5mm),b(0mm),r(0mm),l(0mm)]{pics/4-circles-sol(1)}
\lbl[r]{0,16;$P$}
\lbl[t]{24,17;$Q$}
\lbl[t]{49,19;$Q'$}
\lbl[r]{17.5,20;$A$}
\lbl[lb]{60,31.5;$A'$}
\lbl[lt]{29.7,12;$B$}
\lbl[rt]{38.5,10;$B'$}
\lbl[bl]{27,27;$X$}
\lbl[br]{38.5,31.5;$X'$}
\lbl[r]{18.5,14;$Y$}
\lbl[tl]{60,10;$Y'$}
\end{lpic}
\end{wrapfigure}

\parbf{Exercise~\ref{ex:circumtool}.}
A solution is given on page \pageref{page:solution-for-ex:circumtool}.

\parbf{Exercise~\ref{ex:tangent-circ->parallels}.}
Apply an inversion in a circle with the center at the only point of intersection of the circles;
then use Theorem~\ref{thm:inverse}.

\parbf{Exercise~\ref{ex:4-circles}.}
Label the points of tangency by $X$, $Y$, $A$, $B$, $P$ and $Q$ as on the diagram above.
Apply an inversion with the center at $P$. 
Observe that the two circles which tangent at $P$ become parallel lines and 
the remaining two circles are tangent to each other and these two parallel lines.

Note that the points of tangency $A'$, $B'$, $X'$ and $Y'$ with the parallel lines are vertexes of a square;
in particular they lie on one circle.
These points are images of $A$, $B$, $X$ and $Y$ under the inversion.
By Theorem~\ref{thm:inverse-cline}, the points $A$, $B$, $X$ and $Y$ also lie on one circline.

\parbf{Advanced exercise~\ref{ex:inverse}.} 
Apply the inversion in a circle with center~$A$. 
The point $A$ will go to infinity, the two circles tangent at $A$ will become parallel lines
and the two parallel lines will become circles tangent at~$A$; see the diagram.

\begin{wrapfigure}[7]{i}{48mm}
\begin{lpic}[t(-2mm),b(-1mm),r(0mm),l(0mm)]{pics/ex-inverse(1)}
\lbl[bl]{28,9;$B'$}
\lbl[br]{19,9;$A$}
\end{lpic}
\end{wrapfigure}

It remains to show that the dashed line $AB'$ is parallel to the other two lines.

\parbf{Exercise~\ref{ex:cline-perp-to-two}.}
Let $P_1$ and $P_2$ be the inverses of $P$ 
in $\Omega_1$ and~$\Omega_2$.
Note that the points $P$, $P_1$ and $P_2$ 
are mutually distinct.

Use Theorem~\ref{thm:circumcenter}, to show that there is unique circline $\Gamma$ which passes
thru $P$, $P_1$ and~$P_2$.
Use Corollary~\ref{cor:perp-inverse} to show that
$\Gamma\perp\Omega_1$ and $\Gamma\perp\Omega_2$.
Use Theorem~\ref{thm:perp-inverse} to prove uniqueness.

\parbf{Exercise~\ref{ex:inscribed+inv}.}
Apply Theorem~\ref{lem:inverse-4-angle}\textit{\ref{lem:inverse-4-angle:angle}}, 
Exercise~\ref{ex:quadrilateral}
and Theorem~\ref{thm:inscribed-angle}.

\parbf{Exercise~\ref{ex:centers-of-perp-circles}.}
Let $T$ denotes a point of intersection of $\Omega_1$ and~$\Omega_2$.
Let $P$ be the foot point of $T$ on~$(O_1O_2)$.
Show that
$$\triangle O_1PT
\sim \triangle O_1TO_2
\sim \triangle TPO_2.$$
Conclude that $P$ coincides with the inverses of $O_1$ in $\Omega_2$ and of $O_2$ in~$\Omega_1$.

\parbf{Exercise~\ref{ex:4-th-perp-circ}.}
Since $\Gamma\perp\Omega_1$ and $\Gamma\perp\Omega_2$,
Corollary~\ref{cor:perp-inverse-clines} 
implies that
the circles $\Omega_1$ and $\Omega_2$ are inverted in $\Gamma$ 
to themselves.

Therefore, the points $A$ and $B$ are inverses of each other.

Since $\Omega_3\ni A,B$,
Corollary~\ref{cor:perp-inverse} implies that
$\Omega_3\perp \Gamma$.

\parbf{Exercise~\ref{ex:construction-perp-clines}.}
Follow the solution of Exercise~\ref{ex:cline-perp-to-two}.

%\subsection*{Chapter~\ref{chap:non-euclid}}
\refstepcounter{chapter}
\setcounter{eqtn}{0}

\parbf{Exercise~\ref{ex:abs-bisect=median}}.
Let $D$ denotes the midpoint of~$[BC]$.
Assume $(AD)$ is the angle bisector at~$A$.

Let $A'\in [AD)$ be the point distinct from $A$ such that $AD=A'D$.
Note that $\triangle CAD\cong\triangle BA'D$.
In particular, $\measuredangle BAA'=\measuredangle AA'B$.
It remains to apply Theorem~\ref{thm:isos} for $\triangle ABA'$.

\parbf{Exercise~\ref{ex:abs-inscibed}.}
The statement is evident if $A$, $B$, $C$ and $D$ lie on one line.

In the remaining case, let $O$ denotes the circumcenter.
Apply theorem about isosceles triangle (\ref{thm:isos}) to the triangles 
$AOB$,
$BOC$, 
$COD$, 
$DOA$. 

\textit{(Note that in the Euclidean plane the statement follows from Theorem~\ref{thm:inscribed-quadrilateral} and Exercise~\ref{ex:quadrilateral},
but one cannot use these statements in the neutral plane.)}

\parbf{Exercise~\ref{ex:parallel-abs}.}
Arguing by contradiction, 
assume 
$$2\cdot(\measuredangle ABC+\measuredangle BCD)\equiv0,$$ 
but $(AB)\z\nparallel(C D)$.
Let $Z$ be the point of intersection of $(AB)$ and~$(CD)$.

Note that 
\begin{align*}
2\cdot \measuredangle ABC&\equiv 2\cdot \measuredangle ZBC,
&
2\cdot \measuredangle BCD&\equiv 2\cdot \measuredangle BCZ.
\end{align*}

Apply Proposition~\ref{prop:2sum} to $\triangle ZBC$ and try to arrive to a contradiction.

\begin{wrapfigure}{r}{26mm}
\begin{lpic}[t(2mm),b(0mm),r(0mm),l(0mm)]{pics/SAA(1)}
\lbl[t]{1,0;$B'$}
\lbl[b]{10,29;$A'$}
\lbl[t]{16,0;$C'$}
\lbl[lt]{21,0;$C''$}
\end{lpic}
\end{wrapfigure}

\parbf{Exercise~\ref{ex:SAA}.}
Let $C''\in [B'C')$ be the point such that $B'C''=BC$.

Note that by SAS, $\triangle ABC\cong \triangle A'B'C''$.
Conclude that $\measuredangle B'C'A'\z= \measuredangle B'C''A'$.

Therefore, it is sufficient to show that $C''=C'$.
If $C'\ne C''$ apply Proposition~\ref{prop:2sum} to $\triangle A'C'C''$ and try to arrive to a contradiction.

%(This proof was given in the Euclid's Elements \cite[Book I, Proposition 26]{euclid}.)

\parbf{Exercise~\ref{ex:chev<side}.} 
Use Exercise~\ref{ex:side-angle} and Proposition~\ref{prop:2sum}.

Alternatively, use the same argument as in the solution of Exercise~\ref{ex:inside-outside}.

\parbf{Exercise~\ref{ex:defect}.}
Note that 
$|\measuredangle ADC|+|\measuredangle CDB|=\pi$.
Then apply the definition of the defect.

%\subsection*{Chapter~\ref{chap:poincare}}
\refstepcounter{chapter}
\setcounter{eqtn}{0}

\parbf{Exercise~\ref{ex:ideal-line-unique}.} 
Let $A$ and $B$ be the ideal points of the h-line~$\ell$. 
Note that the center of the Euclidean circle containing $\ell$ lies 
at the intersection of the lines tangent to the absolute at the ideal points of~$\ell$.

\parbf{Exercise~\ref{ex:1ideal-line-unique}.}
Assume $A$ is an ideal point of the h-line $\ell$
and $P\in \ell$.
Let $P'$ denotes the inverse of $P$ in the absolute.
By Corollary~\ref{cor:perp-inverse-clines},
$\ell$ lies in the intersection of the h-plane and the (necessarily unique) circline 
passing thru $P$, $A$ and~$P'$.

\parbf{Exercise~\ref{ex:line/h-line}.} 
Let $\Omega$ and $O$ denote the absolute and its center. 

Let $\Gamma$ be the circline containing~$[PQ]_h$.
Note that $[PQ]_h=[PQ]$ if and only if $\Gamma$ is a line.

Let $P'$ denotes the inverse of $P$ in~$\Omega$.
Note that $O$, $P$ and $P'$ lie on one line.

By the definition of h-line, $\Omega\perp \Gamma$.
By Corollary~\ref{cor:perp-inverse-clines}, $\Gamma$ passes thru $P$ and~$P'$. 
Therefore, $\Gamma$
is a line if and only if it pass thru~$O$.

\parbf{Exercise~\ref{ex:h-dist-eq}.}
Assume that the absolute is a unit circle.

Set $a\z=OX\z=OY$.
Note that $0<a<\tfrac12$ and
\begin{align*}
OX_h&=\ln \tfrac{1+a}{1-a},
&
XY_h&=\ln \tfrac{(1+2\cdot a)\cdot(1-a)}{(1-2\cdot a)\cdot(1+a)}.
\end{align*}
It remains to check that the inequalities 
\[1<
\tfrac{1+a}{1-a}
<
\tfrac{(1+2\cdot a)\cdot(1-a)}{(1-2\cdot a)\cdot(1+a)}\]
hold if $0<a<\tfrac12$.

\parbf{Exercise~\ref{ex:h-perp-unique}.} 
Spell the meaning of terms ``perpendicular'' and ``h-line'' and then apply Exercise~\ref{ex:cline-perp-to-two}.

\parbf{Exercise~\ref{ex:h-reflection}.}
Apply the main observation (\ref{thm:main-observ}\textit{\ref{h-reflect}}).

\parbf{Exercise~\ref{ex:h-circle=circle}.}
Let $X$ and $Y$ denote the points of intersections of $(OP)$ and~$\Delta_\rho'$.
Consider an isometry $(OP)\to\mathbb{R}$ such that $O$ corresponds to $0$.
Let $x$, $y$, $p$ and $\hat p$ denote the real numbers corresponding to $X$, $Y$, $P$ and~$\hat P$.

We can assume that $p>0$ and $x\z<y$.
Note that $\hat p=\tfrac{x+y}2$ and
\[\frac{(1+x)\cdot(1-p)}{(1-x)\cdot(1+p)}=\frac{(1+p)\cdot(1-y)}{(1-p)\cdot(1+y)}.\]
It remains to show that all this implies $0<\hat p <p$.

\begin{wrapfigure}{r}{44mm}
\begin{lpic}[t(-0mm),b(-6mm),r(0mm),l(0mm)]{pics/absolute-triangle-4(1)}
\end{lpic}
\end{wrapfigure}

\parbf{Exercise~\ref{ex:3-h-lines}.} Look at the diagram and think.

\parbf{Advanced exercise~\ref{ex:cosh}.}
By Corollary \ref{cor:invese-comp} and Theorem~\ref{lem:inverse-4-angle},
the right hand sides in the identities 
survive under an inversion in a circle perpendicular to the absolute.

As usual we assume that the absolute is a unit circle.
Let $O$ denotes the h-midpoint of $[PQ]_h$.
By the main observation (\ref{thm:main-observ})
we can assume that $O$ is the center of the absolute.
In this case $O$ is also the Euclidean midpoint of $[PQ]$.

Set $a=OP=OQ$; in this case we have
\begin{align*}PQ&=2\cdot a,
&
PP'=QQ'&=\tfrac1a-a,
\\
P'Q'&=2\cdot \tfrac1a,
&
PQ'=QP'&=\tfrac1a+a.
\end{align*}
and 
\[PQ_h=\ln \tfrac{(1+a)^2}{(1-a)^2}=2\cdot \ln \tfrac{1+a}{1-a}.\]

Therefore
\begin{align*}
\cosh[\tfrac12\cdot PQ_h]
&=\tfrac12\cdot(\tfrac{1+a}{1-a}+\tfrac{1-a}{1+a})=
&&&
\sqrt{\frac{PQ'\cdot P'Q}{PP'\cdot QQ'}}
&=\frac{\frac1a+a}{\frac1a-a}=
\\
&=\tfrac{1+a^2}{1-a^2};
&&&&=\tfrac{1+a^2}{1-a^2}.
\intertext{
Hence the part \textit{(\ref{ex:cosh/2})} follows.
Similarly,
}
\sinh[\tfrac12\cdot PQ_h]
&=\tfrac12\cdot\left(\tfrac{1+a}{1-a}-\tfrac{1-a}{1+a}\right)=
&&&
\sqrt{\frac{PQ\cdot P'Q'}{PP'\cdot QQ'}}
&=\frac{2}{\frac1a-a}=
\\
&=\tfrac{2\cdot a}{1-a^2};
&&&&=\tfrac{2\cdot a}{1-a^2}.
\end{align*} 
Hence the part \textit{(\ref{ex:coshsinh})} follows.

The parts \textit{(\ref{ex:coshtanh})} and \textit{(\ref{ex:coshcosh})} follow from \textit{(\ref{ex:cosh/2})}, \textit{(\ref{ex:coshsinh})}, the definition of hyperbolic tangent and the double-argument identity for hyperbolic cosine, see \ref{double-argument}.

%\subsection*{Chapter~\ref{chap:h-plane}}
\refstepcounter{chapter}
\setcounter{eqtn}{0}

\parbf{Exercise~\ref{ex:small-angle}.}
By triangle inequality, the h-distance from $B$ to $(AC)_h$ is at least 50.
It remains to estimate $|\measuredangle_h ABC|$ using Corollary~\ref{cor:angle-parallelism}.
The inequalities $\cos\phi\z\le 1-\tfrac1{10}\cdot\phi^2$ for $|\phi|<\tfrac\pi2$ and $e^3>10$ should help to finish the proof.

\parbf{Exercise~\ref{ex:side-sup}.}
Note that the angle of parallelism of $B$ to $(CD)_h$ is bigger than $\tfrac\pi4$,
and it converges to $\tfrac\pi4$ as $CD_h\to\infty$.

Applying Proposition~\ref{prop:angle-parallelism},
we get that
$$BC_h<\tfrac12\cdot\ln\frac{1+\frac1{\sqrt{2}}}{1-\frac1{\sqrt{2}}}=\ln\left(1+\sqrt{2}\right).$$

The right hand side is the limit of $BC_h$ if $CD_h\to\infty$.
Therefore, $\ln\left(1+\sqrt{2}\right)$ is the optimal upper bound.

{

\parbf{Exercise~\ref{ex:right-trig-horocycle}.}
As usual, we assume that the absolute is a unit circle.

Let $PQR$ be a hyperbolic triangle
with a right angle at $Q$, such that $PQ_h\z=QR_h$
and the vertices $P$, $Q$ and $R$ 
lie on a horocycle.

Without loss of generality, we may assume that $Q$ is the center of the absolute.
In this case $\measuredangle_hPQR=\measuredangle PQR=\pm\tfrac\pi2$ and $PQ=QR$.

\begin{wrapfigure}{r}{44mm}
\begin{lpic}[t(-7mm),b(-0mm),r(0mm),l(-0mm)]{pics/h-circle-5(1)}
\lbl[tr]{22,20;$Q$}
\lbl[t]{37,20;$P$}
\lbl[br]{21,37;$R$}
\lbl[tr]{1,21;$A$}
\lbl[tl]{43,21;$B$}
\end{lpic}
\end{wrapfigure}

Note that Euclidean circle passing thru $P$, $Q$ and $R$ is tangent to the absolute.
Conclude that $PQ=\tfrac1{\sqrt2}$. 
Apply \ref{lem:O-h-dist} to find $PQ_h$.

\parbf{Exercise~\ref{ex:angle-preserving-hyp}.}
Apply AAA-congruence condition (\ref{thm:AAA}).

\parbf{Exercise~\ref{ex:circum}.}
Apply Proposition~\ref{prop:circum}.
Use that $e>2$ and in particular the function $r\mapsto e^{-r}$ is decreasing.
%\begin{align*}
%\circum_h(r+1)&=\pi\cdot(e^{r+1}-e^{-r-1})=\pi\cdot e\cdot(e^{r}-e^{-r-2})>
%\\
%&>\pi\cdot e\cdot(e^{r}-e^{-r})=e\cdot\circum_h(r)\ge 2\cdot\circum_h(r).
%\end{align*}

\parbf{Exercise~\ref{ex:c+1>a+b}.}
Apply the hyperbolic Pythagorean theorem and the definition of hyperbolic cosine.

}

%\subsection*{Chapter~\ref{chap:trans}}
\refstepcounter{chapter}
\setcounter{eqtn}{0}

\parbf{Exercise~\ref{ex:collinear=affine}.}
Assume that a triple of noncollinear points $P$, $Q$ and $R$ are mapped to one line~$\ell$.
Note that all three lines $(PQ)$, $(QR)$ and $(RP)$ are mapped to~$\ell$.
Therefore, any line which connects two points on these three lines is mapped to~$\ell$.

Note that any point in the plane lies on a line passing thru two distinct points on these three lines.
Therefore, the whole plane is mapped to $\ell$.
The latter contradicts that the map is a bijection.

\parbf{Exercise~\ref{ex:affine-par}.}
Assume the two distinct lines $\ell$ and $m$ 
are mapped to the intersecting lines $\ell'$ and~$m'$.
Let $P'$ denotes their point of intersection.

Let $P$ be the inverse image of~$P'$.
By the definition of affine map, it has to lie on both $\ell$ and $m$;
that is, $\ell$ and $m$ are intersecting.
Hence the result.

\begin{wrapfigure}{r}{20mm}
\begin{lpic}[t(-8mm),b(-3mm),r(0mm),l(-0mm)]{pics/bisection-affine(1)}
\lbl[t]{2,7;$A$}
\lbl[t]{17,10;$B$}
\lbl[br]{8.5,11;$M$}
\end{lpic}
\end{wrapfigure}

\parbf{Exercise~\ref{ex:midpoint-affine}.}
According to the remark before the exercise,
it is sufficient to construct the midpoint of $[AB]$
with a ruler and a parallel tool.

Guess a construction from the diagram.

\parbf{Exercise~\ref{ex:R-hom}.}
Let $O$, $E$, $A$ and $B$ denote the points with the coordinates $(0,0)$, $(1,0)$, $(a,0)$ and $(b,0)$ correspondingly.

To construct a point $W$ with the coordinates $(0,a+b)$, try to construct two parallelograms $ABPQ$ and $BWPQ$.

To construct $Z$ with coordinates $(0,a\cdot b)$
choose a line $(OE')\ne (OE)$
and try to construct the points $A'\in (OE')$
and $Z \in(OE)$
so that 
$\triangle OEE'\z\sim \triangle OAA'$ and $\triangle OE'B\sim \triangle OA'Z$.

\parbf{Exercise~\ref{ex:center-circ-affine}.}
Draw two parallel chords $[XX']$ and~$[YY']$.
Set $Z\z=(XY)\z\cap (X'Y')$ and $Z'= (XY')\cap (X'Y)$.
Note that $(ZZ')$ passes thru the center.

Repeat the same construction for another pair of parallel chords.
The center lies in the intersection of the obtained lines.

\parbf{Exercise~\ref{ex:affine-perp}.}
Assume a construction produces two perpendicular lines.
Apply a shear mapping which changes the angle between the lines.

Note that it transforms the construction to the same construction for other free choices points.
Therefore, this construction does not produce perpendicular lines in general
 (it might be the center only by coincidence).
 
\parbf{Exercise~\ref{ex:inversive-angle}.} Apply \ref{thm:angle-inversion} and \ref{thm:inversions-inversive}.
 
\parbf{Exercise~\ref{ex:reflection/inversive}.}
Fix a line $\ell$.
Choose a circle $\Gamma$ with its center not on~$\ell$.
Let $\Omega$ be the inverse of $\ell$ in $\Gamma$;
note that $\Omega$ is a circle.

Let $\iota_\Gamma$ and $\iota_\Omega$ denote the inversions in $\Gamma$ and~$\Omega$.
Apply \ref{cor:invese-comp} to show that the composition 
$\iota_\Gamma\circ\iota_\Omega\circ\iota_\Gamma$
is the reflection in~$\ell$.

%\subsection*{Chapter~\ref{chap:proj}}
\refstepcounter{chapter}
\setcounter{eqtn}{0}

\parbf{Exercise~\ref{ex:persect}.}
Since $O$, $P$ and $P'$ lie on one line we have that the coordinates of $P'$ are proportional to the coordinates of~$P$.
The $y$ coordinate of $P'$ has to be equal to~$1$.
Therefore, $P'$ has coordinates 
$(\tfrac1y,1,\tfrac zy)$.

\parbf{Exercise~\ref{ex:pappus}.}
Assume that $(AB)$ meets $(A'B')$ at~$O$.
Since $(AB')\parallel (BA')$, we get that $\triangle OAB'\sim\triangle OBA'$
and
\[\frac{OA}{OB}=\frac{OB'}{OA'}.\]

Similarly, since $(AC')\parallel (CA')$, we get that
\[\frac{OA}{OC}=\frac{OC'}{OA'}.\]

Therefore
\[\frac{OB}{OC}=\frac{OC'}{OB'}.\]
Applying the SAS similarity condition, we get that
$\triangle OBC'\sim\triangle OCB'$.
Therefore, $(BC')\parallel (CB')$.

The case $(AB)\parallel(A'B')$ is similar.

\parbf{Exercise~\ref{ex:dual-euclid}.}
Assume the contrary.
Choose two parallel lines $\ell$ and~$m$.
Let $L$ and $M$ be their dual points.
Set $s=(ML)$, then its dual point $S$ has to lie on both $\ell$ and $m$ ---
a contradiction.

\parbf{Exercise~\ref{ex:dula-coordinates}.}
Assume $M=(a,b)$ 
and the line $s$ is given by the equation $p\cdot x+q\cdot y=1$.
Then $M\in s$ is equivalent to $p\cdot a+q\cdot b=1$.

The latter is equivalent to $m\ni S$
where $m$ is the line given by the equation 
$a\cdot x+b\cdot y=1$ and $S=(p,q)$.

To extend this bijection to the whole projective plane, assume that 
(1) the ideal line corresponds to the origin 
and (2) the ideal point given by the pencil of the lines $b\cdot x-a\cdot y=c$ for different values of $c$ corresponds to the line given by the equation $a\cdot x+b\cdot y=0$.

\begin{wrapfigure}{r}{24mm}
\begin{lpic}[t(0mm),b(-5mm),r(0mm),l(-0mm)]{pics/dual-desargues(1)}
\lbl[l]{7.5,28;$P$}
\lbl[r]{8.5,12.6;$P'$}
\lbl[tw]{6,6;$Q$}
\lbl[b]{18,11;$Q'$}
\lbl[tw]{14,7;$R$}
\lbl[br]{3,10;$R'$}
\end{lpic}
\end{wrapfigure}

\parbf{Exercise~\ref{ex:dual-pappus}.}
Assume one set of concurrent lines $a$, $b$, $c$, 
and another set of concurrent lines $a'$, $b'$, $c'$ are given.
Set 
\begin{align*}
P&=b\cap c',
&
Q&=c\cap a',
&
R&=a\cap b',\\
P'&=b'\cap c,
&
Q'&=c'\cap a,
&
R'&=a'\cap b.
\end{align*}
Then the lines $(PP')$, $(QQ')$ and $(RR')$ are concurrent.

\parbf{Exercise~\ref{ex:dual-desargues-construction}.} To solve \textit{(\ref{ex:dual-desargues-construction:a})},
assume $(AA')$ and $(BB')$ are the given lines and $C$ is the given point.
Apply the dual Desargues' theorem (\ref{thm:dual-desargues}) to construct $C'$ so that $(AA')$, $(BB')$ and $(CC')$ are concurrent. 
Since $(AA')\z\parallel (BB')$, 
we get that 
$(AA')\parallel (BB')\parallel (CC')$.

A similar solution can be build on the dual Pappus' theorem, 
see Exercise~\ref{ex:dual-pappus}.

For part \textit{(\ref{ex:dual-desargues-construction:b})}, apply one of the discussed theorems to construct a line parallel to the given one and then apply part \textit{(\ref{ex:dual-desargues-construction:a})}.

\parbf{Exercise~\ref{ex:finite-pp}.}
Let $A$, $B$, $C$ and $D$ 
be the point provided by Axiom~p-\ref{def:proj-axioms:3}.
Given a line $\ell$, we can assume that $A\notin \ell$, 
otherwise permute the labels of the points.
Then by axioms p-\ref{def:proj-axioms:1} and p-\ref{def:proj-axioms:2},
the three lines $(AB)$, $(AC)$ and $(AD)$ intersect $\ell$ at distinct points.
In particular, $\ell$ contains at least three points. 

\parbf{Exercise~\ref{ex:3=3'}.}
Let $A$, $B$, $C$ and $D$ 
be the point provided by Axiom~p-\ref{def:proj-axioms:3}.
Show that the lines $(AB)$, $(BC)$, $(CD)$ and $(DA)$
satisfy Axiom~p-\ref{def:proj-axioms:3}$'$.
The proof of the converse is similar.

\parbf{Exercise~\ref{ex:oder}.} 
Let $\ell$ be a line with $n+1$ points on it.

By Axiom~p-\ref{def:proj-axioms:3}, given any line $m$ there is a point $P$ which does not lie on $\ell$ nor on~$m$.

By axioms p-\ref{def:proj-axioms:1} and p-\ref{def:proj-axioms:2}, there is a bijection between the lines passing thru $P$ and the points on~$\ell$.
In particular, there are exactly $n+1$ lines passing thru~$P$.

The same way there is bijection between the lines passing thru $P$ and the points on~$m$. 
Hence \textit{(\ref{ex:oder:a})} follows.

Fix a point~$X$.
By Axiom~p-\ref{def:proj-axioms:1}, any point $Y$ in the plane lies in a unique line passing thru~$X$.
From part \textit{(\ref{ex:oder:a})}, each such line contains $X$ and yet $n$ point.
Hence \textit{(\ref{ex:oder:b})} follows.

To solve \textit{(\ref{ex:oder:c})}, show that the equation
$n^2+n+1=10$ %???LINE
does not admit an integer solution and then apply part~\textit{(\ref{ex:oder:b})}.

To solve \textit{(\ref{ex:oder:d})}, count the number of lines crossing a given line using the 
part \textit{(\ref{ex:oder:a})} and apply~\textit{(\ref{ex:oder:b})}.

%\subsection*{Chapter~\ref{chap:sphere}}
\refstepcounter{chapter}
\setcounter{eqtn}{0}

\parbf{Exercise~\ref{ex:2(pi/4)=pi/3}.} 
Applying the Pythagorean theorem, we get that
$$
\cos AB_s=\cos AC_s\cdot\cos BC_s=\tfrac12.
$$
Therefore, $AB_s=\tfrac\pi3$.

\begin{wrapfigure}{r}{30mm}
\begin{lpic}[t(-8mm),b(-3mm),r(-2mm),l(0mm)]{pics/tessellation(1)}
\lbl[br]{8,27;$A$}
\lbl[bl]{22,27;$B$}
\lbl[t]{15,23;$C$}
\end{lpic}
\end{wrapfigure}

Alternatively, 
look at the tessellation of the sphere on the picture; 
it is made from 24 copies of $\triangle_s A B C$ and yet 8 equilateral triangles.
From the symmetry of this tessellation, it follows that $[AB]_s$ occupies $\tfrac16$ of the equator.

\parbf{Exercise~\ref{ex:two-stereographics}.}
Note that points on $\Omega$ do not move.
Moreover, the points inside $\Omega$ 
are mapped outside of $\Omega$ and the other way around.

Further, note that this map sends circles to circles;
moreover, the perpendicular circles are mapped to perpendicular circles.
In particular, the circles perpendicular to $\Omega$ are mapped to themselves.

Consider arbitrary point $P\notin\Omega$.
Let $P'$ denotes the inverse of $P$ in~$\Omega$.
Choose two distinct circles which pass thru $P$ and~$P'$.
According to Corollary~\ref{cor:perp-inverse}, 
$\Gamma_1\perp \Omega$ and $\Gamma_2\perp \Omega$.

Therefore, the inversion in $\Omega$ sends $\Gamma_1$ to itself and the same holds for~$\Gamma_2$. 

The image of $P$ has to lie on $\Gamma_1$ and ~$\Gamma_2$.
Since the image of $P$ is distinct from $P$, we get that it has to be ~$P'$.

\parbf{Exercise~\ref{ex:great-circ}.}
Apply Theorem~\ref{thm:inverion-3d}\textit{\ref{thm:inverion-3d:b}}.

\parbf{Exercise~\ref{ex:conform-sphere}.}
Set $z=P'Q'$.
Note that $\tfrac yz\to 1$ as $x\to 0$.

It remains to show that 
$$\lim_{x\to 0} \frac{z}{x}=\frac{2}{1+OP^2}.$$

Recall that the stereographic projection is the inversion in the sphere $\Upsilon$ with the center at the south pole $S$ restricted to the plane $\Pi$.
Show that there is a plane $\Lambda$ passing thru $S$, $P$, $Q$, $P'$ and~$Q'$.
In the plane $\Lambda$, the map $Q\mapsto Q'$ is an inversion in the circle $\Upsilon\cap \Lambda$.

This reduces the problem to Euclidean plane geometry.
The remaining calculations in $\Lambda$ are similar to those in the proof of Lemma~\ref{lem:conformal}.

\parbf{Exercise~\ref{ex:s-medians}.}
\textit{(\ref{ex:s-medians:a})}.
Observe and use that 
$OA'=OB'=OC'$.

\parit{(\ref{ex:s-medians:b}).} Note that the medians of spherical triangle $ABC$ 
map to the medians of Euclidean a triangle $A'B'C'$.
It remains to apply Theorem~\ref{thm:centroid} for $\triangle A'B'C'$.

%\subsection*{Chapter~\ref{chap:klein}}
\refstepcounter{chapter}
\setcounter{eqtn}{0}

\parbf{Exercise~\ref{ex:P-->hat-P}.}
Let $N$, $O$, $S$, $P$, $P'$ and $\hat P$ 
be as on the diagram on page 
\pageref{pic:stereographic_projection-klein}.

Notice that 
$\triangle NOP\sim \triangle NP'S\sim \triangle P'\hat PP$
and
$2\cdot NO\z=NS$.
It remains to do some algebraic manipulations.

{

\begin{wrapfigure}{r}{42mm}
\begin{lpic}[t(-8mm),b(-6mm),r(0mm),l(-0mm)]{pics/h-medians(1)}
\end{lpic}
\end{wrapfigure}

\parbf{Exercise~\ref{ex:hex}.} Consider the bijection $P\z\leftrightarrow \hat P$ of the h-plane with absolute~$\Omega$.
Note that $\hat P\in [A_iB_i]$ if and only if $P\in\Gamma_i$.

\parbf{Exercise~\ref{ex:h-median}.} 
The observation follows since the reflection in the perpendicular bisector of $[PQ]$ is a motion of the Euclidean plane, and a motion of the h-plane as well.

Without loss of generality, we may assume that 
the center of the circumcircle coincides with the center of the absolute.
In this case the h-medians of the triangle coincide with the Euclidean medians.
It remains to apply Theorem~\ref{thm:centroid}.

}

\parbf{Exercise~\ref{ex:klein-perp}.} 
Let $\hat\ell$ and $\hat m$ denote the h-lines in the conformal model which correspond to $\ell$ and $m$.
We need to show that $\hat\ell\perp\hat m$ as arcs in the Euclidean plane.

The point $Z$, where $s$ meets $t$, is the center of the circle $\Gamma$ containing~$\hat\ell$.

If $\bar m$ is passing thru $Z$, then the inversion in $\Gamma$ exchanges the ideal points of~$\hat\ell$.
In particular, $\hat\ell$ maps to itself. 
Hence the result.

{

\begin{wrapfigure}{r}{22mm}
\begin{lpic}[t(-0mm),b(-2mm),r(0mm),l(0mm)]{pics/absolute-triangle-2-k(1)}
\lbl[tr]{10.5,9;$P$}
\lbl[tl]{16.5,9;$Q$}
\lbl[lb]{13,11;{\small $\phi$}}
\end{lpic}
\end{wrapfigure}

\parbf{Exercise~\ref{ex:klein-for-angle-parallelism}.}
Let $Q$ be the foot point of $P$ on the line and $\phi$ be the angle of parallelism. 
We can assume that $P$ is the center of the absolute.
Therefore $PQ=\cos\phi$ and

\[PQ_h=\tfrac12\cdot\ln\frac{1+\cos\phi}{1-\cos\phi}.\]

\parbf{Exercise~\ref{ex:klein-inradius}.} 
Apply Exercise~\ref{ex:klein-for-angle-parallelism} for $\phi=\tfrac\pi3$.

}

\parbf{Exercise~\ref{ex:pyth-h-proj}.}
Note that
\[
b=\tfrac12\cdot\ln\frac{1+t}{1-t};\]
therefore
\[
\cosh b
=
\tfrac12\cdot\left(\sqrt{\tfrac{1+t}{1-t}}+\sqrt{\tfrac{1-t}{1+t}}\right)
=
\frac1{\sqrt{1-t^2}}.
\eqlbl{cosh-b}
\]
The same way we get that
\[\begin{aligned}\cosh c&=\frac1{\sqrt{1-u^2}}.
\end{aligned}
\eqlbl{cosh-c}
\]

Let $X$ and $Y$ are the ideal points of~$(BC)_h$.
Applying the Pythagorean theorem (\ref{thm:pyth}) again,
we get that
$$CX=CY=\sqrt{1-t^2}.$$
Therefore, 
\[
a
=
\tfrac12\cdot\ln\frac{\sqrt{1-t^2}+s}{\sqrt{1-t^2}-s},\]
and
\[
\begin{aligned}
\cosh a&=\tfrac12\cdot\left(
{\sqrt{\frac{\sqrt{1-t^2}+s}{\sqrt{1-t^2}-s}}
+
\sqrt{\frac{\sqrt{1-t^2}-s}{\sqrt{1-t^2}+s}}}\right)=
\\
&=\frac{\sqrt{1-t^2}}{\sqrt{1-t^2-s^2}}=
\\
&=\frac{\sqrt{1-t^2}}{\sqrt{1-u^2}}.
\end{aligned}
\eqlbl{cosh-a}
\]

Finally, note that \ref{cosh-b}, \ref{cosh-c} and \ref{cosh-a} imply the theorem.

\parbf{Exercise~\ref{ex:Boyai-in-Euclid}.}
In the Euclidean plane, the circle $\Gamma_2$ is tangent to $k$; 
that is, the point $T$ of intersection of $\Gamma_2$ and $k$ is unique.
It defines a unique line $(PT)$ parallel to~$\ell$.

%\subsection*{Chapter~\ref{chap:complex}}
\refstepcounter{chapter}
\setcounter{eqtn}{0}

\parbf{Exercise~\ref{ex:|zw|}.} Use that $|z|^2=z\cdot \bar z$ for $z=v$, $w$ and $v\cdot w$.

%(???

\parbf{Exercise~\ref{ex:ptolemy}.} Given a quadrilateral $ABCD$, we can choose the complex coordinates so that $A$ has complex coordinate $0$. 
Rewrite the terms in the Ptolemy's inequality in terms of the complex coordinates $u$, $v$ and $w$ of $B$, $C$ and $D$; apply the identity and the triangle inequality.

%???)

\parbf{Exercise~\ref{ex:3-sum-C}.} 
Let $z$, $v$ and $w$ denote the complex coordinates of $Z$, $V$ and $W$ correspondingly.
Then 
\begin{align*}
\measuredangle ZVW+\measuredangle VWZ+\measuredangle WZV
&\equiv
\arg \tfrac{w-v}{z-v}+\arg \tfrac{z-w}{v-w}+\arg \tfrac{v-z}{w-z}\equiv
\\
&\equiv
\arg \tfrac{(w-v)\cdot(z-w)\cdot(v-z)}{(z-v)\cdot(v-w)\cdot(w-z)}\equiv
\\
&\equiv\arg (-1)\equiv
\\
&\equiv\pi.
\end{align*}

\parbf{Exercise~\ref{ex:C-sim}.}
Note and use that 
\begin{align*}
\measuredangle EOV&=\measuredangle WOZ=\arg v,
&
\frac{OW}{OZ}&=\frac{OZ}{OW}=|v|.
\end{align*}

\parbf{Exercise~\ref{ex:3-squares}.}
Set $\measuredangle EOA=\alpha$, $\measuredangle EOB=\beta$ and $\measuredangle EOC=\gamma$.
Note that
\begin{align*}
\alpha+\beta+\gamma
&\equiv\arg(1+i)+\arg(2+i)+\arg(3+i)=
\\
&\equiv\arg[(1+i)\cdot(2+i)\cdot(3+i)]=
\\
&\equiv\arg [10\cdot i]=
\\
&\equiv\tfrac\pi2.
\end{align*}
Note that the angles are acute and conclude that $\alpha+\beta+\gamma=\tfrac\pi2$.

\parbf{Exercise~\ref{ex:real-cross-ratio}.}
Note that 
\begin{align*}
\arg\frac{(v-u)\cdot(z-w)}{(v-w)\cdot(z-u)}
&\equiv
\arg\frac{v-u}{z-u}
+
\arg\frac{z-w}{v-w}=
\\
&= \measuredangle ZUV+\measuredangle VWZ.
\end{align*}

The statement follows since the value $\tfrac{(v-u)\cdot(z-w)}{(v-w)\cdot(z-u)}$ is real if and only if 
\[2\cdot\arg\frac{(v-u)\cdot(z-w)}{(v-w)\cdot(z-u)}\equiv0.\]

\parbf{Exercise~\ref{ex:6-circles}.}
Check the following identity:
\begin{align*}
\frac{(v-u)\cdot(z-w)}{(v-w)\cdot(z-u)}
\cdot
\frac{(v'-u')\cdot(z'-w')}{(v'-w')\cdot(z'-u')} 
=&
\frac{(v-u)\cdot(u'-v')}{(v-v')\cdot(u'-u)}
\cdot
\frac{(z-w)\cdot(w'-z')}{(z-z')\cdot(w'-w)}
\cdot
\\
\cdot
&
\frac{(v-v')\cdot(w'-w)}{(v-w)\cdot(w'-v')}
\cdot
\frac{(z-z')\cdot(u'-u)}{(z-u)\cdot(u'-z')}.
\end{align*}
By Theorem~\ref{thm:inscribed-quadrilateral-C}, five from six cross ratios in the this identity are real. 
Therefore so is the sixth cross ratio; it remains to apply the theorem again.

\parbf{Exercise~\ref{ex:inverse-Mob}.}
Show that the inverse of each elementary transformation is elementary
and use Proposition~\ref{prop:mob-comp}.

\parbf{Exercise~\ref{ex:3-point-Mob}.}
The fractional linear transformation
\[f(z)=\frac{(z_1-z_\infty)\cdot(z-z_0)}{(z_1-z_0)\cdot(z-z_\infty)}\]
meets the conditions.

To show the uniqueness, assume there is another fractional linear transformation
$g(z)$ which meets the conditions.
Then the composition
$h=g\circ f^{-1}$ 
is a fractional linear transformation; set
$h(z)=\tfrac{a\cdot z+b}{c\cdot z+d}$.

Note that $h(\infty)=\infty$;
therefore, $c=0$.
Further, $h(0)=0$ implies $b=0$.
Finally, since $h(1)=1$, we get that $\tfrac ad=1$.
Therefore, $h$ is the \index{identity map}\emph{identity};
that is, $h(z)=z$ for any~$z$.
It follows that $g=f$.

\parbf{Exercise~\ref{ex:inversion-Mob}.}
Let $Z'$ be the inverse of the point $Z$.
Assume that the circle of the inversion has center $W$ and radius~$r$.
Let $z$, $z'$ and $w$ denote the complex coordinate of the points $Z$, $Z'$ and $W$ correspondingly.

By the definition of inversion:
\begin{align*}
\arg (z-w)&=\arg (z'-w),
&
|z-w|\cdot|z'-w|&=r^2
\end{align*}
It follows that $(\bar z'-\bar w)\cdot ( z- w)= r^2$.
Equivalently,
\[z'=\overline{\left(\frac{\bar w\cdot z+[r^2-|w|^2]}{z- w}\right)}.\]

 
\parbf{Exercise~\ref{ex:C-cross-ratio}.}
Check the statement for each elementary transformation.
Then apply Proposition~\ref{prop:mob-comp}.

\parbf{Exercise~\ref{ex:schwarz-moebius}.}
Note that $f=\tfrac{a\cdot z+b}{c\cdot z+d}$ preserves the unit circle $|z|=1$.
Use Corollary~\ref{cor:invese-comp} and Proposition~\ref{prop:mob-comp} to show that $f$ commutes with the inversion $z\mapsto 1/\bar z$.
In other words, $1/\overline{f(z)}=f(1/\bar z)$ or
\[\frac{\bar c\cdot \bar z+\bar d}{\bar a\cdot \bar z+\bar b}
=\frac{a/\bar z+b}{c/\bar z+d}\]
for any $z\in\hat{\mathbb{C}}$.
The latter identity leads to the required statement. 
The condition $|w|<|v|$ follows since $f(0)\in\mathbb{D}$.

\parbf{Exercise~\ref{ex:schwarz-tanh}.} 
Note that the inverses of the points $z$ and $w$ have complex coordinates $1/\bar z$ and $1/\bar w$.
Apply Exercise \ref{ex:cosh} and simplify.

The second part follows since the function $x\mapsto \tanh(\tfrac12\cdot x)$ is increasing.

\parbf{Exercise~\ref{ex:schwarz}.}
Apply Schwarz--Pick theorem for a function $f$ such that $f(0)\z=0$ and then apply Lemma~\ref{lem:O-h-dist}.

{

\begin{wrapfigure}{r}{30mm}
\begin{lpic}[t(-8mm),b(-0mm),r(0mm),l(0mm)]{pics/appol(1)}
\lbl[br]{6,5;$O$}
\lbl[tl]{25,21;$P$}
\lbl[b]{21,15;$X$}
\end{lpic}
\end{wrapfigure}

%\subsection*{Chapter~\ref{chap:car}}
\refstepcounter{chapter}
\setcounter{eqtn}{0}

\parbf{Exercise~\ref{ex:simple-apollonius}.}
Let $O$ be the point of intersection of the lines.
Draw a line $\ell$ thru the given point $P$ and $O$.
Construct a circle $\Gamma$, tangent to both lines, which crosses~$\ell$. 
Let $X$ denotes one of the points of intersections.

Consider the homothety with the center at $O$ 
which sends $X$ to~$P$.
The image of $\Gamma$ is the needed circle.

}

\parbf{Exercise~\ref{ex:trisect-set-square}.} 
Note that with a set square we can construct a line parallel to given line thru the given point.
It remains to modify the construction in Exercise~\ref{ex:midpoint-affine}.

\parbf{Exercise~\ref{ex:equilateral triangle}.}
Assume that two vertices have rational coordinates, say $(a_1,b_1)$ and $(a_2,b_2)$.
Find the coordinates of the third vertex.
Use that the number $\sqrt{3}$ is irrational
to show that the third vertex is an irrational point.

\begin{wrapfigure}{o}{25mm}
\begin{lpic}[t(-0mm),b(-3mm),r(0mm),l(0mm)]{pics/equleteral-verify(1)}
\end{lpic}
\end{wrapfigure}

\parbf{Exercise~\ref{ex:equilateral triangle-verify}.}
Guess the construction from the diagram.
Prove that it verifies that the triangle is equilateral.

\parbf{Exercise~\ref{ex:center-verify}.} Apply the same argument as in Exercise~\ref{ex:circumtool}.

\parbf{Exercise~\ref{ex:midpoint-proj}.}
Consider the perspective projection as in Exercise~\ref{ex:persect}.
Let $A\z=(1,1,1)$, $B\z=(1,3,1)$ and $M\z=(1,2,1)$.
Note that $M$ is the midpoint of $[AB]$.

Their images are $A'\z=(1,1,1)$, $B'\z=(\tfrac13,1,\tfrac13)$ and $M'\z=(\tfrac12,1,\tfrac12)$.
Clearly, $M'$ is not the midpoint of~$[A'B']$.

\parbf{Exercise~\ref{ex:tangent ruler}.}
The line $v$ polar to $V$ is tangent to~$\Gamma$.
Since $V\in p$, by Claim~\ref{clm:polar}, we get that $P\in v$;
that is, $(PV)=v$.
Hence the statement follows.

{

\begin{wrapfigure}{r}{35mm}
\begin{lpic}[t(-0mm),b(0mm),r(0mm),l(0mm)]{pics/polar-polar(1)}
\lbl[b]{32,21;$P$}
\lbl[l]{23,8;$X'$}
\lbl[l]{23,31.5;$X$}
\lbl[bl]{7,3.5;$Y$}
\lbl[tl]{6,35;$Y'$}
\lbl[lw]{18.2,19.6;$Z$}
\end{lpic}
\end{wrapfigure} %qu

\parbf{Exercise~\ref{ex:concentric-circ}.}
Choose a point $P$ outside of the bigger circle.
Construct the lines dual to $P$ for both circles.
Note that these two lines are parallel. 

Assume that the lines intersect the bigger circle at two pairs of points $X$, $X'$ and $Y, Y'$.
Set $Z=(XY)\cap (X'Y')$.
Note that the line $(PZ)$ passes thru the common center.

The center is the intersection of $(PZ)$ and another line constructed the same way.



%\subsection*{Chapter~\ref{chap:area}}
\refstepcounter{chapter}
\setcounter{eqtn}{0} 

\parbf{Exercise~\ref{ex:triangle-convex}.}
Assume the contrary; 
that is, there is a point $W\in [XY]$ such that $W\notin\solidtriangle ABC$.

}

Without loss of generality, we may assume that $W$ and $A$ lie on opposite sides from the line~$(BC)$.

It imples that both segments $[WX]$ and $[WY]$ intersect $(BC)$.
By Axiom~\ref{def:birkhoff-axioms:1}, $W\in (BC)$ --- a contradiction.

%\parbf{Exercise~\ref{ex:solid-triangle-sum}.} Without loss of generality we may assume that the angles $ACB$, $BAC$ and $CBA$ are positive.
%By Exercise~\ref{ex:signs-PXQ-PYQ}, $X\in \solidtriangle ABC$ if and only if $\measuredangle AXB, \measuredangle BXC,\measuredangle CXA\ge 0$.
%Note that 
%\[\measuredangle AXB+\measuredangle BXC+\measuredangle CXA\equiv0.\]

\parbf{Exercise~\ref{ex:vertex}.} 
To prove the ``only if'' part, consider the line passing thru the vertex which is parallel to the opposite side.

To prove the ``if'' part, use Pasch's theorem (\ref{thm:pasch}).

\parbf{Exercise~\ref{ex:solid-square}.}
Assume the contrary; that is, a solid square $\mathcal{Q}$ can be presented as a union of a finite collection of segments $[A_1B_1],\dots,[A_nB_n]$
and one-point sets $\{C_1\},\dots,\{C_k\}$.

Note that $\mathcal{Q}$ contains an infinite number of mutually nonparallel segments.
Therefore, we can choose a segment $[PQ]$ in $\mathcal{Q}$ 
which is not parallel to any of the segments $[A_1B_1],\dots,[A_nB_n]$.

It follows that $[PQ]$ has at most one common point with each of the sets $[A_iB_i]$ and~$\{C_i\}$.
Since $[PQ]$ contains infinite number of points, we arrive to a contradiction.

\begin{wrapfigure}[9]{r}{26mm}
\begin{lpic}[t(0mm),b(0mm),r(0mm),l(0mm)]{pics/two-parallelograms-sol(1)}
\lbl[tr]{1,7;$A$}
\lbl[t]{12,0;$B$}
\lbl[l]{21,20;$C$}
\lbl[b]{9,27.5;$D$}
\lbl[tl]{16,9;$B'$}
\lbl[b]{18,28.5;$C'$}
\lbl[br]{4,26;$D'$}
\lbl[b]{23,29;$E$}
\end{lpic}
\end{wrapfigure}

\parbf{Exercise~\ref{ex:poly-circ}.} 
First note that among elementary sets
only one-point sets can be subsets of the a circle.
It remains to note that any circle contains an infinite number of points.


\parbf{Exercise~\ref{ex:two-parallelograms}.}
Let $E$ denotes the point of intersection of the lines $(BC)$ and~$(C'D')$.

Use Proposition~\ref{prop:area-parallelogram} to prove the following two identities:
\begin{align*}
\area(\solidsquare AB'ED)
&=\area(\solidsquare ABCD),
\\
\area(\solidsquare AB'ED)
&=\area(\solidsquare AB'C'D').
\end{align*}
Hence the statement follows.

\parbf{Exercise~\ref{ex:three-trig}.}
Without loss of generality, we may assume that the angles $ABC$ and $BCA$ are acute.

Let $A'$ and $B'$ denote the foot points of $A$ and $B$ on $(BC)$ and $(AC)$ correspondingly.
Note that $h_A=AA'$ and $h_B=BB'$.

\begin{wrapfigure}{o}{20mm}
\begin{lpic}[t(-0mm),b(0mm),r(0mm),l(0mm)]{pics/three-areas(1)}
\lbl[tr]{0,0;$A$}
\lbl[tl]{18,0;$B$}
\lbl[b]{6,16;$C$}
\lbl[bl]{12,10;$A'$}
\lbl[br]{3,7;$B'$}
\end{lpic}
\end{wrapfigure}

Note that $\triangle AA'C\sim \triangle BB'C$;
indeed the angle at $C$ is shared and the angles at $A'$ and $B'$ are right.
In particular
\[\frac{AA'}{BB'}=\frac{AC}{BC}\]
or, equivalently,
\[h_A\cdot BC=h_B\cdot AC.\]

Along the same lines, we get that
\[h_C\cdot AB=h_B\cdot AC.\]
Hence the statement follows.

\begin{wrapfigure}{r}{22mm}
\begin{lpic}[t(3mm),b(0mm),r(0mm),l(0mm)]{pics/trigs-in-parallelogram(1)}
\lbl[tr]{0,1;$A$}
\lbl[tl]{15,1;$B$}
\lbl[tl]{19,17;$C$}
\lbl[br]{5,15;$D$}
\lbl[l]{17.5,6.5;$E$}
\lbl[r]{1,6.5;$F$}
\lbl[b]{11.5,8.5;$M$}

\end{lpic}
\end{wrapfigure}

\parbf{Exercise~\ref{ex:half-parallelogram}.}
Draw the line $\ell$ 
thru $M$ parallel to $[AB]$ and $[CD]$;
it subdivides $\solidsquare ABCD$ into two solid parallelograms
which will be denoted by
$\solidsquare ABEF$ and
$\solidsquare CDFE$.
In particular,
\[\area(\solidsquare ABCD)
=
\area(\solidsquare ABEF)+\area(\solidsquare CDFE).\]

By Proposition~\ref{prop:area-parallelogram} and Theorem~\ref{thm:area-of-triangle} we get that 
\begin{align*}
\area(\solidtriangle ABM)&=\tfrac12\cdot\area(\solidsquare ABEF),
\\
\area(\solidtriangle CDM)&=\tfrac12\cdot\area(\solidsquare CDFE)
\end{align*}
and hence the result.

\parbf{Exercise~\ref{ex:area-diag}.}
Let $h_A$ and $h_C$ denote the distances from $A$ and $C$ to the line~$(BD)$ correspondingly.
According to Theorem~\ref{thm:area-of-triangle},
\begin{align*}
\area(\solidtriangle ABM)&=\tfrac12\cdot h_A\cdot BM;
&
\area(\solidtriangle BCM)&=\tfrac12\cdot h_C\cdot BM;
\\
\area(\solidtriangle CDM)&=\tfrac12\cdot h_C\cdot DM;
&
\area(\solidtriangle ABM)&=\tfrac12\cdot h_A\cdot DM.
\end{align*}
Therefore
\begin{align*}
\area(\solidtriangle ABM)\cdot \area(\solidtriangle CDM)
&=\tfrac14 \cdot h_A\cdot h_C\cdot DM\cdot BM=
\\
&=\area(\solidtriangle BCM)\cdot \area(\solidtriangle DAM).
\end{align*}

\parbf{Exercise~\ref{ex:area-inradius}.}
Let $I$ be the incenter of $\triangle ABC$.
Note that $\solidtriangle ABC$
can be subdivided into 
$\solidtriangle IAB$, 
$\solidtriangle IBC$
and $\solidtriangle ICA$.

It remains to apply Theorem~\ref{thm:area-of-triangle} 
to each of these triangles and sum up the results.

\parbf{Exercise~\ref{ex:pyth-2}.}
Assuming $a>b$,
we subdivided $\mathcal{Q}_c$ into $\mathcal{Q}_{a-b}$ and four triangles congruent to~$\mathcal{T}$.
Therefore
\[\area\mathcal{Q}_c=\area\mathcal{Q}_{a-b}+4\cdot\area\mathcal{T}.
\eqlbl{eq:pyth-2}\]

According to Theorem~\ref{thm:area-of-triangle},
$\area\mathcal{T}=\tfrac12\cdot a\cdot b$. %???LINE
Therefore, the identity \ref{eq:pyth-2} can be written as 
\[c^2=(a-b)^2+2\cdot a\cdot b.\]
Simplifying, we get the Pythagorean theorem.

The case $a=b$ is yet simpler.
The case $b>a$ can be done the same way.

\parbf{Exercise~\ref{ex:sum-3-dist}.} 
If $X$ is a point inside of $\triangle ABC$, then $\solidtriangle ABC$ is subdivided into $\solidtriangle ABX$, $\solidtriangle BCX$ and $\solidtriangle CAX$.
Therefore
\[\area(\solidtriangle ABX)
+\area(\solidtriangle BCX)
+\area(\solidtriangle CAX)
=\area(\solidtriangle ABC).\]

Set $a=AB=BC=CA$.
Let $h_1$, $h_2$ and $h_3$ denote the distances from $X$ to the sides $[AB]$, $[BC]$ and~$[CA]$. 
Then by Theorem~\ref{thm:area-of-triangle},
\begin{align*}
\area(\solidtriangle ABX)&=\tfrac12\cdot h_1\cdot a,
\\
\area(\solidtriangle BCX)&=\tfrac12\cdot h_2\cdot a,
\\
\area(\solidtriangle CAX)&=\tfrac12\cdot h_3\cdot a.
\end{align*}
Therefore, 
\[h_1+h_2+h_3=\tfrac2a\cdot\area(\solidtriangle ABC).\]

{
\parbf{Exercise~\ref{ex:area-medians}.}
Let $X$ be a point inside $\triangle ABC$.
Let $Y$ denotes the point of intersection of $(AX)$ and~$[BC]$.

Let $b$ and $c$ denote the distances from $B$ and $C$ to the line~$(AX)$.

\begin{wrapfigure}{r}{37mm}
\begin{lpic}[t(-3mm),b(0mm),r(0mm),l(0mm)]{pics/area-median-sol(1)}
\lbl[r]{8.5,23.5;$X$}
\lbl[tr]{2,36;$A$}
\lbl[tl]{36,6.5;$B$}
\lbl[tr]{1,6.5;$C$}
\lbl[tr]{18,6.5;$Y$}
\lbl[W]{8,12,30;$b$}
\lbl[W]{29,4.5,30;$c$}
\end{lpic}
\end{wrapfigure}

By Theorem~\ref{thm:area-of-triangle}, 
we get the following equivalences:
\begin{align*}
\area (\solidtriangle AXB)&=\area (\solidtriangle AXC),
\\
&\Updownarrow
\\
b&=c,
\\
&\Updownarrow
\\
\area (\solidtriangle AYB)&=\area (\solidtriangle AYC),
\\
&\Updownarrow
\\
BY&=CY.
\end{align*}

}

\parbf{Exercise~\ref{ex:area-medians-2}.}
Let $M$ denotes the intersection of 
two medians $[AA']$ and~$[BB']$.
From Exercise~\ref{ex:area-medians} we have 
\begin{align*}
\area (\solidtriangle ABM)
&=\area (\solidtriangle ACM),
&
\area (\solidtriangle ABM)
&=\area (\solidtriangle CBM).
\end{align*}
Therefore, 
\[\area (\solidtriangle BCM)
=\area (\solidtriangle ACM).\]

According to Exercise~\ref{ex:area-medians},
$M$ lies on the median~$[CC']$.
That is, medians $[AA']$, $[BB']$ and $[CC']$ intersect at one point~$M$.

{

\begin{wrapfigure}[8]{r}{34mm}
\begin{lpic}[t(1mm),b(0mm),r(0mm),l(0mm)]{pics/area-median-sol2(1)}
\lbl[trw]{13,10;$M$}
\lbl[b]{13,25.5;$A$}
\lbl[tl]{31,0;$B$}
\lbl[tr]{1,4;$C$}
\lbl[t]{16,2;$A'$}
\lbl[bl]{23,13;$C'$}
\lbl[br]{7,16;$B'$}
\end{lpic}
\end{wrapfigure}

By Theorem~\ref{thm:area-of-triangle},
we get that
\begin{align*}
\area (\solidtriangle C'AM)
&=\tfrac12\cdot\area (\solidtriangle BAM)\\
&=\tfrac12\cdot\area (\solidtriangle CAM)
\end{align*}

Applying Claim~\ref{clm:area-ratio},
we get that 
\[\frac{MC'}{MC}=\frac{\area (\solidtriangle C'AM)}{\area (\solidtriangle CAM)}=\tfrac12.\]

}

\parbf{Exercise~\ref{ex:circle-is-quadrable}.}
Let $\mathcal{P}_n$ and $\mathcal{Q}_n$ be the solid regular $n$-gons
so that $\Gamma$ is inscribed in $\mathcal{Q}_n$ and circumscribed around~$\mathcal{P}_n$.
Clearly,
\[\mathcal{P}_n\subset\mathcal{D}\subset\mathcal{Q}_n.\]

Show that 
$\tfrac{\area\mathcal{P}_n}{\area\mathcal{Q}_n}=(\cos\tfrac\pi n)^2$;
in particular, 
$$\frac{\area\mathcal{P}_n}{\area\mathcal{Q}_n}\to 1
\quad
\text{as}
\quad
n\to\infty.$$

Next show that $\area\mathcal{Q}_n<100$, say for all $n\ge 100$.

These two statements imply that
$(\area\mathcal{Q}_n-\area\mathcal{P}_n)\to 0$.
Hence the result.


\chapter{Triangle geometry}\label{chap:triangle}
\addtocontents{toc}{\protect\begin{quote}}

Triangle geometry is the study of the properties of triangles, including associated centers and circles.

We discuss the most basic results in triangle geometry, 
mostly to show that we have developed sufficient machinery to prove things.

\section*{Circumcircle and circumcenter}
\addtocontents{toc}{Circumcircle and circumcenter.}

\begin{thm}{Theorem}\label{thm:circumcenter}
Perpendicular bisectors to the sides of any nondegenerate triangle intersect at one point.
\end{thm}

The point of intersection of the perpendicular bisectors is called \index{circumcenter}\emph{circumcenter}.
It is the center of the \index{circumcircle}\emph{circumcircle} of the triangle;
that is, the circle which passes thru all three vertices of the triangle.
The circumcenter of the triangle is usually denoted by~$O$.

\begin{wrapfigure}{o}{25mm}
\begin{lpic}[t(-5mm),b(0mm),r(0mm),l(0mm)]{pics/circum(1)}
\lbl[tr]{0.5,0;$B$}
\lbl[b]{14,33;$A$}
\lbl[tl]{23,0;$C$}
\lbl[t]{11.5,12;$O$}
\lbl[b]{3,19,-30;$\ell$}
\lbl[b]{22,18,17;$m$}
\end{lpic}
\end{wrapfigure}


\parit{Proof.}
Let $\triangle ABC$ be nondegenerate.
Let $\ell$ and $m$ be perpendicular bisectors to sides $[AB]$ and $[AC]$ correspondingly.

Assume $\ell$ and $m$ intersect,
let $O=\ell\cap m$.

Let us apply Theorem~\ref{thm:perp-bisect}.
Since $O\in\ell$, we have $OA=OB$ and since $O\in m$, we have $OA=OC$.
It follows that $OB\z=OC$;
that is, $O$ lies on the perpendicular bisector to~$[B C]$.

It remains to show that $\ell\nparallel m$;
assume the contrary.
Since
$\ell\perp(AB)$ and $m\perp (AC)$, we get that $(AC)\z\parallel (AB)$ 
(see Exercise~\ref{ex:perp-perp}).
Therefore, by Theorem~\ref{perp:ex+un}, $(AC)=(AB)$;
that is, $\triangle ABC$ is degenerate --- a contradiction.
\qeds

\begin{thm}{Exercise}\label{ex:unique-cline}
There is a unique circle which passes thru the vertexes of a given nondegenerate triangle in the Euclidean plane. 
\end{thm}




\section*{Altitudes and orthocenter}
\addtocontents{toc}{Altitudes and orthocenter.}

An \index{altitude}\emph{altitude} of a triangle is a line thru a vertex and perpendicular to the line containing the opposite side.
The term \index{altitude}\emph{altitude} maybe also used for the distance from the vertex to its foot point on the line containing opposite side.

\begin{thm}{Theorem}\label{thm:orthocenter}
The three altitudes of any nondegenerate triangle intersect in a single point.
\end{thm}

The point of intersection of altitudes is called \index{orthocenter}\emph{orthocenter}; 
it is usually denoted by~$H$.


\parit{Proof.}
Let $\triangle A B C$ be nondegenerate.

{

\begin{wrapfigure}{o}{34mm}
\begin{lpic}[t(-2mm),b(0mm),r(-0mm),l(0mm)]{pics/median-triangle(1)}
\lbl[tr]{3,24;$B'$}
\lbl[tr]{6.5,14;$A$}
\lbl[tl]{27,24;$A'$}
\lbl[tl]{20.5,14;$B$}
\lbl[b]{15.5,27.5;$C$}
\lbl[r]{11,4;$C'$}
\lbl[b]{2.5,30.5,-60;$\ell$}
\lbl[b]{30,31,50;$m$}
\lbl[b]{32,26.4;$n$}
\end{lpic}
\end{wrapfigure}

Consider three lines $\ell$, $m$ and $n$
such that 
\begin{align*}
\ell&\parallel(BC),
&
m&\parallel(CA),
&
n&\parallel(AB),
\\
\ell&\ni A,
&
m&\ni B,
&
n&\ni C.
\end{align*}
Since $\triangle A B C$ is nondegenerate,
no pair of the lines $\ell$, $m$ and $n$ is parallel.
Set 
\begin{align*}
A'&=m\cap n,
&
B'&=n\cap \ell,
&
C'&=\ell\cap m.
\end{align*}

}

Note that $\square A B A' C$, $\square B C B' A$ and $\square C B C' A$ are parallelograms.
Applying Lemma~\ref{lem:parallelogram} we get that $\triangle ABC$ is the median triangle of $\triangle A' B' C'$;
that is, $A$, $B$ and $C$ are the midpoints of $[B' C']$, $[C' A']$ and $[A' B']$ correspondingly.

By Exercise \ref{ex:perp-perp},
$(B' C')\parallel (BC)$,
the altitude from $A$ is perpendicular to $[B' C']$ 
and from above it bisects~$[B' C']$.

Hence the altitudes of $\triangle A B C$ 
are also perpendicular bisectors of $\triangle A' B' C'$.
Applying Theorem~\ref{thm:circumcenter}, we get that altitudes of $\triangle ABC$ intersect at one point.
\qeds

\begin{thm}{Exercise}\label{ex:orthic-4}
Assume $H$ is the orthocenter of an acute triangle $A B C$.
Show that $A$ is the orthocenter of $\triangle H B C$.
\end{thm}



\section*{Medians and centroid}
\addtocontents{toc}{Medians and centroid.}

A median of a triangle is the segment joining a vertex to the midpoint of the opposing side. 

\begin{thm}{Theorem}\label{thm:centroid}
The three medians of any nondegenerate triangle intersect in a single point.
Moreover, the point of intersection divides each median in the ratio 2:1.
\end{thm}

The point of intersection of medians is called the \index{centroid}\emph{centroid} of the triangle; 
it is usually denoted by~$M$.

\parit{Proof.}
Consider a nondegenerate triangle $A B C$.
Let $[A A']$ and $[B B']$ be its medians.

According to Exercise~\ref{ex:chevinas}, 
$[A A']$ and $[B B']$ are intersecting. 
Let us denote the point of intersection by $M$.

By SAS, $\triangle B' A' C\z\sim \triangle A B C$ and $A' B'=\tfrac12\cdot A B$.
In particular, 
$\measuredangle A B C= \measuredangle B' A' C$.

\begin{wrapfigure}{o}{28mm}
\begin{lpic}[t(-0mm),b(0mm),r(0mm),l(1mm)]{pics/medians(1)}
\lbl[tr]{1,1;$A$}
\lbl[l]{15,23;$A'$}
\lbl[t]{24,1;$B$}
\lbl[r]{3,23;$B'$}
\lbl[b]{6,45;$C$}
\lbl[t]{10,13;$M$}
\end{lpic}
\end{wrapfigure}

Since $A'$ lies between $B$ and $C$,
we get that $\measuredangle BA'B'+\measuredangle B'A'C=\pi$.
Therefore, 
$$\measuredangle B'A'B+\measuredangle A'BA=\pi.$$
By the transversal property \ref{thm:parallel-2}, 
$(A B)\z\parallel (A' B')$.

Note that $A'$ and $A$ lie on opposite sides from~$(BB')$.
Therefore, by the transversal property \ref{thm:parallel-2},
we get that
$$\measuredangle B'A'M=\measuredangle BAM.$$
The same way we get that
$$\measuredangle A'B'M=\measuredangle ABM.$$
By AA condition,
$\triangle A B M\sim\triangle A' B' M$.

Since $A' B'=\tfrac12\cdot A B$, 
we have
$$\frac{A' M}{A M}=\frac{B' M}{B M}=\frac12.$$
In particular, $M$ divides medians $[A A']$ and $[B B']$ in ratio 2:1.

Note that $M$ is a unique point on $[B B']$ 
such that $$\frac{B' M}{B M}=\frac12.$$
Repeating the same argument for vertices $B$ and $C$ we get that all medians
$[C C']$ and $[B B']$ intersect at~$M$.\qeds

\begin{thm}{Exercise}\label{ex:midle}
Let $\square ABCD$ be a nondegenerate quadrilateral
and $X$, $Y$, $V$, $W$ be the midpoints of its sides 
$[AB]$, $[BC]$, $[CD]$ and~$[DA]$.
Show that $\square XYVW$ is a parallelogram.
\end{thm}

%(???

\section*{Angle bisectors}
\addtocontents{toc}{Angle bisectors.}

If $\measuredangle A B X\equiv-\measuredangle C B X$, 
then we say that the line $(BX)$ {}\emph{bisects} $\angle ABC$,
or line $(BX)$ is the \index{bisector!angle bisector}\emph{bisector} of $\angle ABC$.
If $\measuredangle A B X\equiv\pi-\measuredangle C B X$, then the line $(BX)$ is called the \index{bisector!external bisector}\emph{external bisector} of $\angle ABC$.


\begin{wrapfigure}[9]{o}{39mm}
\begin{lpic}[t(-2mm),b(0mm),r(0mm),l(1mm)]{pics/bisectors(1)}
\lbl[tr]{30,5;$A$}
\lbl[rb]{9,10;$B$}
\lbl[br]{31,29;$C$}
\lbl[b]{30,16.7,17.5;bisector}
\lbl[b]{6.3,26.5,-72.5;external} 
\lbl[t]{4.7,26,-72.5;bisector}
\end{lpic}
\end{wrapfigure}

If $\measuredangle ABA'=\pi$;
that is, if $B$ lies between $A$ and $A'$,
then bisector of $\angle ABC$ is the external bisector of $\angle A' B C$ and the other way around.

Note that the bisector and the external bisector are uniquely defined by the angle.

\begin{thm}{Exercise}\label{ex:perp-bisectors}
Show that for any angle, its bisector and external bisector are perpendicular.
\end{thm}

The bisectors of  $\angle ABC$, $\angle BCA$ and $\angle CAB$ of a nondegenerate triangle $A B C$
are called bisectors of $\triangle A B C$ at vertexes $A$, $B$ and $C$ correspondingly.

\begin{thm}{Lemma}\label{lem:bisect-ratio}
Let $\triangle A B C$ be  a nondegenerate triangle.
Assume that the bisector at the vertex $A$ 
intersects the side $[BC]$ at the point~$D$.
Then 
$$\frac{AB}{AC}=\frac{DB}{DC}.
\eqlbl{bisect-ratio}$$

\end{thm}

\begin{wrapfigure}{o}{28mm}
\begin{lpic}[t(-6mm),b(0mm),r(0mm),l(1mm)]{pics/angle-bisect-ratio-new(1)}
\lbl[br]{14,36;$A$}
\lbl[lt]{24.5,22;$B$}
\lbl[rt]{0.5,22;$C$}
\lbl[tl]{14,21.5;$D$}
\lbl[r]{10,7;$E$}
\lbl[l]{16,3;$\ell$}
\end{lpic}
\end{wrapfigure}

\parit{Proof.}
Let $\ell$ be the line passing thru $C$ that is parallel to~$(AB)$.
Note that $\ell\nparallel (AD)$;
set 
\[E=\ell\cap (AD).\]

Also note that $B$ and $C$ lie on opposite sides of~$(AD)$.
Therefore, by the transversal property (\ref{thm:parallel-2}),
$$\measuredangle BAD=\measuredangle CED.\eqlbl{eq:<BAD=<CED}$$

Further, note that the angles $ADB$ and $EDC$ are vertical; in particular, by \ref{prop:vert} 
$$\measuredangle ADB=\measuredangle EDC.$$

By the AA similarity condition, 
$\triangle ABD\sim \triangle ECD$.
In particular, 
$$\frac{AB}{EC}=\frac{DB}{DC}.\eqlbl{eq:AB/EC=DB/DC}$$

Since $(AD)$ bisects $\angle BAC$, we get that
$\measuredangle BAD=\measuredangle DAC$.
Together with \ref{eq:<BAD=<CED},
it implies that 
$\measuredangle CEA=\measuredangle EAC$.
By Theorem~\ref{thm:isos}, $\triangle ACE$ is isosceles; 
that is, $$EC=AC.$$
Together with \ref{eq:AB/EC=DB/DC}, it implies \ref{bisect-ratio}.
\qeds 



\begin{thm}{Exercise}\label{ex:ext-disect}
Formulate and prove an analog of Lemma~\ref{lem:bisect-ratio} for the external bisector.
\end{thm}


\section*{Equidistant property}
\addtocontents{toc}{Equidistant property.}

Recall that distance from the line $\ell$ to the point $P$ is defined as the distance from $P$ to its foot point on $\ell$; see page~\pageref{distance!from a point to a line}. 

\begin{thm}[\abs]{Proposition}\label{prop:angle-bisect-dist}
Assume $\triangle ABC$ is not degenerate.
Then a point $X$ lies on the bisector or external bisector of $\angle ABC$
if and only if $X$ is equidistant from the lines $(AB)$ and $(BC)$.
\end{thm}

\parit{Proof.}
We can assume that $X$ does not lie on the union of $(AB)$ and $(BC)$.
Otherwise the distance to one of the lines vanish;
in this case $X=B$ is the only point equidistant from the two lines.

{

\begin{wrapfigure}{o}{26mm}
\begin{lpic}[t(-3mm),b(0mm),r(0mm),l(1mm)]{pics/angle-bisect-lemma(1)}
\lbl[bl]{23,11;$A$}
\lbl[br]{1,18;$B$}
\lbl[rb]{21,36;$C$}
%\lbl[t]{13.5,26;$Z$}
\lbl[l]{21,22;$X$}
%\lbl[br]{16,13;$Y$}
\lbl[t]{14,0;$Y$}
\lbl[rb]{8,35.5;$Z$}
\end{lpic}
\end{wrapfigure}

Let $Y$ and $Z$ be the reflections of $X$ in $(AB)$ and $(BC)$ correspondingly.
Note that 
\[Y\ne Z.\]
Otherwise both lines $(AB)$ and $(BC)$ are perpendicular bisectors of $[XY]$, that is $(AB)=(BC)$ which is impossible since $\triangle ABC$ is not degenerate.

By Proposition~\ref{prop:reflection},
\[XB=YB=ZB.\]



Note that $X$ is equidistant from $(AB)$ and $(BC)$ if and only if $XY\z=XZ$.
Applying SSS and then SAS, we get that
$$\begin{aligned}
XY&=XZ.
\\
&\hskip0.4mm\Updownarrow
\\
\triangle BXY&\cong\triangle BXZ.
\\
&\hskip0.4mm\Updownarrow
\\
\measuredangle XBY&= \pm \measuredangle BXZ.
\end{aligned}
$$

}

Since $Y\ne Z$, we get that $\measuredangle XBY\ne \measuredangle BXZ$;
therefore
\[\measuredangle XBY= -\measuredangle BXZ.
\eqlbl{eq:iff-chain}\]

By Proposition~\ref{prop:reflection}, $A$ lies on the bisector of $\angle XBY$
and $B$ lies on the bisector of $\angle XBZ$; that is,
\begin{align*}
2\cdot \measuredangle XBA&\equiv \measuredangle XBY,
&
2\cdot \measuredangle XBC&\equiv \measuredangle XBZ.
\end{align*}
By \ref{eq:iff-chain},
\[2\cdot \measuredangle XBA\equiv -2\cdot \measuredangle XBC.\]
The last identity means either
\[
\measuredangle XBA+\measuredangle XBC\equiv 0
\quad
\text{or}
\quad
\measuredangle XBA+\measuredangle XBC\equiv \pi,
\]
and hence the result.
\qeds

%???)

\section*{Incenter}
\addtocontents{toc}{Incenter.}

\begin{thm}[\abs]{Theorem}\label{thm:incenter}
The angle bisectors of any nondegenerate triangle intersect at one point.
\end{thm}


The point of intersection of bisectors is called the \index{incenter}\emph{incenter} of the triangle; 
it is usually denoted by~$I$.
The point $I$ lies on the same distance from each side.
In particular, it is the center of a circle tangent to each side of triangle.
This circle is called 
the \index{incircle}\emph{incircle} and its radius is called 
the \index{inradius}\emph{inradius} of the triangle.


\begin{wrapfigure}{o}{34mm}
\begin{lpic}[t(-0mm),b(2mm),r(0mm),l(0mm)]{pics/triangle-bisector(0.9)}
\lbl[rt]{2,0;$A$}
\lbl[lt]{34,0;$B$}
\lbl[b]{24.5,56.5;$C$}
\lbl[lb]{21,16;$I$}
\lbl[t]{19.5,0;$Z$}
\lbl[lb]{32.5,21;$A'$}
\lbl[l]{33,16;$X$}
\lbl[r]{8,18;$Y$}
\lbl[rb]{11,23;$B'$}
\end{lpic}
\end{wrapfigure}

\parit{Proof.} 
Let $\triangle ABC$ be a nondegenerate triangle.

Note that the points $B$ and $C$ lie on opposite sides of the bisector of $\angle BAC$.
Hence this bisector intersects $[BC]$ at a point, say~$A'$.

Analogously, there is $B'\in[AC]$ 
such that $(BB')$ bisects $\angle ABC$.

Applying Pasch's theorem (\ref{thm:pasch}) twice
for the triangles $AA'C$ and $BB'C$,
we get that $[AA']$ and $[BB']$ intersect.
Let $I$ denotes the point of intersection.

Let $X$, $Y$ and $Z$ be the foot points of $I$ on  $(B C)$, $(C A)$ and $(A B)$ correspondingly.
Applying Proposition~\ref{prop:angle-bisect-dist}, we get that
$$I Y=I Z=I X.$$
From the same lemma, we get that $I$ lies on the bisector or on the exterior bisector of $\angle B C A$.

The line $(C I)$ intersects $[B B']$,
the points $B$ and $B'$ lie on opposite sides of~$(C I)$.
Therefore,  the angles $I C B'$ and $I C B$ have opposite signs.
Note that $\angle I C A=\angle I C B'$.
Therefore, $(C I)$ cannot be the exterior bisector of $\angle B C A$.
Hence the result.
\qeds

\section*{More exercises}

\begin{thm}{Exercise}\label{ex:bisect=median}
Assume that an angle bisector of a nondegenerate triangle bisects the opposite side. 
Show that the triangle is isosceles.
\end{thm}

\begin{thm}{Exercise}\label{ex:bisect=altitude}
Assume that at one vertex of a nondegenerate triangle the bisector coincides with the altitude.
Show that  the triangle is isosceles.
\end{thm}

\begin{wrapfigure}[5]{r}{26mm}
\begin{lpic}[t(-8mm),b(0mm),r(0mm),l(0mm)]{pics/incircle(1)}
\lbl[rt]{1,1;$A$}
\lbl[tl]{22,1;$B$}
\lbl[b]{17,25;$C$}
\lbl[l]{21,11;$X$}
\lbl[rb]{8,13;$Y$}
\lbl[t]{13.5,1;$Z$}
\end{lpic}
\end{wrapfigure}

\begin{thm}{Exercise}\label{ex:2x=b+c-a}
Assume sides $[B C]$, $[C A]$ and $[A B]$ of $\triangle A B C$ are tangent to the incircle at $X$, $Y$ and $Z$ correspondingly. 
Show that 
$$AY=AZ= \tfrac12\cdot(A B+ A C- B C).$$

\end{thm}

By the definition, the vertexes of \index{triangle!orthic triangle}\index{orthic triangle}\emph{orthic triangle} are the base points of the altitudes of the given triangle.

\begin{thm}{Exercise}\label{ex:orthic-triangle}
Prove that the orthocenter of an acute triangle coincides with the incenter of its orthic triangle.

What should be an analog of this statement for an obtuse triangle?
\end{thm}



\begin{wrapfigure}{r}{30mm}
\begin{lpic}[t(-2mm),b(0mm),r(0mm),l(0mm)]{pics/bisector-parallel(1)}
\lbl[b]{22,13;$A$}
\lbl[t]{27,-1;$B$}
\lbl[t]{1,-1;$C$}
\lbl[t]{18,-1;$D$}
\lbl[lb]{26.5,5;$E$}
\lbl[rb]{8,5;$F$}
\end{lpic}
\end{wrapfigure}

\begin{thm}{Exercise}\label{ex:bisector-parallel} 
Assume that the bisector at $A$ of the triangle $ABC$ intersects the side $[BC]$ at the point $D$;
the line thru $D$ and parallel to $(CA)$ intersects $(AB)$ at the point $E$;
the line thru $E$ and parallel to $(BC)$ intersects $(AC)$ at $F$.
Show that $AE=FC$.
\end{thm}


\addtocontents{toc}{\protect\end{quote}}
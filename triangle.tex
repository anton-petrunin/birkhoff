\chapter{Triangle geometry}\label{chap:triangle}

Triangle geometry is the study of the properties of triangles, including associated centers and circles.

We discuss the most basic results in triangle geometry, 
mostly to show that we have developed sufficient machinery to prove things.

\section{Circumcircle and circumcenter}

\begin{thm}{Theorem}\label{thm:circumcenter}
Perpendicular bisectors to the sides of every nondegenerate triangle intersect at one point.
\end{thm}

The point of intersection of the perpendicular bisectors is called the \index{circumcenter}\emph{circumcenter}.
It is the center of the \index{circumcircle}\emph{circumcircle} of the triangle;
that is, a circle that passes thru all three vertices of the triangle.
The circumcenter of the triangle is usually denoted by~$O$.

\begin{wrapfigure}{o}{29mm}
\centering
\includegraphics{mppics/pic-102}
\end{wrapfigure}


\parit{Proof.}
Let $\triangle ABC$ be nondegenerate.
Let $\ell$ and $m$ be perpendicular bisectors to sides $[AB]$ and $[AC]$ respectively.

Assume $\ell$ intersects $m$, say, at $O$.

Let us apply Theorem~\ref{thm:perp-bisect}.
Since $O\in\ell$, we have $OA\z=OB$ and since $O\in m$, we have $OA\z=OC$.
It follows that $OB\z=OC$;
that is, $O$ lies on the perpendicular bisector of~$[B C]$.

It remains to show that $\ell\nparallel m$;
assume the contrary.
Since
$\ell\perp(AB)$ and $m\perp (AC)$, we get that $(AC)\z\parallel (AB)$ 
(see Exercise~\ref{ex:perp-perp}).
Therefore, by Theorem~\ref{perp:ex+un}, $(AC)=(AB)$;
that is, $\triangle ABC$ is degenerate --- a contradiction.
\qeds

\begin{thm}{Exercise}\label{ex:unique-cline}
Show that there is a unique circle that passes thru the vertices of a given nondegenerate triangle in the Euclidean plane.
\end{thm}



\section{Altitudes and orthocenter}

An \index{altitude}\emph{altitude} of a triangle is a line thru a vertex and perpendicular to the line containing the opposite side.
The term \index{altitude}\emph{altitude} may also be used for the distance from the vertex to its footpoint on the line containing the opposite side.

\begin{thm}{Theorem}\label{thm:orthocenter}
Three altitudes of every nondegenerate triangle intersect at a single point.
\end{thm}

The point of intersection of altitudes is called the \index{orthocenter}\emph{orthocenter}; 
it is usually denoted by~$H$.

{

\begin{wrapfigure}{o}{34mm}
\vskip-4mm
\centering
\includegraphics{mppics/pic-104}
\end{wrapfigure}

\parit{Proof.}
Fix a nondegenerate triangle $A B C$.
Consider three lines $\ell$, $m$, and $n$
such that 
\begin{align*}
\ell&\parallel(BC),
&
m&\parallel(CA),
&
n&\parallel(AB),
\\
\ell&\ni A,
&
m&\ni B,
&
n&\ni C.
\end{align*}
Since $\triangle A B C$ is nondegenerate,
no pair of the lines $\ell$, $m$, and $n$ is parallel.
Let $A'$, $B'$, and $C'$ be the points of intersection of
$m$ and $n$, $n$ and $\ell$, and $\ell$ and $m$, respectively.

}

Note that $\square A B C B'$, $\square B C A C'$, and $\square C A B A'$, are parallelograms.
Applying Lemma~\ref{lem:parallelogram} we get that $\triangle ABC$ is the \index{medial triangle}\emph{medial triangle} of $\triangle A' B' C'$;
that is, $A$, $B$, and $C$ are the midpoints of $[B' C']$, $[C' A']$, and $[A' B']$ respectively.

By Exercise~\ref{ex:perp-perp},
$(B' C')\parallel (BC)$,
the altitude from $A$ is perpendicular to $[B' C']$, 
and from above it bisects~$[B' C']$.

Hence the altitudes of $\triangle A B C$ 
are also perpendicular bisectors of $\triangle A' B' C'$.
Applying Theorem~\ref{thm:circumcenter}, we get that altitudes of $\triangle ABC$ intersect at one point.
\qeds

\begin{thm}{Exercise}\label{ex:orthic-4}
Assume that the orthocenter $H$ of $\triangle ABC$ is distinct from its vertices.
Show that $A$ is the orthocenter of $\triangle H B C$.
\end{thm}

\begin{thm}{Exercise}\label{ex:orthic-sim}
Let $A'$, $B'$, and $C'$ be the footpoints of the corresponding altitudes of an acute triangle $ABC$.
Show that
\[
\triangle ABC \sim \triangle A'B'C \sim \triangle AB'C' \sim \triangle A'BC'.
\]
\end{thm}



\section{Medians and centroid}

A median of a triangle is the segment joining a vertex to the midpoint of the opposing side. 

\begin{thm}{Theorem}\label{thm:centroid}
Three medians of every nondegenerate triangle intersect at a single point.
Moreover, the point of intersection divides each median in the ratio 2:1.
\end{thm}

The point of intersection of medians is called the \index{centroid}\emph{centroid} of the triangle; 
it is usually denoted by~$M$.
In the proof, we will apply exercises \ref{ex:chevinas} and \ref{ex:smililar+parallel}; their complete solutions are given in the hints.

\parit{Proof.}
Consider a nondegenerate triangle $A B C$.
Let $[A A']$ and $[B B']$ be its medians.
According to Exercise~\ref{ex:chevinas}, 
$[A A']$ and $[B B']$ have a point of intersection;
denote it by $M$.

\begin{wrapfigure}{o}{36mm}
\vskip-4mm
\centering
\includegraphics{mppics/pic-106}
\end{wrapfigure}

Draw a line $\ell$ thru $A'$ parallel to $(BB')$.
Applying Exercise~\ref{ex:smililar+parallel} for $\triangle BB'C$ and $\ell$, we get that $\ell$ crosses $[B'C]$ at a point, say $X$, and
\[\frac{CX}{CB'}=\frac{CA'}{CB}=\frac12;\]
that is, $X$ is the midpoint of $[CB']$.

Since $B'$ is the midpoint of $[AC]$ and $X$ is the midpoint of $[B'C]$, we get that 
\[\frac{AB'}{AX}=\frac23.\]

Applying Exercise~\ref{ex:smililar+parallel} for $\triangle XA'A$ and the line $(BB')$, we get that 
\[\frac{AM}{AA'}=\frac{AB'}{AX}=\frac23;\eqlbl{eq:2/3}\]
that is, $M$ divides $[AA']$ in the ratio 2:1.

Condition \ref{eq:2/3} uniquely defines $M$ on $[AA']$.
Repeating the same argument for medians $[AA']$ and $[CC']$, we get that they intersect at~$M$ as well,
hence the result.
\qeds


\begin{thm}{Exercise}\label{ex:midle}
Let $\square ABCD$ be a nondegenerate quadrangle
and $X$, $Y$, $V$, and~$W$ be midpoints of 
$[AB]$, $[BC]$, $[CD]$, and~$[DA]$ respectively.
Show that $\square XYVW$ is a parallelogram.
\end{thm}

\begin{thm}{Advaneced exercise}\label{ex:euler-line}
Show that the orthocenter $H$, centroid $M$, and circumcenter $O$ of every nondegenerate triangle $ABC$ lie on a single line.
Moreover, the centroid divides the segment $[HO]$ in the ratio $2:1$.
\end{thm}

The line in this exercise is called \emph{Euler's line}.


\section{Angle bisectors}

If $\measuredangle A B X\equiv-\measuredangle C B X$, 
then we say that the line $(BX)$ {}\emph{bisects} $\angle ABC$,
or the line $(BX)$ is а \index{bisector!angle bisector}\emph{bisector} of $\angle ABC$.
If $\measuredangle A B X\equiv\pi-\measuredangle C B X$, then the line $(BX)$ is called the \index{bisector!external bisector}\emph{external bisector} of $\angle ABC$.


\begin{wrapfigure}{o}{42mm}
\centering
\includegraphics{mppics/pic-108}
\end{wrapfigure}

If $\measuredangle ABA'=\pi$;
that is, if $B$ lies between $A$ and $A'$,
then the bisector of $\angle ABC$ is the external bisector of $\angle A' B C$ and the other way around.

Note that the bisector and the external bisector are uniquely defined by the angle.

\begin{thm}{Classroom exercise}\label{ex:perp-bisectors}
Show that for every angle, its bisector and external bisector are perpendicular.
\end{thm}

The bisectors of  $\angle ABC$, $\angle BCA$, and $\angle CAB$ of a nondegenerate triangle $A B C$
are called \index{bisector!of the triangle}\emph{bisectors of the triangle} $A B C$ at vertices $A$, $B$, and $C$ respectively.

\begin{thm}{Exercise}\label{ex:bisect=altitude}
Assume that, at one vertex of a nondegenerate triangle, its bisector coincides with its altitude.
Show that  the triangle is isosceles.
\end{thm}

\begin{thm}{Lemma}\label{lem:bisect-ratio}
Let $\triangle A B C$ be  a nondegenerate triangle.
Assume that the bisector at $A$ 
intersects $[BC]$ at~$D$.
Then 
$$\frac{AB}{AC}=\frac{DB}{DC}.
\eqlbl{bisect-ratio}$$

\end{thm}

\begin{wrapfigure}{r}{28mm}
\vskip-6mm
\centering
\includegraphics{mppics/pic-110}
\end{wrapfigure}

\parit{Proof.}
Let $\ell$ be a line passing thru $C$ that is parallel to~$(AB)$.
Note that the lines $\ell$ and $(AD)$ are not parallel;
let $E$ be their point of intersection.

Note also that $B$ and $C$ lie on opposite sides of~$(AD)$.
By the transversal property (\ref{thm:parallel-2}),
$$\measuredangle BAD=\measuredangle CED.\eqlbl{eq:<BAD=<CED}$$

Furthermore, the angles $ADB$ and $EDC$ are vertical; by \ref{prop:vert} we have
$$\measuredangle ADB=\measuredangle EDC.$$

By the AA similarity condition, 
$\triangle ABD\sim \triangle ECD$.
In particular, 
$$\frac{AB}{EC}=\frac{DB}{DC}.\eqlbl{eq:AB/EC=DB/DC}$$

Since $(AD)$ bisects $\angle BAC$, we get that
$\measuredangle BAD=\measuredangle DAC$.
Together with \ref{eq:<BAD=<CED},
it implies that 
$\measuredangle CEA=\measuredangle EAC$.
By Theorem~\ref{thm:isos}, $\triangle ACE$ is isosceles; 
that is, $$EC=AC.$$
Together with \ref{eq:AB/EC=DB/DC}, it implies \ref{bisect-ratio}.
\qeds 

\begin{thm}{Exercise}\label{ex:ext-disect}
Formulate and prove an analog of Lemma~\ref{lem:bisect-ratio} for the external bisector.
\end{thm}

\begin{thm}{Exercise}\label{ex:bisect=median} 
Assume that an angle bisector of a nondegenerate triangle bisects the opposite side. 
Show that the triangle is isosceles.
\end{thm}

{

\begin{wrapfigure}{r}{38mm}
\vskip-5mm
\centering
\includegraphics{mppics/pic-118}
\end{wrapfigure}

\begin{thm}{Exercise}\label{ex:bisector-parallel} 
Assume that the bisector at $A$ of a nondegenerate triangle $ABC$ intersects $[BC]$ at $D$;
a line thru $D$ and parallel to $(CA)$ intersects $(AB)$ at $E$;
a line thru $E$ and parallel to $(BC)$ intersects $(AC)$ at $F$.

Show that 
$AE=FC$.

\end{thm}

}

\section{The equidistant property}

Recall that the distance from a line $\ell$ to a point $P$ is defined as the distance from $P$ to its footpoint on $\ell$; see Section~\ref{sec:perp<oblique}. 

\begin{thm}[\abs]{Proposition}\label{prop:angle-bisect-dist}
Assume $\triangle ABC$ is not degenerate.
Then a point $X$ lies on a bisector or external bisector of $\angle ABC$
if and only if $X$ is equidistant from the lines $(AB)$ and $(BC)$.
\end{thm}


\parit{Proof.}
We can assume that $X$ does not lie on the union of $(AB)$ and $(BC)$.
Otherwise, the distance to one of the lines vanishes;
in this case, $X=B$ is the only point equidistant from the two lines.

Let $Y$ and $Z$ be the reflections of $X$ across $(AB)$ and $(BC)$ respectively.
Note that 
\[Y\ne Z.\]
Otherwise, both lines $(AB)$ and $(BC)$ are perpendicular bisectors of $[XY]$.
Hence $(AB)=(BC)$, but this is impossible since $\triangle ABC$ is not degenerate.

By Proposition~\ref{prop:reflection}, $XB=YB=ZB$.
Note that $X$ is equidistant from $(AB)$ and $(BC)$ if and only if $XY\z=XZ$.


{

\begin{wrapfigure}{r}{24mm}
\centering
\includegraphics{mppics/pic-112}
\end{wrapfigure}

Applying SSS and then SAS, we get that
$$\begin{aligned}
XY&=XZ
\\
&\hskip0.4mm\Updownarrow
\\
\triangle XBY&\cong\triangle XBZ
\\
&\hskip0.4mm\Updownarrow
\\
\measuredangle XBY&\equiv \pm \measuredangle XBZ.
\end{aligned}
$$

Since $Y\ne Z$, we get that $\measuredangle XBY\ne \measuredangle XBZ$.
Therefore $X$ is equidistant from $(AB)$ and $(BC)$ if and only if
\[\measuredangle XBY\equiv -\measuredangle XBZ.
\eqlbl{eq:iff-chain}\]

}

By Proposition~\ref{prop:reflection}, $A$ lies on the bisector of $\angle XBY$,
and $C$ lies on the bisector of $\angle XBZ$; that is,
\begin{align*}
2\cdot \measuredangle XBA&\equiv \measuredangle XBY,
&
2\cdot \measuredangle XBC&\equiv \measuredangle XBZ.
\end{align*}
By \ref{eq:iff-chain},
\[2\cdot \measuredangle XBA\equiv -2\cdot \measuredangle XBC.\]
The last identity means either
\[
\measuredangle XBA+\measuredangle XBC\equiv 0
\quad
\text{or}
\quad
\measuredangle XBA+\measuredangle XBC\equiv \pi
\]
--- hence the result.
\qeds

\section{The incenter}


\begin{thm}[\abs]{Theorem}\label{thm:incenter}
Angle bisectors of every nondegenerate triangle intersect at one point.
\end{thm}

The point of intersection of bisectors is called the \index{incenter}\emph{incenter} of the triangle, 
usually denoted by~$I$.
The point $I$ lies at the same distance from each side.
In particular, it is the center of a circle tangent to each side of the triangle.
This circle is called 
the \index{incircle}\emph{incircle} and its radius is called 
the \index{inradius}\emph{inradius} of the triangle.

\parit{Proof.} 
Let $\triangle ABC$ be a nondegenerate triangle.

Points $B$ and $C$ lie on opposite sides of the bisector of $\angle BAC$.
Hence this bisector intersects $[BC]$ at a point, say~$A'$.

Analogously, there is $B'\in[AC]$ 
such that $(BB')$ bisects $\angle ABC$.


Applying Pasch's theorem (\ref{thm:pasch}) twice
for the triangles $AA'C$ and $BB'C$,
we get that $[AA']$ and $[BB']$ intersect.
Suppose that $I$ denotes the point of intersection.

{

\begin{wrapfigure}{o}{28mm}
\vskip0mm
\centering
\includegraphics{mppics/pic-114}
\end{wrapfigure}

Let $X$, $Y$, and $Z$ be the footpoints of $I$ on  $(B C)$, $(C A)$, and $(A B)$ respectively.
Applying Proposition~\ref{prop:angle-bisect-dist}, we get that
$$I Y=I Z=I X.$$
From the same lemma, we get that $I$ lies on the bisector or on the exterior bisector of $\angle B C A$.

The line $(C I)$ intersects $[B B']$;
points $B$ and $B'$ lie on opposite sides of~$(C I)$.
Therefore, the angles $I C B'$ and $I C B$ have opposite signs.
Note that $\angle I C A=\angle I C B'$.
Therefore, $(C I)$ cannot be the exterior bisector of $\angle B C A$.
Hence the result.
\qeds

}

{

\begin{wrapfigure}{r}{30mm}
\centering
\vskip-2mm
\includegraphics{mppics/pic-116}
\end{wrapfigure}

\begin{thm}{Exercise}\label{ex:2x=b+c-a}
Assume sides $[B C]$, $[C A]$, and $[A B]$ of $\triangle A B C$ are tangent to the incircle at $X$, $Y$, and $Z$ respectively. 
Show that 
$$AY=AZ= \tfrac12\cdot(A B+ A C- B C).$$

\end{thm}

Assume that footpoints $A'$, $B'$, and $C'$ of the altitudes of a given triangle $ABC$ are all mutually distinct.
Then $\triangle A'B'C'$ is called an \index{triangle!orthic triangle}\index{orthic triangle}\emph{orthic triangle} of $\triangle ABC$.

}

\begin{thm}{Exercise}\label{ex:orthic-triangle}
Prove that an orthocenter of an acute triangle coincides with an incenter of its orthic triangle.

What should be an analog of this statement for an obtuse triangle?
\end{thm}

\begin{thm}{Exercise}\label{ex:bisector-incenter}
Let $I$ be the intersection of angle bisectors at $A$ and $B$ of a nondegenerate triangle $ABC$.
Denote by $D$ the intersection of the angle bisector at $A$ with the side $[BC]$.
Show that $\frac{AI}{DI}=\frac{b+c}{a}$,
where $a=BC$, $b=CA$, and $c=AB$.

Use it to build another proof of \ref{thm:incenter}.
\end{thm}




